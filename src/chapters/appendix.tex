\chapter{Appendix}

\section{Properties Subtyping}\label{sec:app:proofs}
%% ======================================================================

\subsection{Decidable Subtyping}\label{subsec:app:proofs:dec}
%% *********************************************************

\begin{lemma}[Context weakening in $\msrop$ (\textbf{\lemref{lem:msr-weakening}})]%
\label{lem:msr-weakening:app}
    The measure of a type signature does not change if the environment
    is extended (in any position) with a variable that occurs neither
    in the signature nor in the environment, i.e.,
    $\forall \tysig, \AEnv, \AEnv'. 
    \forall \varbound{\vany}{\tylb}{\tyub} \text{ with } 
    \lnot \occdflt{\tysig} \text{ and } 
    \lnot \occdflt{\AEnv} \text{ and } \lnot \occdflt{\AEnv'}.$
    \[\tymsr{\concat{\AEnv}{\AEnv'}}{\tysig} = 
        \tymsr{\concat{\AEnv,\varbound{\vany}{\tylb}{\tyub}}{\AEnv'}}{\tysig},\]
    where \concat{\AEnv}{\AEnv''} denotes the concatenation of lists,
    and $\occop$ is defined in~\figref{fig:var-occ}.
\end{lemma}
\begin{proof}
    By strong induction on $n = \size{\AEnv} + \size{\AEnv'} + \size{\tysig}$.

    Case $n = 0$ is not possible: the minimal size of a type signature is 1.

    Base cases for \tyany and \tybot are straightforward; in the base case
    for $\vany'$, environment $\concat{\AEnv}{\AEnv'}$ is empty and
    $\vany \neq \vany'$, 
    meaning that \[\tymsr{\EmptyEnv}{\vany'} = 1 = 
    \tymsr{\varbound{\vany}{\tylb}{\tyub}}{\vany'}.\]
    
    In the inductive step for $n$, the induction hypothesis (IH) states that
    $\forall n'<n. \forall \tysig', \AEnv'', \AEnv'''  \text{ with }
    n' = \size{\AEnv''} + \size{\AEnv'''} + \size{\tysig'}.
    \forall \varbound{\vany}{\tylb}{\tyub} \text{ with } 
    \lnot \occdflt{\tysig'} \text{ and } 
    \lnot \occdflt{\AEnv''} \text{ and } \lnot \occdflt{\AEnv'''}.$
    \[\tymsr{\concat{\AEnv''}{\AEnv'''}}{\tysig'} = 
    \tymsr{\concat{\AEnv'',\varbound{\vany}{\tylb}{\tyub}}{\AEnv'''}}{\tysig'}.\]
    
    Case analysis on \tysig. Base cases \tyany and \tybot are straightforward.
    Cases $\times$, \tyinv\iname{\ldots}, and $\cup$ are also straightforward
    using the induction hypothesis for components of \tysig.
    The remaining cases are:
    \begin{itemize}
        \item Case $\vany'$. Case analysis on $\AEnv'$.
            \begin{itemize}
                \item Case \EmptyEnv. Because $\lnot \occdflt{\vany'},$ we know
                    $\vany \neq \vany'$. Thus,
                    $\tymsr{\AEnv, \varbound{\vany}{\tylb'}{\tyub'}}{\vany'} =
                    \tymsr{\AEnv}{\vany'}$ by definition of $\msrop$.
                \item Case $\AEnv', \varbound{\vany'}{\tylb'}{\tyub'}$.
                    By definition,
                    \[\tymsr{\concat{\AEnv}{\AEnv', \varbound{\vany'}{\tylb'}{\tyub'}}}{\vany'} =
                    1 + \tymsr{\concat{\AEnv}{\AEnv'}}{\tylb'} + 
                    \tymsr{\concat{\AEnv}{\AEnv'}}{\tyub'}.\]
                    Since $\size{\AEnv} + \size{\AEnv'} + \size{\tylb'}\ <\ 
                    \size{\AEnv} + \size{\AEnv', \varbound{\vany'}{\tylb'}{\tyub'}} + \size{\vany'} =
                    \size{\AEnv} + \size{\AEnv'} + \size{\tylb'} + \size{\tyub'} + 1$,
                    the IH applies with $\AEnv'' = \AEnv, \AEnv''' = \AEnv', 
                    \tysig' = \tylb'$, which gives 
                    $\tymsr{\concat{\AEnv}{\AEnv'}}{\tylb'} = 
                    \tymsr{\concat{\AEnv, \varbound{\vany}{\tylb}{\tyub}}{\AEnv'}}{\tylb'}$,
                    and similarly for $\tyub'$. Thus,
                    \[ \tymsr{\concat{\AEnv}{\AEnv', \varbound{\vany'}{\tylb'}{\tyub'}}}{\vany'} =
                    \tymsr{\concat{\AEnv, \varbound{\vany}{\tylb}{\tyub}}{\AEnv', \varbound{\vany'}{\tylb'}{\tyub'}}}{\vany'}. \]
            \end{itemize}
        \item Case \tyexist{\vany'}{\tylb'}{\tyub'}{\tysig}.
            By definition,
            \[
                \begin{array}{c}
                    \tymsr{\concat{\AEnv}{\AEnv'}}{\tyexist{\vany'}{\tylb'}{\tyub'}{\tysig}} \\
                    = \\
                    1 + \tymsr{\concat{\AEnv}{\AEnv'}}{\tylb'} + \msrop(\ldots\tyub') +
                    \tymsr{\concat{\AEnv}{\AEnv', \varbound{\vany'}{\tylb'}{\tyub'}}}{\tysig}.
                \end{array}    
            \]
            Similarly to the last subcase of the $\vany'$ case, the IH applies
            to $\tylb'$ and $\tyub'$.
            Furthermore, since $\size{\AEnv} + \size{\AEnv'} + 
            \size{\tylb'} + \size{\tyub'} + \size{\tysig} <
            \size{\AEnv} + \size{\AEnv'} + 1 + \size{\tylb'} + \size{\tyub'}
            + \size{\tysig},$
            the IH applies to \tysig with $\AEnv'' = \AEnv, 
            \AEnv''' = (\AEnv', \varbound{\vany'}{\tylb'}{\tyub'}), 
            \tysig' = \tysig$.
            All pieces combined, 
            \[\tymsr{\concat{\AEnv}{\AEnv'}}{\tyexist{\vany'}{\tylb'}{\tyub'}{\tysig}} =
            \tymsr{\concat{\AEnv, \varbound{\vany}{\tylb}{\tyub}}{\AEnv'}}{\tyexist{\vany'}{\tylb'}{\tyub'}{\tysig}}.\]
            % and
            % \[\tymsr{\concat{\AEnv, \varbound{\vany}{\tylb}{\tyub}}{\AEnv'}}{\tyexist{\vany'}{\tylb'}{\tyub'}{\tysig}} =
            % 1 + \tymsr{\concat{\AEnv, \varbound{\vany}{\tylb}{\tyub}}{\AEnv'}}{\tylb'} + 
            % \tymsr{\concat{\AEnv, \varbound{\vany}{\tylb}{\tyub}}{\AEnv'}}{\tyub'} +
            % \tymsr{\concat{\AEnv, \varbound{\vany}{\tylb}{\tyub}}{\AEnv', \varbound{\vany'}{\tylb'}{\tyub'}}}{\tysig}.\]
    \end{itemize}
\end{proof}

\subsection{Unification-Free Subtyping}\label{subsec:app:proofs:subty}
%% *********************************************************

\begin{lemma}[Subtyping of \tybot implies arbitrary subtyping
    (\textbf{\lemref{lem:sub-of-bot}})]\label{lem:sub-of-bot:app}
    \[
    \forall \ty, \dctx_{\tybot}, \AEnv.\quad 
    \subtydflt{\ty}{\plug{\dctx_{\tybot}}\tybot}
    \quad\implies\quad
    (\forall \ty', \dctx'.\quad \subtydflt{\plug{\dctx'}{\ty}}{\ty'}).
    \]
\end{lemma}
\begin{proof}
    By induction on the derivation of 
    \subtydflt{\ty}{\plug{\dctx_{\tybot}}\tybot}.
    \begin{itemize}
        \item Case \RST{Bot}
            \subtydflt{\plug\dctx\tybot}{\plug{\dctx_{\tybot}}{\tybot}}
            where $\ty = \plug\dctx\tybot$.

            The case concludes by \RST{Bot}:
            \subtydflt{\plug{\dctx'}{\plug\dctx\tybot}}{\ty'}. 
        \item Case \RST{VarLeft}
            \subtydflt{\plug\dctx\vany}{\plug{\dctx_{\tybot}}{\tybot}}.

            By the form of the rule, \subtydflt{\plug\dctx\tyub}{\plug{\dctx_{\tybot}}{\tybot}}.
            By IH, \subtydflt{\plug{\dctx'}{\plug\dctx\tyub}}{\ty'}.
            Thus, the case concludes by \RST{VarLeft}: 
            \subtydflt{\plug{\dctx'}{\plug\dctx\vany}}{\ty'}.
        \item Case \RST{Tuple}, subcase where
            $\dctx_{\tybot} = \typair{\dctx'_{\tybot}}{\ty'_2}$
            ($\dctx_{\tybot} = \square$ is not possible, and
            $\dctx_{\tybot} = \typair{\ty_1}{\dctx'_{\tybot}}$
            is proved analogously),
            $\ty = \typair{\ty_1}{\ty_2}$:
            \subtydflt{\typair{\ty_1}{\ty_2}}
            {\typair{\plug{\dctx'_{\tybot}}{\tybot}}{\ty'_2}}.

            By the form of the rule, \subtydflt{\ty_1}{\plug{\dctx'_{\tybot}}{\tybot}}.
            By IH, \subtydflt{\plug{\dctx'^h}{\ty_1}}{\ty'} for all $\dctx'^h$,
            so we can take it to be \plug{\dctx'}{\typair{\square}{\ty_2}}.
            Thus, the case concludes by IH: 
            \subtydflt{\plug{\dctx'}{\typair{\ty_1}{\ty_2}}}{\ty'}.
        \item Case \RST{UnionLeft}
            \subtydflt{\plug\dctx{\tyunion{\ty_1}{\ty_2}}}{\plug{\dctx_{\tybot}}{\tybot}}
            where $\ty = \tyunion{\ty_1}{\ty_2}$.
            By the form of the rule, 
            \subtydflt{\plug\dctx{\ty_1}}{\plug{\dctx_{\tybot}}{\tybot}} and
            \subtydflt{\plug\dctx{\ty_2}}{\plug{\dctx_{\tybot}}{\tybot}}.
            By IH, \subtydflt{\plug{\dctx'}{\plug\dctx{\ty_1}}}{\ty'} and
            \subtydflt{\plug{\dctx'}{\plug\dctx{\ty_2}}}{\ty'}.
            Thus, the case concludes by \RST{UnionLeft}: 
            \subtydflt{\plug{\dctx'}{\plug\dctx{\tyunion{\ty_1}{\ty_2}}}}{\ty'}.
    \end{itemize}
    The remaining cases 
    (\RST{Top}, \RST{VarRefl}, \RST{VarRight}, \RST{Inv}, \RST{UnionRight}) 
    are not possible.
\end{proof}

\begin{lemma}[Subtyping of inner union on the right
    (\textbf{\lemref{lem:sub-inner-union-right}})]%
\label{lem:sub-inner-union-right:app}
    $\forall \ty, \dctx', \ty'_1, \ty'_2, \AEnv, \text{ with }$\\
    $\tyvld{}{\AEnv}\ \text{ and }\ \tyvlddflt{\ty, \dctx', \ty'_1, \ty'_2}.$
    \[
        \begin{array}{ccc}
        \subtydflt{\ty}{\plug{\dctx'}{\tyunion{\ty'_1}{\ty'_2}}}\\
        \quad\implies\quad\\
        (\forall \dctx_1, \dctx_2, \text{ with }
        \tyvlddflt{\dctx_1, \dctx_2} \text{ and }
        \subtydflt{\dctx_1}{\dctx_2}.\quad
        \subtydflt
            {\plug{\dctx_1}{\ty}}
            {\tyunion
                {\plug{\dctx_2}{\plug{\dctx'}{\ty'_1}}}
                {\plug{\dctx_2}{\plug{\dctx'}{\ty'_2}}}
            }).
        \end{array}
    \]
\end{lemma}
\begin{proof}
    By induction on the derivation of
    \subtydflt{\ty}{\plug{\dctx'}{\tyunion{\ty'_1}{\ty'_2}}}.
    \begin{itemize}
        \item Case \RST{Bot} by \RST{Bot}.
        \item Case \RST{VarLeft} by the form of the rule, IH, and \RST{VarLeft}.
        \item Case \RST{Tuple}, subcase where
            $\dctx' = \typair{\dctx''}{\ty'}$:
            \subtydflt{\typair{\ty_1}{\ty_2}}
                {\typair{\plug{\dctx''}{\tyunion{\ty'_1}{\ty'_2}}}{\ty'}}.
            By the form of the rule,
            \subtydflt{\ty_1}{\plug{\dctx''}{\tyunion{\ty'_1}{\ty'_2}}} and
            \subtydflt{\ty_2}{\ty'}. 
            
            By IH applied to 
            \subtydflt{\ty_1}{\plug{\dctx''}{\tyunion{\ty'_1}{\ty'_2}}},
            \subtydflt{\plug{\dctx^h_1}{\ty_1}}
                {\tyunion
                    {\plug{\dctx^h_2}{\plug{\dctx''}{\ty'_1}}}
                    {\plug{\dctx^h_2}{\plug{\dctx''}{\ty'_2}}}
                }
            for all $\dctx^h_1, \dctx^h_2$ s.t. $\subtydflt{\dctx^h_1}{\dctx^h_2}$.
            Thus, we can take them to be \plug{\dctx_2}{\typair{\square}{\ty_2}} 
            and \plug{\dctx_2}{\typair{\square}{\ty'}}, 
            respectively, which concludes the case with
            \subtydflt{\plug{\dctx_1}{\typair{\ty_1}{\ty_2}}}
                {\tyunion
                    {\plug{\dctx_2}{\typair{\plug{\dctx''}{\ty'_1}}{\ty_2}}}
                    {\plug{\dctx_2}{\typair{\plug{\dctx''}{\ty'_2}}{\ty'}}}
                }
        \item Case \RST{UnionLeft} by the form of the rule, IH, and \RST{UnionLeft}.
        \item Case \RST{UnionRight}, subcase $i = 1$ where $\dctx' = \square$:
            \subtydflt{\ty}{\tyunion{\ty'_1}{\ty'_2}}.
            By the form of the rule, \subtydflt\ty{\ty'_1}.
            By assumption, \subtydflt{\dctx_1}{\dctx_2}, and thus,
            \subtydflt{\plug{\dctx_1}{\ty}}{\plug{\dctx_2}{\ty'_1}}.
            The case concludes by \RST{UnionRight} with $i=1$:
            \subtydflt{\plug{\dctx_1}{\ty}}
                {\tyunion{\plug{\dctx_2}{\ty'_1}}{\plug{\dctx_2}{\ty'_2}}}.
    \end{itemize}
    The remaining cases 
    (\RST{Top}, \RST{VarRefl}, \RST{VarRight}, \RST{Inv}) 
    are not possible.
\end{proof}

\begin{lemma}[Adding inner union on the right (\textbf{\lemref{}})]%
\label{lem:add-inner-union-right:app}
    $\forall \ty, \dctx', \ty', \AEnv, \text{ with }
    tyvld{}{\AEnv}\ \text{ and }\ \tyvlddflt{\ty, \dctx', \ty'}.$
    \[
        \subtydflt{\ty}{\plug{\dctx'}{\ty'}}
        \quad\implies\quad
        (\forall \ty''.\ \subtydflt{\ty}{\plug{\dctx'}{\tyunion{\ty'}{\ty''}}}).
    \]
\end{lemma}
\begin{proof}
    By induction on the derivation of
    \subtydflt{\ty}{\plug{\dctx'}{\ty'}}.
    \begin{itemize}
        \item Case \RST{Top} where $\dctx'=\square$. By assumption
            \subtydflt{\ty}{\tyany} and \RST{UnionRight} with $i=1$,
            \subtydflt{\ty}{\tyunion{\tyany}{\ty''}}.
        \item Case \RST{Bot} by \RST{Bot}.
        \item Case \RST{VarRefl} where $\dctx'=\square$
            by assumption and \RST{UnionRight} with $i=1$.
        \item Case \RST{VarLeft} by the form of the rule, IH, and \RST{VarLeft}.
        \item Case \RST{VarRight} where $\dctx'=\square$
            by assumption and \RST{UnionRight} with $i=1$.
        \item Case \RST{Tuple}. 
            Subcase $\dctx'$ by assumption and \RST{UnionRight} with $i=1$.
            The other two subcases by the form of the rule, IH, and \RST{Tuple}.
        \item Case \RST{Inv} where $\dctx'=\square$
            by assumption and \RST{UnionRight} with $i=1$.
        \item Case \RST{UnionLeft} by the form of the rule, IH, and \RST{UnionLeft}.
        \item Case \RST{UnionRight} where $\dctx'=\square$
            by assumption and \RST{UnionRight} with $i=1$.
    \end{itemize}
\end{proof}

\clearpage
\section{Semantic Model}\label{sec:app:sem-sub}
%% ======================================================================

Julia's subtype relation was inspired by semantic subtyping.
The description of the original Julia design~\cite{bib:bezanson:julia:2015}
suggested an intuitive interpretation of types as sets of values,
but the interpretation is not well defined.
Furthermore, the treatment of invariant parametric types
in the interpretation does not match
the subtype relation. In particular, for concrete nominal \cjl{Name},
the interpretation
\[
  \interp{\text{\cjl{Name\{}}\ty\text{\cjl{\}}}} =
  \{ x\ |\ \text{\cjl{typeof}}(x) = \text{\cjl{Name\{}}\ty\text{\cjl{\}}} \}   
\]
does not account for the fact that there are multiple syntactic
representations $\ty'$ corresponding to the same interpretation as \ty.
Thus, for example, types \cjl{Vector\{Union\{Int,Any\}\}} and \cjl{Vector\{Any\}}
are not equivalent according to the interpretation,
but they are equivalent in Julia.

\subsection{Tuples and Unions}
%% *********************************************************

In \cite{bib:belyakova:minijl-sub:ftfjp:2019}, I proposed and mechanized in Coq
a semantic interpretation of (a subset of) Julia
types \ty as \textbf{sets of concrete value types} \gty
(or type tags) rather than values, with semantic subtyping defined as
$\interp{\ty} \subseteq \interp{\ty'}$.
For the type language of non-parametric nominal types, tuples, and unions,
a decidable syntactic subtype relation based on disjunctive normal form
coincides with the set inclusion of interpretations.
%$\interp{\ty} \subseteq \interp{\ty'}$.
\figref{fig:sem:ty-simple} presents the interpretation and corresponding
syntactic subtyping in the style of \chapref{chap:dec-sub}
for the type language of base types \iname, covariant tuples, and unions.
$\Sigma$ denotes the set of all value types \gty.

\begin{figure}[t]
\footnotesize
Grammar
\[\begin{array}{rcll}
    \ty &::=& 
        \tyany \Alt \tybot \Alt \iname \Alt
        \typair{\ty_1}{\ty_2} \Alt \tyunion{\ty_1}{\ty_2}
        & \textit{Type} \\
    \\
    \gty &::=& 
        \iname \Alt
        \typair{\gty_1}{\gty_2}
        & \textit{Concrete value type} \\
    \\
    \dctx &::=& \square \Alt
        \typair{\dctxty}{\ty} \Alt \typair{\ty}{\dctxty} 
        & \textit{Distributivity context} \\
\end{array}\]
Interpretation $\interp{\cdot}$
\[
    \begin{array}{ccl}
        \interp{\tyany} &=& \Sigma \\
        \interp{\tybot} &=& \varnothing \\
        \interp{\iname} &=& \{ \iname \} \\
        \interp{\typair{\ty_1}{\ty_2}} &=& 
            \{ \typair{\gty_1}{\gty_2} \ | \ 
            \gty_1 \in \interp{\ty_1},\,\gty_2 \in \interp{\ty_2} \} \\
        \interp{\tyunion{\ty_1}{\ty_2}} &=& 
            \interp{\ty_1} \cup \interp{\ty_2} \\
    \end{array}
\]
Subtyping $\ty \symsub \ty'$\\
\makebox[\textwidth][c]{
\begin{minipage}{\ruleswidth}  
\begin{mathpar}
    \inferrule[]
    { }
    { \ty \symsub \tyany }

    \inferrule[]
    { }
    { \plug\dctx\tybot \symsub \ty' }

    \inferrule[]
    {  }
    { \iname \symsub \iname }
    \\

    \inferrule[]
    { \ty_1 \symsub \ty'_1 \and 
        \ty_2 \symsub \ty'_2 }
    { \typair{\ty_1}{\ty_2} \symsub \typair{\ty'_1}{\ty'_2} }

    \inferrule[]
    { \plug\dctx{\ty_1} \symsub \ty' \and 
        \plug\dctx{\ty_2} \symsub \ty' }
    { \plug\dctx{\tyunion{\ty_1}{\ty_2}} \symsub \ty' }

    \inferrule[]
    { \exists i.\ \ty \symsub \ty'_i }
    { \ty \symsub \tyunion{\ty'_1}{\ty'_2} }
\end{mathpar}
\end{minipage}
}
\caption{Semantic interpretation and decidable subtyping for simple types
}\label{fig:sem:ty-simple}
\end{figure}

In \cite{bib:belyakova:minijl-sub:ftfjp:2019}, I show that the decidable 
syntactic subtype relation is sound and complete with respect
to the interpretation.

\begin{theorem}[]
    $\forall \ty, \ty'.$
    \[
        \begin{array}{c}
            \ty <: \ty' \\
            \iff \\
            (\forall \gty.\quad \gty \in \interp{\ty} \iff
                \gty \in \interp{\ty'}).
        \end{array}
    \]
\end{theorem}

\subsection{Invariant Constructors}
%% *********************************************************

The interpretation in \figref{fig:sem:ty-simple} can be extended to support
invariant type constructors such as \cjl{Vector\{...\}}.
To account for the issue of multiple syntactic representations of types,
in the new system, types are given an indexed interpretation $\interp{\cdot}_n$.
This approach was also mechanized in Coq.

\begin{figure}[t]
\footnotesize
Grammar
\[\begin{array}{rcll}
    \ty &::=& \ldots \Alt \tyinv\iname{\ty_1,\ldots,\ty_n}
        & \textit{Type} \\
    \\
    \gty &::=& \ldots \Alt \tyinv\iname{\ty_1,\ldots,\ty_n}
        & \textit{Concrete value type} \\
\end{array}\]
Interpretation $\interp{\cdot}_w$
\[
    \begin{array}{ccl}
        \interp{\tyany}_w &=& \Sigma \\
        \interp{\tybot}_w &=& \varnothing \\
        \interp{\typair{\ty_1}{\ty_2}}_w &=& 
            \{ \typair{\gty_1}{\gty_2} \ | \ 
            \gty_1 \in \interp{\ty_1}_w,\,\gty_2 \in \interp{\ty_2}_w \} \\
        \interp{\tyunion{\ty_1}{\ty_2}}_w &=& 
            \interp{\ty_1}_w \cup \interp{\ty_2}_w \\
        \\
        \interp{\tyinv\iname{\ty_1,\ldots,\ty_n}}_0 &=& 
            \{ \tyinv\iname{\ty'_1,\ldots,\ty'_n} \} \\
        \interp{\tyinv\iname{\ty_1,\ldots,\ty_n}}_{w+1} &=& 
            \{ \tyinv\iname{\ty'_1,\ldots,\ty'_n} \ |\ 
                \forall i.\; \interp{\ty'_i}_w = \interp{\ty_i}_w \} \\
    \end{array}
\]
Subtyping $\ty \symsub \ty'$\\
\makebox[\textwidth][c]{
\begin{minipage}{\ruleswidth}  
\begin{mathpar}
    ...

    \inferrule[]
    { \ty_i \symsub \ty'_i \and \ty'_i \symsub \ty_i }
    { \tyinv\iname{\ty_1,\ldots,\ty_n} \symsub 
        \tyinv\iname{\ty'_1,\ldots,\ty'_n} }
\end{mathpar}
\end{minipage}
}
\caption{Semantic interpretation and decidable subtyping for
invariant constructors}\label{fig:sem:ty-inv}
\end{figure}

\figref{fig:sem:ty-inv} defines the extension.
Intuitively, the $0$-interpretation is not allowed to look inside invariant
type constructor and contains all value types with the same constructor name.
The higher the interpretation, the more spurious elements are filtered out;
since the equality of the arguments is checked in the lower interpretation,
the definition is well formed.
Then, semantic subtyping is defined as:
\[
  \forall w.\ \interp{\ty}_w \subseteq \interp{\ty'}_w.
\]
Despite the more complex indexed interpretation, the syntactic subtype relation
is still sound and complete.

\begin{theorem}[]
    $\forall \ty, \ty'.$
    \[
        \begin{array}{c}
            \ty <: \ty' \\
            \iff \\
            (\forall w.\ \forall \gty.\quad \gty \in \interp{\ty}_w \iff
                \gty \in \interp{\ty'}_w).
        \end{array}
    \]
\end{theorem}

\subsection{Existential Types}
%% *********************************************************

To match the intuition described in \chapref{chap:julia-sub},
an existential type and type variable can be interpreted as:
\[
\begin{array}{ccl}
    \interpexdflt{\tyexist{\vx}{\tylb}{\tyub}{\ty}} &=& 
        \bigcup_{\interpexdflt{\tylb}\, \subseteq\, \gtyset\,
            \subseteq\, \interpexdflt{\tyub}}
        \interp{\ty}^{\subst{\semsubstvars}{\substel{\vx\,}{\,\gtyset}}}_w \\
    \interpexdflt{\vx} &=& \semsubstvars(\vx), \\
\end{array}
\]
where %$\gtyset \subseteq \Sigma$ and 
\semsubstvars maps variables to interpretation sets.

However, because of invariant constructors, an interpretation cannot be simply 
a set of value type tags\footnote{Tags are assumed to be closed, i.e. do not 
contain free variables.}.
Consider the following example:
\[
    \begin{array}{cl}
        & \interp{\tyexist{\vx}{\tyinv\nref\tyint}{\tyinv\nref\tyint}
            {\tyinv\nvec\vx}}_1 \\
        = & \\
        & \bigcup_{\gtyset = \interp{\tyinv\nref\tyint}_1}
        \interp{\tyinv\nvec\vx}^{\subst{}{\substel{\vx\,}{\,\gtyset}}}_1 \\
        = & \\
        & \{\ 
            \tyinv\nvec\ty \ |\ \interp{\ty}_0 = 
            \interp{\tyinv\nref\tyint}_1
        \ \} \\
        = & \\
        & \{\ 
            \tyinv\nvec\ty \ |\ \interp{\ty}_0 = 
            \{ \tyinv\nref{\ty'}\,|\, \interp{\ty'}_0 = \{ \tyint \} \}
        \ \} \\
        = & \\
        & \{\ 
            \tyinv\nvec\ty \ |\ \interp{\ty}_0 = 
            \{ \tyinv\nref{\tyint}, \tyinv\nref{\tyunion{\tyint}{\tybot}}, \ldots \}
        \ \} \\
    \end{array}
\]
Note that there are no closed types \ty such that
\[
    \interp{\ty}_0 = 
    \{ \tyinv\nref{\tyint}, \tyinv\nref{\tyunion{\tyint}{\tybot}}, 
        \tyinv\nref{\tyunion{\tyint}{\tyint}}, \ldots \},
\]
because the $0$-interpretation of any \nref-type includes arbitrary 
\tyinv\nref{\ty'}, not just \tyint-interpreted $\ty'$.
Therefore, the existential type 
\[
    \tyexist{\vx}{\tyinv\nref\tyint}{\tyinv\nref\tyint}{\tyinv\nvec\vx}
\]
is not a supertype of 
\[ \tyinv\nvec{\tyinv\nref\tyint}, \]
which does not match Julia subtyping.
The crux of the problem is that \vx can be interpreted in a smaller $w$
than it was instantiated with.

In the spirit of~\cite{bib:ahmed:impr-poly:2003},
the domain of interpretation can be redefined as a set of indexed tags
instead of just tags, as presented in \figref{fig:sem:ty-inv-exist}.
In this case, whenever a variable is interpreted in $w$,
all tags originally produced by the potentially higher interpretation
are removed. It is easy to see by induction on \ty that
\[ 
    w' \leq w \quad\implies\quad
    \interp{\ty}^{\semsubstvars}_{w'} \subseteq \interpexdflt{\ty}.
\]

The semantic interpretation can be used to examine soundness
of decidable syntactic subtyping presented in \chapref{sec:dec-sub:defs},
as well as equivalent rewritings discussed in \chapref{sec:eval:results}.

\begin{figure}[t]
\footnotesize
Grammar
\[\begin{array}{rcll}
    \ty, \tylb, \tyub &::=& \ldots \Alt \vx \Alt
        \tyexist{\vx}{\tylb}{\tyub}{\ty}
        & \textit{Type} \\
    \\
    \gty &::=& \ldots \text{ where } \fv{\gty} = \varnothing
        & \textit{Concrete value type} \\
\end{array}\]
Domain
\[
    \hat{\Sigma} = \{ \gty^k \ \Alt\ \gty \in \Sigma,\; k \in \mathbb{N} \}
\]
Interpretation $\interpexdflt{\cdot}$
\[
\begin{array}{ccl}
    \interpexdflt{\tyany} &=& \{ \gty^k \ |\ k \leq w \} \\
    \interpexdflt{\tybot} &=& \varnothing \\
    \interpexdflt{\typair{\ty_1}{\ty_2}} &=& 
        \{ {(\typair{\gty_1}{\gty_2})}^k \ | \ 
        {\gty_1}^k \in \interpexdflt{\ty_1},\,
        {\gty_2}^k \in \interpexdflt{\ty_2} \} \\
    \interpexdflt{\tyunion{\ty_1}{\ty_2}} &=& 
        \interpexdflt{\ty_1} \cup \interpexdflt{\ty_2} \\
    \\
    \interp{\tyinv\iname{\ty_1,\ldots,\ty_n}}^{\semsubstvars}_0 &=& 
        \{ \tyinv\iname{\ty'_1,\ldots,\ty'_n}^0 \} \\
    \interp{\tyinv\iname{\ty_1,\ldots,\ty_n}}^{\semsubstvars}_{w+1} &=& 
        \{ \tyinv\iname{\ty'_1,\ldots,\ty'_n}^{w+1} \ |\ 
            \forall i.\; \interpexdflt{\ty'_i} = \interpexdflt{\ty_i} \}
        \ \cup\ \interp{\tyinv\iname{\ty_1,\ldots,\ty_n}}^{\semsubstvars}_w \\
    \\
    \interpexdflt{\tyexist{\vx}{\tylb}{\tyub}{\ty}} &=& 
        \bigcup\limits_{0 \leq w' \leq w}
        \bigcup\limits_{\interp{\tylb}^{\semsubstvars}_{w'}
            \, \subseteq\, \gtyset\,
            \subseteq\, \interp{\tyub}^{\semsubstvars}_{w'}}
        \interp{\ty}^{\subst{\semsubstvars}{\substel{\vx\,}{\,\gtyset}}}_{w'} \\
    \interpexdflt{\vx} &=& \{ \gty^k \ |\ 
        \gty^k \in \semsubstvars(\vx)\ \text{ and }\ k \leq w \} \\
\end{array}
\]
\caption{Semantic interpretation for
existential types}\label{fig:sem:ty-inv-exist}
\end{figure}


\clearpage
\section{Evaluation}\label{sec:app:eval}
%% ======================================================================

\begin{figure}
\small
%\makebox[\textwidth][c]{
%\begin{minipage}{\ruleswidth}
\begin{minipage}{6.5cm}
    \centering
    Satisfy the restriction
\begin{julia}
Pair{T, T} where T
Tuple{T, Tuple{T, Int}} where T
Tuple{Ref{T}} where T
Tuple{T, Ref{T}} where T
Tuple{Ref{Tuple{T}}} where T
Vector{Union{T, Int}} where T
Ref{Ref{T} where T} where T
Ref{Pair{T, S} where S} where T
Pair{S, Pair{Int, T} where T<:S} where S
Tuple{S, Pair{T, S} where T<:S} where S
Tuple{Ref{T} where T>:S} where S
Ref{T} where T<:(Ref{S} where S)
Tuple{T} where T<:Ref{Ref{<:Any}}
\end{julia}
\end{minipage}
\hspace{1cm}
\begin{minipage}{6cm}
    \centering
    Do not satisfy the restriction
\begin{julia}
Ref{Pair{T, T} where T}
Ref{Tuple{Pair{T, T} where T}}
Tuple{T} where T>:(Pair{S,S} where S)
Vector{Vector{Union{T, Int}} where T}
Vector{Ref{Tuple{T}} where T}
\end{julia}
\end{minipage}
%\end{minipage}}
\caption{Test cases for the analysis of type annotations
}\label{fig:tests:ta-analysis}
\end{figure}

\begin{figure}
%\begin{minipage}{14cm}
\begin{lstlisting}
Error: Couldn't process expression
  e =
   :($(Expr(:$, :d))->begin
             #= none:54 =#
             Base.axes($(Expr(:$, :arraysym)), $(Expr(:$, :d)))
         end)
  err =
   ArgumentError: Not a function definition: :($(Expr(:$, :d))->begin
             #= none:54 =#
             Base.axes($(Expr(:$, :arraysym)), $(Expr(:$, :d)))
         end)
\end{lstlisting}
%\end{minipage}
\caption{An example of a parsing error}\label{fig:evaluation-parse-errors}
\end{figure}

\begin{figure}
\begin{minipage}{12cm}
\begin{lstlisting}
Error: Couldn't process type annotation
    tastr = "Tuple{Union{Document, Node}} where \$(esc.(P)...)"
    err = AssertionError: Unsupported lb-var-ub format
  
Error: Unsupported Expr type annotation
  ty = :(typeof.((year, month, day, yearmonth, ...))...)
  
Error: Couldn't process type annotation
  tastr = "(Tuple{A} where Base.IteratorSize(A)::Base.SizeUnknown) where A"
  err = AssertionError: Unsupported lb-var-ub format
  
Error: Couldn't process type annotation
  tastr = "(((Tuple{(\$T_nameparam){\$N, \$M, \$FT}} ... \$(T_params...)"
  err = AssertionError: Unsupported lb-var-ub format

Error: Couldn't process type annotation
  tastr = "Tuple{Union{map((T->beginn  #= none:302 =#n ...)...}}"
  err = Base.Meta.ParseError("missing comma or ) in argument list")
\end{lstlisting}
\end{minipage}
\caption{Type annotation processing errors}\label{fig:evaluation-process-errors}
\end{figure}
% ty = :(typeof.((year, month, day, yearmonth, monthday, yearmonthday, dayofweek))...)
% tastr = "(Tuple{A} where Base.IteratorSize(A)::Base.SizeUnknown) where A"
% tastr = "(((Tuple{(\$T_nameparam){\$N, \$M, \$FT}} where \$FT) where \$M) where \$N) where \$(T_params...)"
% tastr = "Tuple{Union{map((T->beginn                    #= none:302 =#n                    AbstractGeometry{T}n                end), multipointtypes)...}}"
  
