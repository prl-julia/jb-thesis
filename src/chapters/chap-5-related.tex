\chapter{Related work}\label{chap:5}

\TODO{Semantic subtyping and Java wildcards (several paragraphs).}

Subtyping is typically associated with object-oriented programming languages and
static typing. Within a type system, subtyping enables more flexibility by
allowing an expression of a subtype to be used in place where an expression of a
supertype is expected.


Subtyping and decidability of subtyping are clearly a problem in various languages,
but we will focus on several related subsets.

Decidable and undecidable systems. Subtyping in PL.

\section{Semantic subtyping}
%% ======================================================================

In Julia, subtyping of union and tuple types (maybe also invariant ones?) build
on top of \TODO{Voulon} and has similarities to XDuce/CDuce.
There is a large body of work on semantic subtyping, which extends it to
function and polymorphic types. However, such systems do not deal with
impredicative types.
For example, \citet{frih:sem-sub:2008} describe a framework for defining semantic
subtyping without building a model of the entire language, and then define
decidable subtyping and type checking for a language with dynamic dispatch,
unions, functions, and negation types.\TODO{clean up the description}

In the semantic subtyping approach, types are interpreted as subsets of the
model of the programming

\TODO{\cite{hosoya:xduce:2003, bezanken:cduce:2003}}
