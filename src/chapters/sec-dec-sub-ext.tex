\section{Extended Subtyping}\label{sec:dec-sub:ext}
%% ======================================================================

In this section, I discuss two Julia features that were
omitted from the core language in~\figref{fig:ty-grammar}:
nominal subtyping (\secref{subsec:dec-sub:inheritance}) and
the diagonal rule (\secref{subsec:dec-sub:diagonal}).

\subsection{Nominal subtyping}\label{subsec:dec-sub:inheritance}
%% *********************************************************

\begin{figure}[t] 
\begin{minipage}{5cm}
\begin{lstlisting}
abstract type Real <: Number
end

struct Rational{
    T<:Integer
} <: Real
    num::T
    den::T
end
\end{lstlisting}
\end{minipage}
\hspace{1cm}
\begin{minipage}{5cm}
\begin{lstlisting}
abstract type Ref{T} <: Any
end

struct RefArray{
    T, A<:AbstractArray{T}, R
} <: Ref{T}
    x::A
    ...
end
\end{lstlisting}  
\end{minipage}
\caption{Examples of inheritance in type declarations
}\label{fig:code:inheritance}
\end{figure}

The Julia language supports a limited form of single-parent inheritance:
abstract nominal types can be inherited by both abstract and concrete
types, but concrete types are final and cannot be inherited.
Type parameters of parametric nominal types are invariant;
they can be constrained by non-recursive lower and upper bounds,
and can be referenced from the supertype declaration.
For instance, \cjl{Point\{X<:Real\}} can be declared a subtype of
\cjl{AbstractPoint\{X\}}.
\figref{fig:code:inheritance} provides a few more examples.
A type that is being inherited needs to be defined before the inheriting type,
and mututally recursive type declarations are not supported.

Because of its fairly restricted nature, inheritance in Julia does not
interfere with the decidability of subtyping, unlike, for example,
in Java~\cite{bib:tate:taming-wildcards:2011}.
In particular, the lack of variance annotations, recursive inheritance,
and recursive constraints prevents properties such as expansive
inheritance~\cite{bib:kennedy:nom-sub-var-dec:2007}, which are known to cause
undecidability.
Thus, in Julia, subtyping of nominal types is straightforward: the subtyping
algorithm simply walks the finite inheritance hierarchy, substituting
type arguments of parametric types.
In the specification of Julia
subtyping~\cite{bib:zappa-nardelli:julia-sub:oopsla:2018},
this step is encoded with one extra rule \RLJ{App\_Super},
which is given in~\figref{fig:jlsubex-inheritance}.
In the context of the decidability proof from \secref{subsec:dec-proof},
the measure of nominal types needs to be modified to include the 
measure of the declared supertype, akin to
how~\citet{bib:greenman:f-bound-material-shape:2014}
proved decidability of subtyping for Java with Material-Shape Separation.

Without changes, the framework of restricted existential types presented
in \secref{sec:dec-sub:defs} is suitable for \emph{partial} handling of
inheritance in Julia. For example, consider the following derivation,
where \tyinv\nvec\vx is a declared subtype of \tyinv\nabsvec\vx:\\
\makebox[\textwidth][c]{
\begin{minipage}{\ruleswidth}
\begin{mathpar}
\small
\inferrule[]
{ 
    \inherits{\tyinv\nvec\vx}{\tyinv\nabsvec\vx} \and
    \inferrule[]
    { 
        \subtydflt{\tybot}{\tybot} \and
        \subtydflt{\tyint}{\tyany} \and
    }
    { \subtydflt
        {\tyinv\nabsvec{\rexvarbound{\tybot}{\tyint}}}
        {\tyinv\nabsvec{\rexvarbound{\tybot}{\tyany}}} 
    }
}
{ \subtydflt
    {\tyinv\nvec{\rexvarbound{\tybot}{\tyint}}}
    {\tyinv\nabsvec{\rexvarbound{\tybot}{\tyany}}} 
}
\\
\end{mathpar}
\end{minipage}}
Similarly to \RLJ{App\_Inv}, subtyping is checked by substituting
\vx with \rexvarbound{\tybot}{\tyint} in the supertype declaration
\tyinv\nabsvec\vx.
However, type parameters do not have to be immediate arguments of the 
supertype declaration. For example, the following type declaration is
allowed:
\begin{center}
\begin{minipage}{7cm}
\begin{lstlisting}
struct ZooVec{X} <: AbstractVector{Zoo{X}}
...
end
\end{lstlisting}  
\end{minipage}
\end{center}
In this case, simply substituting \vx with a restricted existential variable
would be \textbf{incorrect}:\\
\makebox[\textwidth][c]{
\begin{minipage}{\ruleswidth}
\begin{mathpar}
\small
\inferrule[]
{ 
    \inherits{\tyinv{\tyname{ZooVec}}\vx}
        {\tyinv\nabsvec{\tyinv{\tyname{Zoo}}{\vx}}} 
    \and
    \AEnv \vdash 
       \tyinv\nabsvec{\tyinv{\tyname{Zoo}}{\rexvarbound{\tybot}{\tyint}}}
       \ ???
}
{ \AEnv \vdash 
    \tyinv{\tyname{ZooVec}}{\rexvarbound{\tybot}{\tyint}}
    \nless:
    \tyinv\nabsvec{\tyinv{\tyname{Zoo}}{\rexvarbound{\tybot}{\tyany}}} 
}
\\
\end{mathpar}
\end{minipage}}
Recall that \tyinv{\tyname{ZooVec}}{\rexvarbound{\tybot}{\tyint}}
denotes an existential type, namely:
\[
    \tyexist{\vy}{\tybot}{\tyint}{\tyinv{\tyname{ZooVec}}\vy}.
\]
Therefore, the proper supertype of this type is
\[
    \tyexist{\vy}{\tybot}{\tyint}
        {\tyinv\nabsvec{\tyinv{\tyname{Zoo}}{\vy}}},
\]
which is different from 
\[
    \tyinv\nabsvec{\tyexist{\vy}{\tybot}{\tyint}{\tyinv{\tyname{Zoo}}\vy}}
\]
denoted by \tyinv\nabsvec{\tyinv{\tyname{Zoo}}{\rexvarbound{\tybot}{\tyint}}}.

\begin{figure}[t!]
\footnotesize
\makebox[\textwidth][c]{
\begin{minipage}{\ruleswidth}
\begin{mathpar}
\\
\fbox{\ljsub{E}{t}{t}{E}}
\\
\inferrule[App\_Inv]
{ E_0 = \mathit{add}(E, \text{Barrier}) \\ 
    \forall\, 0 < i \leq n. \  
    E_{i-1} \vdash \t_i <: \t'_i \vdash E'_i \ \wedge\  
    E'_i \vdash \t'_i <: \t_i \vdash E_i \\
}
{ E \vdash \cstrt{name}{\t_1, \ldots, \t_n} <: 
    \cstrt{name}{\t'_1, \ldots, \t'_n} \vdash \mathit{del}(\text{Barrier}, E_n)}

\inferrule[App\_Super]
{ 
    \cstrt{name}{\T_1, \ldots, \T_n} <: \t'' \ \in\ tds \\
    E \vdash \t''[\t_1/\T_1, \ldots, \t_n/\T_n] <: \t' \vdash E'
}
{ E \vdash \cstrt{name}{\t_1, \ldots, \t_n} <: 
    \t' \vdash E'}
\end{mathpar}
\end{minipage}}
\caption{Julia subtyping: nominal types}\label{fig:jlsubex-inheritance}
\begin{tablenotes}[para]
\small
\centering
An excerpt from \cite{bib:zappa-nardelli:julia-sub:oopsla:2018}, extends
\figref{fig:jlsubex}.
\end{tablenotes}
\end{figure}

The simplest solution to this problem would be to disallow instantiating
type variables such as \vx in \tyname{ZooVec} (that is, variables that
occur in the supertype but not as immediate arguments)
with restricted existentials.
Thus, \tyinv{\tyname{ZooVec}}\tyint and
\tyexist{\vx}{\tybot}{\tyint}{\tyinv{\tyname{ZooVec}}\vx} would be allowed,
whereas \tyinv{\tyname{ZooVec}}{\rexvarbound{\tybot}{\tyint}} would not.

\begin{figure}
\footnotesize
\makebox[\textwidth][c]{
\begin{minipage}{\ruleswidth}
\begin{mathpar}
    \fbox{\subtydflt{\ty}{\ty}}
    \\

    \inferrule[\RST{InvLeft}]
    { \vx \text{ fresh} \\
      \subty{\AEnv, \varbound{\vx}{\tylb}{\tyub}}
        {\tyinv\iname{\ldots,\vx,\ldots}}
        {\tyinv{\iname'}{\rexvar'_1,\ldots,\rexvar'_m}} }
    { \subtydflt
        {\tyinv\iname{\ldots,\rexvarbound{\tylb}{\tyub},\ldots}}
        {\tyinv{\iname'}{\rexvar'_1,\ldots,\rexvar'_m}} }

    \inferrule[\RST{InvSuper}]
    { \inherits{\tyinv\iname{\vy_1,\ldots,\vy_n}}{\ty''}
      \\
      \subtydflt
        {\subst{\ty''}{\substel{\vy_1}{\ty_1},\ldots,\substel{\vy_1}{\ty_n}}}
        {\tyinv{\iname'}{\rexvar'_1,\ldots,\rexvar'_m}} } 
    { \subtydflt
        {\tyinv\iname{\ty_1,\ldots,\ty_n}}
        {\tyinv{\iname'}{\rexvar'_1,\ldots,\rexvar'_m}} }
    \\
    \fbox{\subtyctrdflt{\ty}{\ty}} 
    \\

    \inferrule[\RSC{InvLeft-UL}]
    { \vx \text{ fresh} \\
      \subtyctrL{\AEnv, \varbound{\vx}{\tylb}{\tyub}}{\UEnvD}
        {\tyinv\iname{\ldots,\vx,\ldots}}
        {\tyinv{\iname'}{\rexvar'_1,\ldots,\rexvar'_m}}
        {\CSet} }
    { \subtyctrLdflt
        {\tyinv\iname{\ldots,\rexvarbound{\tylb}{\tyub},\ldots}}
        {\tyinv{\iname'}{\rexvar'_1,\ldots,\rexvar'_m}} }

    \inferrule[\RSC{InvLeft-UR}]
    { \vx \text{ fresh} \\
      \subtyctrR{\AEnv, \varbound{\vx}{\tylb}{\tyub}}{\UEnvD}
        {\tyinv\iname{\ldots,\vx,\ldots}}
        {\tyinv{\iname'}{\rexvar'_1,\ldots,\rexvar'_m}}
        {\CSet}  \\
      \CSet' = \dischctrdflt }
    { \subtyctrRdfltenv
        {\tyinv\iname{\ldots,\rexvarbound{\tylb}{\tyub},\ldots}}
        {\tyinv{\iname'}{\rexvar'_1,\ldots,\rexvar'_m}}
        {\CSet'} }

    \inferrule[\RSC{InvSuper}]
    { \inherits{\tyinv\iname{\vy_1,\ldots,\vy_n}}{\ty''}
      \\
      \subtyctrdflt
        {\tyinv{\iname''}{\subst{\ty''_1}{\substel{\vy_1}{\ty_1},\ldots},\ldots}}
        {\tyinv{\iname'}{\rexvar'_1,\ldots,\rexvar'_m}} }
    { \subtyctrdflt
        {\tyinv\iname{\ty_1,\ldots,\ty_n}}
        {\tyinv{\iname'}{\rexvar'_1,\ldots,\rexvar'_m}} }
\end{mathpar}
\end{minipage}}
\caption{Additional subtyping rules to support inheritance
    %Note: \dctx stands for \dctxty.
}\label{fig:subtyping-inheritance}
\begin{tablenotes}[para]
\small
This figure extends unification-free subtyping (\figref{fig:subtyping-base})
and constrained subtyping (\figref{fig:subtyping-constrained}).
Signature subtyping (\figref{fig:subtyping-tysigs}) remains unchanged.
$\dischctrop$ is defined in \figref{fig:ctr-discharge}.
\end{tablenotes}
\end{figure}

A more general solution would be to open all restricted existential types
before walking up the inheritance hierarchy, similar
to~\cite{bib:tate:mixed-site:2013}. Thus, the example above would
proceed as follows:\\
\makebox[\textwidth][c]{
\begin{minipage}{\ruleswidth}
\begin{mathpar}
\small
\inferrule[]
{ 
    \inferrule[]
    { 
        \inherits{\tyinv{\tyname{ZooVec}}\vx}
            {\tyinv\nabsvec{\tyinv{\tyname{Zoo}}{\vx}}}
        \\
        \AEnv, \varbound{\vy}{\tybot}{\tyint} \vdash
            \tyinv\nabsvec{\tyinv{\tyname{Zoo}}{\vy}}
            \nless:
            \tyinv\nabsvec{\tyinv{\tyname{Zoo}}{\rexvarbound{\tybot}{\tyany}}} 
    }
    { \vy \text{ fresh} \and
      \AEnv, \varbound{\vy}{\tybot}{\tyint} \vdash
        \tyinv{\tyname{ZooVec}}{\vy}    
        \nless:
        \tyinv\nabsvec{\tyinv{\tyname{Zoo}}{\rexvarbound{\tybot}{\tyany}}} }
}
{ \AEnv \vdash 
    \tyinv{\tyname{ZooVec}}{\rexvarbound{\tybot}{\tyint}}
    \nless:
    \tyinv\nabsvec{\tyinv{\tyname{Zoo}}{\rexvarbound{\tybot}{\tyany}}} 
}
\\
\end{mathpar}
\end{minipage}}
Additional subtyping rules that support this approach to handling inheritance
are given in \figref{fig:subtyping-inheritance}.
The rules \RST{InvSuper} and \RSC{InvSuper} are similar to \RLJ{App\_Super}:
they walk the inheritance hierarchy, substituting type arguments;
importantly, these rules can be applied only when all the arguments of the
left-hand side invariant constructor are tight,
i.e., $\ty_i$ rather than $\rexvar_i$.
The rule \RST{InvLeft} of unification-free subtyping is similar to
\RSS{InvLeft}: it simply opens a restricted existential type variable.
The remaining rules of constrained subtyping are more interesting:
\begin{itemize}
    \item \RSC{InvLeft-UL} handles the case when unification variables are on
        the left. Although the bounds \tylb and \tyub may contain unification
        variables, the fresh variable \vx is treated as universal and added
        to \AEnv. Since \vx does not occur in the right-hand side type, 
        unification variables from its bounds cannot ``leak'' to the right.
    \item \RSC{InvLeft-UR} handles the case when unification variables are on
        the right. The fresh variable \vx can occur in generated constraints,
        so it needs to be discharged before the constraints are propagated.
        The discharge algorithm is given in \figref{fig:ctr-discharge}.
        Intuitively, since \vx is introduced after unification variables,
        constraints need to be satisfied for all possible valid instantiations
        of \vx. To reflect this, covariant occurrences of \vx in lower-bound
        (upper-bound) constraints are replaced with the upper bound
        (lower bound) of \vx. If \vx occurs invariantly, it has to be
        tightly bound (\subtydflt{\tyub}{\tylb}), or else discharge 
        and subtyping fail.
\end{itemize}

\begin{figure}
\footnotesize
\centering
\begin{minipage}{.7\linewidth}
\begin{algorithm}[H]
    \SetKwProg{DischargeCtrs}{Discharge}{}{}

    \DischargeCtrs{$(\AEnv;\varbound{\vx}{\tylb}{\tyub};\CSet)$}{
        \uIf{$\subtydflt{\tyub}{\tylb}$}{
            \KwRet{$\subst{\CSet}{\substel{\vx}{\tylb}}$} 
        }
        \Else{
            $\CSet_{\tyub} \gets 
                \ctrset{\ctrsub{\covsubst{\vx}{\tyub}{\tylb'}}{\va} \,|\, \ctrsub{\tylb'}{\va} \in \CSet} $ \;
            $\CSet_{\tylb} \gets 
                \ctrset{\ctrsub{\va}{\covsubst{\vx}{\tylb}{\tyub'}} \,|\, \ctrsub{\va}{\tyub'} \in \CSet} $ \;
            \KwRet{$\CSet_{\tyub} \cup \CSet_{\tylb}$}
        }
    }
    %$\substvars \gets \emptysubst$
\end{algorithm}  
\end{minipage}
\[
\begin{array}{lcl}
    \hline
    \multicolumn{3}{c}{\covsubstdflt{\ty}} \\ 
    \hline 
    \covsubstdflt{\tyany} &=& \tyany\\
    \covsubstdflt{\tybot} &=& \tybot\\
    \covsubstdflt{\vany} &=& \ty_{\vany}\\
    \covsubstdflt{\vany'} &=& \vany'\\
    \covsubstdflt{\typair{\ty_1}{\ty_2}} &=& 
        \typair{\covsubstdflt{\ty_1}}{\covsubstdflt{\ty_2}}\\
    \covsubstdflt{\tyinv\iname{\ldots}} &=&
        \text{if } \occdflt{\tyinv\iname{\ldots}}
        \text{ then } \failure
        \text{ else } \tyinv\iname{\ldots}\\
    \covsubstdflt{\tyunion{\ty_1}{\ty_2}} &=& 
        \tyunion{\covsubstdflt{\ty_1}}{\covsubstdflt{\ty_2}}\\
\end{array}
\]
\caption{Variable discharge algorithm \dischctrdflt}\label{fig:ctr-discharge}      
\end{figure}

The following example illustrates \RSC{InvLeft-UL}:\\
\makebox[\textwidth][c]{
\begin{minipage}{\ruleswidth}
\begin{mathpar}
\small
\inferrule[]
{ 
    \inferrule[]
    {
        \inferrule[]
        {  
            \subtyctrR{\AEnv, \varbound{\vx}{\va}{\tyint}}{\va}
                {\tyint}{\va}
                {\ctrsngl{\tyint}{\va}}  
        }
        { 
            \subtyctrR{\AEnv, \varbound{\vx}{\va}{\tyint}}{\va}
                {\tyint}{\vx}
                {\ctrsngl{\tyint}{\va}}  
        }
        \and
        \subtyctrL{\AEnv, \varbound{\vx}{\va}{\tyint}}{\va}
            {\vx}{\tyany}{\EmptyCSet} 
    }
    {
        \subtyctrL{\AEnv, \varbound{\vx}{\va}{\tyint}}{\va}
            {\tyinv\nref\vx}
            {\tyinv\nref{\rexvarbound{\tyint}{\tyany}}}
            {\ctrsngl{\tyint}{\va}}  
    }
}
{   
    \subtyctrL{\AEnv}{\va}{\tyinv\nref{\rexvarbound{\va}{\tyint}}}
        {\tyinv\nref{\rexvarbound{\tyint}{\tyany}}}
        {\ctrsngl{\tyint}{\va}}
}
\\
\end{mathpar}
\end{minipage}}

The following example illustrates \RSC{InvLeft-UR} where
the variable discharge succeeds:\\
\makebox[\textwidth][c]{
\begin{minipage}{\ruleswidth}
\begin{mathpar}
\small
\inferrule[]
{ 
    \inferrule[]
    {
        \subtyctrL{\AEnv, \varbound{\vx}{\tyint}{\tyany}}{\va}
            {\tybot}{\vx}{\EmptyCSet} 
        \and
        \subtyctrR{\AEnv, \varbound{\vx}{\tyint}{\tyany}}{\va}
            {\vx}{\va}
            {\ctrsngl{\vx}{\va}}  
    }
    {
        \subtyctrR{\AEnv, \varbound{\vx}{\tyint}{\tyany}}{\va}
            {\tyinv\nref\vx}
            {\tyinv\nref{\rexvarbound{\tybot}{\va}}}
            {\ctrsngl{\vx}{\va}}  
    }
    \\\\
    \dischctr{\AEnv}{\varbound{\vx}{\tyint}{\tyany}}{\ctrsngl{\vx}{\va}} =
        \ctrsngl{\tyany}{\va}
}
{   
    \subtyctrR{\AEnv}{\va}{\tyinv\nref{\rexvarbound{\tyint}{\tyany}}}
        {\tyinv\nref{\rexvarbound{\tybot}{\va}}}
        {\ctrsngl{\tyany}{\va}}
}
\\
\end{mathpar}
\end{minipage}}

The following example illustrates \RSC{InvLeft-UR} where
the variable discharge fails:\\
\makebox[\textwidth][c]{
\begin{minipage}{\ruleswidth}
\begin{mathpar}
\small
\inferrule[]
{ 
    \inferrule[]
    {
        \subtyctrR{\AEnv, \varbound{\vx}{\tyint}{\tyany}}{\va}
            {\tyinv\nabsvec{\tyinv{\tyname{Zoo}}{\vx}}}
            {\tyinv\nabsvec\va}
            {\ctrset{\ctrsub{\tyinv{\tyname{Zoo}}{\vx}}{\va}, 
                \ctrsub{\va}{\tyinv{\tyname{Zoo}}{\vx}}}}
    }
    {
        \subtyctrR{\AEnv, \varbound{\vx}{\tyint}{\tyany}}{\va}
            {\tyinv{\tyname{ZooVec}}{\vx}}
            {\tyinv\nabsvec\va}
            {\ctrset{\ctrsub{\tyinv{\tyname{Zoo}}{\vx}}{\va}, 
                \ctrsub{\va}{\tyinv{\tyname{Zoo}}{\vx}}}}
    }
    \\\\
    \dischctr{\AEnv}{\varbound{\vx}{\tyint}{\tyany}}
        {\ctrset{\ctrsub{\tyinv{\tyname{Zoo}}{\vx}}{\va}, 
            \ctrsub{\va}{\tyinv{\tyname{Zoo}}{\vx}}}} = \failure
}
{   
    \AEnv \Alt \va \ \vdash\ \tyinv{\tyname{ZooVec}}{\rexvarbound{\tyint}{\tyany}}
        \nlessdot\!\!\bullet \tyinv\nabsvec\va
}
\\
\end{mathpar}
\end{minipage}}

The syntax of type declarations and extended validity rules
are given in \figref{fig:validity-inheritance}.
Note that the supertype cannot be a restricted existential type:
\tyinv{\aname}{\ty_1,\ldots,\ty_m} is an invariant constructor with
tight type arguments.
The validity of types is checked with respect to an implicit datatype table.
In a type declaration, type parameters $\vx_i$ may reference previous
parameters: this is handled by checking the validity of type parameters
in \tyvld{\TyTable}{\tydecl} using \tyvld{}{\AEnv} from
\figref{fig:ty-tysig-validity}.
Similarly to declared bounds of existential variables, bounds of type parameters
need to be consistent.

\begin{figure}
\footnotesize
\[
\begin{array}{rcll}
    \iname
    &::=& & \textit{Invariant constructor name} \\
    &\Alt& \cname & \text{concrete} \\
    &\Alt& \aname & \text{abstract} \\
\end{array}
\]
\[
\begin{array}{rcll}
    \TyTable &::=& \varnothing \Alt \TyTable,\tydecl 
        & \textit{Datatype table} \\
    \tydecl 
        &::=&
        \inherits{\tyinv{\iname}{\varbound{\vx_1}{\tylb_1}{\tyub_1},
            \,\ldots,\,\varbound{\vx_n}{\tylb_n}{\tyub_n}}
        }{\ubdecl}
        & \textit{Type declaration} \\
    \ubdecl
        &::=& \tyany 
        \Alt \tyinv{\aname}{\ty_1,\ldots,\ty_m}
        & \textit{Supertype} \\ 
\end{array}
\]
\begin{mathpar}
    \fbox{\tyvld{}{\TyTable}}
    \\

    \inferrule*[]
    { }
    { \tyvld{}{\varnothing} }

    \inferrule*[]
    { \tyvld{}{\TyTable} \and \tyvld{\TyTable}{\tydecl} }
    { \tyvld{}{\TyTable,\tydecl } }
    \\

    \fbox{\tyvld{\TyTable}{\tydecl}}
    \\

    \inferrule*[]
    { \tyvld{}{\varbound{\vx_1}{\tylb_1}{\tyub_1},\,\ldots}
      \and 
      \ubdecl=\tyinv\aname{\ldots} \implies \aname \text{ is defined in } \TyTable
      \and
      \tyvld{\varbound{\vx_1}{\tylb_1}{\tyub_1},\,\ldots}{\ubdecl}   
    }
    { \tyvld{\TyTable}{\inherits{\tyinv{\iname}{
            \varbound{\vx_1}{\tylb_1}{\tyub_1},\,\ldots}
        }{\ubdecl}} }
    \\

    \fbox{\tyvlddflt{\ty}}
    \\

    \inferrule*[]
    {   
        \tyvlddflt{\tylb} \and \tyvlddflt{\tyub} \and
        \vx \text{ fresh} \and
        \tyvld{\AEnv, \varbound{\vx}{\tylb}{\tyub}}
            {\tyinv\iname{\ldots,\vx,\ldots}} 
    }
    { 
      \tyvlddflt{\tyinv\iname{\ldots,\rexvarbound{\tylb}{\tyub},\ldots}}
    }

    \inferrule*[]
    { 
        \inherits{\tyinv{\iname}{\varbound{\vx_1}{\tylb_1}{\tyub_1},\,\ldots}
        }{\ubdecl} 
        \and
        \tyvlddflt{\subst{\ubdecl}{\substel{\vx_1}{\ty_1},\,\ldots}}
    }
    { 
        \tyvlddflt{\tyinv\iname{\ty_1,\,\ldots,}} 
    }
\end{mathpar}
\caption{Validity of type declarations and types in the presence of inheritance
(extends \figref{fig:ty-grammar} and \figref{fig:ty-tysig-validity})
}\label{fig:validity-inheritance}
\end{figure}


\subsection{Diagonal Rule}\label{subsec:dec-sub:diagonal}
%% *********************************************************

As discussed in \secref{sec:julia-sub:overview},
Julia provides special support for a generic programming pattern
where method arguments are expected to be of the same concrete type.
For example, in the existential type \cjl{Tuple\{T,T\} where T},
type variable \cjl{T} can be instantiated only with concrete types.
This is called the \textbf{diagonal rule}. 
%The rule applies only to top-level existential types.
% JB: this is no longer true!

In this work, the diagonal rule is modeled explicitly, using a new kind of
type variables---\textbf{concrete variable}. 
In type signatures, existential types specify the kind of the bound variable
explicitly:
\[
    \tysig ::= \ldots \Alt
        \tyexist{\vany{:}\varkind}{\tylb}{\tyub}{\tysig}
\]
The kind annotation,
\[\varkind ::= \kindany \Alt \kindval,\]
allows for two options: 
\begin{itemize}
    \item kind \kindany denotes a regular variable that can be instantiated with any type;
    \item kind \kindval denotes a {concrete} variable that can be instantiated 
        only with a concrete type.
\end{itemize}

All earlier-presented definitions and reasoning apply
to \kindany-kinded vars without modifications.

Because concrete variables have to be instantiated with concrete types,
the only allowed declared lower bound is \tybot. To achieve tight bounds,
a concrete type can be chosen for the upper bound.
Thus, the validity is extended as follows:
\begin{mathpar}
\inferrule*[]
    { \tyvlddflt{\tyub} \and
        \tyvld{\AEnv, \varbound{\varval{\vany}}{\tybot}{\tyub}}{\tysig} }
    { \tyvlddflt{\tyexist{\varval{\vany}}{\tybot}{\tyub}{\tysig}} }
\end{mathpar}

\figref{fig:concrete} defines concreteness \tyvaldflt{\ty}:
type \ty is concrete if it is
a concrete variable,
a concrete invariant constructor with tight arguments,
a tuple of concrete types,
a union of equivalent concrete types,
or a union of a concrete and a bottom type.
The concreteness judgment is used by the subtyping algorithm.

\begin{figure}[t]
\footnotesize
\makebox[\textwidth][c]{
\begin{minipage}{\ruleswidth}
\begin{mathpar}
    \fbox{\tyvaldflt{\ty}}
    \\

    \inferrule*[]
    { \varbound{\varval{\vany}}{\tylb}{\tyub} \in \AEnv }
    { \tyvaldflt{\vany} }

    \inferrule*[]
    { }
    { \tyvaldflt{\tyinv\cname{\ty_1,\ldots,\ty_n}} }

    \inferrule*[]
    { \tyvaldflt{\ty_1} \and \tyvaldflt{\ty_2} }
    { \tyvaldflt{\typair{\ty_1}{\ty_2}} }

    \inferrule*[]
    { 
        \tyvaldflt{\ty_1} \and \tyvaldflt{\ty_2} \and 
        \subtydflt{\ty_1}{\ty_2} \and \subtydflt{\ty_2}{\ty_1}
    }
    { \tyvaldflt{\tyunion{\ty_1}{\ty_2}} }

    \inferrule*[]
    { 
        i \neq j \and \tyvaldflt{\ty_i} \and \subtydflt{\ty_j}{\tybot}
    }
    { \tyvaldflt{\tyunion{\ty_1}{\ty_2}} }
\end{mathpar}
\end{minipage}}
\caption{Type concreteness}\label{fig:concrete}
\end{figure}

The subtyping algorithm requires minimal modifications, 
as defined in \figref{fig:subty-concrete}.
All previously defined rules are applicable to both kinds of variables.
The two new rules cover the tight-bound case for a concrete variable: as the only 
valid instantiation of the variable is \tyub, it can be used
instead of \tybot to check subtyping with the concrete variable on the right.
The same reasoning explains the new cases for intersection $\sqcap_\AEnv$.
Constraints for concrete unification variables are generated in the same way
as for regular variables. The key difference is in the constraints resolution
algorithm: the smallest type satisfying the constraints, 
$\ty_{\va} = \mcup_i \tylb_i$,
needs to be concrete\footnote{
    It is possible that there are no lower-bound constraints $\tylb_i$,
    in which case instantiating \va with \tybot would technically be incorrect.
    However, the absence of generated lower bounds means that any instantiation
    within the declared bounds would satisfy the constraints, so it can 
    be picked arbitrarily; in this case,
    the only necessary condition is that
    $\lnot (\subtydflt{\substvars(\tyub)}{\tybot})$,
    as otherwise, no concrete instantiations would be valid.
} in order for constraints resolution to succeed.

\begin{figure}[t]
\footnotesize
\makebox[\textwidth][c]{
\begin{minipage}{\ruleswidth}  
\begin{mathpar}
    \fbox{\subtydflt{\ty}{\ty}}
    \qquad\qquad
    \fbox{\subtyctrdflt{\ty}{\ty}}
    \\

    \inferrule[\RST{VarRight-ConcreteTight}]
    { \varbound{\varval{\vany}}{\tybot}{\tyub} \in \AEnv \\
        \tyvaldflt{\tyub} \\\\
        \subtydflt\ty{\tyub} }
    { \subtydflt{\ty}{\vany} }

    \inferrule[\RSC{VarRight-ConcreteTight}]
    { \varbound{\varval{\vany}}{\tybot}{\tyub} \in \AEnv \\
        \tyvaldflt{\tyub} \\\\
        \subtyctrdflt\ty{\tyub} }
    { \subtyctrdflt{\ty}{\vany} }
\\
\end{mathpar}
\[
\begin{array}{ccccll}
    \hline
    \multicolumn{6}{c}{\tymeetdflt{\ty}{\ty'}} \\ 
    \hline 
    \vany &\sqcap_{\AEnv}& \ty' &=& 
        \tymeetdflt{\tyub}{\ty'} & 
        \text{ where } \varbound{\varval{\vany}}{\tybot}{\tyub} \in \AEnv 
        \text{ and } \tyvaldflt{\tyub} \\
    \ty &\sqcap_{\AEnv}& \vany &=& 
        \tymeetdflt{\ty}{\tyub} & 
        \text{ where } \varbound{\varval{\vany}}{\tybot}{\tyub} \in \AEnv 
        \text{ and } \tyvaldflt{\tyub} \\
    \\
\end{array}
\]
\end{minipage}}
\begin{minipage}{6.5cm}
\begin{algorithm}[H]
    \SetKwProg{SolveCtrs}{Solve}{}{}

    \SolveCtrs{$(\AEnv;\,\UEnv,\varbound{\va{:}\!\kindany\!}{\tylb}{\tyub};\,\CSet)$}{
        $\CSet_{\va} \gets 
            \ctrset{\ctrsub{\tylb'}{\va} \in \CSet} 
            \cup 
            \ctrset{\ctrsub{\va}{\tyub'} \in \CSet} $ \;
        $\CSet' \gets \CSet \setminus \CSet_{\va}$ \;
        $\UEnvD \gets \dom{\UEnv}$\;
        \ForEach{$\ctrsub{\tylb_i}{\va}, \ctrsub{\va}{\tyub_j} \in \CSet_{\va}$}{%
            \subtydflt{\tylb_i}{\tyub_j}
        }
        \ForEach{$\ctrsub{\tylb_i}{\va} \in \CSet_{\va}$}{%
            \subtyctrRdfltenv{\tylb_i}{\tyub}{\CSet_{\tylb_i}}
        }
        \ForEach{$\ctrsub{\va}{\tyub_j} \in \CSet_{\va}$}{%
            \subtyctrLdfltenv{\tylb}{\tyub_j}{\CSet_{\tyub_j}}
        }
        $\substvars \gets \solvectr{\AEnv}{\UEnv}{
            \CSet' \mcup_i \CSet_{\tylb_i} \mcup_j \CSet_{\tyub_j}
        }$\;
        $\ty_{\va} \gets \substvars(\tylb) \mcup_i \tylb_i$\;
        \BlankLine
        \KwRet{$\subst\substvars{\substel{\va}{\ty_{\va}}}$}
    }
    %$\substvars \gets \emptysubst$
\end{algorithm}  
\end{minipage}
\hspace{0.2cm}
\begin{minipage}{6.5cm}
\begin{algorithm}[H]
    \SetKwProg{SolveCtrs}{Solve}{}{}

    \SolveCtrs{$(\AEnv;\,\UEnv,\varbound{\va{:}\!\kindval\!}{\tybot}{\tyub};\,\CSet)$}{
        $\CSet_{\va} \gets 
            \ctrset{\ctrsub{\tylb'}{\va} \in \CSet} 
            \cup 
            \ctrset{\ctrsub{\va}{\tyub'} \in \CSet} $ \;
        $\CSet' \gets \CSet \setminus \CSet_{\va}$ \;
        $\UEnvD \gets \dom{\UEnv}$\;
        \ForEach{$\ctrsub{\tylb_i}{\va}, \ctrsub{\va}{\tyub_j} \in \CSet_{\va}$}{%
            \subtydflt{\tylb_i}{\tyub_j}
        }
        \ForEach{$\ctrsub{\tylb_i}{\va} \in \CSet_{\va}$}{%
            \subtyctrRdfltenv{\tylb_i}{\tyub}{\CSet_{\tylb_i}}
        }
        \BlankLine\BlankLine
        \BlankLine\BlankLine
        $\substvars \gets \solvectr{\AEnv}{\UEnv}{
            \CSet' \mcup_i \CSet_{\tylb_i} \mcup_j \CSet_{\tyub_j}
        }$\;
        $\ty_{\va} \gets \mcup_i \tylb_i$\;
        \BlankLine
        $\tyvaldflt{\ty_{\va}}$\;
        \BlankLine
        \KwRet{$\subst\substvars{\substel{\va}{\ty_{\va}}}$}
    }
    %$\substvars \gets \emptysubst$
\end{algorithm}  
\end{minipage}
\caption{Subtyping with concrete variables}\label{fig:subty-concrete}
\begin{tablenotes}[para]
\small
    Extends \figref{fig:subtyping-constrained} and \figref{fig:ty-join-meet},
    and replaces \figref{fig:ctr-solve}.
    Constraints resolution algorithm for regular variables is exactly the same
    as the algorithm in \figref{fig:ctr-solve}; it is provided here for
    convenience of comparison.
\end{tablenotes}
\end{figure}