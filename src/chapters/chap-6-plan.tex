\chapter{Plan of work}\label{chap:6}

I am currently working on designing a subtype relation
and proving its decidability, reflexivity, and transitivity.
I have also extended the interpretation of types to account for type variables.
To evaluate the practicality of the subtype relation,
I will perform a static analysis of type annotations in registered Julia
packages (about 9000 as of March 2023).

In July 2023, I intend to submit a POPL 2024 paper on decidable subtyping
and complete my thesis, defending in August 2023.

\begin{table}[h]
  \caption{Schedule}
  \vspace*{0.25em}
  \centering\footnotesize
  \begin{tabular}{c|ccccc}
  \toprule
  & May & June & July & August \\
  \midrule
  proofs \& evaluation & X & X & & \\
  paper & X & X & & \\
  thesis & X & X & X & \\
  defense & & & & X \\
\end{tabular}
\end{table}

% I plan to work on developing a decidable subtyping specification for the Julia
% language and write a research paper on that.
% As outlined in \chapref{chap:3}, I envision the type language to be close to the
% one currently used in Julia, with some restrictions that ensure decidability
% but support the majority of the existing code base.
% As a first step, I will look into restricting lower bounds of existential type
% variables, for lower bounds make it possible to encode the undecidable
% system \FSub in Julia. Furthermore, taking into account the inconsistent
% treatment of types due to the diagonal rule,
% and the importance of concrete types for optimizations in the JIT compiler,
% I propose to explicitly distinguish between regular existential types
% and existential types where the type variable ranges over only concrete types.

% I intend to do the technical work on decidable subtyping and submit a paper
% for POPL 2023, with the deadline in July 2022.
% After that, I intend to collaborate with the Julia developers on incorporating
% decidable subtyping into the Julia language and work on the thesis.
% I expect that the thesis will be completed in a year from the proposal time,
% around February 2023.


% \begin{figure}
% \small
% \makebox[\textwidth]{
% \begin{tabular}{l@{\hspace{4mm}}l}
%   $\begin{array}{rcll}
%     \ty
%       &::=& & \textit{Type annotations} \\
%       &\Alt& \tyany & \text{top type} \\
%       &\Alt& \tybot & \text{bottom type} \\
%       &\Alt& \typair{\ty_1}{\ty_2}
%                     & \text{covariant pair} \\
%       &\Alt& \tyinv\iname\tys
%                     & \text{invariant constr.} \\
%       &\Alt& \tyexist{X}{\cty_l}{\ty_u}{\ty}
%                     & \text{existential type} \\
%       &\Alt& \tvx   & \text{type variable} \\
%       &\Alt& \gtyexist{X}{\ty_u}{\ty}
%                     & \text{concretely-exist. type} \\
%       &\Alt& \gvx   & \text{concrete type var.} \\
%       &\Alt& \tyunion{\ty_1}{\ty_2}
%                     & \text{union type} \\
%     \\
%     \cty  &::=& \ty  \Alt \fv\ty  = \varnothing & \textit{Closed types} \\
%   \end{array}$
% &
%   $\begin{array}{rcll}
%     \gty
%       &::=& & \textit{Type tags} \\
%       &\Alt& \tyinv\cname\tys
%                     & \text{concr. inv. constr.} \\
%       &\Alt& \typair{\gty_1}{\gty_2}
%                     & \text{concrete pair} \\
%       &\Alt& \gvx   & \text{concrete type var.} \\
%     \\\\
%     \iname
%       &::=& & \textit{User-defined names} \\
%       &\Alt& \cname & \text{concrete} \\
%       &\Alt& \aname & \text{abstract} \\
      
%     \\\\
%     \cgty &::=& \gty \Alt \fv\gty = \varnothing & \textit{Closed tags} \\
%   \end{array}$
% \end{tabular}
% }\caption{Syntax}\label{fig:syntax}
% \end{figure}

% \begin{figure}
% \small

%   \[ \VEnv ::= \EmptyEnv \Alt \VEnv, \var{X} \]

% \begin{mathpar}
%   \fbox{\wlscpd{\ty}}
%   \\

%   \inferrule{ }
%   { \wlscpd{\tyany} }

%   \inferrule{ }
%   { \wlscpd{\tybot} }
%   \\

%   \inferrule
%   { \wlscpd{\ty_1} \and \wlscpd{\ty_2} }
%   { \wlscpd{ \typair{\ty_1}{\ty_2} } }

%   \inferrule
%   { \forall i.\ \cfbox{light-gray}{\wlfrscpd{\ty_i}} }
%   { \wlscpd{ \tyinv\iname\tys } }

%   \inferrule
%   { \wlscpd{\ty_1} \and \wlscpd{\ty_2} }
%   { \wlscpd{ \tyunion{\ty_1}{\ty_2} } }
%   \\

%   \inferrule
%   { \wlscp{\EmptyEnv}{\cty_l} \and \wlscpd{\ty_u} \and \wlscp{\VEnv,\vx}{\ty} }
%   { \wlscpd{ \tyexist{X}{\cty_l}{\ty_u}{\ty} } }

%   \inferrule
%   { \wlscpd{\ty_u} \and \wlscp{\VEnv,\vx}{\ty} }
%   { \wlscpd{ \gtyexist{X}{\ty_u}{\ty} } }
  
%   \inferrule
%   { \vx \in \dom\VEnv }
%   { \wlscpd{\avx} }
%   \\

%   \fbox{\wlfrscpd{\ty}}
%   \\

%   \inferrule{ }
%   { \wlfrscpd{\tyany} }

%   \inferrule{ }
%   { \wlfrscpd{\tybot} }
%   \\

%   \inferrule
%   { \wlfrscpd{\ty_1} \and \wlfrscpd{\ty_2} }
%   { \wlfrscpd{ \typair{\ty_1}{\ty_2} } }

%   \inferrule
%   { \forall i.\ \wlfrscpd{\ty_i} }
%   { \wlfrscpd{ \tyinv\iname\tys } }

%   \inferrule
%   { \cfbox{light-gray}{$\wlscp{\colorbox{light-gray}{\EmptyEnv}}{\ty_1}$} \and
%     \cfbox{light-gray}{$\wlscp{\colorbox{light-gray}{\EmptyEnv}}{\ty_2}$} }
%   { \wlfrscpd{ \tyunion{\ty_1}{\ty_2} } }
%   \\

%   \inferrule
%   { \cfbox{light-gray}{\wlscp{\EmptyEnv}{\cty_l}} \and
%     \cfbox{light-gray}{$\wlscp{\colorbox{light-gray}{\EmptyEnv}}{\ty_u}$} \and
%     \cfbox{light-gray}{$\wlscp{\colorbox{light-gray}{\vx}}{\ty}$} }
%   { \wlfrscpd{ \tyexist{X}{\cty_l}{\ty_u}{\ty} } }

%   \inferrule
%   { \cfbox{light-gray}{$\wlscp{\colorbox{light-gray}{\EmptyEnv}}{\ty_u}$} \and
%     \cfbox{light-gray}{$\wlscp{\colorbox{light-gray}{\vx}}{\ty}$} }
%   { \wlfrscpd{ \gtyexist{X}{\ty_u}{\ty} } }

%   \inferrule
%   { \vx \in \dom\VEnv }
%   { \wlfrscpd{\avx} }
  
% \end{mathpar}

% \caption{Well scopedness}\label{fig:well-scope}
% \end{figure}


\begin{figure}
\small

\begin{array}{rcll}
  \ty & ::= & & 
\end{array}

\caption{TODO}
\end{figure}

