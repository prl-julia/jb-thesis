\chapter{Related work}\label{chap:5}

Subtyping is typically associated with object-oriented programming languages and
static type systems. As a part of a type system, subtyping allows
an expression of a more specific type (subtype) to be used in place
where an expression of a more general type (supertype) is expected,
enabling more benign programs to be typable.
Subtyping can also be used at run time
in both statically and dynamically typed languages to perform
type tests, casts, and dynamic dispatch.

There is a wide variety of research related to subtyping, but this chapter
focuses primarily on work that concerns decidability of subtyping,
semantic subtyping, and subtyping of Java generics with wildcards,
as Julia's subtype relation was inspired by the latter
two~\cite{bib:bezanson:julia:2015}.
In particular, the treatment of tuple and union types aligns with the semantic
subtyping approach, and bounded existential types are commonly used
as a model of wildcards.


\section{Decidability of subtyping}
%% ======================================================================

Even in the context of statically typed languages,
decidability is a desirable property of static type systems and
subtype relations, but establishing decidability can be challenging.
For example, the \emph{un}decidability of checking subtyping for Java generics
was established only fairly recently by~\citet{grigore:java-undec:2017},
and the previously suspected undecidability of subtyping and type checking for
the core calculus of Scala 3,
DOT (Dependent Object Calculus)~\cite{amin:dot:2016},
was proven by~\citet{hu:dot-undec:2020}.

Typically, undecidability is shown by a reduction from a known undecidable
problem. For example, the halting problem for Turing machines is used
to prove the undecidability of Java generics~\cite{grigore:java-undec:2017}
% the culprit is recursive inheritance in Java generics
as well as System \FSub~\cite{pierce:bound-sub-undec:1992}.
\FSub~\cite{cardelli:fsub:1991} is a language with bounded universal types,
where the undecidability of subtyping is caused by the combination of
contravariance and rebounding of type variables in the context.
%There, rebounding of type variables in the rule for subtyping universal types
%leads to the growth of type variable context.
Proofs by reduction from \FSub are commonly used to establish
undecidability in languages with some form of bounded quantification.
For instance, the undecidability of DOT~\cite{hu:dot-undec:2020} is shown
by reducing a variant of \FSub to \DSub, a restriction of DOT featuring type
members and path-dependent types. A similar approach is used
by~\citet{mackay:path-dep-dec:2020} for \WyvCore, another simplification of
DOT featuring path-dependent and recursive types.
In the case of Julia, the undecidability of subtyping also follows from its
power to encode a variant of \FSub,
as discussed in~\secref{sec:julia-sub:undec}.
%In the context of languages with nominal inheritance and variance,
%\citet{kennedy:nom-sub-var-dec:2007} showed the undecidability of subtyping
%by a reduction from the Post correspondence problem.

In practice, benefits of having a more expressive type system might outweigh
the cost of undecidability, which amounts to the compiler crashing or
not terminating on some input programs.
If the undecidability arises only in rare, contrived cases,
being able to successfully type check more benign programs
%at the cost of rare compiler crashes
might be preferable to reliably rejecting those
programs with a decidable but less expressive type system.

However, when subtyping is used at run time, the cost of undecidability is
higher because a crash or non-termination occurs during program execution
rather than compilation.
In mainstream programming languages that require run-time subtyping checks,
run-time types are often more restrictive than static ones,
which simplifies the corresponding subtyping problem.
For example, in the .NET intermediate language,
the decidability of subtyping was shown for ground types,
which are used at run time~\cite{kennedy:nom-sub-var-dec:2007}\footnote{Due
to the lack of documentation, it is not clear whether subtyping between ground
types is still decidable in \CSharp as of 2023.}.
In the case of Java generics, where static subtyping is undecidable,
the type erasure mechanism allows for decidable run-time subtyping.
In Julia, however, it is the \emph{run-time} subtyping that is undecidable,
yet subtyping is heavily used by the dynamic semantics.

Identifying decidable fragments of undecidable type systems and subtype
relations remains an important challenge.
In some cases, it might be possible to recover
decidability by restricting the system in a way that does not affect most
practical programs. For example, the material-shape separation for Java
generics~\cite{greenman:f-bound-material-shape:2014}
enables decidability of subtyping by limiting F-bounded polymorphism %~\cite{canning:f-bound:1989}
(i.e., recursive constraints on type variables)
to the subset of types, called shapes, that are used exclusively as constraints;
this separation appears to be in agreement with an industry-wide practice.
\citet{mackay:path-dep-dec:2020} extend the material-shape separation
to the context of path-dependent and recursive types. %, with additional
%restrictions to ensure decidability.
Earlier, \citet{kennedy:nom-sub-var-dec:2007} identified three decidable
fragments of undecidable subtyping in the context of nominal inheritance
and variance without F-bounded polymorphism: the fragments can be obtained by
restricting either contravariance, or expansive class tables,
or multiple instantiation inheritance.
Aiming for decidable subtyping, Julia designers deliberately restricted
the language to \emph{not} support F-bounded polymorphism, 
contravariant constructors, and multiple inheritance.
Based on those restrictions, \citet{bib:bezanson:julia:2015} conjectured
the decidability of subtyping but pointed out that
the combination of invariant constructors
and contravariance in lower bounds of existential types
%one of the undecidability ingredients of \FSub.
is akin to the source of undecidability in~\FSub;
the conjecture relied on the fact that left- and right-hand side variables
are treated assymetrically in Julia.
Still, Julia subtyping is powerful enough to encode
the undecidable \FSub.

A number of decidable subtype relations,
e.g. \WyvSelf~\cite{mackay:path-dep-dec:2020}
and Kernel \DSub~\cite{hu:dot-undec:2020},
adopt subtyping rules inspired by a decidable, restricted variant of \FSub
called Kernel~\FSub~\cite{cardelli:types-poly:1985}.
The downside of such relations is their rejecting more subtyping judgments
than is desirable in practice.
A strictly more powerful than Kernel \DSub system
Strong Kernel \DSub~\cite{hu:dot-undec:2020} is also decidable.
This relation splits the type variable context into two parts, for the left-
and right-hand sides of the subtyping judgment, thus avoiding the problematic
rebounding of type variables in a single context;
the same idea applies to \FSub to obtain the decidable, Strong Kernel \FSub.
A similar approach with context splitting is used
by~\citet{mackay:path-dep-dec:2020} for decidable subtyping in \WyvFix.
In both \WyvFix and Strong Kernel systems,
an undesired consequence of maintaining two separate contexts
is that the resulting subtype relations lack transitivity.
Another decidable variant of \FSub, called \FSubR, was proposed
by~\citet{mackay:bound-poly-sub-dec:2020}. There, %contravariance is allowed
rebounding of type variables is allowed
only for type bounds that do not themselves contain bounded quantification,
with unrestricted types resorting to the Kernel subtyping rule.
With this syntactic restriction, \FSubR admits some useful judgments
rejected by Kernel \FSub, yet does not lose transitivity.

\section{Semantic subtyping}
%% ======================================================================

In semantic subtyping~\cite{frih:sem-sub:2008}, types are given a set-theoretic
interpretation, and subtyping is defined as the set inclusion on
interpretations.
For example, \citet{hosoya:reg-types-XML:2000} define a static type system for
an XML processing language and interpret types as sets of valid XML documents.
Thus, semantic subtyping provides an intuitive mental
model of types to the users. In addition, it automatically satisfies
useful properties such as reflexivity and transitivity.

However, semantic subtyping still needs a decision procedure, and that is where
the complexity typically comes from, especially if efficiency matters.
For example, in the Julia language, semantic subtyping of union and tuple types
%inspired by the semantic approach
relies on a space-efficient algorithm that lazily explores the disjunctive
normal form~\cite{chung:julia-sub-algo:2019}.
In~\cite{frisch:sem-sub:2002}, which supports not only union, but also
intersection and negation types, subtyping $t \leq s$ is checked via
the emptiness test $t \land \lnot s \approx \bot$ and also relies on
the disjunctive normal form.
Related to Julia's treatment of right-hand side existential variables as 
unification variables, \citet{castagna:sem-poly-inf:2015} generate and solve
subtype constraints as a part of type inference process in their set-theoretic
framework with union, intersection, and negation types.
\citet{schimpf:set-types-erlang:2023} successfully apply the
same framework to typing Erlang.
However, the approach to constraint solving does not translate to Julia
due to the presence of invariant type constructors.%, which prevents the
%possibility of rewriting constraints into the right form.

While types such as unions and tuples have a straightforward
semantic interpretation, other types that are common for statically typed
languages can be challenging to interpret.
To this end, \citet{frisch:sem-sub:2002} provide an interpretation for
function types, and \citet{castagna:sem-poly:2011} interpret predicative
parametric polymorphism. In the latter, the interpretation needs to be
parameterized by semantic assignments of type variables.
In the context of object-oriented languages,
\citet{dardha:sem-sub-obj:2016} propose an integration of structural types and
semantic subtyping, and \citet{ancona:sem-sub-imp:2016} study semantic
subtyping for imperative languages with mutable fields.
In the case of the Julia language, %which does not have function types and is not
%concerned with static type checking,
the key challenge is the combination of
impredicative bounded existential types and invariant type constructors.

\section{Java wildcards}
%% ======================================================================

The wildcards mechanism of Java generics~\cite{torgersen:wildcards:2004}
provides use-site variance of parametric
types~\cite{thorup:unif-genericity:1999}.
For example, a variable of type \code{List<? \textbf{extends} Number>}
can be assigned any list with the element type that is a subtype of \code{Number};
using the variable, elements of the list can be safely read at type \code{Number}.
Use-site variance has been recognized as a restricted form of bounded existential
types~\cite{igarashi:variance:2002}.
Thus, the wildcard-typed list above represents an existential
type \code{$\exists$X<:Number.List<X>}.
There have been multiple formalizations of Java wildcards,
such as WildFJ~\cite{torgersen:wildfj:2005} and
TameFJ~\cite{cameron:java-wildcards:2008}, but they
were focused on type soundness rather than a decidable subtyping algorithm.
\citet{smith:java-type-inf:2008} found inconsistencies in Java's type inference
and subtyping algorithms and proposed a solution using a limited form of union
types, with a conjecture on the decidability of subtyping.

Subtyping Java wildcards is challenging and, as shown
by~\citet{grigore:java-undec:2017},
undecidable.
\citet{wehr:dec-bounded-exist:2009} identify multiple undecidable subtype
relations for bounded existential types in formal models inspired by Java.
\citet{tate:taming-wildcards:2011} highlight multiple sources of 
non-termination in Java subtyping,
e.g. the presence of recursive constraints on type variables
and wildcards being allowed in the inheritance hierarchy;
by making practical restrictions on the Java language,
they provide a terminating subtyping algorithm.
Julia lacks all the Java features that were linked to non-termination of
subtyping but supports union and existential types that are not present
in Java's surface language.
