\chapter{Research problem}

As discussed in \chapref{chap:2}, Julia relies on a complex language of types
and subtyping for multiple dynamic dispatch, and subtyping turns out to be
undecidable. In practice, this can manifest with a \cjl{StackOverflowError}
during program execution, as subtyping is used at run time for several purposes:
to resolve function calls, to process new method definitions, and even during
JIT compilation.

There are a number of issues related to subtyping on the Julia bug tracker.
For example,
\href{https://github.com/JuliaLang/julia/issues/41948}{\code{\#41948}}\footnote{
    \url{https://github.com/JuliaLang/julia/issues/41948}
} reports \cjl{StackOverflowError} caused by a function definition,
and
\href{https://github.com/JuliaLang/julia/issues/33137}{\code{\#33137}}\footnote{
    \url{https://github.com/JuliaLang/julia/issues/33137}
} points out surprising behavior related to the diagonal rule.
As of December 2021, there are 22 open and 114 closed issues labeled with ``bug''
and ``types and dispatch''. For context, ``bug'' and ``codegen'' are assigned to
6 open and 76 closed issues, and just ``bug'' is assigned to 213 open and 2477
closed issues.

The main research problem I am going to tackle in this work is
\emph{decidable subtyping for the Julia language}.
The goal is to develop a language of types and a corresponding specification
of subtyping that is decidable but not unreasonably restrictive for the language.
The latter means that the majority of types in the existing Julia packages
are supported by the proposal.
Furthermore, the proposed type language should be usable by the rest of the
Julia compiler.
To tackle the problem, I will answer the following questions:
\begin{enumerate}
    \item \emph{How are types used and what operations on types need to be supported?}
      Clearly, types are used as type annotations in method definitions, with
      subtyping being part of the dispatch resolution.
      Beyond that, Julia is known to rely on type inference for optimizations
      during JIT compilation. Thus, for example, \citet{TODO}, which describes the
      original Julia design, mentions that besides subtyping, the type inference
      algorithm needs meet, join, and widening operators.
      As \cite{TODO} is generally outdated, I plan to 
\end{enumerate}
