\chapter{Research problems}

As discussed in \chapref{chap:2}, Julia relies on a complex language of types
and subtyping for multiple dynamic dispatch, and subtyping turns out to be
undecidable. In practice, this can manifest with a \cjl{StackOverflowError}
during program execution, as subtyping is used at run time for several purposes:
to resolve function calls, to process new method definitions, and even during
JIT compilation.

There are a number of issues related to subtyping on the Julia bug tracker.
For example,
\href{https://github.com/JuliaLang/julia/issues/41948}{\code{\#41948}}\footnote{
    \url{https://github.com/JuliaLang/julia/issues/41948}
} reports \cjl{StackOverflowError} caused by a function definition,
and
\href{https://github.com/JuliaLang/julia/issues/33137}{\code{\#33137}}\footnote{
    \url{https://github.com/JuliaLang/julia/issues/33137}
} points out surprising behavior related to the diagonal rule.
As of December 2021, there are 22 open and 114 closed issues labeled with ``bug''
and ``types and dispatch''. For context, ``bug'' and ``codegen'' are assigned to
6 open and 76 closed issues, and just ``bug'' is assigned to 213 open and 2477
closed issues.
