\section{Proofs}

\subsection{Decidability of Subtyping}\label{subsec:dec-proof}
%% ======================================================================

\begin{figure}
\footnotesize
\[
\begin{array}{lcl}
    \hline
    \multicolumn{3}{c}{\tymsrdflt{\ty}} \\ 
    \hline 
    \tymsrdflt{\tyany} &::=& 1\\
    \tymsrdflt{\tybot} &::=& 1\\
    \tymsr{\,\EmptyEnv\,}{\vany} &::=& 1\\
    \tymsr{\AEnv, \varbound{\vany}{\tylb}{\tyub}}{\vany} &::=& 
        1 + \tymsrdflt{\tylb} + \tymsrdflt{\tyub}\\
    \tymsr{\AEnv, \varbound{\vany'}{\tylb}{\tyub}}{\vany} &::=& 
        \tymsrdflt{\vany} \\
    \tymsrdflt{\typair{\ty_1}{\ty_2}} &::=& 
        1 + \tymsrdflt{\ty_1} + \tymsrdflt{\ty_2}\\
    \tymsrdflt{\tyinv\iname{\rexvar_1,\ldots,\rexvar_n}} &::=&
        1 + \tymsrdflt{\rexvar_1} + \ldots + \tymsrdflt{\rexvar_n}\\
    \tymsrdflt{\tyunion{\ty_1}{\ty_2}} &::=& 
        1 + \tymsrdflt{\ty_1} + \tymsrdflt{\ty_2}\\
    \\
    \hline
    \multicolumn{3}{c}{\tymsrdflt{\rexvar}} \\ 
    \hline 
    \tymsrdflt{\rexvarbound{\ty}{\ty}} &::=& \tymsrdflt{\ty}\\
    \tymsrdflt{\rexvarbound{\tylb}{\tyub}} &::=& 
        2\times(1 + \tymsrdflt{\tylb} + \tymsrdflt{\tyub})\\
    \\
    \hline
    \multicolumn{3}{c}{\tymsrdflt{\dctx}} \\ 
    \hline 
    \tymsrdflt{\square} &::=& 0\\
    \tymsrdflt{\typair{\dctx}{\ty}} &::=& 
        1 + \tymsrdflt{\dctx} + \tymsrdflt{\ty}\\
    \tymsrdflt{\typair{\ty}{\dctx}} &::=& 
        1 + \tymsrdflt{\ty} + \tymsrdflt{\dctx}\\
    \\
    \hline
    \multicolumn{3}{c}{\tymsrdflt{\tysig}} \\ 
    \hline 
    \tymsrdflt{\tyany} &::=& 1\\
    \multicolumn{3}{l}{\ldots} \\
    \tymsrdflt{\tyexist{\vany}{\tylb}{\tyub}{\tysig}} &::=& 
        1 + \tymsrdflt{\tylb} + \tymsrdflt{\tyub} + 
        \tymsr{\AEnv,\varbound{\vany}{\tylb}{\tyub}}{\tysig}\\
    \\
    \hline
    \multicolumn{3}{c}{\tymsrdflt{\dctxsig}} \\ 
    \hline 
    \tymsrdflt{\square} &::=& 0\\
    \tymsrdflt{\typair{\dctxsig}{\tysig}} &::=& 
        1 + \tymsrdflt{\dctxsig} + \tymsrdflt{\tysig}\\
    \tymsrdflt{\typair{\tysig}{\dctxsig}} &::=& 
        1 + \tymsrdflt{\tysig} + \tymsrdflt{\dctxsig}\\
\end{array}
\]
\caption{Measure of types and type signatures}\label{fig:ty-measure}
\end{figure}

To show the decidability of the subtyping algorithm,
we will use the measure $\msrop$ of types and type signatures,
as defined in \figref{fig:ty-measure}.
The measure function is defined recursively and is
similar to the syntactic size,
except for the treatment of type variables: for variables from~\AEnv,
the measure of a variable includes the measures of its bounds.
The definition of the measure for restricted existential variables \rexvar
reflects the fact that \tyinv\iname{\rexvar_1,\ldots,\rexvar_n} represents both
invariant constructors and restricted existential types.
When $\rexvar_i$ represents a single type $\ty_i$,
i.e. $\rexvar_i = \rexvarbound{\ty_i}{\ty_i}$,
its measure is simply the measure of the type $\ty_i$ in
\tyinv\iname{\ldots,\ty_i,\ldots}.
Otherwise, $\rexvar_i = \rexvarbound{\tylb_i}{\tyub_i}$ represents
an existential type with a single occurrence of the bound variable,
\tyexist{\vany_i}{\tylb_i}{\tyub_i}{\tyinv\iname{\ldots,\vany_i,\ldots}},
and measures accordingly, comprising both the binding and its single occurrence.

Note that the measure function itself always terminates and evaluates to a
positive integer. This is the case because for every recursive call
\tymsr{\AEnv'}{\ty'} of \tymsrdflt{\ty}, 
the combined syntactic size of the arguments $\size{\ty'} + \size{\AEnv'}$
is strictly smaller than $\size{\ty} + \size{\AEnv}$.
The same is true for recursive calls
\tymsr{\AEnv'}{\tysig'} of \tymsrdflt{\tysig}.
The syntactic size is defined in~\figref{fig:ty-size}.

\begin{figure}
\footnotesize
\[
\begin{array}{lcl}
    \hline
    \multicolumn{3}{c}{\size{\ty}} \\ 
    \hline 
    \size{\tyany} &::=& 1\\
    \size{\tybot} &::=& 1\\
    \size{\vany}  &::=& 1\\
    \size{\typair{\ty_1}{\ty_2}}  &::=& 1 + \size{\ty_1} + \size{\ty_2}\\
    \size{\tyinv\iname{\rexvar_1,\ldots,\rexvar_n}} &::=&
        1 + \size{\rexvar_1} + \ldots + \size{\rexvar_n}\\
    \size{\tyunion{\ty_1}{\ty_2}} &::=& 1 + \size{\ty_1} + \size{\ty_2}\\
    \\
    \hline
    \multicolumn{3}{c}{\size{\rexvar}} \\ 
    \hline 
    \size{\rexvarbound{\tylb}{\tyub}} &::=& \size{\tylb} + \size{\tyub}\\
    \\
    \hline
    \multicolumn{3}{c}{\size{\dctx}} \\ 
    \hline 
    \size{\square} &::=& 0\\
    \size{\typair{\dctx}{\ty}} &::=& 
        1 + \size{\dctx} + \size{\ty}\\
    \size{\typair{\ty}{\dctx}} &::=& 
        1 + \size{\ty} + \size{\dctx}\\
    \\
    \hline
    \multicolumn{3}{c}{\size{\tysig}} \\ 
    \hline 
    \size{\tyany} &::=& 1\\
    \multicolumn{3}{l}{\ldots} \\
    \size{\tyexist{\vany}{\tylb}{\tyub}{\tysig}} &::=& 
        1 + \size{\tylb} + \size{\tyub} + \size{\tysig}\\
    \\
    \hline
    \multicolumn{3}{c}{\size{\AEnv}} \\ 
    \hline 
    \size{\EmptyEnv} &::=& 0 \\
    \size{\AEnv, \varbound{\vany}{\tylb}{\tyub}} &::=& 
        \size{\AEnv} + \size{\tylb} + \size{\tyub}\\
    \\
    \hline
    \multicolumn{3}{c}{\size{\dctxsig}} \\ 
    \hline 
    \size{\square} &::=& 0\\
    \size{\typair{\dctxsig}{\tysig}} &::=& 
        1 + \size{\dctxsig} + \size{\tysig}\\
    \size{\typair{\tysig}{\dctxsig}} &::=& 
        1 + \size{\tysig} + \size{\dctxsig}\\
\end{array}
\]
\caption{Syntactic size}\label{fig:ty-size}
\end{figure}

In what follows, we will implicitly use the following facts about
distributivity contexts (proofs are straightforward by induction):
\begin{itemize}
    \item $\plug{\dctx}{\ty} = \ty';$
    \item $\plug{\dctx}{\dctx'} = \dctx'';$
    \item $\plug{\dctx}{\plug{\dctx'}{\ty}} = 
        \plug{(\plug{\dctx}{\dctx'})}{\ty};$
    \item $\subtydflt{\dctx}{\dctx'} \land \subtydflt{\ty}{\ty'}
        \implies \subtydflt{\plug{\dctx}{\ty}}{\plug{\dctx'}{\ty'}};$
    \item $\size{\plug{\dctx}{\ty}} = \size{\dctx} + \size{\ty};$
    \item $\tymsrdflt{\plug{\dctx}{\ty}} = \tymsrdflt{\dctx} + \tymsrdflt{\ty};$
    \item $\tymsrdflt{\plug{\dctxsig}{\tysig}} = 
        \tymsrdflt{\dctxsig} + \tymsrdflt{\tysig};$
    \item $\substvars(\plug\dctx\ty) = 
        \plug{\substvars(\dctx)}{\substvars(\ty)}; $    
    \item \TODO{more as needed}
\end{itemize}


\begin{theorem}{Termination of\ \ \subtydflt{\ty}{\ty'}.}%
\label{thm:subty-terminates}
    The subtyping algorithm built from the rules of subtyping of types
    \subtydflt{\ty}{\ty'} terminates.
\end{theorem}
\begin{proof}
    It follows from the fact that for each subtyping rule, 
    the measure of each premise \subtydflt{\ty_p}{\ty'_p}
    is strictly smaller than the measure 
    of the conclusion \subtydflt{\ty}{\ty'}, i.e.
    \[\tymsrdflt{\ty_p} + \tymsrdflt{\ty'_p} \quad<\quad 
    \tymsrdflt{\ty} + \tymsrdflt{\ty'}.\]

    For example, in the case \RST{VarLeft},
    \[\tymsrdflt{\plug\dctx\tyub} + \tymsrdflt{\ty'} < 
    \tymsrdflt{\plug\dctx\vany} + \tymsrdflt{\ty'}\]
    because \[\tymsrdflt{\tyub} < \tymsrdflt{\vany} = 
        1 + \tymsrdflt{\tylb} + \tymsrdflt{\tyub}.\]
\end{proof}

\begin{theorem}{Termination of\ \ \subtyctrdflt{\ty}{\ty'}.}%
\label{thm:subtyctr-terminates}
    The subtyping algorithm built from the rules of
    constrained subtyping of types
    \subtyctrdflt{\ty}{\ty'} terminates.
\end{theorem}
\begin{proof}
    Similarly to the previous theorem, the measure decreases, i.e.
    \[\tymsrdflt{\ty_p} + \tymsrdflt{\ty'_p} \quad<\quad 
    \tymsrdflt{\ty} + \tymsrdflt{\ty'}\]
    for each premise \subtyctrdflt{\ty_p}{\ty'_p}
    of the conclusion \subtyctrdflt{\ty}{\ty'}.
    
    The only interesting case is \RSC{UVar-UnionRight}.
    By the variable names convention~\ref{def:var-names}, $\va \notin \AEnv$.
    Therefore, $\tymsrdflt{\va} = \tymsrdflt{\va_1} = \tymsrdflt{\va_2}$.
\end{proof}

\begin{theorem}{Termination of\ \solvectrdflt.}%
\label{thm:solvectr-terminates}
    The constraints resolution algorithm \solvectrdflt terminates.
\end{theorem}
\begin{proof}
    It follows from the fact that:
    \begin{enumerate}
        \item subtyping algorithms \subtydflt{\ty}{\ty'} and 
            \subtyctrdflt{\ty}{\ty'} used to check the consistency of 
            constraints terminate;
        \item the argument \UEnv of the only recursive call to $\solvectrop$
            is strictly smaller than that of the original call.
    \end{enumerate} 
\end{proof}


\begin{figure}
\footnotesize
\[
\begin{array}{lcl}
    \hline
    \multicolumn{3}{c}{\occdflt{\ty}} \\ 
    \hline 
    \occdflt{\tyany} &::=& \false\\
    \occdflt{\tybot} &::=& \false\\
    \occdflt{\vany}  &::=& \true\\
    \occdflt{\vany'}  &::=& \false\\
    \occdflt{\typair{\ty_1}{\ty_2}}  &::=& \occdflt{\ty_1} \lor \occdflt{\ty_2}\\
    \occdflt{\tyinv\iname{\rexvar_1,\ldots,\rexvar_n}} &::=&
        \occdflt{\rexvar_1} \lor \ldots \lor \occdflt{\rexvar_n}\\
    \occdflt{\tyunion{\ty_1}{\ty_2}} &::=& \occdflt{\ty_1} \lor \occdflt{\ty_2}\\
    \\
    \hline
    \multicolumn{3}{c}{\occdflt{\rexvar}} \\ 
    \hline 
    \occdflt{\rexvarbound{\tylb}{\tyub}} &::=& \occdflt{\tylb} \lor \occdflt{\tyub}\\
    \\
    \hline
    \multicolumn{3}{c}{\occdflt{\tysig}} \\ 
    \hline 
    \occdflt{\tyany} &::=& \false\\
    \multicolumn{3}{l}{\ldots} \\
    \occdflt{\tyexist{\vany}{\tylb}{\tyub}{\tysig}} &::=& \true\\
    \occdflt{\tyexist{\vany'}{\tylb}{\tyub}{\tysig}} &::=& 
        \occdflt{\tylb} \lor \occdflt{\tyub} \lor \occdflt{\tysig}\\
    \\
    \hline
    \multicolumn{3}{c}{\occdflt{\AEnv}} \\ 
    \hline 
    \occdflt{\EmptyEnv} &::=& \false \\
    \occdflt{\AEnv, \varbound{\vany}{\tylb}{\tyub}} &::=& \true\\
    \occdflt{\AEnv, \varbound{\vany'}{\tylb}{\tyub}} &::=& 
        \occdflt{\AEnv} \lor \occdflt{\tylb} \lor \occdflt{\tyub}\\
\end{array}
\]
\caption{Occurrence of a variable}\label{fig:var-occ}
\end{figure}

\begin{lemma}{Context weakening in $\msrop$.}%
\label{lem:msr-weakening}
    The measure of a type signature does not change if the environment
    is extended (in any position) with a variable that occurs neither
    in the signature nor in the environment, i.e.,
    $\forall \tysig, \AEnv, \AEnv'. 
    \forall \varbound{\vany}{\tylb}{\tyub} \text{ s.t. } 
    \lnot \occdflt{\tysig} \land 
    \lnot \occdflt{\AEnv} \land \lnot \occdflt{\AEnv'}.$
    \[\tymsr{\concat{\AEnv}{\AEnv'}}{\tysig} = 
        \tymsr{\concat{\AEnv,\varbound{\vany}{\tylb}{\tyub}}{\AEnv'}}{\tysig},\]
    where \concat{\AEnv}{\AEnv''} denotes the concatenation of lists,
    and $\occop$ is defined in~\figref{fig:var-occ}.
\end{lemma}
\begin{proof}
    By strong induction on $n = \size{\AEnv} + \size{\AEnv'} + \size{\tysig}$.

    Case $n = 0$ is not possible, as the minimal size of a type signature is 1.
    
    In the inductive step for $n$, the induction hypothesis (IH) states that
    $\forall n'<n. \forall \tysig', \AEnv'', \AEnv'''  \text{ s.t. }
    n' = \size{\AEnv''} + \size{\AEnv'''} + \size{\tysig'}.
    \forall \varbound{\vany}{\tylb}{\tyub} \text{ s.t. } 
    \lnot \occdflt{\tysig'} \land 
    \lnot \occdflt{\AEnv''} \land \lnot \occdflt{\AEnv'''}.$
    \[\tymsr{\concat{\AEnv''}{\AEnv'''}}{\tysig'} = 
    \tymsr{\concat{\AEnv'',\varbound{\vany}{\tylb}{\tyub}}{\AEnv'''}}{\tysig'}.\]
    
    Case analysis on \tysig. Base cases \tyany and \tybot are straightforward.
    Cases $\times$, \tyinv\iname{\ldots}, and $\cup$ are also straightforward
    using the induction hypothesis for components of \tysig.
    The remaining cases are:
    \begin{itemize}
        \item Case $\vany'$. Case analysis on $\AEnv'$.
            \begin{itemize}
                \item Case \EmptyEnv. Because $\lnot \occdflt{\vany'},$ we know
                    $\vany \neq \vany'$. Thus,
                    $\tymsr{\AEnv, \varbound{\vany}{\tylb'}{\tyub'}}{\vany'} =
                    \tymsr{\AEnv}{\vany'}$ by definition of $\msrop$.
                \item Case $\AEnv', \varbound{\vany'}{\tylb'}{\tyub'}$.
                    By definition,
                    \[\tymsr{\concat{\AEnv}{\AEnv', \varbound{\vany'}{\tylb'}{\tyub'}}}{\vany'} =
                    1 + \tymsr{\concat{\AEnv}{\AEnv'}}{\tylb'} + 
                    \tymsr{\concat{\AEnv}{\AEnv'}}{\tyub'}.\]
                    Since $\size{\AEnv} + \size{\AEnv'} + \size{\tylb'}\ <\ 
                    \size{\AEnv} + \size{\AEnv', \varbound{\vany'}{\tylb'}{\tyub'}} + \size{\vany'} =
                    \size{\AEnv} + \size{\AEnv'} + \size{\tylb'} + \size{\tyub'} + 1$,
                    the IH applies with $\AEnv'' = \AEnv, \AEnv''' = \AEnv', 
                    \tysig' = \tylb'$, which gives 
                    $\tymsr{\concat{\AEnv}{\AEnv'}}{\tylb'} = 
                    \tymsr{\concat{\AEnv, \varbound{\vany}{\tylb}{\tyub}}{\AEnv'}}{\tylb'}$,
                    and similarly for $\tyub'$. Thus,
                    \[ \tymsr{\concat{\AEnv}{\AEnv', \varbound{\vany'}{\tylb'}{\tyub'}}}{\vany'} =
                    \tymsr{\concat{\AEnv, \varbound{\vany}{\tylb}{\tyub}}{\AEnv', \varbound{\vany'}{\tylb'}{\tyub'}}}{\vany'}. \]
            \end{itemize}
        \item Case \tyexist{\vany'}{\tylb'}{\tyub'}{\tysig}.
            By definition,
            \[\tymsr{\concat{\AEnv}{\AEnv'}}{\tyexist{\vany'}{\tylb'}{\tyub'}{\tysig}} =
            1 + \tymsr{\concat{\AEnv}{\AEnv'}}{\tylb'} + \msrop(\ldots\tyub') +
            \tymsr{\concat{\AEnv}{\AEnv', \varbound{\vany'}{\tylb'}{\tyub'}}}{\tysig}.\]
            Similarly to the last subcase of the $\vany'$ case, the IH applies
            to $\tylb'$ and $\tyub'$.
            Furthermore, since $\size{\AEnv} + \size{\AEnv'} + 
            \size{\tylb'} + \size{\tyub'} + \size{\tysig} <
            \size{\AEnv} + \size{\AEnv'} + 1 + \size{\tylb'} + \size{\tyub'}
            + \size{\tysig},$
            the IH applies to \tysig with $\AEnv'' = \AEnv, 
            \AEnv''' = (\AEnv', \varbound{\vany'}{\tylb'}{\tyub'}), 
            \tysig' = \tysig$.
            All pieces combined, 
            \[\tymsr{\concat{\AEnv}{\AEnv'}}{\tyexist{\vany'}{\tylb'}{\tyub'}{\tysig}} =
            \tymsr{\concat{\AEnv, \varbound{\vany}{\tylb}{\tyub}}{\AEnv'}}{\tyexist{\vany'}{\tylb'}{\tyub'}{\tysig}}.\]
            % and
            % \[\tymsr{\concat{\AEnv, \varbound{\vany}{\tylb}{\tyub}}{\AEnv'}}{\tyexist{\vany'}{\tylb'}{\tyub'}{\tysig}} =
            % 1 + \tymsr{\concat{\AEnv, \varbound{\vany}{\tylb}{\tyub}}{\AEnv'}}{\tylb'} + 
            % \tymsr{\concat{\AEnv, \varbound{\vany}{\tylb}{\tyub}}{\AEnv'}}{\tyub'} +
            % \tymsr{\concat{\AEnv, \varbound{\vany}{\tylb}{\tyub}}{\AEnv', \varbound{\vany'}{\tylb'}{\tyub'}}}{\tysig}.\]
    \end{itemize}
\end{proof}

\begin{theorem}{Termination of\ \ \subtysigdflt{\tysig}{\tysig'}.}%
\label{thm:subtysig-terminates}
    The subtyping algorithm built from the rules of
    subtyping of type signatures
    \subtysigdflt{\tysig}{\tysig'} terminates.
\end{theorem}
\begin{proof}
    It follows from the fact that for each subtyping rule, 
    the measure of each premise \subtysigdflt{\tysig_p}{\tysig'_p}
    is strictly smaller than the measure 
    of the conclusion \subtysigdflt{\tysig}{\tysig'}, i.e.
    \[\tymsr{\AEnv'}{\tysig_p} + \tymsr{\concat{\AEnv'}{\UEnv'}}{\tysig'_p} \quad<\quad 
    \tymsr{\AEnv}{\tysig} + \tymsr{\concat{\AEnv}{\UEnv}}{\tysig'}.\]

    Most of the cases are similar to the cases of~\thmref{thm:subty-terminates}
    on the termination of \subtydflt{\ty}{\ty'}.
    The remaining cases are:
    \begin{itemize}
        \item \RSS{InvLeft}. Since \vx is a fresh variable,
            by~\lemref{lem:msr-weakening} (weakening),
            $\tymsr{\AEnv, \varbound{\vx}{\tylb}{\tyub}}
                {\plug\dctxsig{\tyinv\iname{\ldots}}} = 
            \tymsrdflt{\plug\dctxsig{\tyinv\iname{\ldots}}},$ and also
            $\tymsr{\concat{\AEnv, \varbound{\vx}{\tylb}{\tyub}}{\UEnv}}{\tysig'}
            = \tymsr{\concat{\AEnv}{\UEnv}}{\tysig'}.$

            By the definition of $\msrop$,
            \[\tymsr{\AEnv, \varbound{\vx}{\tylb}{\tyub}}{\vx} =
            1 + \tymsrdflt{\tylb} + \tymsrdflt{\tyub} <
            2\times(1 + \tymsrdflt{\tylb} + \tymsrdflt{\tyub}) =
            \tymsrdflt{\rexvarbound{\tylb}{\tyub}},\]
            which concludes the case.
        \item \RSS{ExistLeft}. By the variables names
            convention~\defref{def:var-names}, \vx is different from variables
            in \AEnv, \UEnv, as well as bound variables of \dctxsig, \tysig,
            and $\tysig'$. Therefore, by~\lemref{lem:msr-weakening} (weakening),
            $\tymsr{\AEnv, \varbound{\vx}{\tylb}{\tyub}}{\dctxsig} = 
            \tymsrdflt{\dctxsig}$, and similarly for $\tysig'$.
            By the definition of $\msrop$,
            \[\tymsrdflt{\tyexist{\vx}{\tylb}{\tyub}{\tysig}} = 1 +
                \tymsrdflt{\tylb} + \tymsrdflt{\tyub} + 
                \tymsr{\AEnv, \varbound{\vx}{\tylb}{\tyub}}{\tysig},\]
            which is strictly larger than
            \tymsr{\AEnv, \varbound{\vx}{\tylb}{\tyub}}{\tysig} in the premise,
            which concludes the case.
        \item \RSS{ExistRight}. Similarly to \RSS{ExistLeft}.
        \item \RSS{Types}. The first premise,
            \subtyctrR{\AEnv}{\dom\UEnv}{\ty}{\ty'}{\CSet},
            terminates by \thmref{thm:subtyctr-terminates}.
            Since the constraints resolution \solvectrdflt terminates
            by \thmref{thm:solvectr-terminates}, the entire step also terminates.
    \end{itemize}
\end{proof}


\subsection{Properties of Subtyping of Types}%
\label{subsec:props-subty-proof}
%% ======================================================================

%We will use an implicit assumption that all types are valid
%in the given context.

\begin{theorem}{Reflexivity of subtyping of types.}\label{thm:sub-ty-refl}
    $
        \forall \ty, \AEnv, \text{ s.t. } \tyvlddflt{\ty}.\quad
        \subtydflt{\ty}{\ty}.
    $
\end{theorem}
\begin{proof}
    By induction on the structure of \ty.
    \begin{itemize}
        \item Case \tyany by \RST{Top}.
        \item Case \tybot by \RST{Bot}.
        \item Case \vany by \RST{VarRefl}.
        \item Case \typair{\ty_1}{\ty_2} by IH and \RST{Tuple}.
        \item Case \tyinv\iname{\rexvar_1,\ldots,\rexvar_n} by IH on
            $\tylb_i, \tyub_i$, and \RST{Inv}.
        \item Case \tyunion{\ty_1}{\ty_2} by IH, \RST{UnionRight},
            and \RST{UnionLeft}.
    \end{itemize}
\end{proof}

\begin{lemma}{Subtyping of \tybot implies arbitrary subtyping.}\label{lem:sub-of-bot}
    \[
    \forall \ty, \dctx_{\tybot}, \AEnv.\quad 
    \subtydflt{\ty}{\plug{\dctx_{\tybot}}\tybot}
    \quad\implies\quad
    (\forall \ty', \dctx'.\quad \subtydflt{\plug{\dctx'}{\ty}}{\ty'}).
    \]
\end{lemma}
\begin{proof}
    By induction on the derivation of 
    \subtydflt{\ty}{\plug{\dctx_{\tybot}}\tybot}.
    \begin{itemize}
        \item Case \RST{Bot}
            \subtydflt{\plug\dctx\tybot}{\plug{\dctx_{\tybot}}{\tybot}}
            where $\ty = \plug\dctx\tybot$.

            The case concludes by \RST{Bot}:
            \subtydflt{\plug{\dctx'}{\plug\dctx\tybot}}{\ty'}. 
        \item Case \RST{VarLeft}
            \subtydflt{\plug\dctx\vany}{\plug{\dctx_{\tybot}}{\tybot}}.

            By inversion, \subtydflt{\plug\dctx\tyub}{\plug{\dctx_{\tybot}}{\tybot}}.
            By IH, \subtydflt{\plug{\dctx'}{\plug\dctx\tyub}}{\ty'}.
            Thus, the case concludes by \RST{VarLeft}: 
            \subtydflt{\plug{\dctx'}{\plug\dctx\vany}}{\ty'}.
        \item Case \RST{Tuple}, subcase where
            $\dctx_{\tybot} = \typair{\dctx'_{\tybot}}{\ty'_2}$
            ($\dctx_{\tybot} = \square$ is not possible, and
            $\dctx_{\tybot} = \typair{\ty_1}{\dctx'_{\tybot}}$
            is proved analogously),
            $\ty = \typair{\ty_1}{\ty_2}$:
            \subtydflt{\typair{\ty_1}{\ty_2}}
            {\typair{\plug{\dctx'_{\tybot}}{\tybot}}{\ty'_2}}.

            By inversion, \subtydflt{\ty_1}{\plug{\dctx'_{\tybot}}{\tybot}}.
            By IH, \subtydflt{\plug{\dctx'^h}{\ty_1}}{\ty'} for all $\dctx'^h$,
            so we can take it to be \plug{\dctx'}{\typair{\square}{\ty_2}}.
            Thus, the case concludes by IH: 
            \subtydflt{\plug{\dctx'}{\typair{\ty_1}{\ty_2}}}{\ty'}.
        \item Case \RST{UnionLeft}
            \subtydflt{\plug\dctx{\tyunion{\ty_1}{\ty_2}}}{\plug{\dctx_{\tybot}}{\tybot}}
            where $\ty = \tyunion{\ty_1}{\ty_2}$.
            By inversion, 
            \subtydflt{\plug\dctx{\ty_1}}{\plug{\dctx_{\tybot}}{\tybot}} and
            \subtydflt{\plug\dctx{\ty_2}}{\plug{\dctx_{\tybot}}{\tybot}}.
            By IH, \subtydflt{\plug{\dctx'}{\plug\dctx{\ty_1}}}{\ty'} and
            \subtydflt{\plug{\dctx'}{\plug\dctx{\ty_2}}}{\ty'}.
            Thus, the case concludes by \RST{UnionLeft}: 
            \subtydflt{\plug{\dctx'}{\plug\dctx{\tyunion{\ty_1}{\ty_2}}}}{\ty'}.
    \end{itemize}
    The remaining cases 
    (\RST{Top}, \RST{VarRefl}, \RST{VarRight}, \RST{Inv}, \RST{UnionRight}) 
    are not possible.
\end{proof}

\begin{lemma}{Subtyping of inner union on the right.}%
\label{lem:sub-inner-union-right}
    $\forall \ty, \dctx', \ty'_1, \ty'_2, \AEnv, \text{ s.t. }
    \tyvlddflt{\ty, \dctx', \ty'_1, \ty'_2}.$
    \[
        \begin{array}{ccc}
        \subtydflt{\ty}{\plug{\dctx'}{\tyunion{\ty'_1}{\ty'_2}}}\\
        \quad\implies\quad\\
        (\forall \dctx_1, \dctx_2, \text{ s.t. }
        \tyvlddflt{\dctx_1, \dctx_2} \land
        \subtydflt{\dctx_1}{\dctx_2}.\quad
        \subtydflt
            {\plug{\dctx_1}{\ty}}
            {\tyunion
                {\plug{\dctx_2}{\plug{\dctx'}{\ty'_1}}}
                {\plug{\dctx_2}{\plug{\dctx'}{\ty'_2}}}
            }).
        \end{array}
    \]
\end{lemma}
\begin{proof}
    By induction on the derivation of
    \subtydflt{\ty}{\plug{\dctx'}{\tyunion{\ty'_1}{\ty'_2}}}.
    \begin{itemize}
        \item Case \RST{Bot} by \RST{Bot}.
        \item Case \RST{VarLeft} by inversion, IH, and \RST{VarLeft}.
        \item Case \RST{Tuple}, subcase where
            $\dctx' = \typair{\dctx''}{\ty'}$:
            \subtydflt{\typair{\ty_1}{\ty_2}}
                {\typair{\plug{\dctx''}{\tyunion{\ty'_1}{\ty'_2}}}{\ty'}}.
            By inversion,
            \subtydflt{\ty_1}{\plug{\dctx''}{\tyunion{\ty'_1}{\ty'_2}}} and
            \subtydflt{\ty_2}{\ty'}. 
            
            By IH applied to 
            \subtydflt{\ty_1}{\plug{\dctx''}{\tyunion{\ty'_1}{\ty'_2}}},
            \subtydflt{\plug{\dctx^h_1}{\ty_1}}
                {\tyunion
                    {\plug{\dctx^h_2}{\plug{\dctx''}{\ty'_1}}}
                    {\plug{\dctx^h_2}{\plug{\dctx''}{\ty'_2}}}
                }
            for all $\dctx^h_1, \dctx^h_2$ s.t. $\subtydflt{\dctx^h_1}{\dctx^h_2}$.
            Thus, we can take them to be \plug{\dctx_2}{\typair{\square}{\ty_2}} 
            and \plug{\dctx_2}{\typair{\square}{\ty'}}, 
            respectively, which concludes the case with
            \subtydflt{\plug{\dctx_1}{\typair{\ty_1}{\ty_2}}}
                {\tyunion
                    {\plug{\dctx_2}{\typair{\plug{\dctx''}{\ty'_1}}{\ty_2}}}
                    {\plug{\dctx_2}{\typair{\plug{\dctx''}{\ty'_2}}{\ty'}}}
                }
        \item Case \RST{UnionLeft} by inversion, IH, and \RST{UnionLeft}.
        \item Case \RST{UnionRight}, subcase $i = 1$ where $\dctx' = \square$:
            \subtydflt{\ty}{\tyunion{\ty'_1}{\ty'_2}}.
            By inversion, \subtydflt\ty{\ty'_1}.
            By assumption, \subtydflt{\dctx_1}{\dctx_2}, and thus,
            \subtydflt{\plug{\dctx_1}{\ty}}{\plug{\dctx_2}{\ty'_1}}.
            The case concludes by \RST{UnionRight} with $i=1$:
            \subtydflt{\plug{\dctx_1}{\ty}}
                {\tyunion{\plug{\dctx_2}{\ty'_1}}{\plug{\dctx_2}{\ty'_2}}}.
    \end{itemize}
    The remaining cases 
    (\RST{Top}, \RST{VarRefl}, \RST{VarRight}, \RST{Inv}) 
    are not possible.
\end{proof}

\begin{lemma}{Adding inner union on the right.}%
\label{lem:add-inner-union-right}
    $\forall \ty, \dctx', \ty', \AEnv, \text{ s.t. }
    \tyvlddflt{\ty, \dctx', \ty'}.$
    \[
        \subtydflt{\ty}{\plug{\dctx'}{\ty'}}
        \quad\implies\quad
        (\forall \ty''.\ \subtydflt{\ty}{\plug{\dctx'}{\tyunion{\ty'}{\ty''}}}).
    \]
\end{lemma}
\begin{proof}
    By induction on the derivation of
    \subtydflt{\ty}{\plug{\dctx'}{\ty'}}.
    \begin{itemize}
        \item Case \RST{Top} where $\dctx'=\square$. By assumption
            \subtydflt{\ty}{\tyany} and \RST{UnionRight} with $i=1$,
            \subtydflt{\ty}{\tyunion{\tyany}{\ty''}}.
        \item Case \RST{Bot} by \RST{Bot}.
        \item Case \RST{VarRefl} where $\dctx'=\square$
            by assumption and \RST{UnionRight} with $i=1$.
        \item Case \RST{VarLeft} by inversion, IH, and \RST{VarLeft}.
        \item Case \RST{VarRight} where $\dctx'=\square$
            by assumption and \RST{UnionRight} with $i=1$.
        \item Case \RST{Tuple}. 
            Subcase $\dctx'$ by assumption and \RST{UnionRight} with $i=1$.
            The other two subcases by inversion, IH, and \RST{Tuple}.
        \item Case \RST{Inv} where $\dctx'=\square$
            by assumption and \RST{UnionRight} with $i=1$.
        \item Case \RST{UnionLeft} by inversion, IH, and \RST{UnionLeft}.
        \item Case \RST{UnionRight} where $\dctx'=\square$
            by assumption and \RST{UnionRight} with $i=1$.
    \end{itemize}
\end{proof}

\begin{lemma}{Subtyping of union on the right.}%
\label{lem:sub-union-right}
    $\forall \ty, \dctx', \ty'_1, \ty'_2, \AEnv, \text{ s.t. }
    \tyvlddflt{\ty, \dctx', \ty'_1, \ty'_2}.$
    \[
        \subtydflt{\ty}{\tyunion{\plug{\dctx'}{\ty'_1}}{\plug{\dctx'}{\ty'_2}}}
        \quad\implies\quad
        \subtydflt{\ty}{\plug{\dctx'}{\tyunion{\ty'_1}{\ty'_2}}}.
    \]
\end{lemma}
\begin{proof}
    By induction on the derivation of
    \subtydflt{\ty}{\tyunion{\plug{\dctx'}{\ty'_1}}{\plug{\dctx'}{\ty'_2}}}.
    Four cases are possible: \RST{Bot}, \RST{VarLeft}, \RST{UnionLeft}, and
    \RST{UnionRight}. The first three are analogous to the cases of
    \lemref{lem:sub-inner-union-right} (subtyping inner union on the right): 
    inversion, IH, constructor.
    The remaining case is \RST{UnionRight}, subcase $i=1$.
    By inversion, \subtydflt{\ty}{\plug{\dctx'}{\ty'_1}}.
    Thus, the case concludes by \lemref{lem:add-inner-union-right} (adding inner
    union on the right):
    \subtydflt{\ty}{\plug{\dctx'}{\tyunion{\ty'_1}{\ty'_2}}}.
    % bot by bot
    % VarLeft by inversion, IH, VarLeft
    % UnionLeft by inversion, IH, UnionLeft
\end{proof}

\begin{lemma}{Context weakening in subtyping of types.}%
\label{lem:subty-weakening}
    $\forall \AEnv, \ty, \ty'.\ \forall \AEnv' \text{ s.t. }$\\
    $\dom{\AEnv'} \cap \dom{\AEnv} = \varnothing \land 
    (\forall \vany \in \dom{\AEnv'}. 
        \lnot \occ{\vany}{\ty} \land \lnot \occ{\vany}{\ty}),$
    \[ \subtydflt{\ty}{\ty'} \quad\implies\quad 
    \subty{\concat{\AEnv}{\AEnv'}}{\ty}{\ty'}. \]
\end{lemma}
\begin{proof}
    Straightforward induction on the derivation of \subtydflt{\ty}{\ty'}.
\end{proof}


\begin{theorem}{Transitivity of subtyping of types.}\label{thm:sub-ty-trans}
    $\forall \ty_1, \ty_2, \ty_3, \AEnv, \text{ s.t. }
    \tyvlddflt{\ty_1, \ty_2, \ty_3} \land \tyvld{\,}{\AEnv}.$
    \[
        \subtydflt{\ty_1}{\ty_2} \land \subtydflt{\ty_2}{\ty_3}
        \quad\implies\quad
        \subtydflt{\ty_1}{\ty_3}.
    \]
\end{theorem}
\begin{proof}
    By strong induction on
    $n = \tymsrdflt{\ty_1} + 2\times\tymsrdflt{\ty_2} + \tymsrdflt{\ty_3}$.
    Cases $n = 1..3$ are not possible as the minimal measure of a type is 1.
    In the inductive step for $n$, the induction hypothesis (IH) states that
    $\forall n'<n. \forall \ty'_1, \ty'_2, \ty'_3 \text{ s.t. }
    n' = \tymsrdflt{\ty'_1} + 2\times\tymsrdflt{\ty'_2} + \tymsrdflt{\ty'_3},$
    it holds that
    \[
        \subtydflt{\ty'_1}{\ty'_2} \land \subtydflt{\ty'_2}{\ty'_3}
        \quad\implies\quad
        \subtydflt{\ty'_1}{\ty'_3}.
    \]
    Case analysis on \subtydflt{\ty_2}{\ty_3} (the most interesting cases are
    highlighted in bold).
    \begin{itemize}
        \item Case \RST{Top} \subtydflt{\ty_2}{\tyany} where $\ty_3 = \tyany$
            concludes by \RST{Top}: \subtydflt{\ty_1}{\tyany}.
        \item Case \RSS{Bot} \subtydflt{\plug\dctx\tybot}{\ty_3}
            where $\ty_2=\plug\dctx\tybot$.
            The case concludes by \lemref{lem:sub-of-bot}
            (subtyping of \tybot) applied
            to the assumption \subtydflt{\ty_1}{\plug\dctx\tybot}
            with $\dctx' = \square, \ty' = \ty_3$:
            \subtydflt{\ty_1}{\ty_3}.
        \item Case \RST{VarRefl} \subtydflt{\vany}{\vany}.
            Since $\ty_2=\ty_3=\vany$, the case concludes by the assumption
            \subtydflt{\ty_1}{\vany}.
        \item \textbf{Case \RST{VarLeft}} \subtydflt{\plug\dctx\vany}{\ty_3} 
            where $\ty_2 = \plug\dctx\vany$.
            By inversion, $\varbound{\vany}{\tylb}{\tyub} \in \AEnv$ and
            \subtydflt{\plug\dctx{\tyub}}{\ty_3}.
            By assumption, \subtydflt{\ty_1}{\plug\dctx\vany}.
            We will need the following auxiliary fact.

            \begin{lemma}\label{lem:sub-var-right-sub-ub}
                \subtydflt{\ty_1}{\plug\dctx\tyub}.
            \end{lemma}
            \begin{proof}
                By induction on \subtydflt{\ty_1}{\plug\dctx\vany}.
                \begin{itemize}
                    \item Case \RST{Bot} by \RST{Bot}.
                    \item Case \RST{VarRefl} \subtydflt{\vany}{\vany}.
                        By \thmref{thm:sub-ty-refl} (reflexivity),
                        \subtydflt{\tyub}{\tyub}.
                        Thus, by \RST{VarLeft}, \subtydflt{\vany}{\tyub}.
                    \item Case \RST{VarLeft} 
                        \subtydflt{\plug{\dctx'}{\vany'}}{\plug\dctx\vany}.
                        By inversion, $\varbound{\vany'}{\tylb'}{\tyub'} \in \AEnv$
                        and \subtydflt{\plug{\dctx'}{\tyub'}}{\plug\dctx\vany}.
                        By IH,
                        \subtydflt{\plug{\dctx'}{\tyub'}}{\plug\dctx\tyub}.
                        Thus, by \RST{VarLeft},
                        \subtydflt{\plug{\dctx'}{\vany'}}{\plug\dctx\tyub}.
                    \item \textbf{Case \RST{VarRight}} \subtydflt{\ty_1}{\vany}.
                        By inversion, $\varbound{\vany}{\tylb}{\tyub} \in \AEnv$
                        and \subtydflt{\ty_1}{\tylb}.
                        By inversion of the assumptions \tyvlddflt{\vany} and
                        \tyvld{\,}{\AEnv} and by~\lemref{lem:subty-weakening}
                        (context weakening), we have \subtydflt{\tylb}{\tyub}.

                        Since $\ty_2 = \vany$ and $\tymsrdflt{\vany} = 
                        1 + \tymsrdflt{\tylb} + \tymsrdflt{\tyub}$, we have
                        $\tymsrdflt{\ty_1} + 2\times\tymsrdflt{\tylb} + 
                        \tymsrdflt{\tyub} < \tymsrdflt{\ty_1} + 
                        2\times\tymsrdflt{\vany} + \tymsrdflt{\ty_3}$. Thus,
                        IH for \emph{transitivity} is applicable to 
                        \subtydflt{\ty_1}{\tylb} and \subtydflt{\tylb}{\tyub},
                        which concludes the case with \subtydflt{\ty_1}{\tyub}.
                    \item Case \RST{Tuple}, subcase
                        $\dctx = \typair{\dctx'}{\ty_{22}}$
                        ($\dctx = \square$ is not possible, and
                        $\dctx = \typair{\ty_{21}}{\dctx'}$
                        is proved analogously), i.e.
                        \subtydflt
                            {\typair{\ty_{11}}{\ty_{12}}}
                            {\typair{\plug{\dctx'}\vany}{\ty_{22}}}.
                        By inversion, \subtydflt{\ty_{11}}{\plug{\dctx'}\vany}
                        and \subtydflt{\ty_{12}}{\ty_{22}}.
                        By IH, \subtydflt{\ty_{11}}{\plug{\dctx'}\tyub}.
                        Thus, by \RST{Tuple},
                        \subtydflt
                            {\typair{\ty_{11}}{\ty_{12}}}
                            {\typair{\plug{\dctx'}\tyub}{\ty_{22}}}.
                    \item Case \RST{UnionLeft} is proved analogously 
                        to \RST{Tuple}: by inversion, IH, and \RST{UnionLeft}.
                \end{itemize}
                The remaining cases
                (\RST{Top}, \RST{Tuple}, \RST{Inv}, \RST{UnionRight}) 
                are not possible.
            \end{proof}

            Be \lemref{lem:sub-var-right-sub-ub} above,
            \subtydflt{\ty_1}{\plug\dctx\tyub}.
            Since $\ty_2 = \plug\dctx\vany$ and 
            $\tymsrdflt{\tyub} < \tymsrdflt{\vany}$,
            IH is applicable to \subtydflt{\ty_1}{\plug\dctx\tyub} and 
            \subtydflt{\plug\dctx\tyub}{\ty_3}, which concludes the case with
            \subtydflt{\ty_1}{\ty_3}.
        \item Case \RST{VarRight} \subtydflt{\ty_2}{\vany}.
            By inversion, $\varbound{\vany}{\tylb}{\tyub} \in \AEnv$ and
            \subtydflt{\ty_2}{\tylb}.
            Since $\tymsrdflt{\tylb} < \tymsrdflt{\vany}$, by IH,
            \subtydflt{\ty_1}{\tylb}.
            Thus, by \RST{VarRight}, \subtydflt{\ty_1}{\vany}.
        \item Case \RST{Tuple} 
            \subtydflt{\typair{\ty_{21}}{\ty_{22}}}{\typair{\ty_{31}}{\ty_{32}}}
            where $\ty_i = \typair{\ty_{i1}}{\ty_{i2}}$.
            Case analysis on \subtydflt{\ty_1}{\typair{\ty_{21}}{\ty_{22}}}.
            \begin{itemize}
                \item Case \RST{Bot} by \RST{Bot}.
                \item Case \RST{VarLeft} by inversion, IH on 
                    \subtydflt{\plug\dctx\tyub}{\ty_2} and
                    \subtydflt{\ty_2}{\ty_3}, and \RST{VarLeft} on
                    \subtydflt{\plug\dctx\tyub}{\ty_3}.
                \item Case \RST{Tuple} by inversion, IH on
                    \subtydflt{\ty_{1j}}{\ty_{2j}} and 
                    \subtydflt{\ty_{2j}}{\ty_{3j}}, and \RST{Tuple}.
                \item Case \RST{UnionLeft} 
                    \subtydflt{\plug\dctx{\tyunion{\ty_{11}}{\ty_{12}}}}{\ty_2}
                    by inversion, IH on
                    \subtydflt{\plug\dctx{\ty_{1j}}}{\ty_2} and
                    \subtydflt{\ty_2}{\ty_3}, and \RST{UnionLeft}.
            \end{itemize}
            The remaining cases
            (\RST{Top}, \RST{VarRefl}, \RST{VarRight}, \RST{Inv}, \RST{UnionRight}) 
            are not possible.
        \item Case \RST{Inv} is proved similarly to \RST{Tuple}, with
            possible cases of \subtydflt{\ty_1}{\ty_2} being
            \RST{Bot}, \RST{VarLeft}, \RST{Inv}, and \RST{UnionLeft}.
        \item \textbf{Case \RST{UnionLeft}} 
            \subtydflt{\plug\dctx{\tyunion{\ty_{21}}{\ty_{22}}}}{\ty_3}.
            By \lemref{lem:sub-inner-union-right} applied to
            \subtydflt{\ty_1}{\plug\dctx{\tyunion{\ty_{21}}{\ty_{22}}}}
            with $\dctx_1=\square,\dctx_2=\square$,
            \subtydflt{\ty_1}{\tyunion{\plug\dctx{\ty_{21}}}{\plug\dctx{\ty_{22}}}}.
            Case analysis on the latter.
            \begin{itemize}
                \item Case \RST{Bot} by \RST{Bot}.
                \item Case \RST{VarLeft} where $\ty_1 = \plug{\dctx'}\vany$.
                    By inversion, $\varbound{\vany}{\tylb}{\tyub} \in \AEnv$ and
                    \subtydflt{\plug{\dctx'}\tyub}
                    {\tyunion{\plug\dctx{\ty_{21}}}{\plug\dctx{\ty_{22}}}}.
                    By \lemref{lem:sub-union-right} applied to the latter,
                    \subtydflt{\plug{\dctx'}\tyub}
                    {\plug\dctx{\tyunion{\ty_{21}}{\ty_{22}}}}.

                    Since $\tymsrdflt{\tyub} < \tymsrdflt{\vany}$, by IH,
                    \subtydflt{\plug{\dctx'}\tyub}{\ty_3}.
                    Thus, the case concludes by \RST{VarLeft}:
                    \subtydflt{\plug{\dctx'}\vany}{\ty_3}.
                \item Case \RST{UnionLeft} similarly to \RST{VarLeft}.
                \item Case \RST{UnionRight}, subcase $i=1$.
                    By inversion, \subtydflt{\ty_1}{\plug\dctx{\ty_{21}}}.
                    By inversion of the outer case assumption
                    \subtydflt{\plug\dctx{\tyunion{\ty_{21}}{\ty_{22}}}}{\ty_3},
                    \subtydflt{\plug\dctx{\ty_{21}}}{\ty_3}.
                    Since $\tymsrdflt{\plug\dctx{\ty_{21}}} < 
                    \tymsrdflt{\plug\dctx{\tyunion{\ty_{21}}{\ty_{22}}}}$,
                    by IH, \subtydflt{\ty_1}{\ty_3}.
            \end{itemize}
            The remaining cases
            (\RST{Top}, \RST{VarRefl}, \RST{VarRight}, \RST{Tuple}, \RST{Inv}) 
            are not possible.
        \item Case \RST{UnionRight} 
            \subtydflt{\ty_2}{\tyunion{\ty_{31}}{\ty_{32}}} where 
            $\ty_3 = \tyunion{\ty_{31}}{\ty_{32}}$, subcase $i=1$.
            By inversion, \subtydflt{\ty_2}{\ty_{31}}. Since
            $\tymsrdflt{\ty_{31}} < \tymsrdflt{\tyunion{\ty_{31}}{\ty_{32}}}$,
            by IH, \subtydflt{\ty_1}{\ty_{31}}.
            Thus, the case concludes by \RST{UnionRight}.
    \end{itemize}
\end{proof}

%% This fact we use implicitly
% \begin{lemma}{Preservation of subtyping by distributivity context.}%
%     \[
%         \forall \AEnv, \dctx, \ty, \ty'.\quad
%         \subtydflt{\ty}{\ty'} \quad\implies\quad
%         \subtydflt{\plug\dctx\ty}{\plug\dctx{\ty'}}.
%      \]
% \end{lemma}
% \begin{proof}
%     Straightforward by induction on the structure of \dctx,
%     using the assumption \subtydflt{\ty}{\ty'} in the base case
%     and \thmref{thm:sub-ty-refl} (reflexivity) in the inductive cases.
% \end{proof}

\begin{theorem}{Soundness of subtyping of types with respect to a subsitution.}%
\label{thm:subty-sound-subst}
    $\forall \AEnv, \ty, \ty' \text{ s.t. } \tyvlddflt{\ty, \ty'}.\ 
     \forall \AEnv', \substvars \text{ s.t. } \vldinenvdflt{\substvars}.$
     \[ 
        \subtydflt{\ty}{\ty'} \quad\implies\quad
        \subty{\AEnv'}{\substvars(\ty)}{\substvars(\ty')}.
     \]
\end{theorem}
\begin{proof}
    By induction on the derivation of \subtydflt{\ty}{\ty'}.

    Cases \RST{Top} and \RST{Bot} are straightforward by \RST{Top} and
    \RST{Bot}, respectively. 

    Case \RST{VarRefl} by \thmref{thm:sub-ty-refl} (reflexivity):
    \subty{\AEnv'}{\substvars(\vany)}{\substvars(\vany)}.

    Cases \RST{Tuple}, \RST{Inv}, \RST{UnionLeft}, and \RST{UnionRight} are
    straightforward using inversion, the inductive hypothesis, and constructor.
    For example, consider the case \RST{Tuple}
    \subtydflt{\typair{\ty_1}{\ty_2}}{\typair{\ty'_1}{\ty'_2}}.
    By inversion, \subtydflt{\ty_i}{\ty'_i}.
    By IH, \subty{\AEnv'}{\substvars(\ty_i)}{\substvars(\ty'_i)}.
    The case concludes by \RST{Tuple} with 
    \[\subty{\AEnv'}{\typair{\substvars(\ty_1)}{\substvars(\ty_2)}}
    {\typair{\substvars(\ty'_1)}{\substvars(\ty'_2)}}\]
    and the fact that $\substvars(\typair{\ty_1}{\ty_2}) = 
    \typair{\substvars(\ty_1)}{\substvars(\ty_2)}.$

    The remaining cases \RST{VarLeft} and \RST{VarRight} are similar.
    For example, consider \RST{VarLeft} \subtydflt{\plug{\dctx}\vany}{\ty'}.
    By inversion, $\varbound{\vany}{\tylb}{\tyub} \in \AEnv$ and
    \subtydflt{\plug\dctx{\tyub}}{\ty'}.
    By IH, \subty{\AEnv'}{\substvars(\plug\dctx{\tyub})}{\substvars(\ty')}.
    By inversion of the assumption \vldinenvdflt{\substvars}, we know
    \subty{\AEnv'}{\substvars(\vany)}{\substvars(\tyub)}.
    Since $\substvars(\plug\dctx{\tyub}) = 
    \plug{\substvars(\dctx)}{\substvars(\tyub)}$ and
    \subty{\AEnv'}{\substvars(\dctx)}{\substvars(\dctx)} by reflexivity,
    we have \subty{\AEnv'}{\plug{\substvars(\dctx)}{\substvars(\vany)}}
    {\plug{\substvars(\dctx)}{\substvars(\tyub)}}.
    The case concludes by \thmref{thm:sub-ty-trans} (transitivity) with the
    middle type $\substvars(\plug\dctx{\tyub})$:
    \[\subty{\AEnv'}{\substvars(\plug\dctx{\vany})}{\substvars(\ty')}.\]
\end{proof}


\subsection{Properties of Constrained Subtyping of Types}%
\label{subsec:props-subtyctr-proof}
%% ======================================================================

\begin{lemma}{Constrained subtyping coincides with subtyping on
    unification-free types.}%
\label{lem:subtyctr-subty}
    $\forall \AEnv, \ty, \ty' \text{ s.t. } 
    \tyvld{}{\AEnv} \ \land\ \tyvlddflt{\ty} \ \land\ \tyvlddflt{\ty'}$ 
    the following holds:
    \begin{enumerate}
        \item $\forall \UEnvD.\ \subtyctrdflt{\ty}{\ty'} \quad\implies\quad
            \CSet = \EmptyCSet\ \land\ \subtydflt{\ty}{\ty'};$
        \item $\subtydflt{\ty}{\ty'} \quad\implies\quad 
            \forall \UEnvD.\ \subtyctrdfltenv{\ty}{\ty'}{\EmptyCSet}.$
    \end{enumerate}
\end{lemma}
\begin{proof}
    Straightforward by induction on the derivation of:
    \begin{enumerate}
        \item \subtyctrdflt{\ty}{\ty'} (more precisely, by mutual induction
            on \subtyctrLdflt{\ty}{\ty'} and \subtyctrRdflt{\ty}{\ty'});
        \item \subtydflt{\ty}{\ty'}.
    \end{enumerate}

    In the first case, the rules \RSC{UBot}, \RSC{UVarLeft}, \RSC{UVarRight},
    and \RSC{UVar-UnionRight} could not have been used to build the derivation
    because they refer a unification variable \va from \UEnvD,
    and both \ty and $\ty'$ are valid in \AEnv alone.
    All other rules of constrained subtyping have matching subtyping rules.

    In the second case, all subtyping rules have matching rules of constrained
    subtyping.

    The assumption \tyvld{}{\AEnv} and \lemref{lem:subty-weakening} (context
    weakening) allow for concluding 
    \tyvlddflt{\plug\dctx\tyub} and \tyvlddflt{\tylb} 
    to apply the inductive hypothesis in the rules
    \RSC{VarLeft}/\RST{VarLeft} and \RSC{VarRight}/\RST{VarRight}.
\end{proof}

\begin{theorem}{Soundness of constrained subtyping.}%
\label{thm:subtyctr-sound}
\TODO{Any other validity requirements?}
    $\forall \AEnv, \UEnvD, \ty, \ty' \text{ s.t. }$
    \[\subtyctrRdflt{\ty}{\ty'} \ \land\ 
        \tyvlddflt{\ty} \ \land\  \tyunfvlddflt{\ty'}\] 
    or
    \[\subtyctrLdflt{\ty}{\ty'} \ \land\ 
        \tyunfvlddflt{\ty} \ \land\  \tyvlddflt{\ty'},\]
    $\forall \substvars \text{ s.t. } 
        \dom{\substvars} \supseteq \UEnvD \ \land\ 
        \dom{\substvars} \cap \dom{\AEnv} = \varnothing \ \land\ 
        \vldinenv{\AEnv}{\CSet}{\substvars}.$
     \[ 
        \subtydflt{\substvars(\ty)}{\substvars(\ty')}.
     \]
\end{theorem}
\begin{proof}
    By strong induction on $\tymsrdflt{\ty} + \tymsrdflt{\ty'}.$
    \begin{itemize}
        \item Case \RSC{Top} \subtyctrdfltenv{\ty}{\tyany}{\EmptyCSet}.
            By definition, $\substvars(\tyany) = \tyany$.
            Thus, the case concludes by \RST{Top}:
            \subtydflt{\substvars(\ty)}{\tyany}.
        \item Case \RSC{Bot} by \RST{Bot}, similarly to \RSC{Top}.
        \item Case \RSC{UBot}
            \subtyctrLdfltenv{\plug\dctx\va}{\ty'}{\ctrsngl\va\tybot}.
            Since $\substvars(\plug\dctx\va) = 
                \plug{\substvars(\dctx)}{\substvars(\va)} = 
                \plug{\dctx'}{\substvars(\va)}$ for some $\dctx'$
            and by assumption, \subtydflt{\substvars(\va)}{\tybot},
            we have \subtydflt{\plug{\dctx'}{\substvars(\va)}}{\plug{\dctx'}\tybot}.
            By \RST{Bot}, \subtydflt{\plug{\dctx'}\tybot}{\substvars(\ty')}.
            Therefore, the case concludes by transitivity:
            \subtydflt{\plug{\dctx'}{\substvars(\va)}}{\substvars(\ty')}.
        \item Case \RSC{VarRefl} by \RST{VarRefl}, for $\substvars(\vx) = \vx.$
        \item Case \RSC{UVarLeft} 
            \subtyctrLdfltenv{\va}{\ty'}{\ctrsngl{\va}{\ty'}} by assumption
            \subtydflt{\substvars(\va)}{\ty'}, since $\substvars(\ty') = \ty'$
            due to \tyvlddflt{\ty'}.
        \item Case \RSC{UVarRight} similarly to \RSC{UVarLeft}.
        \item Case \RSC{VarLeft} (\RSC{VarRight}) by inversion, IH, and
            \RST{VarLeft} (\RST{VarRight}), for $\substvars(\vx) = \vx$
            and $\substvars(\tyub) = \tyub$ ($\substvars(\tylb) = \tylb$).
        \item Cases \RSC{Tuple}, \RSC{Inv}, \RSC{UnionLeft}, and \RSC{UnionRight}
            are all similar: by inversion, IH, and the corresponding subtyping
            constructor.
        \item \textbf{Case \RSC{UVar-UnionRight}} 
            \subtyctrLdfltenv{\plug\dctx{\va}}{\tyunion{\ty'_1}{\ty'_2}}
            {\CSet'_1 \cup \CSet'_2 \cup \CSet'} .
    \end{itemize}
\end{proof}
