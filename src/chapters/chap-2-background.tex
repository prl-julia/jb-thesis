\chapter{Background}

\section{Overview of the Julia Language}

Julia is a high-level, dynamic programming language for scientific computing~\cite{TODO}.
It was designed to provide good performance as well as productivity features and
the ease of use~\cite{TODO}.
For performance, Julia relies on an optimizing JIT compiler.
For productivity, the language provides garbage collection, dynamic typing, and
multiple dispatch. 

\tdef{Multiple dynamic dispatch}~\cite{TODO} is the core paradigm of the Julia
language. The dispatch mechanism allows a function (called \emph{generic
function}) to have multiple implementations (called \emph{methods}) tailored to
different argument types. For example, the following code snippet shows several
method definitions of the addition function \cjl{(+)}:
\begin{julia}
+(x::Float64, y::Float64) = Base.add_float(x, y)
+(z::Complex, w::Complex) = Complex(real(z) + real(w), imag(z) + imag(w))
+(m::Missing, n::Number)  = missing
+(r1::LinRange{T}, r2::LinRange{T}) where T = ...
\end{julia}
For every function call, using \emph{run-time} types of the arguments,
the dispatch mechanism either selects the \emph{most specific applicable} method,
or throws an error if such a single best method does not exist.
\secref{TODO} describes this process in more details.

TODO: talk about types.

%Notably, 
