\chapter{Introduction}

Julia is a dynamic, high-level, high-performance programming
language~\cite{bib:bezanson:julia-fresh:2017},
originally designed for scientific computing.
At the core of Julia's design is multiple dynamic dispatch, which relies
on an expressive language of type annotations to tailor function implementations
to different argument types. The type language includes set-theoretic unions, 
distributive tuples, parametric invariant types, and impredicative existential 
types. Multiple dispatch is resolved at run time
with a complex subtyping algorithm.
However, subtyping in Julia is undecidable. In practice, when an undecidable
subtyping check is encountered, program execution fails 
with an uninformative \cjl{StackOverflowError}.

In this dissertation, I propose a decidable subtype relation for a restricted
language of Julia types. Namely, the restriction limits existential types 
inside invariant constructors to existential types expressible with 
use-site variance, similar to Java wildcards.
In a corpus of 9K registered Julia packages, 99.99\% of 2M 
statically identifiable type annotations satisfy the proposed restriction.
Thus, the restriction provides a viable approach for evolving the Julia
language towards decidable subtyping.

\section{Subtyping in Julia}
%% ======================================================================

To encourage a high level of code reuse and extensibility,
Julia is designed around \emph{symmetric multiple dynamic dispatch}.
Multiple dispatch allows a function to have multiple implementations, called
methods, tailored to different argument types; at run time, a function
call is dispatched to the most specific method for the given arguments.
%Thus, with multiple dispatch, programmers can write code
% \TODO{generic code plus optimized special cases}

To deliver performance, Julia relies on a type-specializing just-in-time (JIT)
compiler: with few exceptions\footnote{%
\url{https://docs.julialang.org/en/v1/manual/performance-tips/\#Be-aware-of-when-Julia-avoids-specializing}
}, %~\cite{TODO}, link to Julia docs on avoiding specialization
every time a method is called with a new set of argument types,
it is specialized for those types, creating a new method
definition~\cite{bib:pelenitsyn:type-stability:oopsla:2021}.
Therefore, even for functions with a single user-defined method,
more methods can accrue during the program execution,
adding work for the dispatch mechanism.

% Thus, although Julia is dynamically typed and does not require
% programmers to provide type annotations,
% types have a profound role in the language semantics and performance.

Multiple dispatch is resolved with the use of a subtype relation.
Namely, for every function call, subtyping is checked between
the argument types and method signatures,
as well as between method signatures,
to pick the most specific applicable method.
While most run-time value types are nominal, %e.g. \cjl{Vector\{Int64\}},
the language of type annotations is more complex and expressive. 
In particular, inspired by semantic subtyping
where types are interpreted as sets,
Julia supports covariant tuples that distribute over unions.
For example, according to the semantic approach, the type 
\cjl{Tuple\{String, Union\{Int64, UInt64\}\}} represents a set of binary
tuples where the first component is a string, and the second component is
either a signed or unsigned integer. Therefore, both
\cjl{Tuple\{String, Int64\}} and \cjl{Tuple\{String, UInt64\}} are subtypes
of the above type, and that type is equivalent to the union of the two
tuples, \cjl{Union\{Tuple\{String, Int64\}, Tuple\{String, UInt64\}\}}.
In addition to finite unions, Julia supports existential types, referred to as
union-all types or iterated unions in the language. 
For example, \cjl{Vector\{T\} where T}
represents an infinite union of vectors \cjl{Vector\{t\}} for all
instantiations \cjl{t} of the type variable \cjl{T}.
A~more detailed account of Julia types and subtyping
is provided in~\chapref{chap:background}.
% to represent parametric method definitions
%choosing an unusual combination of type language features.

Despite intentional simplifications in the type language,
e.g. the lack of recursive bounds\footnote{
  For example, \texttt{X \textbf{implements} Comparable<X>} in Java. 
} on type variables,
subtyping in Julia is
\emph{undecidable}~\cite{bib:chung:type-julia-thesis:2023}.
%due to %the support for
%impredicative bounded existential types.
%Simply disallowing such types would not be practical,
%as Julia users rely on them to describe heterogeneous data.
% For example, \cjl{Vector\{Vector\{T\} where T<:Number\}}
% describes a heterogeneous vector of vectors of numeric values
% where inner vectors can have different element types.
Although relying on undecidable subtyping is not unprecedented
for a statically typed language (see Scala~\cite{bib:hu:dot-undec:2020}
for an example),
%with an expressive type system such as
the price of undecidability
is higher in Julia, since it
can manifest at almost any point during program \emph{execution}.
%as subtyping over a complex language of run-time types and type annotations
%is an integral part of Julia's dynamic semantics.
In particular, the run-time system relies on undecidable subtyping
to resolve function calls,
process new method definitions~\cite{bib:belyakova:world-age:oopsla:2020},
manipulate data (e.g. when adding an element to a container),
as well as JIT compile methods~\cite{bib:pelenitsyn:type-stability:oopsla:2021}.
In practice, the undecidability
leads to a run-time crash with a \cjl{StackOverflowError}.
Such an issue can be particularly hard to debug,
because the error is uninformative: neither the problematic subtyping query
nor stack trace is reported.

A number of issues related to subtyping have been reported
on the Julia bug tracker. For example,
\href{https://github.com/JuliaLang/julia/issues/41948}{\code{\#41948}}\footnote{
    \url{https://github.com/JuliaLang/julia/issues/41948}
} reports a \cjl{StackOverflowError} caused by a function definition,
which is likely linked to undecidability;
\href{https://github.com/JuliaLang/julia/issues/33137}{\code{\#33137}}\footnote{
    \url{https://github.com/JuliaLang/julia/issues/33137}
} points out an inconsistency in subtyping; and % related to the diagonal rule.
\href{https://github.com/JuliaLang/julia/issues/24166}{\code{\#24166}}\footnote{
    \url{https://github.com/JuliaLang/julia/issues/24166} 
} (now fixed) reports a problem with reflexivity and transitivity.
Overall, there are 105 open/704 closed issues labeled with ``types and
dispatch'' as of March 2023,
with 13 open/138 closed being also labeled with ``bug''
(not every issue is properly labeled as a bug,
e.g. the aforementioned
\href{https://github.com/JuliaLang/julia/issues/24166}{\code{\#24166}} is not).
\tabref{tab:julia-issues-stats} provides a few more data points for comparison:
for example, there are 8 open/86 closed ``codegen'' bugs
and 1 open/15 closed ``GC'' bugs.
Thus, type-related concerns, including the undecidability of subtyping,
%are not purely theoretical and
constitute a non-negligible portion of problems in the Julia implementation.

\begin{table}[t]
\begin{threeparttable}
\centering\footnotesize
\caption{Open/closed issues on the Julia bug tracker
(March~2023)}\label{tab:julia-issues-stats}

\begin{tabular}{c|ccccc}
 & types and dispatch & codegen & GC & macros & <any label> \\
\midrule
<any label> &
  92/414 & 56/246 & 24/51 & 27/36 & 3551/19238 \\
bug &
  13/138 & 8/86 & 1/15 & 5/11 & 226/2664
\end{tabular}

\begin{tablenotes}[para]
\small
Axes represent labels the issues are marked with: for example, the cell
24/51 in the row \emph{<any label>} and column \emph{GC} corresponds to all
issues labeled with {``GC''}, whereas 1/15 in the row \emph{bug} and column 
\emph{GC} corresponds to {``GC''} issues that are also labeled with {``bug''}.
\end{tablenotes}
\end{threeparttable}
\end{table}

\section{Thesis and Contributions}
%% ======================================================================

Due to Julia's ubiquitous use of subtyping at run time,
the undecidability of subtyping is consequential.
It is unlikely, as I discuss in \secref{sec:julia-sub:undec},
that a compatible\footnote{\emph{Compatible} means \emph{compatible with the
current Julia subtyping}, i.e., produces the same answer
as Julia subtyping whenever Julia subtyping terminates without an error.}
decidable subtype relation exists for the entirety of Julia
types. However, my thesis is that
\begin{quotation}\noindent\emph{%
  the Julia language can be evolved to provide for decidable subtyping while
  requiring minimal effort for migrating existing code.
}\end{quotation}

Because the type language and its associated subtype relation 
impact the set of valid Julia programs and their dynamic semantics,
replacing subtyping can have profound effects on programmer experience 
as well as the ability to migrate existing code.
Thus, not every decidable subtyping would be a viable candidate for
Julia's evolution.

In this dissertation, I show that there is
a restriction on Julia's type language that allows for decidable subtyping
and supports 99.99\% of statically identifiable type annotations
in registered Julia packages.
Namely, the restriction limits existential types inside invariant constructors
to existential types expressible with use-site variance,
similar to Java wildcards~\cite{bib:torgersen:wildcards:2004}.

The dissertation provides an overview of Julia
(\chapref{chap:background} and \secref{sec:julia-sub:overview})
and makes the following contributions:
\begin{itemize}
  \item \textbf{Specification of Julia
    subtyping}~\citep*{bib:zappa-nardelli:julia-sub:oopsla:2018}
    (\secref{sec:julia-sub:lambda-julia}).
    The specification covers most of Julia, with the only exception being
    variable-length tuples.
  \item \textbf{Decidable subtyping} for a restricted language of Julia 
    types~\citep*{bib:belyakova:julia-sub-dec:draft} (\chapref{chap:dec-sub}).
    The restriction retains most of Julia types, limiting only existential types
    inside invariant constructors.
  \item \textbf{Evaluation of the migration effort} required to switch to 
    the restricted type language (\chapref{chap:eval}). 
    In the corpus of all 9K registered Julia packages, 99.99\% of source code 
    type annotations satisfy the restriction.
\end{itemize}
Furthermore, \appref{sec:app:sem-sub} discusses a semantic interpretation
of types, extending \cite{bib:belyakova:minijl-sub:ftfjp:2019}.

Related to this dissertation, I worked on formalizing several aspects of the
Julia language:
\begin{itemize}
  \item \citep*{bib:belyakova:world-age:oopsla:2020} presents 
    world-age semantics, which enables dispatch optimization in the presence
    of \texttt{eval} by delaying visibility of dynamically generated methods.
  \item \citep*{bib:pelenitsyn:type-stability:oopsla:2021} defines type
    stability---the property which allows for efficient compilation of code with
    multiple dispatch.
\end{itemize}
