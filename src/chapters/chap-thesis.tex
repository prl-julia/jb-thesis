\chapter{Thesis Statement}\label{chap:3}

Because of the undecidability of subtyping, which is an integral part of Julia's
dynamic semantics, a Julia program can
unexpectedly crash during the execution.
One way to address this problem would be to use a decidable subtype relation
in place of the existing, undecidable one.
However, as subtyping impacts the set of valid programs and their semantics,
this can have profound effects on user experience with the language.
%the language expressiveness, usability, and performance.
%Thus, not any decidable subtype relation would be worthwhile a major change in
%the language.

My thesis is:
\begin{quotation}\emph{
  The Julia language can be evolved to provide for decidable subtyping while
  requiring minimal effort for migrating existing code.
}\end{quotation}

To validate the thesis, I will:
\begin{enumerate}
  \item design a new subtype relation based on the one currently used by Julia
    and prove it decidable;
  \item estimate the migration effort by conducting a static analysis of
    registered Julia packages and suggesting code migration strategies.
\end{enumerate}

% I conjecture that the migration effort can be used to assess the impact of the
% new subtype relation on expressiveness and usability:
% if most of the existing code remains valid in the new language, 
% then the impact on the expressiveness and usability is limited.

An implementation of the subtype relation and understanding its impact on
Julia's performance are beyond the scope of this thesis and
remain future work.

% Furthermore, I suggest that Julia types are given a set-theoretic interpretation,
% with the subtype relation matching set inclusion on the interpretations.
% Although Julia was inspired by semantic subtyping, the existing subtype relation
% is not consistent with the semantic approach: for example, the type 
% \cjl{Tuple\{Int, Union\{\}\}} (a covariant tuple of an integer and the bottom type)
% is not considered a subtype of the bottom type despite the fact that there are
% no values of type \cjl{Tuple\{Int, Union\{\}\}}.

\paragraph*{Preliminary work.}
In my thesis research so far,
I have collaborated on reconstructing a specification of Julia subtyping
(OOPSLA 2018~\cite{bib:zappa-nardelli:julia-sub:oopsla:2018}),
and defined and mechanized a set-theoretic model of a subset of Julia types
(FTfJP 2019~\cite{bib:belyakova:minijl-sub:ftfjp:2019});
more details about these efforts can be found in \chapref{chap:4}.
I have also collaborated on modeling Julia's dynamic semantics, including
\cjl{eval}~\cite{bib:belyakova:world-age:oopsla:2020}
and the JIT compiler~\cite{bib:pelenitsyn:type-stability:oopsla:2021},
which illuminated the role of types and subtyping
in the language.

% Because efficiency is important for Julia, and the way it is achieved is by JIT
% compilation, we explore the role of types in the JIT compiler. It turns out that
% the JIT heavily relies on type inference \TODO{OOPSLA 21}. Therefore, we need to
% understand what happens with types and which operations need to be supported on
% types. This will be our final guiding principle.
