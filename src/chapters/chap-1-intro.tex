\chapter{Introduction}

Julia is a dynamic, high-level, yet high-performance programming
language~\cite{TODO},
originally designed for scientific computing.
Julia aims to solve the two-language problem, combining the convenience of
productivity languages such as Python and R, and performance of languages such
as Fortran and C~\cite{TODO}.

Notably, to allow for a high level of code reuse and extensibility,
Julia is designed around \emph{symmetric multiple dynamic dispatch}.
Multiple dispatch allows a function to have multiple implementations, called
methods, tailored to different argument types; at run time, a function
call is dispatched to the most specific method for the given arguments.
To deliver performance, Julia relies on a type-specializing just-in-time (JIT)
compiler: with few exceptions~\cite{TODO},
every time a method is called with a new set of argument types,
it is specialized for those types, creating a new method definition~\cite{TODO}.
Therefore, even for functions with a single user-defined method,
more methods may accrue during the program execution,
adding work for the dispatch mechanism.

% Thus, although Julia is dynamically typed and does not require
% programmers to provide type annotations,
% types have a profound role in the language semantics and performance.

Multiple dispatch is resolved with the use of a subtype relation.
In particular, for every function call, subtyping is checked between
the argument types and method type signatures,
as well as between method type signatures.
While most run-time values are represented by nominal types,
the language of type annotations is inspired by semantic subtyping
and supports union types that distribute over covariant tuples.
Furthermore, Julia uses existential types to represent
parametric method definitions.
With all the features combined, Julia's type language
does not have a straightforward analogue in the literature.

%to identify the most specific applicable method.

Despite intentional simplifications, e.g. disallowing recursive bounds on type
variables, subtyping in Julia is \emph{undecidable}.
The key issue is the support for impredicative bounded existential types
such as \cjl{Vector\{Vector\{T\} where T<:Number\}}, which
describes a vector of vectors of numeric values where inner vectors
can have different element types.
A more detailed account of subtyping is provided in~\chapref{chap:2}.
% Although Julia does not have a static type system, the language of type
% annotations is surprisingly expressive.

Although relying on undecidable subtyping is not unusual
for a statically typed language
with an expressive type system such as Scala~\ref{TODO},
in the dynamic Julia language, undecidability manifests at run time and thus,
has a much higher cost to the user.
Namely, as every function call is resolved by multiple dispatch via subtyping,
any function call may lead to a run-time crash.

% Although Julia does not have a static type system,
% the dispatch mechanism is built around an expressive language of types
% and subtyping.
% For example, Julia provides impredicative, bounded existential types,
% which can be used to model wildcards in Java.
% Furthermore, drawing inspiration from semantic subtyping,
% union and existential types distribute over covariant tuple types.
% A more detailed account of dispatch and subtyping is provided in~\chapref{chap:2}.
% Additionally, Julia supports a special kind of existential
% types to allow for a common use case of dispatched methods: so-called diagonal
% rule, where an existential variable is allowed to be instantiated only with a
% concrete type.

% As it turns out, Julia substyping is actually \emph{undecidable},
% which is a problem because it is used at run time to resolve function calls,
% process function definitions, and even during JIT compilation.
% Despite similarities of parts of the type language to other systems
% such as semantic subtyping and Java wildcards, the particular combination
% of features in Julia does not have a straightforward analogue in the literature.

In this thesis, I propose to develop
a \emph{decidable and practical} subtyping relation for the Julia language.
%,as outlined in~\chapref{chap:3}.

%
%\chapref{chap:4} reports preliminary results,
%and \chapref{chap:5} describes the related work.


% both concrete user-defined
% types that describe values, and abstract types that can be used in method type
% annotations to specify multiple dispatch. Dispatch is implemented using
% subtyping in this language of types.
% The language of types is surprisingly complex, and there is no clean
% specification of subtyping available to Julia users. The only reference is
% effectively the implementation of subtyping, which is written in obscure C (or
% C++?) code and meant to be efficient. The implementation is not very readable,
% and the reason it needs to be fast is that subtyping is used at run-time.
% As any function call in Julia is a dynamically dispatched call, and dispatch
% relies on subtyping, it needs to be fast.

% However, even the most efficient implementation of subtyping is not enough to
% make the language fast. Dispatched calls can hinder optimizations. Therefore,
% Julia's JIT compiler tries to remove dispatched calls and perform further
% optimizations. In doing so, Julia heavily relies on type inference at run-time.
% When inferred types are precise enough to ``statically'' dispatch calls, the JIT
% will do so. Again, this means that subtyping is used. But furthermore, the types
% are used by the type inference algorithm to guide optimizations.

% As the primary purpose of types in Julia at the surface level is to describe
% behaviors, the language of types is an unusual one. For example, it provides
% impredicative exsitential types with bounded quantification. Furthermore,
% covariant tuples and unions are distributive in Julia, which is similar to
% semantic subtyping. Additionally, Julia supports a special kind of existential
% types to allow for a common use case of dispatched methods: so-called diagonal
% rule, where an existential variable is allowed to be instantiated only with a
% concrete type. Given all this, there are concerns of understandability and
% properties of the type language and subtyping in Julia.

% As the first step in understanding the language of types, we tried to reverse
% engineer the implementation of subtyping and provide a more readable
% specification. This work is reported in \TODO{OOPSLA 18}.
% While we did come up with subtyping rules that describe subtyping, we found
% several issues: for example, subtyping was not transitive. There were also
% several bugs in the implementation of subtyping. Furthermore, we found that the
% diagonal rule works in such a way that the meaning of a type changes depending
% on a subtyping relation it is checked against.
% Most importantly, subtyping turns out to be undecidable in Julia. This means
% that in practice, the user may face with a stack overflow at run-time because
% subtyping is used for dispatching.
% Another problem is that the rules we provided are hard to reason about to prove
% transitivity, for example.
% Thus, our first goal is to develop a specification of subtyping that is
% decidable and can be reasoned about.

% The way existential types are used in Julia are reminiscent of models of Java
% wildcards. The key difference is that Julia's existential types are
% impredicative. Interestingly, in the hope to avoid undecidability, Julia
% developers intentionally restricted bounded quantification to, for example,
% forbid recursive constraints on type variables.
% Thus, some approaches to making subtyping decidable are not applicable, e.g.
% material-shape separation \TODO{understand and explain}.

% Ideally, subtyping should also be intuitive for the users. So we explore the
% applicability of a semantic subtyping approach to the type language. We will
% identify a property that we call tag-based semantic subtyping, and we will
% strive to support this property.

% Because efficiency is important for Julia, and the way it is achieved is by JIT
% compilation, we explore the role of types in the JIT compiler. It turns out that
% the JIT heavily relies on type inference \TODO{OOPSLA 21}. Therefore, we need to
% understand what happens with types and which operations needs to be supported on
% types. This will be our final guiding principle.
