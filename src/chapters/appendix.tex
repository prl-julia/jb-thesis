\chapter{Appendix}

\section{Properties Subtyping}\label{sec:app:proofs}
%% ======================================================================

\subsection{Decidable Subtyping}\label{subsec:app:proofs:dec}
%% *********************************************************

\begin{lemma}{Context weakening in $\msrop$ (\textbf{\lemref{lem:msr-weakening}}).}%
\label{lem:msr-weakening:app}
    The measure of a type signature does not change if the environment
    is extended (in any position) with a variable that occurs neither
    in the signature nor in the environment, i.e.,
    $\forall \tysig, \AEnv, \AEnv'. 
    \forall \varbound{\vany}{\tylb}{\tyub} \text{ s.t. } 
    \lnot \occdflt{\tysig} \land 
    \lnot \occdflt{\AEnv} \land \lnot \occdflt{\AEnv'}.$
    \[\tymsr{\concat{\AEnv}{\AEnv'}}{\tysig} = 
        \tymsr{\concat{\AEnv,\varbound{\vany}{\tylb}{\tyub}}{\AEnv'}}{\tysig},\]
    where \concat{\AEnv}{\AEnv''} denotes the concatenation of lists,
    and $\occop$ is defined in~\figref{fig:var-occ}.
\end{lemma}
\begin{proof}
    By strong induction on $n = \size{\AEnv} + \size{\AEnv'} + \size{\tysig}$.

    Case $n = 0$ is not possible: the minimal size of a type signature is 1.

    Base cases for \tyany and \tybot are straightforward; in the base case
    for $\vany'$, environment $\concat{\AEnv}{\AEnv'}$ is empty and
    $\vany \neq \vany'$, 
    meaning that \[\tymsr{\EmptyEnv}{\vany'} = 1 = 
    \tymsr{\varbound{\vany}{\tylb}{\tyub}}{\vany'}.\]
    
    In the inductive step for $n$, the induction hypothesis (IH) states that
    $\forall n'<n. \forall \tysig', \AEnv'', \AEnv'''  \text{ s.t. }
    n' = \size{\AEnv''} + \size{\AEnv'''} + \size{\tysig'}.
    \forall \varbound{\vany}{\tylb}{\tyub} \text{ s.t. } 
    \lnot \occdflt{\tysig'} \land 
    \lnot \occdflt{\AEnv''} \land \lnot \occdflt{\AEnv'''}.$
    \[\tymsr{\concat{\AEnv''}{\AEnv'''}}{\tysig'} = 
    \tymsr{\concat{\AEnv'',\varbound{\vany}{\tylb}{\tyub}}{\AEnv'''}}{\tysig'}.\]
    
    Case analysis on \tysig. Base cases \tyany and \tybot are straightforward.
    Cases $\times$, \tyinv\iname{\ldots}, and $\cup$ are also straightforward
    using the induction hypothesis for components of \tysig.
    The remaining cases are:
    \begin{itemize}
        \item Case $\vany'$. Case analysis on $\AEnv'$.
            \begin{itemize}
                \item Case \EmptyEnv. Because $\lnot \occdflt{\vany'},$ we know
                    $\vany \neq \vany'$. Thus,
                    $\tymsr{\AEnv, \varbound{\vany}{\tylb'}{\tyub'}}{\vany'} =
                    \tymsr{\AEnv}{\vany'}$ by definition of $\msrop$.
                \item Case $\AEnv', \varbound{\vany'}{\tylb'}{\tyub'}$.
                    By definition,
                    \[\tymsr{\concat{\AEnv}{\AEnv', \varbound{\vany'}{\tylb'}{\tyub'}}}{\vany'} =
                    1 + \tymsr{\concat{\AEnv}{\AEnv'}}{\tylb'} + 
                    \tymsr{\concat{\AEnv}{\AEnv'}}{\tyub'}.\]
                    Since $\size{\AEnv} + \size{\AEnv'} + \size{\tylb'}\ <\ 
                    \size{\AEnv} + \size{\AEnv', \varbound{\vany'}{\tylb'}{\tyub'}} + \size{\vany'} =
                    \size{\AEnv} + \size{\AEnv'} + \size{\tylb'} + \size{\tyub'} + 1$,
                    the IH applies with $\AEnv'' = \AEnv, \AEnv''' = \AEnv', 
                    \tysig' = \tylb'$, which gives 
                    $\tymsr{\concat{\AEnv}{\AEnv'}}{\tylb'} = 
                    \tymsr{\concat{\AEnv, \varbound{\vany}{\tylb}{\tyub}}{\AEnv'}}{\tylb'}$,
                    and similarly for $\tyub'$. Thus,
                    \[ \tymsr{\concat{\AEnv}{\AEnv', \varbound{\vany'}{\tylb'}{\tyub'}}}{\vany'} =
                    \tymsr{\concat{\AEnv, \varbound{\vany}{\tylb}{\tyub}}{\AEnv', \varbound{\vany'}{\tylb'}{\tyub'}}}{\vany'}. \]
            \end{itemize}
        \item Case \tyexist{\vany'}{\tylb'}{\tyub'}{\tysig}.
            By definition,
            \[
                \begin{array}{c}
                    \tymsr{\concat{\AEnv}{\AEnv'}}{\tyexist{\vany'}{\tylb'}{\tyub'}{\tysig}} \\
                    = \\
                    1 + \tymsr{\concat{\AEnv}{\AEnv'}}{\tylb'} + \msrop(\ldots\tyub') +
                    \tymsr{\concat{\AEnv}{\AEnv', \varbound{\vany'}{\tylb'}{\tyub'}}}{\tysig}.
                \end{array}    
            \]
            Similarly to the last subcase of the $\vany'$ case, the IH applies
            to $\tylb'$ and $\tyub'$.
            Furthermore, since $\size{\AEnv} + \size{\AEnv'} + 
            \size{\tylb'} + \size{\tyub'} + \size{\tysig} <
            \size{\AEnv} + \size{\AEnv'} + 1 + \size{\tylb'} + \size{\tyub'}
            + \size{\tysig},$
            the IH applies to \tysig with $\AEnv'' = \AEnv, 
            \AEnv''' = (\AEnv', \varbound{\vany'}{\tylb'}{\tyub'}), 
            \tysig' = \tysig$.
            All pieces combined, 
            \[\tymsr{\concat{\AEnv}{\AEnv'}}{\tyexist{\vany'}{\tylb'}{\tyub'}{\tysig}} =
            \tymsr{\concat{\AEnv, \varbound{\vany}{\tylb}{\tyub}}{\AEnv'}}{\tyexist{\vany'}{\tylb'}{\tyub'}{\tysig}}.\]
            % and
            % \[\tymsr{\concat{\AEnv, \varbound{\vany}{\tylb}{\tyub}}{\AEnv'}}{\tyexist{\vany'}{\tylb'}{\tyub'}{\tysig}} =
            % 1 + \tymsr{\concat{\AEnv, \varbound{\vany}{\tylb}{\tyub}}{\AEnv'}}{\tylb'} + 
            % \tymsr{\concat{\AEnv, \varbound{\vany}{\tylb}{\tyub}}{\AEnv'}}{\tyub'} +
            % \tymsr{\concat{\AEnv, \varbound{\vany}{\tylb}{\tyub}}{\AEnv', \varbound{\vany'}{\tylb'}{\tyub'}}}{\tysig}.\]
    \end{itemize}
\end{proof}

\subsection{Unification-Free Subtyping}\label{subsec:app:proofs:subty}
%% *********************************************************

\begin{lemma}{Subtyping of \tybot implies arbitrary subtyping
    (\textbf{\lemref{lem:sub-of-bot}}).}\label{lem:sub-of-bot:app}
    \[
    \forall \ty, \dctx_{\tybot}, \AEnv.\quad 
    \subtydflt{\ty}{\plug{\dctx_{\tybot}}\tybot}
    \quad\implies\quad
    (\forall \ty', \dctx'.\quad \subtydflt{\plug{\dctx'}{\ty}}{\ty'}).
    \]
\end{lemma}
\begin{proof}
    By induction on the derivation of 
    \subtydflt{\ty}{\plug{\dctx_{\tybot}}\tybot}.
    \begin{itemize}
        \item Case \RST{Bot}
            \subtydflt{\plug\dctx\tybot}{\plug{\dctx_{\tybot}}{\tybot}}
            where $\ty = \plug\dctx\tybot$.

            The case concludes by \RST{Bot}:
            \subtydflt{\plug{\dctx'}{\plug\dctx\tybot}}{\ty'}. 
        \item Case \RST{VarLeft}
            \subtydflt{\plug\dctx\vany}{\plug{\dctx_{\tybot}}{\tybot}}.

            By inversion, \subtydflt{\plug\dctx\tyub}{\plug{\dctx_{\tybot}}{\tybot}}.
            By IH, \subtydflt{\plug{\dctx'}{\plug\dctx\tyub}}{\ty'}.
            Thus, the case concludes by \RST{VarLeft}: 
            \subtydflt{\plug{\dctx'}{\plug\dctx\vany}}{\ty'}.
        \item Case \RST{Tuple}, subcase where
            $\dctx_{\tybot} = \typair{\dctx'_{\tybot}}{\ty'_2}$
            ($\dctx_{\tybot} = \square$ is not possible, and
            $\dctx_{\tybot} = \typair{\ty_1}{\dctx'_{\tybot}}$
            is proved analogously),
            $\ty = \typair{\ty_1}{\ty_2}$:
            \subtydflt{\typair{\ty_1}{\ty_2}}
            {\typair{\plug{\dctx'_{\tybot}}{\tybot}}{\ty'_2}}.

            By inversion, \subtydflt{\ty_1}{\plug{\dctx'_{\tybot}}{\tybot}}.
            By IH, \subtydflt{\plug{\dctx'^h}{\ty_1}}{\ty'} for all $\dctx'^h$,
            so we can take it to be \plug{\dctx'}{\typair{\square}{\ty_2}}.
            Thus, the case concludes by IH: 
            \subtydflt{\plug{\dctx'}{\typair{\ty_1}{\ty_2}}}{\ty'}.
        \item Case \RST{UnionLeft}
            \subtydflt{\plug\dctx{\tyunion{\ty_1}{\ty_2}}}{\plug{\dctx_{\tybot}}{\tybot}}
            where $\ty = \tyunion{\ty_1}{\ty_2}$.
            By inversion, 
            \subtydflt{\plug\dctx{\ty_1}}{\plug{\dctx_{\tybot}}{\tybot}} and
            \subtydflt{\plug\dctx{\ty_2}}{\plug{\dctx_{\tybot}}{\tybot}}.
            By IH, \subtydflt{\plug{\dctx'}{\plug\dctx{\ty_1}}}{\ty'} and
            \subtydflt{\plug{\dctx'}{\plug\dctx{\ty_2}}}{\ty'}.
            Thus, the case concludes by \RST{UnionLeft}: 
            \subtydflt{\plug{\dctx'}{\plug\dctx{\tyunion{\ty_1}{\ty_2}}}}{\ty'}.
    \end{itemize}
    The remaining cases 
    (\RST{Top}, \RST{VarRefl}, \RST{VarRight}, \RST{Inv}, \RST{UnionRight}) 
    are not possible.
\end{proof}

\begin{lemma}{Subtyping of inner union on the right
    (\textbf{\lemref{lem:sub-inner-union-right}}).}%
\label{lem:sub-inner-union-right:app}
    $\forall \ty, \dctx', \ty'_1, \ty'_2, \AEnv, \text{ s.t. }$\\
    $\tyvld{}{\AEnv}\ \land\ \tyvlddflt{\ty, \dctx', \ty'_1, \ty'_2}.$
    \[
        \begin{array}{ccc}
        \subtydflt{\ty}{\plug{\dctx'}{\tyunion{\ty'_1}{\ty'_2}}}\\
        \quad\implies\quad\\
        (\forall \dctx_1, \dctx_2, \text{ s.t. }
        \tyvlddflt{\dctx_1, \dctx_2} \land
        \subtydflt{\dctx_1}{\dctx_2}.\quad
        \subtydflt
            {\plug{\dctx_1}{\ty}}
            {\tyunion
                {\plug{\dctx_2}{\plug{\dctx'}{\ty'_1}}}
                {\plug{\dctx_2}{\plug{\dctx'}{\ty'_2}}}
            }).
        \end{array}
    \]
\end{lemma}
\begin{proof}
    By induction on the derivation of
    \subtydflt{\ty}{\plug{\dctx'}{\tyunion{\ty'_1}{\ty'_2}}}.
    \begin{itemize}
        \item Case \RST{Bot} by \RST{Bot}.
        \item Case \RST{VarLeft} by inversion, IH, and \RST{VarLeft}.
        \item Case \RST{Tuple}, subcase where
            $\dctx' = \typair{\dctx''}{\ty'}$:
            \subtydflt{\typair{\ty_1}{\ty_2}}
                {\typair{\plug{\dctx''}{\tyunion{\ty'_1}{\ty'_2}}}{\ty'}}.
            By inversion,
            \subtydflt{\ty_1}{\plug{\dctx''}{\tyunion{\ty'_1}{\ty'_2}}} and
            \subtydflt{\ty_2}{\ty'}. 
            
            By IH applied to 
            \subtydflt{\ty_1}{\plug{\dctx''}{\tyunion{\ty'_1}{\ty'_2}}},
            \subtydflt{\plug{\dctx^h_1}{\ty_1}}
                {\tyunion
                    {\plug{\dctx^h_2}{\plug{\dctx''}{\ty'_1}}}
                    {\plug{\dctx^h_2}{\plug{\dctx''}{\ty'_2}}}
                }
            for all $\dctx^h_1, \dctx^h_2$ s.t. $\subtydflt{\dctx^h_1}{\dctx^h_2}$.
            Thus, we can take them to be \plug{\dctx_2}{\typair{\square}{\ty_2}} 
            and \plug{\dctx_2}{\typair{\square}{\ty'}}, 
            respectively, which concludes the case with
            \subtydflt{\plug{\dctx_1}{\typair{\ty_1}{\ty_2}}}
                {\tyunion
                    {\plug{\dctx_2}{\typair{\plug{\dctx''}{\ty'_1}}{\ty_2}}}
                    {\plug{\dctx_2}{\typair{\plug{\dctx''}{\ty'_2}}{\ty'}}}
                }
        \item Case \RST{UnionLeft} by inversion, IH, and \RST{UnionLeft}.
        \item Case \RST{UnionRight}, subcase $i = 1$ where $\dctx' = \square$:
            \subtydflt{\ty}{\tyunion{\ty'_1}{\ty'_2}}.
            By inversion, \subtydflt\ty{\ty'_1}.
            By assumption, \subtydflt{\dctx_1}{\dctx_2}, and thus,
            \subtydflt{\plug{\dctx_1}{\ty}}{\plug{\dctx_2}{\ty'_1}}.
            The case concludes by \RST{UnionRight} with $i=1$:
            \subtydflt{\plug{\dctx_1}{\ty}}
                {\tyunion{\plug{\dctx_2}{\ty'_1}}{\plug{\dctx_2}{\ty'_2}}}.
    \end{itemize}
    The remaining cases 
    (\RST{Top}, \RST{VarRefl}, \RST{VarRight}, \RST{Inv}) 
    are not possible.
\end{proof}

\begin{lemma}{Adding inner union on the right (\textbf{\lemref{}}).}%
\label{lem:add-inner-union-right:app}
    $\forall \ty, \dctx', \ty', \AEnv, \text{ s.t. }
    tyvld{}{\AEnv}\ \land\ \tyvlddflt{\ty, \dctx', \ty'}.$
    \[
        \subtydflt{\ty}{\plug{\dctx'}{\ty'}}
        \quad\implies\quad
        (\forall \ty''.\ \subtydflt{\ty}{\plug{\dctx'}{\tyunion{\ty'}{\ty''}}}).
    \]
\end{lemma}
\begin{proof}
    By induction on the derivation of
    \subtydflt{\ty}{\plug{\dctx'}{\ty'}}.
    \begin{itemize}
        \item Case \RST{Top} where $\dctx'=\square$. By assumption
            \subtydflt{\ty}{\tyany} and \RST{UnionRight} with $i=1$,
            \subtydflt{\ty}{\tyunion{\tyany}{\ty''}}.
        \item Case \RST{Bot} by \RST{Bot}.
        \item Case \RST{VarRefl} where $\dctx'=\square$
            by assumption and \RST{UnionRight} with $i=1$.
        \item Case \RST{VarLeft} by inversion, IH, and \RST{VarLeft}.
        \item Case \RST{VarRight} where $\dctx'=\square$
            by assumption and \RST{UnionRight} with $i=1$.
        \item Case \RST{Tuple}. 
            Subcase $\dctx'$ by assumption and \RST{UnionRight} with $i=1$.
            The other two subcases by inversion, IH, and \RST{Tuple}.
        \item Case \RST{Inv} where $\dctx'=\square$
            by assumption and \RST{UnionRight} with $i=1$.
        \item Case \RST{UnionLeft} by inversion, IH, and \RST{UnionLeft}.
        \item Case \RST{UnionRight} where $\dctx'=\square$
            by assumption and \RST{UnionRight} with $i=1$.
    \end{itemize}
\end{proof}

\clearpage
\section{Semantic Model}\label{sec:app:sem-sub}
%% ======================================================================

Julia's subtype relation was inspired by semantic subtyping.
The description of the original Julia design~\cite{bib:bezanson:julia:2015}
suggested an intuitive interpretation of types as sets of values,
but the interpretation is not well defined.
Furthermore, the treatment of invariant parametric types
in the interpretation does not match
the subtype relation. In particular, for concrete nominal \cjl{Name},
\[
  \interp{\text{\cjl{Name\{}}\ty\text{\cjl{\}}}} =
  \{ x\ |\ \text{\cjl{typeof}}(x) = \text{\cjl{Name\{}}\ty\text{\cjl{\}}} \}   
\]
does not account for the fact that there are multiple syntactic
representations $\ty'$ corresponding to the same interpretation as \ty.
Thus, for example, types \cjl{Vector\{Union\{Int,Any\}\}} and \cjl{Vector\{Any\}}
are not equivalent according to the interpretation,
but they are equivalent in Julia.

In \cite{bib:belyakova:minijl-sub:ftfjp:2019}, I proposed and mechanized in Coq
a semantic interpretation of (a subset of) Julia
types \ty as \textbf{sets of concrete value types} \gty
(or type tags) rather than values.
For the type language of non-parametric nominal types, tuples, and unions,
a decidable syntactic subtyping based on disjunctive normal form
coincides with the set inclusion of interpretations.
%$\interp{\ty} \subseteq \interp{\ty'}$.
\figref{fig:sem:ty-simple} presents the interpretation and the corresponding
syntactic, decidable subtyping in the style of \chapref{chap:dec-sub}
for that language of types.
$\Sigma$ denotes the set of all value types \gty.

\begin{figure}[t]
\footnotesize
Grammar
\[\begin{array}{rcll}
    \ty &::=& 
        \tyany \Alt \tybot \Alt \iname \Alt
        \typair{\ty_1}{\ty_2} \Alt \tyunion{\ty_1}{\ty_2}
        & \textit{Type} \\
    \\
    \gty &::=& 
        \iname \Alt
        \typair{\ty_1}{\ty_2}
        & \textit{Concrete value type} \\
    \\
    \dctx &::=& \square \Alt
        \typair{\dctxty}{\ty} \Alt \typair{\ty}{\dctxty} 
        & \textit{Distributivity context} \\
\end{array}\]
Interpretation $\interp{\cdot}$
\[
    \begin{array}{ccl}
        \interp{\tyany} &=& \Sigma \\
        \interp{\tybot} &=& \varnothing \\
        \interp{\iname} &=& \{ \iname \} \\
        \interp{\typair{\ty_1}{\ty_2}} &=& 
            \{ \typair{\gty_1}{\gty_2} \ | \ 
            \gty_1 \in \interp{\ty_1}, \gty_2 \in \interp{\ty_2} \} \\
        \interp{\tyunion{\ty_1}{\ty_2}} &=& 
            \interp{\ty_1} \cup \interp{\ty_2} \\
    \end{array}
\]
Subtyping $\ty \symsub \ty'$

\makebox[\textwidth][c]{
\begin{minipage}{\ruleswidth}  
\begin{mathpar}
    \inferrule[]
    { }
    { \ty \symsub \tyany }

    \inferrule[]
    { }
    { \plug\dctx\tybot \symsub \ty' }

    \inferrule[]
    {  }
    { \iname \symsub \iname }
    \\

    \inferrule[]
    { \ty_1 \symsub \ty'_1 \and 
        \ty_2 \symsub \ty'_2 }
    { \typair{\ty_1}{\ty_2} \symsub \typair{\ty'_1}{\ty'_2} }

    \inferrule[]
    { \plug\dctx{\ty_1} \symsub \ty' \and 
        \plug\dctx{\ty_2} \symsub \ty' }
    { \plug\dctx{\tyunion{\ty_1}{\ty_2}} \symsub \ty' }

    \inferrule[]
    { \exists i.\ \ty \symsub \ty'_i }
    { \ty \symsub \tyunion{\ty'_1}{\ty'_2} }
\end{mathpar}
\end{minipage}
}
\caption{Semantic interpretation and decidable subtyping for simple types
}\label{fig:sem:ty-simple}
\end{figure}


\clearpage
\section{Evaluation}\label{sec:app:eval}
%% ======================================================================

\begin{figure}
\small
%\makebox[\textwidth][c]{
%\begin{minipage}{\ruleswidth}
\begin{minipage}{6.5cm}
    \centering
    Satisfy the restriction
\begin{julia}
Pair{T, T} where T
Tuple{T, Tuple{T, Int}} where T
Tuple{Ref{T}} where T
Tuple{T, Ref{T}} where T
Tuple{Ref{Tuple{T}}} where T
Vector{Union{T, Int}} where T
Ref{Ref{T} where T} where T
Ref{Pair{T, S} where S} where T
Pair{S, Pair{Int, T} where T<:S} where S
Tuple{S, Pair{T, S} where T<:S} where S
Tuple{Ref{T} where T>:S} where S
Ref{T} where T<:(Ref{S} where S)
Tuple{T} where T<:Ref{Ref{<:Any}}
\end{julia}
\end{minipage}
\hspace{1cm}
\begin{minipage}{6cm}
    \centering
    Do not satisfy the restriction
\begin{julia}
Ref{Pair{T, T} where T}
Ref{Tuple{Pair{T, T} where T}}
Tuple{T} where T>:(Pair{S,S} where S)
Vector{Vector{Union{T, Int}} where T}
Vector{Ref{Tuple{T}} where T}
\end{julia}
\end{minipage}
%\end{minipage}}
\caption{Test cases for the analysis of type annotations
}\label{fig:tests:ta-analysis}
\end{figure}

\begin{figure}
%\begin{minipage}{14cm}
\begin{lstlisting}
Error: Couldn't process expression
  e =
   :($(Expr(:$, :d))->begin
             #= none:54 =#
             Base.axes($(Expr(:$, :arraysym)), $(Expr(:$, :d)))
         end)
  err =
   ArgumentError: Not a function definition: :($(Expr(:$, :d))->begin
             #= none:54 =#
             Base.axes($(Expr(:$, :arraysym)), $(Expr(:$, :d)))
         end)
\end{lstlisting}
%\end{minipage}
\caption{An example of a parsing error}\label{fig:evaluation-parse-errors}
\end{figure}

\begin{figure}
\begin{minipage}{12cm}
\begin{lstlisting}
Error: Couldn't process type annotation
    tastr = "Tuple{Union{Document, Node}} where \$(esc.(P)...)"
    err = AssertionError: Unsupported lb-var-ub format
  
Error: Unsupported Expr type annotation
  ty = :(typeof.((year, month, day, yearmonth, ...))...)
  
Error: Couldn't process type annotation
  tastr = "(Tuple{A} where Base.IteratorSize(A)::Base.SizeUnknown) where A"
  err = AssertionError: Unsupported lb-var-ub format
  
Error: Couldn't process type annotation
  tastr = "(((Tuple{(\$T_nameparam){\$N, \$M, \$FT}} ... \$(T_params...)"
  err = AssertionError: Unsupported lb-var-ub format

Error: Couldn't process type annotation
  tastr = "Tuple{Union{map((T->beginn  #= none:302 =#n ...)...}}"
  err = Base.Meta.ParseError("missing comma or ) in argument list")
\end{lstlisting}
\end{minipage}
\caption{Type annotation processing errors}\label{fig:evaluation-process-errors}
\end{figure}
% ty = :(typeof.((year, month, day, yearmonth, monthday, yearmonthday, dayofweek))...)
% tastr = "(Tuple{A} where Base.IteratorSize(A)::Base.SizeUnknown) where A"
% tastr = "(((Tuple{(\$T_nameparam){\$N, \$M, \$FT}} where \$FT) where \$M) where \$N) where \$(T_params...)"
% tastr = "Tuple{Union{map((T->beginn                    #= none:302 =#n                    AbstractGeometry{T}n                end), multipointtypes)...}}"
  
