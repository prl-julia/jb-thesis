\RequirePackage{silence} % :-\
  \WarningFilter{scrreprt}{Usage of package `titlesec'}
  \WarningFilter{titlesec}{Non standard sectioning command}
\documentclass[11pt,
  footinclude,headinclude,
  abstract=on
]{scrreprt}
%paper=b5

%% %%%%%%%%%%%%%%%%%%%%%%%%%%%%%%%%%%%%%%%%%%%%%%%%%%%%%%%%%%%%%%%%%%%%%%
%% ************************************************************
%% ==================================================

%% General
%% %%%%%%%%%%%%%%%%%%%%%%%%%%%%%%%%%%%%%%%%%%%%%%%%%%%%%%%%%%%%%%%%%%%%%%

\usepackage[T1]{fontenc}
\usepackage[linedheaders=true]{classicthesis} % ,manychapters
%\usepackage[osf]{libertine}
\usepackage[english]{babel}

%% Bibliography
%% %%%%%%%%%%%%%%%%%%%%%%%%%%%%%%%%%%%%%%%%%%%%%%%%%%%%%%%%%%%%%%%%%%%%%%

\usepackage{natbib}
%% Settings copied from the ACM style
\bibliographystyle{style/ACM-Reference-Format}{}
\setcitestyle{%
    authoryear,%
    open={[},close={]},citesep={;},%
    aysep={},yysep={,},%
    notesep={, }}
\let\citeN\cite
\let\cite\citep

%% Style
%% %%%%%%%%%%%%%%%%%%%%%%%%%%%%%%%%%%%%%%%%%%%%%%%%%%%%%%%%%%%%%%%%%%%%%%

\usepackage{style/julia}
\usepackage{xspace}
\usepackage{xcolor}

%% Math
%% %%%%%%%%%%%%%%%%%%%%%%%%%%%%%%%%%%%%%%%%%%%%%%%%%%%%%%%%%%%%%%%%%%%%%%

\usepackage{stmaryrd}
\usepackage{amsmath}


\newcommand{\TODO}[1]{\textcolor{red}{\textbf{TODO:} #1}}

\begin{document}

\title{Decidable Subtyping\\ of Existential Types\\for the Julia Language}
% \subtitle{Restricting Existential Types Inside Invariant Constructors\\ 
% to Use-Site Variance}

\author{Julia Belyakova}

\date{\normalsize%
Northeastern University\\
Khoury College of Computer Sciences\\
Boston, Massachusetts, USA\\
August 2023
}

\maketitle


\begin{abstract}
\TODO{abstract}
% Julia is a dynamic, high-level, yet high-performance programming language
% for scientific computing.
% To encourage a high level of code reuse and extensibility, Julia is
% designed around symmetric multiple dynamic dispatch, which allows functions
% to have multiple implementations tailored to different argument types.
% To resolve multiple dispatch, Julia relies on a subtype relation over a complex
% language of run-time types and type annotations, which include parametric types,
% untagged unions that distribute over covariant tuples, and impredicative
% existential types.
% Notably, Julia's subtyping is undecidable.
% Unlike in statically typed languages
% with undecidable subtyping, e.g. Scala, where the undecidability manifests at
% compile time, in Julia, it leads to a run-time stack overflow error. 
% This can happen
% at almost any point during the program execution, as subtyping is used to
% resolve every function call.
% In this thesis, I propose to develop a decidable subtype relation
% for the Julia language,
% such that migrating existing code would require minimal effort.
% In particular, to provide for decidability, I will restrict the language of type
% annotations. To evaluate the migration effort, I will check whether
% type annotations used in registered Julia packages conform to the proposed
% restriction.

\end{abstract}


%% ======================================================================
%% *********************************************************

\section*{Proposal requirements}

The thesis proposal should include the relevant background materials from
literature and clearly specify the research problems to be attacked, the
techniques to be used, and a schedule of milestones towards completion.
Typically, the thesis proposal should not exceed 15 pages, excluding appendices
and bibliography.

% \section{Outline}

% \begin{enumerate}
%   \item Introduction: overview of the problem and the proposal?
%   \item Background: brief intro to Julia and an overview of types.
%   \item Research problems: what ids the role of types and subtyping in Julia?,
%     can we define a reasonable restriction on types to achieve decidable but
%     still useful subtyping relation? (With a plan to attach the problems)
%   \item Related work: wildcards and existential types, decidability of bounded
%     quantification, union and intersection types.
%   \item Preliminary work: undesired properties of subtyping, confusion with
%     diagonal types, tag-based semantic subtyping, model of Julia's JIT.
%   \item Schedule of milestones.
% \end{enumerate}

\chapter{Introduction}

\section{A Section}

\TODO{text} dsadsa

In this section, we will discuss the \emph{first} topic.
TOTO12
\chapter{Research problem}\label{chap:3}

As discussed in \chapref{chap:2}, Julia relies on a complex language of types
and subtyping for multiple dynamic dispatch, and subtyping turns out to be
undecidable. In practice, this can manifest with a \cjl{StackOverflowError}
during program execution, as subtyping is used at run time for several purposes:
to resolve function calls, to process new method definitions, and even during
JIT compilation. A number of issues related to subtyping have been reported
on the Julia bug tracker. For example,
\href{https://github.com/JuliaLang/julia/issues/41948}{\code{\#41948}}\footnote{
    \url{https://github.com/JuliaLang/julia/issues/41948}
} reports \cjl{StackOverflowError} caused by a function definition,
and
\href{https://github.com/JuliaLang/julia/issues/33137}{\code{\#33137}}\footnote{
    \url{https://github.com/JuliaLang/julia/issues/33137}
} points out an inconsistency related to the diagonal rule.
Overall, there are 22 open and 114 closed issues labeled with both ``bug''
and ``types and dispatch'' (as of December~2021). For context, ``bug'' and
``codegen'' are assigned to 6 open and 76 closed issues, and overall, ``bug''
is assigned to 213 open and 2477 closed issues.
This demonstrates that type-related concerns, including undecidability of
subtyping, are not purely theoretical: they manifest in the user code and
constitute a non-negligible portion of problems in the Julia implementation.

The main research problem I am going to focus on is
\tdef{decidable subtyping for the Julia language}.
The goal is to find a decidable subtyping specification for a type language
that is close enough to the one currently used in Julia
and is not unreasonably restrictive.
The latter means that the majority of types in the existing Julia packages
are supported by the proposed specification.
Furthermore, the type language should be suitable for the use in the rest of
the Julia compiler.
To tackle the problem, I will answer the following questions:
\begin{enumerate}
    \item \emph{How are types used in Julia and what operations on types
      need to be supported?}
      Clearly, types are used as type annotations in method definitions, with
      subtyping being part of the dispatch resolution.
      But beyond that, Julia is known to rely on type inference for optimizations
      during JIT compilation. According to \citet{TODO}, which describes the
      original Julia design, the type inference algorithm needs subtyping
      but also meet, join, and widening operators.
      As \cite{TODO} is generally outdated, the current state of Julia needs
      to be reviewed to identify what operations on types the compiler relies on.
    \item \emph{How complicated are types used in practice?}
      To make subtyping decidable, I will likely need to restrict the type
      language in some way. At the same time, the restriction should not be
      prohibitively strong for the existing code base. Thus, an analysis of types
      currently used in practice can provide guidance for possible restrictions.
\end{enumerate}

\chapter{Background}

\section{A Section}

\chapter{Preliminary work}\label{chap:4}

%In this chapter, I briefly describe preliminary work
%related to subtyping and the role of types in Julia. %run-time semantics of Julia.

\paragraph{A reconstruction of Julia subtyping.} 
This work is the result of a collaborative effort to provide a
readable specification of Julia subtyping.
Initially, an incomplete (and soon outdated) definition of subtyping existed only
in~\cite{bib:bezanson:julia:2015}, with the actual implementation of subtyping ($\sim$3000 lines of
heavily optimized C~code) being the only reference point.

In~\cite{bib:zappa-nardelli:julia-sub:oopsla:2018}, we define subtyping in the form of a judgment
\[
  \ljsub{E}{t}{t'}{E'}.
\]
Here, $E$ is a type variable environment that contains two kinds of variables,
forall (also called left) and exist (also called right).
A forall variable is added to the environment
when an existential type appears on the left of the subtyping judgment,
as in \cjl{(Vector\{X\} where X) <: t'}. Such variables never change in the
environment, which corresponds to the intuition that for a left-hand side
existential type,
subtyping should hold for all possible instantiations of the type variable.
However, when an existential type appears on the right,
as in \cjl{t <: Vector\{Q\} where Q}, the variable is
added as an exist variable. During subtype checking, such variables may accrue
subtype constraints in addition to their declared bounds; if all the constraints
are consistent, subtyping succeeds, which corresponds to the intuition that
for a right-hand side existential type, subtyping holds
if there exists a valid instantiation of the type variable. 
%Constraints on possible instantiations take the form of updated
%variable bounds, and these updates produce the environment~$E'$.
Exist-variables with updated bounds are recorded in the output environment~$E'$.
For example, if \cjl{Q} is an exist variable with bounds
\cjl{Union\{\}<:Q<:Any}, then
\[
  \ljsub{\mathtt{Q}_\text{\cjl{Union\{\}}}^\text{\cjl{Any}}}
    {\text{\cjl{Int}}}{\mathtt{Q}}
    {\mathtt{Q}_\text{\cjl{Int}}^\text{\cjl{Any}}},
\]
i.e., \cjl{Int<:Q} results in updating \cjl{Q} to \cjl{Int<:Q<:Any}.

Having reconstructed the specification of Julia subtyping,
we were able to identify its undecidability. Furthermore, this work highlighted
several other problems, e.g. the lack of transitivity,
which I plan to address in the thesis.
%For example, the diagonal rule (discussed in \secref{sec:julia-sub:overview})
%is applied not to an isolated type, but based on the subtyping check with
%another type. As a result, the same existential type may be constrained to
%a concrete type variable in one subtyping judgment but not the other.
%This makes it hard, if not impossible, to reason about properties of subtyping
%such as reflexivity and transitivity.

\paragraph{Tag-based semantic subtyping for non-existential types.}
Julia's subtype relation was inspired by semantic subtyping.
The description of the original Julia design~\cite{bib:bezanson:julia:2015}
suggested an intuitive interpretation of types as sets of values,
but the interpretation is not well defined.
Furthermore, the treatment of invariant parametric types
in the interpretation does not match
the subtype relation. In particular, for concrete nominal \cjl{Name},
\[
  \interp{\text{\cjl{Name\{}}\ty\text{\cjl{\}}}} =
  \{ x\ |\ \text{\cjl{typeof}}(x) = \text{\cjl{Name\{}}\ty\text{\cjl{\}}} \}   
\]
does not account for the fact that there are multiple syntactic
representations $\ty'$ corresponding to the same interpretation as \ty.
Thus, for example, types \cjl{Vector\{Union\{Int,Any\}\}} and \cjl{Vector\{Any\}}
are not equivalent according to the interpretation,
but they are equivalent in Julia.

In \cite{bib:belyakova:minijl-sub:ftfjp:2019}, I proposed and mechanized in Coq
a semantic interpretation of (a subset of) Julia
types \ty as sets of type tags \gty (i.e. concrete types) rather than values.
For the type language of non-parametric nominal types, tuples, and unions,
a decidable syntactic subtyping based on disjunctive normal form
coincides with the set inclusion of interpretations
$\interp{\ty} \subseteq \interp{\ty'}$.
% While the treatment of tuples and unions is completely standard, nominal types
% are not as straightforward: interpreting an abstract type as a set of
% its concrete subtypes (e.g. $\interp{\text{\cjl{Signed}}} =
% \{\text{\cjl{Int8, Int16, ..., Int128}}\}$) would lead to brittle programs.
% Namely, because subtyping is used at run time to process method definitions and 
% dynamic dispatch, defining a new concrete subtype of an abstract type would
% change the subtyping relation and
% could cause the existing code to execute differently.
%It is worth emphasizing that Julia subtyping contradicts the set-theoretic
%interpretation in its treatment of the bottom type \cjl{Union\{\}}.
%Semantically, both \cjl{Union\{\}} and \cjl{Tuple\{Union\{\}\}} represent the
%empty set and are thus equivalent. However, in Julia,
%\cjl{Tuple\{Union\{\}\}} is not a subtype of \cjl{Union\{\}}.

Following up on this work, I extended the interpretation to support
invariant type constructors such as \cjl{Vector\{...\}}.
To account for the issue of multiple syntactic representations of types,
in the new system, types are given an indexed interpretation $\interp{\cdot}_n$.
In particular, a concrete type constructor \cjl{Name\{...\}} is interpreted as
\[
\begin{array}{lcl}
  \interp{\text{\cjl{Name\{}}\ty\text{\cjl{\}}}}_{n+1} & = &
  \{ \text{\cjl{Name\{}}\ty'\text{\cjl{\}}}
  \ |\ \interp{\ty'}_n = \interp{\ty}_n \}    \\
  \interp{\text{\cjl{Name\{}}\ty\text{\cjl{\}}}}_{0} & = &
  \{ \text{\cjl{Name\{}}\ty'\text{\cjl{\}}} \},
\end{array}
\]
and subtyping is defined as
\[
  \ty <: \ty' \quad \equiv \quad
  \forall n.\ \interp{\ty}_n \subseteq \interp{\ty'}_n.
\]
Despite the more complex indexed interpretation, the subtype relation has
an equivalent, decidable syntactic definition $\ty \leq \ty'$ where
$\text{\cjl{Name\{}}\ty\text{\cjl{\}}} \leq
\text{\cjl{Name\{}}\ty'\text{\cjl{\}}}$
holds if $\ty \leq \ty'$ and $\ty' \leq \ty$.
This extension is also mechanized in Coq.

% \paragraph{A model of Julia's JIT~\cite{TODO}.}
% Using an abstract machine modeled after Julia's JIT compiler, this work formally
% defines the property of a method definition that impacts the ability of
% the compiler to optimize the code: the property is called type stability.
% Although type stability is not directly related to the topic of this thesis
% proposal, having defined a model of the JIT compiler, I found that
% (1)~the correctness of JIT compilation depends on the soundness of type
% inference, and (2)~concrete types and the ability of type inference to infer
% concrete types of expressions are conducive to optimizations.
% These observations emphasize that despite Julia being a dynamic language,
% subtyping and the type language are an integral part of the compiler
% and Julia's performance.

\chapter{Related work}\label{chap:5}

Subtyping is typically associated with object-oriented programming languages and
static type systems. As a part of a type system, subtyping allows
an expression of a more specific type (subtype) to be used in place
where an expression of a more general type (supertype) is expected,
enabling more benign programs to be typable.

Although decidability is desirable for a static type system, establishing
decidability can be challenging and may take years.
For example, the \emph{un}decidability of Java generics was 
established only fairly recently by~\citet{grigore:java-undec:2017},
and the previously suspected undecidability of the core calculus of Scala 3,
DOT, was proven by~\citet{hu:dot-undec:2020}.
In both cases, the undecidability of the type system is a consequence of
undecidable subtyping.
Practically, the undecidability of a type system manifests itself with
the compiler crashing or not terminating. In some cases,
benefits of having a more expressive type system might outweigh the cost of
undecidability: if the undecidability arises only in rare, contrived cases,
being able to type more benign programs at the cost of rare compiler crashes
might be preferable to being able to type fewer programs
with a decidable but less expressive type system.
Identifying decidable fragments of undecidable type systems remains
an important challenge~\cite{mackay:path-dep-dec:2020,
hu:dot-undec:2020,mackay:bound-poly-sub-dec:2020,kennedy:nom-sub-var-dec:2007}.

Apart from a static type system,
subtyping can also be used at run time for type tests and casts,
in which case its decidability becomes more important: otherwise, a crash or
non-termination could happen during program execution rather than compilation.
Run-time types can be more restrictive than static types, which
simplifies the corresponding subtyping problem.
For example, subtyping between ground types in the .NET intermediate
language is decidable~\cite{kennedy:nom-sub-var-dec:2007},
and with the type erasure mechanism in Java generics,
its run-time subtyping is decidable too.

\TODO{Talk about issues that cause undecidability and F-bounded polymorphism.}

\TODO{Semantic subtyping and Java wildcards (several paragraphs).}


Subtyping and decidability of subtyping are clearly a problem in various languages,
but we will focus on several related subsets.

Decidable and undecidable systems. Subtyping in PL..

\section{Semantic subtyping}
%% ======================================================================

In Julia, subtyping of union and tuple types (maybe also invariant ones?) build
on top of \TODO{Voulon} and has similarities to XDuce/CDuce.
There is a large body of work on semantic subtyping, which extends it to
function and polymorphic types. However, such systems do not deal with
impredicative types.
For example, \citet{frih:sem-sub:2008} describe a framework for defining semantic
subtyping without building a model of the entire language, and then define
decidable subtyping and type checking for a language with dynamic dispatch,
unions, functions, and negation types.\TODO{clean up the description}

In the semantic subtyping approach, types are interpreted as subsets of the
model of the programming

\TODO{\cite{hosoya:xduce:2003, bezanken:cduce:2003}}

%\chapter{Plan of work}\label{chap:6}

I am currently working on designing a subtype relation
and proving its decidability, reflexivity, and transitivity.
I have also extended the interpretation of types to account for type variables.
To evaluate the practicality of the subtype relation,
I will perform a static analysis of type annotations in registered Julia
packages (about 9000 as of March 2023).

In July 2023, I intend to submit a POPL 2024 paper on decidable subtyping
and complete my thesis, defending in August 2023.

\begin{table}[h]
  \caption{Schedule}
  \vspace*{0.25em}
  \centering\footnotesize
  \begin{tabular}{c|ccccc}
  \toprule
  & May & June & July & August \\
  \midrule
  proofs \& evaluation & X & X & & \\
  paper & X & X & & \\
  thesis & X & X & X & \\
  defense & & & & X \\
\end{tabular}
\end{table}

% I plan to work on developing a decidable subtyping specification for the Julia
% language and write a research paper on that.
% As outlined in \chapref{chap:3}, I envision the type language to be close to the
% one currently used in Julia, with some restrictions that ensure decidability
% but support the majority of the existing code base.
% As a first step, I will look into restricting lower bounds of existential type
% variables, for lower bounds make it possible to encode the undecidable
% system \FSub in Julia. Furthermore, taking into account the inconsistent
% treatment of types due to the diagonal rule,
% and the importance of concrete types for optimizations in the JIT compiler,
% I propose to explicitly distinguish between regular existential types
% and existential types where the type variable ranges over only concrete types.

% I intend to do the technical work on decidable subtyping and submit a paper
% for POPL 2023, with the deadline in July 2022.
% After that, I intend to collaborate with the Julia developers on incorporating
% decidable subtyping into the Julia language and work on the thesis.
% I expect that the thesis will be completed in a year from the proposal time,
% around February 2023.


% \begin{figure}
% \small
% \makebox[\textwidth]{
% \begin{tabular}{l@{\hspace{4mm}}l}
%   $\begin{array}{rcll}
%     \ty
%       &::=& & \textit{Type annotations} \\
%       &\Alt& \tyany & \text{top type} \\
%       &\Alt& \tybot & \text{bottom type} \\
%       &\Alt& \typair{\ty_1}{\ty_2}
%                     & \text{covariant pair} \\
%       &\Alt& \tyinv\iname\tys
%                     & \text{invariant constr.} \\
%       &\Alt& \tyexist{X}{\cty_l}{\ty_u}{\ty}
%                     & \text{existential type} \\
%       &\Alt& \tvx   & \text{type variable} \\
%       &\Alt& \gtyexist{X}{\ty_u}{\ty}
%                     & \text{concretely-exist. type} \\
%       &\Alt& \gvx   & \text{concrete type var.} \\
%       &\Alt& \tyunion{\ty_1}{\ty_2}
%                     & \text{union type} \\
%     \\
%     \cty  &::=& \ty  \Alt \fv\ty  = \varnothing & \textit{Closed types} \\
%   \end{array}$
% &
%   $\begin{array}{rcll}
%     \gty
%       &::=& & \textit{Type tags} \\
%       &\Alt& \tyinv\cname\tys
%                     & \text{concr. inv. constr.} \\
%       &\Alt& \typair{\gty_1}{\gty_2}
%                     & \text{concrete pair} \\
%       &\Alt& \gvx   & \text{concrete type var.} \\
%     \\\\
%     \iname
%       &::=& & \textit{User-defined names} \\
%       &\Alt& \cname & \text{concrete} \\
%       &\Alt& \aname & \text{abstract} \\
      
%     \\\\
%     \cgty &::=& \gty \Alt \fv\gty = \varnothing & \textit{Closed tags} \\
%   \end{array}$
% \end{tabular}
% }\caption{Syntax}\label{fig:syntax}
% \end{figure}

% \begin{figure}
% \small

%   \[ \VEnv ::= \EmptyEnv \Alt \VEnv, \var{X} \]

% \begin{mathpar}
%   \fbox{\wlscpd{\ty}}
%   \\

%   \inferrule{ }
%   { \wlscpd{\tyany} }

%   \inferrule{ }
%   { \wlscpd{\tybot} }
%   \\

%   \inferrule
%   { \wlscpd{\ty_1} \and \wlscpd{\ty_2} }
%   { \wlscpd{ \typair{\ty_1}{\ty_2} } }

%   \inferrule
%   { \forall i.\ \cfbox{light-gray}{\wlfrscpd{\ty_i}} }
%   { \wlscpd{ \tyinv\iname\tys } }

%   \inferrule
%   { \wlscpd{\ty_1} \and \wlscpd{\ty_2} }
%   { \wlscpd{ \tyunion{\ty_1}{\ty_2} } }
%   \\

%   \inferrule
%   { \wlscp{\EmptyEnv}{\cty_l} \and \wlscpd{\ty_u} \and \wlscp{\VEnv,\vx}{\ty} }
%   { \wlscpd{ \tyexist{X}{\cty_l}{\ty_u}{\ty} } }

%   \inferrule
%   { \wlscpd{\ty_u} \and \wlscp{\VEnv,\vx}{\ty} }
%   { \wlscpd{ \gtyexist{X}{\ty_u}{\ty} } }
  
%   \inferrule
%   { \vx \in \dom\VEnv }
%   { \wlscpd{\avx} }
%   \\

%   \fbox{\wlfrscpd{\ty}}
%   \\

%   \inferrule{ }
%   { \wlfrscpd{\tyany} }

%   \inferrule{ }
%   { \wlfrscpd{\tybot} }
%   \\

%   \inferrule
%   { \wlfrscpd{\ty_1} \and \wlfrscpd{\ty_2} }
%   { \wlfrscpd{ \typair{\ty_1}{\ty_2} } }

%   \inferrule
%   { \forall i.\ \wlfrscpd{\ty_i} }
%   { \wlfrscpd{ \tyinv\iname\tys } }

%   \inferrule
%   { \cfbox{light-gray}{$\wlscp{\colorbox{light-gray}{\EmptyEnv}}{\ty_1}$} \and
%     \cfbox{light-gray}{$\wlscp{\colorbox{light-gray}{\EmptyEnv}}{\ty_2}$} }
%   { \wlfrscpd{ \tyunion{\ty_1}{\ty_2} } }
%   \\

%   \inferrule
%   { \cfbox{light-gray}{\wlscp{\EmptyEnv}{\cty_l}} \and
%     \cfbox{light-gray}{$\wlscp{\colorbox{light-gray}{\EmptyEnv}}{\ty_u}$} \and
%     \cfbox{light-gray}{$\wlscp{\colorbox{light-gray}{\vx}}{\ty}$} }
%   { \wlfrscpd{ \tyexist{X}{\cty_l}{\ty_u}{\ty} } }

%   \inferrule
%   { \cfbox{light-gray}{$\wlscp{\colorbox{light-gray}{\EmptyEnv}}{\ty_u}$} \and
%     \cfbox{light-gray}{$\wlscp{\colorbox{light-gray}{\vx}}{\ty}$} }
%   { \wlfrscpd{ \gtyexist{X}{\ty_u}{\ty} } }

%   \inferrule
%   { \vx \in \dom\VEnv }
%   { \wlfrscpd{\avx} }
  
% \end{mathpar}

% \caption{Well scopedness}\label{fig:well-scope}
% \end{figure}


\begin{figure}
\small

\begin{array}{rcll}
  \ty & ::= & & 
\end{array}

\caption{TODO}
\end{figure}



\bibliography{bib/jv,bib/jl-lang,bib/all}

\end{document}
