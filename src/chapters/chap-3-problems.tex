\chapter{Research problem}

As discussed in \chapref{chap:2}, Julia relies on a complex language of types
and subtyping for multiple dynamic dispatch, and subtyping turns out to be
undecidable. In practice, this can manifest with a \cjl{StackOverflowError}
during program execution, as subtyping is used at run time for several purposes:
to resolve function calls, to process new method definitions, and even during
JIT compilation. A number of issues related to subtyping have been reported
on the Julia bug tracker. For example,
\href{https://github.com/JuliaLang/julia/issues/41948}{\code{\#41948}}\footnote{
    \url{https://github.com/JuliaLang/julia/issues/41948}
} reports \cjl{StackOverflowError} caused by a function definition,
and
\href{https://github.com/JuliaLang/julia/issues/33137}{\code{\#33137}}\footnote{
    \url{https://github.com/JuliaLang/julia/issues/33137}
} points out an inconsistency related to the diagonal rule.
Overall, there are 22 open and 114 closed issues labeled with both ``bug''
and ``types and dispatch'' (as of December~2021). For context, ``bug'' and
``codegen'' are assigned to 6 open and 76 closed issues, and overall, ``bug''
is assigned to 213 open and 2477 closed issues.
This demonstrates that type-related concerns, including undecidability of
subtyping, are not purely theoretical: they manifest in the user code and
constitute a non-negligible portion of problems in the Julia implementation.

The main research problem I am going to focus on is
\tdef{decidable subtyping for the Julia language}.
The goal is to find a decidable subtyping specification for a type language
that is close enough to the one currently used in Julia
and is not unreasonably restrictive.
The latter means that the majority of types in the existing Julia packages
are supported by the proposed specification.
Furthermore, the type language should be suitable for the use in the rest of
the Julia compiler.
To tackle the problem, I will answer the following questions:
\begin{enumerate}
    \item \emph{How are types used in Julia and what operations on types
      need to be supported?}
      Clearly, types are used as type annotations in method definitions, with
      subtyping being part of the dispatch resolution.
      But beyond that, Julia is known to rely on type inference for optimizations
      during JIT compilation. According to \citet{TODO}, which describes the
      original Julia design, the type inference algorithm needs subtyping
      but also meet, join, and widening operators.
      As \cite{TODO} is generally outdated, the current state of Julia needs
      to be reviewed to identify what operations on types the compiler relies on.
    \item \emph{How complicated are types used in practice?}
      To make subtyping decidable, I will likely need to restrict the type
      language in some way. At the same time, the restriction should not be
      prohibitively strong for the existing code base. Thus, an analysis of types
      currently used in practice can provide guidance for possible restrictions.
\end{enumerate}
