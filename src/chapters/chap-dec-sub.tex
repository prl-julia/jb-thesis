\chapter{Decidable Subtyping of Existential Types}%
\label{chap:dec-sub}

\TODO{intro}

\section{Definition}%
\label{sec:dec-sub:defs}
%% ======================================================================

%% Grammar
%% *********************************************************

The key to decidable subtyping is to restrict existential types within
invariant constructors to use-site variance of the form
\tyinv\iname{\rexvarbound{\ty_1}{\ty_1},\ldots},
similar to Java wildcards.
Unrestricted existential types are supported only at the top level,
in type signatures \tysig.

Types and type signatures are defined in \figref{fig:ty-grammar}.

\tyinv\iname{\ty_1,\ldots} is a shorthand for
\tyinv\iname{\rexvarbound{\ty_1}{\ty_1},\ldots}.

\begin{figure}[t]
\footnotesize
\makebox[\textwidth][c]{
\begin{tabular}{l@{\hspace{4mm}}l}
    $\begin{array}{rcll}
        \tysig
        &::=& & \textit{Type signature} \\
        &\Alt& \tyany & \text{top} \\
        &\Alt& \tybot & \text{bottom} \\
        &\Alt& \vany
            & \text{type variable} \\
        &\Alt& \typair{\tysig_1}{\tysig_2}
            & \text{covariant tuple} \\        
        &\Alt& \tyinv\iname\rexvars
            & \text{invariant constr.} \\
        &\Alt& \tyunion{\tysig_1}{\tysig_2}
            & \text{union} \\
        &\Alt& \tyexist{\vany}{\tylb}{\tyub}{\tysig}
            & \text{existential} \\
        \\
        \vany
        &::=& & \textit{Type variable} \\
        &\Alt& \vx, \vy, \ldots & \text{universal var.} \\
        &\Alt& \va, \vb, \ldots & \text{unification var.} \\
        \\
        \rexvar
        &::=& \rexvarbound{\tylb}{\tyub} & \textit{Restricted existential var.} \\
    \end{array}$ 
    &
    $\begin{array}{rcll}
        \ty, l, u
        &::=& & \textit{Type} \\
        &\Alt& \tyany & \text{top} \\
        &\Alt& \tybot & \text{bottom} \\
        &\Alt& \vany
            & \text{type variable} \\
        &\Alt& \typair{\ty_1}{\ty_2}
            & \text{covariant tuple} \\        
        &\Alt& \tyinv\iname\rexvars
            & \text{invariant constr.} \\
        &\Alt& \tyunion{\ty_1}{\ty_2}
            & \text{union} \\
        \\ \\ \\ \\ \\ \\ \\ \\
    \end{array}$
\end{tabular}
}
\caption{Grammar of type signatures and types}\label{fig:ty-grammar}
\end{figure}

\begin{figure}
\footnotesize
\[
\begin{array}{lccccccl}
    \dctxtysig &::=& \square &
        \Alt& \typair{\dctxtysig}{\tysig} &
        \Alt& \typair{\tysig}{\dctxtysig} &
        \textit{Distributivity context for type signatures}
    \\
    \dctxty &::=& \square &
        \Alt& \typair{\dctxty}{\ty} &
        \Alt& \typair{\ty}{\dctxty} &
        \textit{Distributivity context for types}
    \\
    \\
    \AEnv, \UEnv &::=& \EmptyEnv &
        \Alt& \AEnv, \varbound{\vany}{\tylb}{\tyub} &
        & & 
        \textit{Type variable environment}
    \\
    \UEnvD &::=& \EmptyEnv &
        \Alt& \UEnvD, \vany &
        & & 
        \textit{Type variable list}
\end{array}
\]
\caption{Auxiliary definitions}\label{fig:subty-aux}
\end{figure}
% \\
% \\
% \sty &::=& \tyany \Alt \tybot \Alt \vany &
%     \Alt& \typair{\sty_1}{\sty_2} &
%     \Alt& \tyinv\iname\rexvars &
%     \textit{Type without top-level unions}


%% Subtyping
%% *********************************************************

For simplicity of the presentation, we assume the following variable names 
convention based on Barendregt's convention:

\begin{definition}[Variable names convention]\label{def:var-names}
    Everywhere in definitions and proofs, the following conditions hold:
    \begin{itemize}
        \item all bound variables in \ty, \tysig are different from each other 
            and from free variables;
        \item all variables in \AEnv, \UEnv, \UEnvD are different 
            from each other;
        \item whenever both \AEnv and \UEnv or \AEnv and \UEnvD
            appear in the same judgment, 
            $\dom{\AEnv} \cap \dom{\UEnv} = \varnothing$ and
                $\dom{\AEnv} \cap \dom{\UEnvD} = \varnothing$;
        % \item whenever both \AEnv and \UEnv or \AEnv and \UEnvD
        %     appear in the same judgment:
        %     \begin{itemize}
        %         \item $\dom{\AEnv} \cap \dom{\UEnv} = \varnothing$ and
        %             $\dom{\AEnv} \cap \dom{\UEnvD} = \varnothing$;
        %         \item \AEnv contains only universal variables \vx, \vy, etc.,
        %             with their bounds also containing only universal variables;
        %         \item \UEnv and \UEnvD contain only unification variables
        %             \va, \vb, etc., with their bounds (for \UEnv) also
        %             containing only unification variables.
        %     \end{itemize}
        \item whenever a variable environment and a type/type signature
            appear in the same judgment, bound variables of the type/type
            signature are different from variables in the environment;
        \item whenever multiple types/type signatures appear in the same
            judgment, their bound variables are different.
    \end{itemize}
    These conditions can be maintained by alpha-renaming.
\end{definition}

To visually aid the perception of inference rules,
we use \vx, \vy, etc. to denote universal variables
and \va, \vb, etc. to denote unification variables.
Unification variables are constrained variables.

% The convention about using different sets of variables for \AEnv and \UEnv/\UEnvD
% is not necessary, but it visually aids the perception of inference rules:
% whenever \va or \vb is used in a judgment,
% it is known to belong to \UEnv or \UEnvD.
% Without this convention, we would need to always check which environment
% the variable belongs to in order to discern its meaning.

Subtyping starts with
$\subtysigdflt\tysig{\tysig'}$, subtyping of type signatures.
Here, we introduce all explicit existential variables into environments:
variables from \tysig into \AEnv (universal), 
and variables from $\tysig'$ into \UEnv (unification).
The intuition is that for all valid instantiations of \AEnv exist valid
instantiations of \UEnv.

Once all existentials are processed, we run constraints-generating
subtyping $\subtyctrdflt\ty{\ty'}$. It stands for either:
\begin{itemize}
    \item $\subtyctrRdflt\ty{\ty'}$ when unification variables are on the right, or
    \item $\subtyctrLdflt\ty{\ty'}$ when unification variables are on the left.
\end{itemize}
We use \UEnvD rather than \UEnv to emphasize that declared bounds of unification
variables are irrelevant for the judgment.

Once all constraints are generated, they are resolved by \solvectrdflt
primarily with $\subtydflt\ty{\ty'}$,
subtyping of types that are free from unification vars.
There is a caveat that declared bounds of unification vars need to be consistent
with collected constraints, which is why constrained subtyping is used
in constraints resolution.

\begin{figure}
\footnotesize
\begin{mathpar}
    \fbox{\subtydflt{\ty}{\ty}}
    \\

    \inferrule[\RST{Top}]
    { }
    { \subtydflt{\ty}{\tyany} }

    \inferrule[\RST{Bot}]
    { }
    { \subtydflt{\plug\dctx\tybot}{\ty'} }
    \\

    \inferrule[\RST{VarRefl}]
    { \varbound{\vany}{\tylb}{\tyub} \in \AEnv }
    { \subtydflt{\vany}{\vany} }

    \inferrule[\RST{VarLeft}]
    { \varbound{\vany}{\tylb}{\tyub} \in \AEnv \and
        \subtydflt{\plug\dctx{\tyub}}{\ty'} }
    { \subtydflt{\plug{\dctx}\vany}{\ty'} }

    \inferrule[\RST{VarRight}]
    { \varbound{\vany}{\tylb}{\tyub} \in \AEnv \and
        \subtydflt\ty{\tylb} }
    { \subtydflt{\ty}{\vany} }
    \\

    \inferrule[\RST{Tuple}]
    { \subtydflt{\ty_1}{\ty'_1} \and \subtydflt{\ty_2}{\ty'_2} }
    { \subtydflt{\typair{\ty_1}{\ty_2}}{\typair{\ty'_1}{\ty'_2}} }

    % \inferrule*[]
    % { \inherits
    %     {\tyinv{\iname}{\vx_1,\ldots,\vx_n}}
    %     {\tyinv{\iname'}{\ty_1,\ldots,\ty_m}} \and
    %   \forall j \in 1..m.\ 
    %   \subtydflt
    %     {\subst{\ty_j}
    %         {\substel{\vx_1}{\rexvar_1},\ldots,\substel{\vx_n}{\rexvar_n}}}
    %     {\rexvar'_j} }
    % { \subtydflt
    %     {\tyinv\iname{\rexvar_1,\ldots,\rexvar_n}}
    %     {\tyinv{\iname'}{\rexvar'_1,\ldots,\rexvar'_m}} }

    \inferrule[\RST{Inv}]
    { \forall i \in 1..n. \and \subtydflt{\rexvar_i}{\rexvar'_i} }
    { \subtydflt
        {\tyinv\iname{\rexvar_1,\ldots,\rexvar_n}}
        {\tyinv\iname{\rexvar'_1,\ldots,\rexvar'_n}} }

    \inferrule[\RST{UnionLeft}]
    { \subtydflt{\plug\dctx{\ty_1}}{\ty'} \and 
        \subtydflt{\plug\dctx{\ty_2}}{\ty'} }
    { \subtydflt{\plug\dctx{\tyunion{\ty_1}{\ty_2}}}{\ty'} }

    \inferrule[\RST{UnionRight}]
    { \exists i.\ \ \subtydflt\ty{\ty'_i} }
    { \subtydflt{\ty}{\tyunion{\ty'_1}{\ty'_2}} }

    \\
    \fbox{\subtydflt{\rexvar}{\rexvar}}
    \\
    \inferrule*[]
    { \subtydflt{\tylb'}{\tylb} \and \subtydflt{\tyub}{\tyub'} }
    { \subtydflt
        {\rexvarbound{\tylb}{\tyub}}
        {\rexvarbound{\tylb'}{\tyub'}} }
    %
    \\
    \fbox{\subtydflt{\dctx}{\dctx}}
    \\

    \inferrule*[]
    { }
    { \subtydflt\square\square }

    \inferrule*[]
    { \subtydflt{\dctx_1}{\dctx'_1} \and \subtydflt{\ty_2}{\ty'_2} }
    { \subtydflt{\typair{\dctx_1}{\ty_2}}{\typair{\dctx'_1}{\ty'_2}} }

    \inferrule*[]
    { \subtydflt{\ty_1}{\ty'_1} \and \subtydflt{\dctx_2}{\dctx'_2} }
    { \subtydflt{\typair{\ty_1}{\dctx_2}}{\typair{\ty'_1}{\dctx'_2}} }
\end{mathpar}
\caption{Subtyping of types (free from unification variables)
    %Note: \dctx stands for \dctxty.
}\label{fig:subtyping-base}
\end{figure}


\begin{figure}
\footnotesize
\makebox[\textwidth][c]{
\begin{minipage}{\ruleswidth}  
\[
\begin{array}{ccccll}
    \hline
    \multicolumn{6}{c}{\tyjoindflt{\ty}{\ty'}} \\ 
    \hline 
    \ty &\sqcup_{\AEnv}& \ty' &=& \ty' & \text{if } \subtydflt{\ty}{\ty'} \\
    \ty &\sqcup_{\AEnv}& \ty' &=& \ty  & \text{if } \subtydflt{\ty'}{\ty} \\
    \ty &\sqcup_{\AEnv}& \ty' &=& \tyunion{\ty}{\ty'} & \text{otherwise} \\
    \\
    \hline
    \multicolumn{6}{c}{\tymeetdflt{\ty}{\ty'}} \\ 
    \hline 
    \ty &\sqcap_{\AEnv}& \ty' &=& \ty  & \text{ if } \subtydflt{\ty}{\ty'} \\
    \ty &\sqcap_{\AEnv}& \ty' &=& \ty' & \text{ if } \subtydflt{\ty'}{\ty} \\
    \tyunion{\ty_1}{\ty_2} &\sqcap_{\AEnv}& \ty' &=& 
        \tyunion{(\tymeetdflt{\ty_1}{\ty'})}{(\tymeetdflt{\ty_2}{\ty'})} &  \\
    \ty &\sqcap_{\AEnv}& \tyunion{\ty'_1}{\ty'_2} &=& 
        \tyunion{(\tymeetdflt{\ty}{\ty'_1})}{(\tymeetdflt{\ty}{\ty'_2})} &  \\
    \vany &\sqcap_{\AEnv}& \ty' &=& 
        \tymeetdflt{\tylb}{\ty'} & 
        \text{ where } \varbound{\vany}{\tylb}{\tyub} \in \AEnv \\
    \ty &\sqcap_{\AEnv}& \vany &=& 
        \tymeetdflt{\ty}{\tylb} & 
        \text{ where } \varbound{\vany}{\tylb}{\tyub} \in \AEnv \\
    \typair{\ty_1}{\ty_2} &\sqcap_{\AEnv}& \typair{\ty'_1}{\ty'_2} &=& 
        \typair{(\tymeetdflt{\ty_1}{\ty'_1})}{(\tymeetdflt{\ty_2}{\ty'_2})} &  \\
    \tyinv\iname{\ldots,\rexvarbound{\tylb_i}{\tyub_i},\ldots} 
        &\sqcap_{\AEnv}& \tyinv\iname{\ldots,\rexvarbound{\tylb'_i}{\tyub'_i},\ldots} &=& 
        \tyinv\iname{\ldots,\rexvarbound{(\tyjoindflt{\tylb_i}{\tylb'_i})}
            {(\tymeetdflt{\tyub_i}{\tyub'_i})},\ldots} &  \\
    & & & & \multicolumn{2}{l}{\text{where } \forall i.\ 
        \subtydflt{(\tyjoindflt{\tylb_i}{\tylb'_i})}
        {(\tymeetdflt{\tyub_i}{\tyub'_i})} } \\
    \ty &\sqcap_{\AEnv}& \ty' &=& \tybot & \text{otherwise} \\
\end{array}
\]
\end{minipage}}
\caption{Join ($\sqcup_{\AEnv}$) and meet ($\sqcap_{\AEnv}$) of types.
Both definitions should be read top to bottom, i.e. earlier cases have
precedence over later cases.
Some cases (e.g. where \ty or $\ty'$ are \tybot or \tyany, 
and where both are the same type variable)
are omitted because they are covered by \subtydflt{\ty}{\ty'} and 
\subtydflt{\ty'}{\ty}.}\label{fig:ty-join-meet}
\end{figure}

\begin{figure}
\footnotesize
\makebox[\textwidth][c]{
\begin{minipage}{\ruleswidth}  
\begin{mathpar}
    \fbox{\subtyctrdflt{\ty}{\ty}} 
    \\

    \inferrule[\RSC{Top}]
    { }
    { \subtyctrdfltenv{\ty}{\tyany}{\EmptyCSet} }

    \inferrule[\RSC{Bot}]
    { }
    { \subtyctrdfltenv{\plug\dctx\tybot}{\ty'}{\EmptyCSet} }

    \colorbox{light-gray}{$
    \inferrule[\RSC{UBot}]
    { \va \in \UEnvD }
    { \subtyctrLdfltenv{\plug\dctx\va}{\ty'}{\ctrsngl\va\tybot} }
    $}
    \\

    \inferrule[\RSC{VarRefl}]
    { \varbound{\vx}{\tylb}{\tyub} \in \AEnv }
    { \subtyctrdfltenv{\vx}{\vx}{\EmptyCSet} }

    \colorbox{light-gray}{$
    \inferrule[\RSC{UVarLeft}]
    { \va \in \UEnvD }
    { \subtyctrLdfltenv{\va}{\ty'}{\ctrsngl{\va}{\ty'}} }
    $}

    \colorbox{light-gray}{$
    \inferrule[\RSC{UVarRight}]
    { \va \in \UEnvD }
    { \subtyctrRdfltenv{\ty}{\va}{\ctrset{\ctrsub{\ty}{\va}}} }
    $}

    \inferrule[\RSC{VarLeft}]
    { \varbound{\vx}{\tylb}{\tyub} \in \AEnv \and
        \subtyctrdflt{\plug\dctx{\tyub}}{\ty'} }
    { \subtyctrdflt{\plug\dctx\vx}{\ty'} }

    \inferrule[\RSC{VarRight}]
    { \varbound{\vx}{\tylb}{\tyub} \in \AEnv \and
        \subtyctrdflt\ty{\tylb} }
    { \subtyctrdflt{\ty}{\vx} }
    \\

    \inferrule[\RSC{Tuple}]
    { \subtyctrdfltenv{\ty_1}{\ty'_1}{\CSet_1} \and 
        \subtyctrdfltenv{\ty_2}{\ty'_2}{\CSet_2} }
    { \subtyctrdfltenv{\typair{\ty_1}{\ty_2}}{\typair{\ty'_1}{\ty'_2}}
        {\CSet_1 \cup \CSet_2} }

    \inferrule[\RSC{Inv}]
    { \forall i \in 1..n. \and 
        \subtyctrdfltenv{\rexvar_i}{\rexvar'_i}{\CSet_i} }
    { \subtyctrdfltenv
        {\tyinv\iname{\rexvar_1,\ldots,\rexvar_n}}
        {\tyinv\iname{\rexvar'_1,\ldots,\rexvar'_n}}
        {\mcup_{i=1}^n \CSet_i} }
    \\

    \inferrule[\RSC{UnionLeft}]
    { \subtyctrdfltenv{\plug\dctx{\ty_1}}{\ty'}{\CSet_1} \and 
        \subtyctrdfltenv{\plug\dctx{\ty_2}}{\ty'}{\CSet_2} }
    { \subtyctrdfltenv{\plug\dctx{\tyunion{\ty_1}{\ty_2}}}{\ty'}
        {\CSet_1 \cup \CSet_2} }

    \inferrule[\RSC{UnionRight}]
    { \exists i.\ \ \subtyctrdflt\ty{\ty'_i} }
    { \subtyctrdflt{\ty}{\tyunion{\ty'_1}{\ty'_2}} }

    \colorbox{light-gray}{$
    \inferrule[\RSC{UVar-UnionRight}]
    {   \va \in \UEnvD \and \va_1, \va_2 \text{ fresh} \and
        \subtyctrL{\AEnv}{\UEnvD,\va_1}{\plug\dctx{\va_1}}{\ty'_1}{\CSet_1} \and
        \subtyctrL{\AEnv}{\UEnvD,\va_2}{\plug\dctx{\va_2}}{\ty'_2}{\CSet_2} \\ 
        \CSet_1 = \CSet'_1 \mcup_{i=1}^n \ctrset{\ctrsub{\va_1}{\tyub_1^i}} 
            \and \va_1 \notin \CSet'_1 \and
        \CSet_2 = \CSet'_2 \mcup_{j=1}^m \ctrset{\ctrsub{\va_2}{\tyub_2^j}} 
            \and \va_2 \notin \CSet'_2  \\
        \CSet'\ \ =\ \ 
            \ctrsngl{\va}{\tyunion{\msqcap_{i=1}^n \tyub_1^i}
                {\msqcap_{j=1}^m \tyub_2^j}} 
            \text{ if } n\geq1,m\geq1
        \text{\ or\ }
        \mcup_{i=1}^n \ctrset{\ctrsub{\va_1}{u_1^i}}\text{ if } m=0
        \text{\ or\ }
        \mcup_{j=1}^m \ctrset{\ctrsub{\va_2}{u_2^j}}\text{ if } n=0
    }
    { \subtyctrLdfltenv{\plug\dctx{\va}}{\tyunion{\ty'_1}{\ty'_2}}
        {\CSet'_1 \cup \CSet'_2 \cup \CSet'} }
    $}
%
    \\
    \fbox{\subtyctrdflt{\rexvar}{\rexvar}}
    \\

    \inferrule*[]
    { \subtyctrRdfltenv{\tylb'}{\tylb}{\CSet_l} \and 
        \subtyctrLdfltenv{\tyub}{\tyub'}{\CSet_u} }
    { \subtyctrLdfltenv
        {\rexvarbound{\tylb}{\tyub}}
        {\rexvarbound{\tylb'}{\tyub'}}
        {\CSet_l \cup \CSet_u} }

    \inferrule*[]
    { \subtyctrLdfltenv{\tylb'}{\tylb}{\CSet_l} \and 
        \subtyctrRdfltenv{\tyub}{\tyub'}{\CSet_u} }
    { \subtyctrRdfltenv
        {\rexvarbound{\tylb}{\tyub}}
        {\rexvarbound{\tylb'}{\tyub'}}
        {\CSet_l \cup \CSet_u} }
\end{mathpar}
\end{minipage}}
\caption{Constrained subtyping of types.
    Note: %\dctx stands for \dctxty;
    every rule with the symbol $\symsubctr$ is a shorthand for two rules, 
    one where all occurrences of $\symsubctr$ are replaced with 
    $\symsubctrL$, and another with $\symsubctrR$.
    Thus, the figure defines two mutually recursive relations,
    \subtyctrLdflt{\ty}{\ty'} and \subtyctrRdflt{\ty}{\ty'}.
    %$\symsubctr$ stands for either $\symsubctrL$ or $\symsubctrR$,
    %and within one rule, all occurrences of $\symsubctr$ denote the same relation
}\label{fig:subtyping-constrained}
\end{figure}
%% old definition of UVar-UnionRight: unclear how to prove its soundness
% \colorbox{light-gray}{$
% \inferrule[\RSC{UVar-UnionRight}]
% {   \va \in \UEnvD \and \va_1, \va_2 \text{ fresh} \and
%     \subtyctrL{\AEnv}{\UEnvD,\va_1}{\plug\dctx{\va_1}}{\ty'_1}{\CSet_1} \and
%     \subtyctrL{\AEnv}{\UEnvD,\va_2}{\plug\dctx{\va_2}}{\ty'_2}{\CSet_2} \\ 
%     \CSet_1 = \CSet'_1 \mcup_{i=1}^n \ctrset{\ctrsub{\va_1}{\tyub_1^i}} 
%         \and \va_1 \notin \CSet'_1 \and
%     \CSet_2 = \CSet'_2 \mcup_{j=1}^m \ctrset{\ctrsub{\va_2}{\tyub_2^j}} 
%         \and \va_2 \notin \CSet'_2  \\
%     \CSet'\ \ =\ \ \mcup_{i=1,j=1}^{i=n,j=m}
%         \ctrset{\ctrsub{\va}{\tyunion{\tyub_1^i}{\tyub_2^j}}}
%         \text{ if } n\geq1,m\geq1
%     \text{\ or\ }
%     \mcup_{i=1}^n \ctrset{\ctrsub{\va_1}{u_1^i}}\text{ if } m=0
%     \text{\ or\ }
%     \mcup_{j=1}^m \ctrset{\ctrsub{\va_2}{u_2^j}}\text{ if } n=0
% }
% { \subtyctrLdfltenv{\plug\dctx{\va}}{\tyunion{\ty'_1}{\ty'_2}}
%     {\CSet'_1 \cup \CSet'_2 \cup \CSet'} }
% $}


\begin{figure}
\footnotesize
\makebox[\textwidth][c]{
\begin{minipage}{\ruleswidth}  
\begin{mathpar}
    \fbox{\subtysigdflt{\tysig}{\tysig}}
    \\

    \inferrule[\RSS{Top}]
    { }
    { \subtysigdflt{\tysig}{\tyany} }

    \inferrule[\RSS{Bot}]
    { }
    { \subtysigdflt{\plug\dctxsig\tybot}{\tysig'} }
    \\

    \inferrule[\RSS{VarLeft}]
    { \varbound{\vx}{\tylb}{\tyub} \in \AEnv \and
        \subtysigdflt{\plug\dctxsig{\tyub}}{\tysig'} }
    { \subtysigdflt{\plug\dctxsig\vx}{\tysig'} }

    \inferrule[\RSS{UnionLeft}]
    { \subtysigdflt{\plug\dctxsig{\tysig_1}}{\tysig'} \and 
        \subtysigdflt{\plug\dctxsig{\tysig_2}}{\tysig'}}
    { \subtysigdflt{\plug\dctxsig{\tyunion{\tysig_1}{\tysig_2}}}{\tysig'} }
    \\

    \inferrule[\RSS{InvLeft}]
    { \vx \text{ fresh} \and 
        \subtysig{\AEnv, \varbound{\vx}{\tylb}{\tyub}}{\UEnv}
        {\plug\dctxsig{\tyinv\iname{\ldots,\vx,\ldots}}}{\tysig'} }
    { \subtysigdflt{\plug\dctxsig{
        \tyinv\iname{\ldots,\rexvarbound{\tylb}{\tyub},\ldots}}}{\tysig'} }

    \inferrule[\RSS{ExistLeft}]
    { \subtysig{\AEnv, \varbound{\vx}{\tylb}{\tyub}}{\UEnv}{\plug\dctxsig\tysig}{\tysig'} }
    { \subtysigdflt{\plug\dctxsig{\tyexist{\vx}{\tylb}{\tyub}{\tysig}}}{\tysig'} }


    \inferrule[\RSS{UnionRight}]
    { \exists i.\ \subtysigdflt{\tysig}{\plug\dctxsig{\tysig'_i}} }
    { \subtysigdflt{\tysig}{\plug\dctxsig{\tyunion{\tysig'_1}{\tysig'_2}}} }

    \inferrule[\RSS{ExistRight}]
    { \subtysig{\AEnv}{\UEnv, \varbound{\va}{\tylb}{\tyub}}{\tysig}{\plug\dctxsig{\tysig'}} }
    { \subtysigdflt{\tysig}{\plug\dctxsig{\tyexist{\va}{\tylb}{\tyub}{\tysig'}}} }
    \\

    \inferrule[\RSS{Types}]
    { \subtyctrR{\AEnv}{\dom\UEnv}{\ty}{\ty'}{\CSet} \and
        \solvectrdflt = \substvars }
    { \subtysigdflt{\ty}{\ty'} }
\end{mathpar}
\end{minipage}}
\caption{Subtyping of type signatures
    %Note: \dctx stands for \dctxtysig.
}\label{fig:subtyping-tysigs}
\end{figure}


\begin{figure}
\footnotesize
\centering
\begin{minipage}{.6\linewidth}
\begin{algorithm}[H]
    \SetKwProg{SolveCtrs}{Solve}{}{}

    \SolveCtrs{$(\AEnv;\,\EmptyEnv;\,\CSet)$}{
        \KwRet{$\emptysubst$}
    }
    \BlankLine
    \SolveCtrs{$(\AEnv;\,\UEnv,\varbound{\va}{\tylb}{\tyub};\,\CSet)$}{
        $\CSet_{\va} \gets 
            \ctrset{\ctrsub{\tylb'}{\va} \,|\, \ctrsub{\tylb'}{\va} \in \CSet} 
            \cup 
            \ctrset{\ctrsub{\va}{\tyub'} \,|\, \ctrsub{\va}{\tyub'} \in \CSet} $ \;
        $\CSet' \gets \CSet \setminus \CSet_{\va}$ \;
        $\UEnvD \gets \dom{\UEnv}$\;
        \lForEach{$\ctrsub{\tylb_i}{\va}, \ctrsub{\va}{\tyub_j} \in \CSet_{\va}$}{%
            \subtydflt{\tylb_i}{\tyub_j}
        }
        \lForEach{$\ctrsub{\tylb_i}{\va} \in \CSet_{\va}$}{%
            \subtyctrRdfltenv{\tylb_i}{\tyub}{\CSet_{\tylb_i}}
        }
        \lForEach{$\ctrsub{\va}{\tyub_j} \in \CSet_{\va}$}{%
            \subtyctrLdfltenv{\tylb}{\tyub_j}{\CSet_{\tyub_j}}
        }
        $\substvars \gets \solvectr{\AEnv}{\UEnv}{
            \CSet' \mcup_i \CSet_{\tylb_i} \mcup_j \CSet_{\tyub_j}
        }$\;
        \KwRet{$\subst\substvars{\substel{\va}{
            \substvars(\tylb) \mcup_i \tylb_i
        }}$}
    }
    %$\substvars \gets \emptysubst$
\end{algorithm}  
\end{minipage}
\caption{Constraints resolution algorithm \solvectrdflt}\label{fig:ctr-solve}      
\end{figure}


%% Validity
%% *********************************************************

\begin{figure}
\footnotesize
\begin{mathpar}
    \fbox{\tyvlddflt{\ty}}
    \\

    \inferrule*[]
    { }
    { \tyvlddflt{\tyany} }

    \inferrule*[]
    { }
    { \tyvlddflt{\tybot} }

    \inferrule*[]
    { \varbound{\vany}{\tylb}{\tyub} \in \AEnv }
    { \tyvlddflt{\vany} }
    \\

    \inferrule*[]
    { \tyvlddflt{\ty_1} \and \tyvlddflt{\ty_2} }
    { \tyvlddflt{\typair{\ty_1}{\ty_2}} }

    \inferrule*[]
    { \iname \text{ has arity } n \and
      \forall i \in 1..n.\ \tyvlddflt{\rexvar_i} }
    { \tyvlddflt{\tyinv\iname{\rexvar_1,\ldots,\rexvar_n}} }

    \inferrule*[]
    { \tyvlddflt{\ty_1} \and \tyvlddflt{\ty_2} }
    { \tyvlddflt{\tyunion{\ty_1}{\ty_2}} }
    %
    \\
    \fbox{\tyvlddflt{\rexvar}}
    \\

    \inferrule*[]
    { \tyvlddflt{\tylb} \and \tyvlddflt{\tyub} \and 
        \subtydflt{\tylb}{\tyub} }
    { \tyvlddflt{\rexvarbound{\tylb}{\tyub}} }
    %
    \\
    \fbox{\tyvlddflt{\dctx}}
    \\

    \inferrule*[]
    { }
    { \tyvlddflt\square }

    \inferrule*[]
    { \tyvlddflt{\dctx} \and \tyvlddflt{\ty} }
    { \tyvlddflt{\typair{\dctx}{\ty}} }

    \inferrule*[]
    { \tyvlddflt{\ty} \and \tyvlddflt{\dctx} }
    { \tyvlddflt{\typair{\ty}{\dctx}} }
    %
    \\
    \fbox{\tyvlddflt{\tysig}}
    \\

    \inferrule*[]
    { }
    { \tyvlddflt{\tyany} }

    \ldots

    \inferrule*[]
    { \tyvlddflt{\tysig_1} \and \tyvlddflt{\tysig_2} }
    { \tyvlddflt{\tyunion{\tysig_1}{\tysig_2}} }

    \inferrule*[]
    { \tyvlddflt{\tylb} \and \tyvlddflt{\tyub} \and
        \subtydflt{\tylb}{\tyub} \and
        \tyvld{\AEnv, \varbound{\vany}{\tylb}{\tyub}}{\tysig} }
    { \tyvlddflt{\tyexist{\vany}{\tylb}{\tyub}{\tysig}} }
    %
    \\
    \fbox{\tyunfvlddflt{\ty}}
    \\

    \inferrule*[]
    { \varbound{\vx}{\tylb}{\tyub} \in \AEnv }
    { \tyunfvlddflt{\vx} }

    \inferrule*[]
    { \va \in \UEnvD }
    { \tyunfvlddflt{\va} }

    \inferrule*[]
    { }
    { \tyunfvlddflt{\tyany} }

    \ldots

    \inferrule*[]
    { \tyunfvlddflt{\ty_1} \and \tyunfvlddflt{\ty_2} }
    { \tyunfvlddflt{\tyunion{\ty_1}{\ty_2}} }
    %
    \\
    \fbox{\tyvld{}{\AEnv}}
    \\

    \inferrule*[]
    { }
    { \tyvld{}{\EmptyEnv} }

    \inferrule*[]
    { \tyvld{}{\AEnv} \and 
        \tyvlddflt{\tylb} \and \tyvlddflt{\tyub} \and
        \subtydflt{\tylb}{\tyub} }
    { \tyvld{}{\AEnv, \varbound{\vany}{\tylb}{\tyub}} }
\end{mathpar}
\caption{Validity of types and type signatures}\label{fig:ty-tysig-validity}
\end{figure}
% \inferrule*[]
% { }
% { \tyvlddflt{\tybot} }

% \inferrule*[]
% { \varbound{\vany}{\tylb}{\tyub} \in \AEnv }
% { \tyvlddflt{\vany} }

% \inferrule*[]
% { \iname \text{ has arity } n \and
%   \forall i \in 1..n.\ \tyvlddflt{\rexvar_i} }
% { \tyvlddflt{\tyinv\iname{\rexvar_1,\ldots,\rexvar_n}} }

% \inferrule*[]
% { \tyvlddflt{\tysig_1} \and \tyvlddflt{\tysig_2} }
% { \tyvlddflt{\typair{\tysig_1}{\tysig_2}} }

\begin{figure}
\footnotesize
\begin{mathpar}
    \fbox{\vldinenvdflt{\substvars}}
    \\

    \inferrule*[]
    { \forall \varbound{\vany}{\tylb}{\tyub} \in \AEnv. \and
        \tyvld{\AEnv'}{\substvars(\vany)} \and
        \subty{\AEnv'}{\substvars(\tylb)}{\substvars(\vany)} \and 
        \subty{\AEnv'}{\substvars(\vany)}{\substvars(\tyub)} }
    { \vldinenvdflt{\substvars} }
    
    \\
    \fbox{\vldinenv{\AEnv}{\CSet}{\substvars}}
    \\

    \inferrule*[]
    { \forall \ctrsub{\ty}{\va} \in \CSet.\ \ 
        \subtydflt{\ty}{\substvars(\va)} \and 
      \forall \ctrsub{\va}{\ty'} \in \CSet.\ \  
        \subtydflt{\substvars(\va)}{\ty'} }
    { \vldinenv{\AEnv}{\CSet}{\substvars} }
\end{mathpar}
\caption{Validity of substitution with respect to environment and constrains}%
\label{fig:substuvars-validity}
\end{figure}

Substitution is defined in the usual manner.

\textbf{The subtyping algorithm} is given by subtyping rules in
Figures~\ref{fig:subtyping-base,fig:subtyping-constrained,fig:subtyping-tysigs}
along with the constraints resolution algorithm in \figref{fig:ctr-solve}.
The rules are not syntax-directed because there may be multiple rules
applicable to a pair of types or type signatures,
but they are \emph{analytic}~\cite{bib:martin-lof:analytic-synthetic:1994}:
there is a finite number of applicable rules, and the premises of each rule
are comprised of the subcomponents of its conclusion.

\paragraph*{Overview of proofs.}
Subtyping is checked only for valid types/signatures.
The validity check ensures the absence of recursive constraints on type
variables as well as consistency of variable bounds.
For subtyping of types, we assume \tyvlddflt{\ty} and \tyvlddflt{\ty'}.
For subtyping of type signatures, we assume
\tyvlddflt{\tysig} and \tyvld{\concat{\AEnv}{\UEnv}}{\tysig'}.

\textbf{Termination.} The measure of a type includes the measure of its bounds.
In all subtyping judgments, every recursive call is made with a smaller measure.
For constraints resolution, every recursive call is made with a smaller \UEnv.

\textbf{Notes.}
Exists of unions is equivalent to the union of exists (in Julia subtyping),
so treating parts of the union on the right independently is correct
in subtyping of signatures.

Getting the upper bound of a universal variable might be needed at either the
signature or the type level, that's why the rules have some duplicates.
The key is to get rid of all top-level existentials.

Because the same semantic type can be represented with multiple syntactic types,
e.g. with an explicit existential vs restricted existential, it may be necessary
to open restricted existentials as top-level variables, but it is not always
necessary. That's why the rule for signature subtyping of invariant
constructor in a distibutivity context doesn't require all type arguments to be
specified as variables.

\section{Proofs}

\subsection{Decidability of Subtyping}\label{subsec:dec-proof}
%% ======================================================================

\begin{figure}
\footnotesize
\[
\begin{array}{lcl}
    \hline
    \multicolumn{3}{c}{\tymsrdflt{\ty}} \\ 
    \hline 
    \tymsrdflt{\tyany} &::=& 1\\
    \tymsrdflt{\tybot} &::=& 1\\
    \tymsr{\,\EmptyEnv\,}{\vany} &::=& 1\\
    \tymsr{\AEnv, \varbound{\vany}{\tylb}{\tyub}}{\vany} &::=& 
        1 + \tymsrdflt{\tylb} + \tymsrdflt{\tyub}\\
    \tymsr{\AEnv, \varbound{\vany'}{\tylb}{\tyub}}{\vany} &::=& 
        \tymsrdflt{\vany} \\
    \tymsrdflt{\typair{\ty_1}{\ty_2}} &::=& 
        1 + \tymsrdflt{\ty_1} + \tymsrdflt{\ty_2}\\
    \tymsrdflt{\tyinv\iname{\rexvar_1,\ldots,\rexvar_n}} &::=&
        1 + \tymsrdflt{\rexvar_1} + \ldots + \tymsrdflt{\rexvar_n}\\
    \tymsrdflt{\tyunion{\ty_1}{\ty_2}} &::=& 
        1 + \tymsrdflt{\ty_1} + \tymsrdflt{\ty_2}\\
    \\
    \hline
    \multicolumn{3}{c}{\tymsrdflt{\rexvar}} \\ 
    \hline 
    \tymsrdflt{\rexvarbound{\ty}{\ty}} &::=& \tymsrdflt{\ty}\\
    \tymsrdflt{\rexvarbound{\tylb}{\tyub}} &::=& 
        2\times(1 + \tymsrdflt{\tylb} + \tymsrdflt{\tyub})\\
    \\
    \hline
    \multicolumn{3}{c}{\tymsrdflt{\dctx}} \\ 
    \hline 
    \tymsrdflt{\square} &::=& 0\\
    \tymsrdflt{\typair{\dctx}{\ty}} &::=& 
        1 + \tymsrdflt{\dctx} + \tymsrdflt{\ty}\\
    \tymsrdflt{\typair{\ty}{\dctx}} &::=& 
        1 + \tymsrdflt{\ty} + \tymsrdflt{\dctx}\\
    \\
    \hline
    \multicolumn{3}{c}{\tymsrdflt{\tysig}} \\ 
    \hline 
    \tymsrdflt{\tyany} &::=& 1\\
    \multicolumn{3}{l}{\ldots} \\
    \tymsrdflt{\tyexist{\vany}{\tylb}{\tyub}{\tysig}} &::=& 
        1 + \tymsrdflt{\tylb} + \tymsrdflt{\tyub} + 
        \tymsr{\AEnv,\varbound{\vany}{\tylb}{\tyub}}{\tysig}\\
    \\
    \hline
    \multicolumn{3}{c}{\tymsrdflt{\dctxsig}} \\ 
    \hline 
    \tymsrdflt{\square} &::=& 0\\
    \tymsrdflt{\typair{\dctxsig}{\tysig}} &::=& 
        1 + \tymsrdflt{\dctxsig} + \tymsrdflt{\tysig}\\
    \tymsrdflt{\typair{\tysig}{\dctxsig}} &::=& 
        1 + \tymsrdflt{\tysig} + \tymsrdflt{\dctxsig}\\
\end{array}
\]
\caption{Measure of types and type signatures}\label{fig:ty-measure}
\end{figure}

To show the decidability of the subtyping algorithm,
we will use the measure $\msrop$ of types and type signatures,
as defined in \figref{fig:ty-measure}.
The measure function is defined recursively and is
similar to the syntactic size,
except for the treatment of type variables: for variables from~\AEnv,
the measure of a variable includes the measures of its bounds.
The definition of the measure for restricted existential variables \rexvar
reflects the fact that \tyinv\iname{\rexvar_1,\ldots,\rexvar_n} represents both
invariant constructors and restricted existential types.
When $\rexvar_i$ represents a single type $\ty_i$,
i.e. $\rexvar_i = \rexvarbound{\ty_i}{\ty_i}$,
its measure is simply the measure of the type $\ty_i$ in
\tyinv\iname{\ldots,\ty_i,\ldots}.
Otherwise, $\rexvar_i = \rexvarbound{\tylb_i}{\tyub_i}$ represents
an existential type with a single occurrence of the bound variable,
\tyexist{\vany_i}{\tylb_i}{\tyub_i}{\tyinv\iname{\ldots,\vany_i,\ldots}},
and measures accordingly, comprising both the binding and its single occurrence.

Note that the measure function itself always terminates and evaluates to a
positive integer. This is the case because for every recursive call
\tymsr{\AEnv'}{\ty'} of \tymsrdflt{\ty}, 
the combined syntactic size of the arguments $\size{\ty'} + \size{\AEnv'}$
is strictly smaller than $\size{\ty} + \size{\AEnv}$.
The same is true for recursive calls
\tymsr{\AEnv'}{\tysig'} of \tymsrdflt{\tysig}.
The syntactic size is defined in~\figref{fig:ty-size}.

\begin{figure}
\footnotesize
\[
\begin{array}{lcl}
    \hline
    \multicolumn{3}{c}{\size{\ty}} \\ 
    \hline 
    \size{\tyany} &::=& 1\\
    \size{\tybot} &::=& 1\\
    \size{\vany}  &::=& 1\\
    \size{\typair{\ty_1}{\ty_2}}  &::=& 1 + \size{\ty_1} + \size{\ty_2}\\
    \size{\tyinv\iname{\rexvar_1,\ldots,\rexvar_n}} &::=&
        1 + \size{\rexvar_1} + \ldots + \size{\rexvar_n}\\
    \size{\tyunion{\ty_1}{\ty_2}} &::=& 1 + \size{\ty_1} + \size{\ty_2}\\
    \\
    \hline
    \multicolumn{3}{c}{\size{\rexvar}} \\ 
    \hline 
    \size{\rexvarbound{\tylb}{\tyub}} &::=& \size{\tylb} + \size{\tyub}\\
    \\
    \hline
    \multicolumn{3}{c}{\size{\dctx}} \\ 
    \hline 
    \size{\square} &::=& 0\\
    \size{\typair{\dctx}{\ty}} &::=& 
        1 + \size{\dctx} + \size{\ty}\\
    \size{\typair{\ty}{\dctx}} &::=& 
        1 + \size{\ty} + \size{\dctx}\\
    \\
    \hline
    \multicolumn{3}{c}{\size{\tysig}} \\ 
    \hline 
    \size{\tyany} &::=& 1\\
    \multicolumn{3}{l}{\ldots} \\
    \size{\tyexist{\vany}{\tylb}{\tyub}{\tysig}} &::=& 
        1 + \size{\tylb} + \size{\tyub} + \size{\tysig}\\
    \\
    \hline
    \multicolumn{3}{c}{\size{\AEnv}} \\ 
    \hline 
    \size{\EmptyEnv} &::=& 0 \\
    \size{\AEnv, \varbound{\vany}{\tylb}{\tyub}} &::=& 
        \size{\AEnv} + \size{\tylb} + \size{\tyub}\\
    \\
    \hline
    \multicolumn{3}{c}{\size{\dctxsig}} \\ 
    \hline 
    \size{\square} &::=& 0\\
    \size{\typair{\dctxsig}{\tysig}} &::=& 
        1 + \size{\dctxsig} + \size{\tysig}\\
    \size{\typair{\tysig}{\dctxsig}} &::=& 
        1 + \size{\tysig} + \size{\dctxsig}\\
\end{array}
\]
\caption{Syntactic size}\label{fig:ty-size}
\end{figure}

In what follows, we will implicitly use the following facts about
distributivity contexts (proofs are straightforward by induction):
\begin{itemize}
    \item $\plug{\dctx}{\ty} = \ty';$
    \item $\plug{\dctx}{\dctx'} = \dctx'';$
    \item $\plug{\dctx}{\plug{\dctx'}{\ty}} = 
        \plug{(\plug{\dctx}{\dctx'})}{\ty};$
    \item $\subtydflt{\dctx}{\dctx'} \land \subtydflt{\ty}{\ty'}
        \implies \subtydflt{\plug{\dctx}{\ty}}{\plug{\dctx'}{\ty'}};$
    \item $\size{\plug{\dctx}{\ty}} = \size{\dctx} + \size{\ty};$
    \item $\tymsrdflt{\plug{\dctx}{\ty}} = \tymsrdflt{\dctx} + \tymsrdflt{\ty};$
    \item $\tymsrdflt{\plug{\dctxsig}{\tysig}} = 
        \tymsrdflt{\dctxsig} + \tymsrdflt{\tysig};$
    \item $\substvars(\plug\dctx\ty) = 
        \plug{\substvars(\dctx)}{\substvars(\ty)}; $    
    \item \TODO{more as needed}
\end{itemize}


\begin{theorem}{Termination of\ \ \subtydflt{\ty}{\ty'}.}%
\label{thm:subty-terminates}
    The subtyping algorithm built from the rules of subtyping of types
    \subtydflt{\ty}{\ty'} terminates.
\end{theorem}
\begin{proof}
    It follows from the fact that for each subtyping rule, 
    the measure of each premise \subtydflt{\ty_p}{\ty'_p}
    is strictly smaller than the measure 
    of the conclusion \subtydflt{\ty}{\ty'}, i.e.
    \[\tymsrdflt{\ty_p} + \tymsrdflt{\ty'_p} \quad<\quad 
    \tymsrdflt{\ty} + \tymsrdflt{\ty'}.\]

    For example, in the case \RST{VarLeft},
    \[\tymsrdflt{\plug\dctx\tyub} + \tymsrdflt{\ty'} < 
    \tymsrdflt{\plug\dctx\vany} + \tymsrdflt{\ty'}\]
    because \[\tymsrdflt{\tyub} < \tymsrdflt{\vany} = 
        1 + \tymsrdflt{\tylb} + \tymsrdflt{\tyub}.\]
\end{proof}


\begin{lemma}{Termination of join ($\sqcup_{\AEnv}$) and meet ($\sqcap_{\AEnv}$).}%
\label{lem:meet-terminates}
    $\forall \AEnv, \ty, \ty'$ both
    \tyjoindflt{\ty}{\ty'} and \tymeetdflt{\ty}{\ty'} terminate.
\end{lemma}
\begin{proof}
    \tyjoindflt{\ty}{\ty'} terminates because subtyping of types terminates
    by~\thmref{thm:subty-terminates}.

    \tymeetdflt{\ty}{\ty'} terminates because subtyping of types terminates
    and for each recursive call \tymeetdflt{\ty_r}{\ty'_r} of 
    \tymeetdflt{\ty}{\ty'}, the measure $\tymsrdflt{\ty_r} + \tymsrdflt{\ty'_r}$ 
    is strictly smaller than $\tymsrdflt{\ty} + \tymsrdflt{\ty'}$.
\end{proof}

\begin{theorem}{Termination of\ \ \subtyctrdflt{\ty}{\ty'}.}%
\label{thm:subtyctr-terminates}
    The subtyping algorithm built from the rules of
    constrained subtyping of types
    \subtyctrdflt{\ty}{\ty'} terminates.
\end{theorem}
\begin{proof}
    Similarly to the previous theorem, the measure decreases, i.e.
    \[\tymsrdflt{\ty_p} + \tymsrdflt{\ty'_p} \quad<\quad 
    \tymsrdflt{\ty} + \tymsrdflt{\ty'}\]
    for each premise \subtyctrdflt{\ty_p}{\ty'_p}
    of the conclusion \subtyctrdflt{\ty}{\ty'}.
    
    The only interesting case is \RSC{UVar-UnionRight}.
    By the variable names convention~\ref{def:var-names}, $\va \notin \AEnv$.
    Therefore, $\tymsrdflt{\va} = \tymsrdflt{\va_1} = \tymsrdflt{\va_2}$
    and the measure of the left-hand side type is the same in both premises
    and the conclusion,
    while the measure of the right-hand side type in both premises 
    is strictly smaller than in conclusion.
    Furthermore, $\msqcap_{i=1}^n \tyub_1^i$ and $\msqcap_{j=1}^m \tyub_2^j$
    terminate by~\lemref{lem:meet-terminates}.
\end{proof}

\begin{theorem}{Termination of\ \solvectrdflt.}%
\label{thm:solvectr-terminates}
    The constraints resolution algorithm \solvectrdflt terminates.
\end{theorem}
\begin{proof}
    It follows from the fact that:
    \begin{enumerate}
        \item subtyping algorithms \subtydflt{\ty}{\ty'} and 
            \subtyctrdflt{\ty}{\ty'} used to check the consistency of 
            constraints terminate;
        \item the argument \UEnv of the only recursive call to $\solvectrop$
            is strictly smaller than that of the original call.
    \end{enumerate} 
\end{proof}


\begin{figure}
\footnotesize
\[
\begin{array}{lcl}
    \hline
    \multicolumn{3}{c}{\occdflt{\ty}} \\ 
    \hline 
    \occdflt{\tyany} &::=& \false\\
    \occdflt{\tybot} &::=& \false\\
    \occdflt{\vany}  &::=& \true\\
    \occdflt{\vany'}  &::=& \false\\
    \occdflt{\typair{\ty_1}{\ty_2}}  &::=& \occdflt{\ty_1} \lor \occdflt{\ty_2}\\
    \occdflt{\tyinv\iname{\rexvar_1,\ldots,\rexvar_n}} &::=&
        \occdflt{\rexvar_1} \lor \ldots \lor \occdflt{\rexvar_n}\\
    \occdflt{\tyunion{\ty_1}{\ty_2}} &::=& \occdflt{\ty_1} \lor \occdflt{\ty_2}\\
    \\
    \hline
    \multicolumn{3}{c}{\occdflt{\rexvar}} \\ 
    \hline 
    \occdflt{\rexvarbound{\tylb}{\tyub}} &::=& \occdflt{\tylb} \lor \occdflt{\tyub}\\
    \\
    \hline
    \multicolumn{3}{c}{\occdflt{\tysig}} \\ 
    \hline 
    \occdflt{\tyany} &::=& \false\\
    \multicolumn{3}{l}{\ldots} \\
    \occdflt{\tyexist{\vany}{\tylb}{\tyub}{\tysig}} &::=& \true\\
    \occdflt{\tyexist{\vany'}{\tylb}{\tyub}{\tysig}} &::=& 
        \occdflt{\tylb} \lor \occdflt{\tyub} \lor \occdflt{\tysig}\\
    \\
    \hline
    \multicolumn{3}{c}{\occdflt{\AEnv}} \\ 
    \hline 
    \occdflt{\EmptyEnv} &::=& \false \\
    \occdflt{\AEnv, \varbound{\vany}{\tylb}{\tyub}} &::=& \true\\
    \occdflt{\AEnv, \varbound{\vany'}{\tylb}{\tyub}} &::=& 
        \occdflt{\AEnv} \lor \occdflt{\tylb} \lor \occdflt{\tyub}\\
\end{array}
\]
\caption{Occurrence of a variable}\label{fig:var-occ}
\end{figure}

\begin{lemma}{Context weakening in $\msrop$.}%
\label{lem:msr-weakening}
    The measure of a type signature does not change if the environment
    is extended (in any position) with a variable that occurs neither
    in the signature nor in the environment, i.e.,
    $\forall \tysig, \AEnv, \AEnv'. 
    \forall \varbound{\vany}{\tylb}{\tyub} \text{ s.t. } 
    \lnot \occdflt{\tysig} \land 
    \lnot \occdflt{\AEnv} \land \lnot \occdflt{\AEnv'}.$
    \[\tymsr{\concat{\AEnv}{\AEnv'}}{\tysig} = 
        \tymsr{\concat{\AEnv,\varbound{\vany}{\tylb}{\tyub}}{\AEnv'}}{\tysig},\]
    where \concat{\AEnv}{\AEnv''} denotes the concatenation of lists,
    and $\occop$ is defined in~\figref{fig:var-occ}.
\end{lemma}
\begin{proof}
    By strong induction on $n = \size{\AEnv} + \size{\AEnv'} + \size{\tysig}$.

    Case $n = 0$ is not possible, as the minimal size of a type signature is 1.
    
    In the inductive step for $n$, the induction hypothesis (IH) states that
    $\forall n'<n. \forall \tysig', \AEnv'', \AEnv'''  \text{ s.t. }
    n' = \size{\AEnv''} + \size{\AEnv'''} + \size{\tysig'}.
    \forall \varbound{\vany}{\tylb}{\tyub} \text{ s.t. } 
    \lnot \occdflt{\tysig'} \land 
    \lnot \occdflt{\AEnv''} \land \lnot \occdflt{\AEnv'''}.$
    \[\tymsr{\concat{\AEnv''}{\AEnv'''}}{\tysig'} = 
    \tymsr{\concat{\AEnv'',\varbound{\vany}{\tylb}{\tyub}}{\AEnv'''}}{\tysig'}.\]
    
    Case analysis on \tysig. Base cases \tyany and \tybot are straightforward.
    Cases $\times$, \tyinv\iname{\ldots}, and $\cup$ are also straightforward
    using the induction hypothesis for components of \tysig.
    The remaining cases are:
    \begin{itemize}
        \item Case $\vany'$. Case analysis on $\AEnv'$.
            \begin{itemize}
                \item Case \EmptyEnv. Because $\lnot \occdflt{\vany'},$ we know
                    $\vany \neq \vany'$. Thus,
                    $\tymsr{\AEnv, \varbound{\vany}{\tylb'}{\tyub'}}{\vany'} =
                    \tymsr{\AEnv}{\vany'}$ by definition of $\msrop$.
                \item Case $\AEnv', \varbound{\vany'}{\tylb'}{\tyub'}$.
                    By definition,
                    \[\tymsr{\concat{\AEnv}{\AEnv', \varbound{\vany'}{\tylb'}{\tyub'}}}{\vany'} =
                    1 + \tymsr{\concat{\AEnv}{\AEnv'}}{\tylb'} + 
                    \tymsr{\concat{\AEnv}{\AEnv'}}{\tyub'}.\]
                    Since $\size{\AEnv} + \size{\AEnv'} + \size{\tylb'}\ <\ 
                    \size{\AEnv} + \size{\AEnv', \varbound{\vany'}{\tylb'}{\tyub'}} + \size{\vany'} =
                    \size{\AEnv} + \size{\AEnv'} + \size{\tylb'} + \size{\tyub'} + 1$,
                    the IH applies with $\AEnv'' = \AEnv, \AEnv''' = \AEnv', 
                    \tysig' = \tylb'$, which gives 
                    $\tymsr{\concat{\AEnv}{\AEnv'}}{\tylb'} = 
                    \tymsr{\concat{\AEnv, \varbound{\vany}{\tylb}{\tyub}}{\AEnv'}}{\tylb'}$,
                    and similarly for $\tyub'$. Thus,
                    \[ \tymsr{\concat{\AEnv}{\AEnv', \varbound{\vany'}{\tylb'}{\tyub'}}}{\vany'} =
                    \tymsr{\concat{\AEnv, \varbound{\vany}{\tylb}{\tyub}}{\AEnv', \varbound{\vany'}{\tylb'}{\tyub'}}}{\vany'}. \]
            \end{itemize}
        \item Case \tyexist{\vany'}{\tylb'}{\tyub'}{\tysig}.
            By definition,
            \[\tymsr{\concat{\AEnv}{\AEnv'}}{\tyexist{\vany'}{\tylb'}{\tyub'}{\tysig}} =
            1 + \tymsr{\concat{\AEnv}{\AEnv'}}{\tylb'} + \msrop(\ldots\tyub') +
            \tymsr{\concat{\AEnv}{\AEnv', \varbound{\vany'}{\tylb'}{\tyub'}}}{\tysig}.\]
            Similarly to the last subcase of the $\vany'$ case, the IH applies
            to $\tylb'$ and $\tyub'$.
            Furthermore, since $\size{\AEnv} + \size{\AEnv'} + 
            \size{\tylb'} + \size{\tyub'} + \size{\tysig} <
            \size{\AEnv} + \size{\AEnv'} + 1 + \size{\tylb'} + \size{\tyub'}
            + \size{\tysig},$
            the IH applies to \tysig with $\AEnv'' = \AEnv, 
            \AEnv''' = (\AEnv', \varbound{\vany'}{\tylb'}{\tyub'}), 
            \tysig' = \tysig$.
            All pieces combined, 
            \[\tymsr{\concat{\AEnv}{\AEnv'}}{\tyexist{\vany'}{\tylb'}{\tyub'}{\tysig}} =
            \tymsr{\concat{\AEnv, \varbound{\vany}{\tylb}{\tyub}}{\AEnv'}}{\tyexist{\vany'}{\tylb'}{\tyub'}{\tysig}}.\]
            % and
            % \[\tymsr{\concat{\AEnv, \varbound{\vany}{\tylb}{\tyub}}{\AEnv'}}{\tyexist{\vany'}{\tylb'}{\tyub'}{\tysig}} =
            % 1 + \tymsr{\concat{\AEnv, \varbound{\vany}{\tylb}{\tyub}}{\AEnv'}}{\tylb'} + 
            % \tymsr{\concat{\AEnv, \varbound{\vany}{\tylb}{\tyub}}{\AEnv'}}{\tyub'} +
            % \tymsr{\concat{\AEnv, \varbound{\vany}{\tylb}{\tyub}}{\AEnv', \varbound{\vany'}{\tylb'}{\tyub'}}}{\tysig}.\]
    \end{itemize}
\end{proof}

\begin{theorem}{Termination of\ \ \subtysigdflt{\tysig}{\tysig'}.}%
\label{thm:subtysig-terminates}
    The subtyping algorithm built from the rules of
    subtyping of type signatures
    \subtysigdflt{\tysig}{\tysig'} terminates.
\end{theorem}
\begin{proof}
    It follows from the fact that for each subtyping rule, 
    the measure of each premise \subtysigdflt{\tysig_p}{\tysig'_p}
    is strictly smaller than the measure 
    of the conclusion \subtysigdflt{\tysig}{\tysig'}, i.e.
    \[\tymsr{\AEnv'}{\tysig_p} + \tymsr{\concat{\AEnv'}{\UEnv'}}{\tysig'_p} \quad<\quad 
    \tymsr{\AEnv}{\tysig} + \tymsr{\concat{\AEnv}{\UEnv}}{\tysig'}.\]

    Most of the cases are similar to the cases of~\thmref{thm:subty-terminates}
    on the termination of \subtydflt{\ty}{\ty'}.
    The remaining cases are:
    \begin{itemize}
        \item \RSS{InvLeft}. Since \vx is a fresh variable,
            by~\lemref{lem:msr-weakening} (weakening),
            $\tymsr{\AEnv, \varbound{\vx}{\tylb}{\tyub}}
                {\plug\dctxsig{\tyinv\iname{\ldots}}} = 
            \tymsrdflt{\plug\dctxsig{\tyinv\iname{\ldots}}},$ and also
            $\tymsr{\concat{\AEnv, \varbound{\vx}{\tylb}{\tyub}}{\UEnv}}{\tysig'}
            = \tymsr{\concat{\AEnv}{\UEnv}}{\tysig'}.$

            By the definition of $\msrop$,
            \[\tymsr{\AEnv, \varbound{\vx}{\tylb}{\tyub}}{\vx} =
            1 + \tymsrdflt{\tylb} + \tymsrdflt{\tyub} <
            2\times(1 + \tymsrdflt{\tylb} + \tymsrdflt{\tyub}) =
            \tymsrdflt{\rexvarbound{\tylb}{\tyub}},\]
            which concludes the case.
        \item \RSS{ExistLeft}. By the variables names
            convention~\defref{def:var-names}, \vx is different from variables
            in \AEnv, \UEnv, as well as bound variables of \dctxsig, \tysig,
            and $\tysig'$. Therefore, by~\lemref{lem:msr-weakening} (weakening),
            $\tymsr{\AEnv, \varbound{\vx}{\tylb}{\tyub}}{\dctxsig} = 
            \tymsrdflt{\dctxsig}$, and similarly for $\tysig'$.
            By the definition of $\msrop$,
            \[\tymsrdflt{\tyexist{\vx}{\tylb}{\tyub}{\tysig}} = 1 +
                \tymsrdflt{\tylb} + \tymsrdflt{\tyub} + 
                \tymsr{\AEnv, \varbound{\vx}{\tylb}{\tyub}}{\tysig},\]
            which is strictly larger than
            \tymsr{\AEnv, \varbound{\vx}{\tylb}{\tyub}}{\tysig} in the premise,
            which concludes the case.
        \item \RSS{ExistRight}. Similarly to \RSS{ExistLeft}.
        \item \RSS{Types}. The first premise,
            \subtyctrR{\AEnv}{\dom\UEnv}{\ty}{\ty'}{\CSet},
            terminates by \thmref{thm:subtyctr-terminates}.
            Since the constraints resolution \solvectrdflt terminates
            by \thmref{thm:solvectr-terminates}, the entire step also terminates.
    \end{itemize}
\end{proof}


\subsection{Properties of Subtyping of Types}%
\label{subsec:props-subty-proof}
%% ======================================================================

%We will use an implicit assumption that all types are valid
%in the given context.

\begin{theorem}{Reflexivity of subtyping of types.}\label{thm:sub-ty-refl}
    $
        \forall \ty, \AEnv, \text{ s.t. } \tyvlddflt{\ty}.\quad
        \subtydflt{\ty}{\ty}.
    $
\end{theorem}
\begin{proof}
    By induction on the structure of \ty.
    \begin{itemize}
        \item Case \tyany by \RST{Top}.
        \item Case \tybot by \RST{Bot}.
        \item Case \vany by \RST{VarRefl}.
        \item Case \typair{\ty_1}{\ty_2} by IH and \RST{Tuple}.
        \item Case \tyinv\iname{\rexvar_1,\ldots,\rexvar_n} by IH on
            $\tylb_i, \tyub_i$, and \RST{Inv}.
        \item Case \tyunion{\ty_1}{\ty_2} by IH, \RST{UnionRight},
            and \RST{UnionLeft}.
    \end{itemize}
\end{proof}

\begin{lemma}{Subtyping of \tybot implies arbitrary subtyping.}\label{lem:sub-of-bot}
    \[
    \forall \ty, \dctx_{\tybot}, \AEnv.\quad 
    \subtydflt{\ty}{\plug{\dctx_{\tybot}}\tybot}
    \quad\implies\quad
    (\forall \ty', \dctx'.\quad \subtydflt{\plug{\dctx'}{\ty}}{\ty'}).
    \]
\end{lemma}
\begin{proof}
    By induction on the derivation of 
    \subtydflt{\ty}{\plug{\dctx_{\tybot}}\tybot}.
    \begin{itemize}
        \item Case \RST{Bot}
            \subtydflt{\plug\dctx\tybot}{\plug{\dctx_{\tybot}}{\tybot}}
            where $\ty = \plug\dctx\tybot$.

            The case concludes by \RST{Bot}:
            \subtydflt{\plug{\dctx'}{\plug\dctx\tybot}}{\ty'}. 
        \item Case \RST{VarLeft}
            \subtydflt{\plug\dctx\vany}{\plug{\dctx_{\tybot}}{\tybot}}.

            By inversion, \subtydflt{\plug\dctx\tyub}{\plug{\dctx_{\tybot}}{\tybot}}.
            By IH, \subtydflt{\plug{\dctx'}{\plug\dctx\tyub}}{\ty'}.
            Thus, the case concludes by \RST{VarLeft}: 
            \subtydflt{\plug{\dctx'}{\plug\dctx\vany}}{\ty'}.
        \item Case \RST{Tuple}, subcase where
            $\dctx_{\tybot} = \typair{\dctx'_{\tybot}}{\ty'_2}$
            ($\dctx_{\tybot} = \square$ is not possible, and
            $\dctx_{\tybot} = \typair{\ty_1}{\dctx'_{\tybot}}$
            is proved analogously),
            $\ty = \typair{\ty_1}{\ty_2}$:
            \subtydflt{\typair{\ty_1}{\ty_2}}
            {\typair{\plug{\dctx'_{\tybot}}{\tybot}}{\ty'_2}}.

            By inversion, \subtydflt{\ty_1}{\plug{\dctx'_{\tybot}}{\tybot}}.
            By IH, \subtydflt{\plug{\dctx'^h}{\ty_1}}{\ty'} for all $\dctx'^h$,
            so we can take it to be \plug{\dctx'}{\typair{\square}{\ty_2}}.
            Thus, the case concludes by IH: 
            \subtydflt{\plug{\dctx'}{\typair{\ty_1}{\ty_2}}}{\ty'}.
        \item Case \RST{UnionLeft}
            \subtydflt{\plug\dctx{\tyunion{\ty_1}{\ty_2}}}{\plug{\dctx_{\tybot}}{\tybot}}
            where $\ty = \tyunion{\ty_1}{\ty_2}$.
            By inversion, 
            \subtydflt{\plug\dctx{\ty_1}}{\plug{\dctx_{\tybot}}{\tybot}} and
            \subtydflt{\plug\dctx{\ty_2}}{\plug{\dctx_{\tybot}}{\tybot}}.
            By IH, \subtydflt{\plug{\dctx'}{\plug\dctx{\ty_1}}}{\ty'} and
            \subtydflt{\plug{\dctx'}{\plug\dctx{\ty_2}}}{\ty'}.
            Thus, the case concludes by \RST{UnionLeft}: 
            \subtydflt{\plug{\dctx'}{\plug\dctx{\tyunion{\ty_1}{\ty_2}}}}{\ty'}.
    \end{itemize}
    The remaining cases 
    (\RST{Top}, \RST{VarRefl}, \RST{VarRight}, \RST{Inv}, \RST{UnionRight}) 
    are not possible.
\end{proof}

\begin{lemma}{Subtyping of inner union on the right.}%
\label{lem:sub-inner-union-right}
    $\forall \ty, \dctx', \ty'_1, \ty'_2, \AEnv, \text{ s.t. }
    \tyvlddflt{\ty, \dctx', \ty'_1, \ty'_2}.$
    \[
        \begin{array}{ccc}
        \subtydflt{\ty}{\plug{\dctx'}{\tyunion{\ty'_1}{\ty'_2}}}\\
        \quad\implies\quad\\
        (\forall \dctx_1, \dctx_2, \text{ s.t. }
        \tyvlddflt{\dctx_1, \dctx_2} \land
        \subtydflt{\dctx_1}{\dctx_2}.\quad
        \subtydflt
            {\plug{\dctx_1}{\ty}}
            {\tyunion
                {\plug{\dctx_2}{\plug{\dctx'}{\ty'_1}}}
                {\plug{\dctx_2}{\plug{\dctx'}{\ty'_2}}}
            }).
        \end{array}
    \]
\end{lemma}
\begin{proof}
    By induction on the derivation of
    \subtydflt{\ty}{\plug{\dctx'}{\tyunion{\ty'_1}{\ty'_2}}}.
    \begin{itemize}
        \item Case \RST{Bot} by \RST{Bot}.
        \item Case \RST{VarLeft} by inversion, IH, and \RST{VarLeft}.
        \item Case \RST{Tuple}, subcase where
            $\dctx' = \typair{\dctx''}{\ty'}$:
            \subtydflt{\typair{\ty_1}{\ty_2}}
                {\typair{\plug{\dctx''}{\tyunion{\ty'_1}{\ty'_2}}}{\ty'}}.
            By inversion,
            \subtydflt{\ty_1}{\plug{\dctx''}{\tyunion{\ty'_1}{\ty'_2}}} and
            \subtydflt{\ty_2}{\ty'}. 
            
            By IH applied to 
            \subtydflt{\ty_1}{\plug{\dctx''}{\tyunion{\ty'_1}{\ty'_2}}},
            \subtydflt{\plug{\dctx^h_1}{\ty_1}}
                {\tyunion
                    {\plug{\dctx^h_2}{\plug{\dctx''}{\ty'_1}}}
                    {\plug{\dctx^h_2}{\plug{\dctx''}{\ty'_2}}}
                }
            for all $\dctx^h_1, \dctx^h_2$ s.t. $\subtydflt{\dctx^h_1}{\dctx^h_2}$.
            Thus, we can take them to be \plug{\dctx_2}{\typair{\square}{\ty_2}} 
            and \plug{\dctx_2}{\typair{\square}{\ty'}}, 
            respectively, which concludes the case with
            \subtydflt{\plug{\dctx_1}{\typair{\ty_1}{\ty_2}}}
                {\tyunion
                    {\plug{\dctx_2}{\typair{\plug{\dctx''}{\ty'_1}}{\ty_2}}}
                    {\plug{\dctx_2}{\typair{\plug{\dctx''}{\ty'_2}}{\ty'}}}
                }
        \item Case \RST{UnionLeft} by inversion, IH, and \RST{UnionLeft}.
        \item Case \RST{UnionRight}, subcase $i = 1$ where $\dctx' = \square$:
            \subtydflt{\ty}{\tyunion{\ty'_1}{\ty'_2}}.
            By inversion, \subtydflt\ty{\ty'_1}.
            By assumption, \subtydflt{\dctx_1}{\dctx_2}, and thus,
            \subtydflt{\plug{\dctx_1}{\ty}}{\plug{\dctx_2}{\ty'_1}}.
            The case concludes by \RST{UnionRight} with $i=1$:
            \subtydflt{\plug{\dctx_1}{\ty}}
                {\tyunion{\plug{\dctx_2}{\ty'_1}}{\plug{\dctx_2}{\ty'_2}}}.
    \end{itemize}
    The remaining cases 
    (\RST{Top}, \RST{VarRefl}, \RST{VarRight}, \RST{Inv}) 
    are not possible.
\end{proof}

\begin{lemma}{Adding inner union on the right.}%
\label{lem:add-inner-union-right}
    $\forall \ty, \dctx', \ty', \AEnv, \text{ s.t. }
    \tyvlddflt{\ty, \dctx', \ty'}.$
    \[
        \subtydflt{\ty}{\plug{\dctx'}{\ty'}}
        \quad\implies\quad
        (\forall \ty''.\ \subtydflt{\ty}{\plug{\dctx'}{\tyunion{\ty'}{\ty''}}}).
    \]
\end{lemma}
\begin{proof}
    By induction on the derivation of
    \subtydflt{\ty}{\plug{\dctx'}{\ty'}}.
    \begin{itemize}
        \item Case \RST{Top} where $\dctx'=\square$. By assumption
            \subtydflt{\ty}{\tyany} and \RST{UnionRight} with $i=1$,
            \subtydflt{\ty}{\tyunion{\tyany}{\ty''}}.
        \item Case \RST{Bot} by \RST{Bot}.
        \item Case \RST{VarRefl} where $\dctx'=\square$
            by assumption and \RST{UnionRight} with $i=1$.
        \item Case \RST{VarLeft} by inversion, IH, and \RST{VarLeft}.
        \item Case \RST{VarRight} where $\dctx'=\square$
            by assumption and \RST{UnionRight} with $i=1$.
        \item Case \RST{Tuple}. 
            Subcase $\dctx'$ by assumption and \RST{UnionRight} with $i=1$.
            The other two subcases by inversion, IH, and \RST{Tuple}.
        \item Case \RST{Inv} where $\dctx'=\square$
            by assumption and \RST{UnionRight} with $i=1$.
        \item Case \RST{UnionLeft} by inversion, IH, and \RST{UnionLeft}.
        \item Case \RST{UnionRight} where $\dctx'=\square$
            by assumption and \RST{UnionRight} with $i=1$.
    \end{itemize}
\end{proof}

\begin{lemma}{Subtyping of union on the right.}%
\label{lem:sub-union-right}
    $\forall \ty, \dctx', \ty'_1, \ty'_2, \AEnv, \text{ s.t. }
    \tyvlddflt{\ty, \dctx', \ty'_1, \ty'_2}.$
    \[
        \subtydflt{\ty}{\tyunion{\plug{\dctx'}{\ty'_1}}{\plug{\dctx'}{\ty'_2}}}
        \quad\implies\quad
        \subtydflt{\ty}{\plug{\dctx'}{\tyunion{\ty'_1}{\ty'_2}}}.
    \]
\end{lemma}
\begin{proof}
    By induction on the derivation of
    \subtydflt{\ty}{\tyunion{\plug{\dctx'}{\ty'_1}}{\plug{\dctx'}{\ty'_2}}}.
    Four cases are possible: \RST{Bot}, \RST{VarLeft}, \RST{UnionLeft}, and
    \RST{UnionRight}. The first three are analogous to the cases of
    \lemref{lem:sub-inner-union-right} (subtyping inner union on the right): 
    inversion, IH, constructor.
    The remaining case is \RST{UnionRight}, subcase $i=1$.
    By inversion, \subtydflt{\ty}{\plug{\dctx'}{\ty'_1}}.
    Thus, the case concludes by \lemref{lem:add-inner-union-right} (adding inner
    union on the right):
    \subtydflt{\ty}{\plug{\dctx'}{\tyunion{\ty'_1}{\ty'_2}}}.
    % bot by bot
    % VarLeft by inversion, IH, VarLeft
    % UnionLeft by inversion, IH, UnionLeft
\end{proof}

\begin{lemma}{Context weakening in subtyping of types.}%
\label{lem:subty-weakening}
    $\forall \AEnv, \ty, \ty'.\ \forall \AEnv' \text{ s.t. }$\\
    $\dom{\AEnv'} \cap \dom{\AEnv} = \varnothing \land 
    (\forall \vany \in \dom{\AEnv'}. 
        \lnot \occ{\vany}{\ty} \land \lnot \occ{\vany}{\ty}),$
    \[ \subtydflt{\ty}{\ty'} \quad\implies\quad 
    \subty{\concat{\AEnv}{\AEnv'}}{\ty}{\ty'}. \]
\end{lemma}
\begin{proof}
    Straightforward induction on the derivation of \subtydflt{\ty}{\ty'}.
\end{proof}


\begin{theorem}{Transitivity of subtyping of types.}\label{thm:sub-ty-trans}
    $\forall \ty_1, \ty_2, \ty_3, \AEnv, \text{ s.t. }
    \tyvlddflt{\ty_1, \ty_2, \ty_3} \land \tyvld{\,}{\AEnv}.$
    \[
        \subtydflt{\ty_1}{\ty_2} \land \subtydflt{\ty_2}{\ty_3}
        \quad\implies\quad
        \subtydflt{\ty_1}{\ty_3}.
    \]
\end{theorem}
\begin{proof}
    By strong induction on
    $n = \tymsrdflt{\ty_1} + 2\times\tymsrdflt{\ty_2} + \tymsrdflt{\ty_3}$.
    Cases $n = 1..3$ are not possible as the minimal measure of a type is 1.
    In the inductive step for $n$, the induction hypothesis (IH) states that
    $\forall n'<n. \forall \ty'_1, \ty'_2, \ty'_3 \text{ s.t. }
    n' = \tymsrdflt{\ty'_1} + 2\times\tymsrdflt{\ty'_2} + \tymsrdflt{\ty'_3},$
    it holds that
    \[
        \subtydflt{\ty'_1}{\ty'_2} \land \subtydflt{\ty'_2}{\ty'_3}
        \quad\implies\quad
        \subtydflt{\ty'_1}{\ty'_3}.
    \]
    Case analysis on \subtydflt{\ty_2}{\ty_3} (the most interesting cases are
    highlighted in bold).
    \begin{itemize}
        \item Case \RST{Top} \subtydflt{\ty_2}{\tyany} where $\ty_3 = \tyany$
            concludes by \RST{Top}: \subtydflt{\ty_1}{\tyany}.
        \item Case \RSS{Bot} \subtydflt{\plug\dctx\tybot}{\ty_3}
            where $\ty_2=\plug\dctx\tybot$.
            The case concludes by \lemref{lem:sub-of-bot}
            (subtyping of \tybot) applied
            to the assumption \subtydflt{\ty_1}{\plug\dctx\tybot}
            with $\dctx' = \square, \ty' = \ty_3$:
            \subtydflt{\ty_1}{\ty_3}.
        \item Case \RST{VarRefl} \subtydflt{\vany}{\vany}.
            Since $\ty_2=\ty_3=\vany$, the case concludes by the assumption
            \subtydflt{\ty_1}{\vany}.
        \item \textbf{Case \RST{VarLeft}} \subtydflt{\plug\dctx\vany}{\ty_3} 
            where $\ty_2 = \plug\dctx\vany$.
            By inversion, $\varbound{\vany}{\tylb}{\tyub} \in \AEnv$ and
            \subtydflt{\plug\dctx{\tyub}}{\ty_3}.
            By assumption, \subtydflt{\ty_1}{\plug\dctx\vany}.
            We will need the following auxiliary fact.

            \begin{lemma}\label{lem:sub-var-right-sub-ub}
                \subtydflt{\ty_1}{\plug\dctx\tyub}.
            \end{lemma}
            \begin{proof}
                By induction on \subtydflt{\ty_1}{\plug\dctx\vany}.
                \begin{itemize}
                    \item Case \RST{Bot} by \RST{Bot}.
                    \item Case \RST{VarRefl} \subtydflt{\vany}{\vany}.
                        By \thmref{thm:sub-ty-refl} (reflexivity),
                        \subtydflt{\tyub}{\tyub}.
                        Thus, by \RST{VarLeft}, \subtydflt{\vany}{\tyub}.
                    \item Case \RST{VarLeft} 
                        \subtydflt{\plug{\dctx'}{\vany'}}{\plug\dctx\vany}.
                        By inversion, $\varbound{\vany'}{\tylb'}{\tyub'} \in \AEnv$
                        and \subtydflt{\plug{\dctx'}{\tyub'}}{\plug\dctx\vany}.
                        By IH,
                        \subtydflt{\plug{\dctx'}{\tyub'}}{\plug\dctx\tyub}.
                        Thus, by \RST{VarLeft},
                        \subtydflt{\plug{\dctx'}{\vany'}}{\plug\dctx\tyub}.
                    \item \textbf{Case \RST{VarRight}} \subtydflt{\ty_1}{\vany}.
                        By inversion, $\varbound{\vany}{\tylb}{\tyub} \in \AEnv$
                        and \subtydflt{\ty_1}{\tylb}.
                        By inversion of the assumptions \tyvlddflt{\vany} and
                        \tyvld{\,}{\AEnv} and by~\lemref{lem:subty-weakening}
                        (context weakening), we have \subtydflt{\tylb}{\tyub}.

                        Since $\ty_2 = \vany$ and $\tymsrdflt{\vany} = 
                        1 + \tymsrdflt{\tylb} + \tymsrdflt{\tyub}$, we have
                        $\tymsrdflt{\ty_1} + 2\times\tymsrdflt{\tylb} + 
                        \tymsrdflt{\tyub} < \tymsrdflt{\ty_1} + 
                        2\times\tymsrdflt{\vany} + \tymsrdflt{\ty_3}$. Thus,
                        IH for \emph{transitivity} is applicable to 
                        \subtydflt{\ty_1}{\tylb} and \subtydflt{\tylb}{\tyub},
                        which concludes the case with \subtydflt{\ty_1}{\tyub}.
                    \item Case \RST{Tuple}, subcase
                        $\dctx = \typair{\dctx'}{\ty_{22}}$
                        ($\dctx = \square$ is not possible, and
                        $\dctx = \typair{\ty_{21}}{\dctx'}$
                        is proved analogously), i.e.
                        \subtydflt
                            {\typair{\ty_{11}}{\ty_{12}}}
                            {\typair{\plug{\dctx'}\vany}{\ty_{22}}}.
                        By inversion, \subtydflt{\ty_{11}}{\plug{\dctx'}\vany}
                        and \subtydflt{\ty_{12}}{\ty_{22}}.
                        By IH, \subtydflt{\ty_{11}}{\plug{\dctx'}\tyub}.
                        Thus, by \RST{Tuple},
                        \subtydflt
                            {\typair{\ty_{11}}{\ty_{12}}}
                            {\typair{\plug{\dctx'}\tyub}{\ty_{22}}}.
                    \item Case \RST{UnionLeft} is proved analogously 
                        to \RST{Tuple}: by inversion, IH, and \RST{UnionLeft}.
                \end{itemize}
                The remaining cases
                (\RST{Top}, \RST{Tuple}, \RST{Inv}, \RST{UnionRight}) 
                are not possible.
            \end{proof}

            Be \lemref{lem:sub-var-right-sub-ub} above,
            \subtydflt{\ty_1}{\plug\dctx\tyub}.
            Since $\ty_2 = \plug\dctx\vany$ and 
            $\tymsrdflt{\tyub} < \tymsrdflt{\vany}$,
            IH is applicable to \subtydflt{\ty_1}{\plug\dctx\tyub} and 
            \subtydflt{\plug\dctx\tyub}{\ty_3}, which concludes the case with
            \subtydflt{\ty_1}{\ty_3}.
        \item Case \RST{VarRight} \subtydflt{\ty_2}{\vany}.
            By inversion, $\varbound{\vany}{\tylb}{\tyub} \in \AEnv$ and
            \subtydflt{\ty_2}{\tylb}.
            Since $\tymsrdflt{\tylb} < \tymsrdflt{\vany}$, by IH,
            \subtydflt{\ty_1}{\tylb}.
            Thus, by \RST{VarRight}, \subtydflt{\ty_1}{\vany}.
        \item Case \RST{Tuple} 
            \subtydflt{\typair{\ty_{21}}{\ty_{22}}}{\typair{\ty_{31}}{\ty_{32}}}
            where $\ty_i = \typair{\ty_{i1}}{\ty_{i2}}$.
            Case analysis on \subtydflt{\ty_1}{\typair{\ty_{21}}{\ty_{22}}}.
            \begin{itemize}
                \item Case \RST{Bot} by \RST{Bot}.
                \item Case \RST{VarLeft} by inversion, IH on 
                    \subtydflt{\plug\dctx\tyub}{\ty_2} and
                    \subtydflt{\ty_2}{\ty_3}, and \RST{VarLeft} on
                    \subtydflt{\plug\dctx\tyub}{\ty_3}.
                \item Case \RST{Tuple} by inversion, IH on
                    \subtydflt{\ty_{1j}}{\ty_{2j}} and 
                    \subtydflt{\ty_{2j}}{\ty_{3j}}, and \RST{Tuple}.
                \item Case \RST{UnionLeft} 
                    \subtydflt{\plug\dctx{\tyunion{\ty_{11}}{\ty_{12}}}}{\ty_2}
                    by inversion, IH on
                    \subtydflt{\plug\dctx{\ty_{1j}}}{\ty_2} and
                    \subtydflt{\ty_2}{\ty_3}, and \RST{UnionLeft}.
            \end{itemize}
            The remaining cases
            (\RST{Top}, \RST{VarRefl}, \RST{VarRight}, \RST{Inv}, \RST{UnionRight}) 
            are not possible.
        \item Case \RST{Inv} is proved similarly to \RST{Tuple}, with
            possible cases of \subtydflt{\ty_1}{\ty_2} being
            \RST{Bot}, \RST{VarLeft}, \RST{Inv}, and \RST{UnionLeft}.
        \item \textbf{Case \RST{UnionLeft}} 
            \subtydflt{\plug\dctx{\tyunion{\ty_{21}}{\ty_{22}}}}{\ty_3}.
            By \lemref{lem:sub-inner-union-right} applied to
            \subtydflt{\ty_1}{\plug\dctx{\tyunion{\ty_{21}}{\ty_{22}}}}
            with $\dctx_1=\square,\dctx_2=\square$,
            \subtydflt{\ty_1}{\tyunion{\plug\dctx{\ty_{21}}}{\plug\dctx{\ty_{22}}}}.
            Case analysis on the latter.
            \begin{itemize}
                \item Case \RST{Bot} by \RST{Bot}.
                \item Case \RST{VarLeft} where $\ty_1 = \plug{\dctx'}\vany$.
                    By inversion, $\varbound{\vany}{\tylb}{\tyub} \in \AEnv$ and
                    \subtydflt{\plug{\dctx'}\tyub}
                    {\tyunion{\plug\dctx{\ty_{21}}}{\plug\dctx{\ty_{22}}}}.
                    By \lemref{lem:sub-union-right} applied to the latter,
                    \subtydflt{\plug{\dctx'}\tyub}
                    {\plug\dctx{\tyunion{\ty_{21}}{\ty_{22}}}}.

                    Since $\tymsrdflt{\tyub} < \tymsrdflt{\vany}$, by IH,
                    \subtydflt{\plug{\dctx'}\tyub}{\ty_3}.
                    Thus, the case concludes by \RST{VarLeft}:
                    \subtydflt{\plug{\dctx'}\vany}{\ty_3}.
                \item Case \RST{UnionLeft} similarly to \RST{VarLeft}.
                \item Case \RST{UnionRight}, subcase $i=1$.
                    By inversion, \subtydflt{\ty_1}{\plug\dctx{\ty_{21}}}.
                    By inversion of the outer case assumption
                    \subtydflt{\plug\dctx{\tyunion{\ty_{21}}{\ty_{22}}}}{\ty_3},
                    \subtydflt{\plug\dctx{\ty_{21}}}{\ty_3}.
                    Since $\tymsrdflt{\plug\dctx{\ty_{21}}} < 
                    \tymsrdflt{\plug\dctx{\tyunion{\ty_{21}}{\ty_{22}}}}$,
                    by IH, \subtydflt{\ty_1}{\ty_3}.
            \end{itemize}
            The remaining cases
            (\RST{Top}, \RST{VarRefl}, \RST{VarRight}, \RST{Tuple}, \RST{Inv}) 
            are not possible.
        \item Case \RST{UnionRight} 
            \subtydflt{\ty_2}{\tyunion{\ty_{31}}{\ty_{32}}} where 
            $\ty_3 = \tyunion{\ty_{31}}{\ty_{32}}$, subcase $i=1$.
            By inversion, \subtydflt{\ty_2}{\ty_{31}}. Since
            $\tymsrdflt{\ty_{31}} < \tymsrdflt{\tyunion{\ty_{31}}{\ty_{32}}}$,
            by IH, \subtydflt{\ty_1}{\ty_{31}}.
            Thus, the case concludes by \RST{UnionRight}.
    \end{itemize}
\end{proof}

%% This fact we use implicitly
% \begin{lemma}{Preservation of subtyping by distributivity context.}%
%     \[
%         \forall \AEnv, \dctx, \ty, \ty'.\quad
%         \subtydflt{\ty}{\ty'} \quad\implies\quad
%         \subtydflt{\plug\dctx\ty}{\plug\dctx{\ty'}}.
%      \]
% \end{lemma}
% \begin{proof}
%     Straightforward by induction on the structure of \dctx,
%     using the assumption \subtydflt{\ty}{\ty'} in the base case
%     and \thmref{thm:sub-ty-refl} (reflexivity) in the inductive cases.
% \end{proof}

\begin{theorem}{Soundness of subtyping of types with respect to subsitution.}%
\label{thm:subty-sound-subst}
    $\forall \AEnv, \ty, \ty' \text{ s.t. } \tyvlddflt{\ty, \ty'}.\ 
     \forall \AEnv', \substvars \text{ s.t. } \vldinenvdflt{\substvars}.$
     \[ 
        \subtydflt{\ty}{\ty'} \quad\implies\quad
        \subty{\AEnv'}{\substvars(\ty)}{\substvars(\ty')}.
     \]
\end{theorem}
\begin{proof}
    By induction on the derivation of \subtydflt{\ty}{\ty'}.

    Cases \RST{Top} and \RST{Bot} are straightforward by \RST{Top} and
    \RST{Bot}, respectively. 

    Case \RST{VarRefl} by \thmref{thm:sub-ty-refl} (reflexivity):
    \subty{\AEnv'}{\substvars(\vany)}{\substvars(\vany)}.

    Cases \RST{Tuple}, \RST{Inv}, \RST{UnionLeft}, and \RST{UnionRight} are
    straightforward using inversion, the induction hypothesis, and constructor.
    For example, consider the case \RST{Tuple}
    \subtydflt{\typair{\ty_1}{\ty_2}}{\typair{\ty'_1}{\ty'_2}}.
    By inversion, \subtydflt{\ty_i}{\ty'_i}.
    By IH, \subty{\AEnv'}{\substvars(\ty_i)}{\substvars(\ty'_i)}.
    The case concludes by \RST{Tuple} with 
    \[\subty{\AEnv'}{\typair{\substvars(\ty_1)}{\substvars(\ty_2)}}
    {\typair{\substvars(\ty'_1)}{\substvars(\ty'_2)}}\]
    and the fact that $\substvars(\typair{\ty_1}{\ty_2}) = 
    \typair{\substvars(\ty_1)}{\substvars(\ty_2)}.$

    The remaining cases \RST{VarLeft} and \RST{VarRight} are similar.
    For example, consider \RST{VarLeft} \subtydflt{\plug{\dctx}\vany}{\ty'}.
    By inversion, $\varbound{\vany}{\tylb}{\tyub} \in \AEnv$ and
    \subtydflt{\plug\dctx{\tyub}}{\ty'}.
    By IH, \subty{\AEnv'}{\substvars(\plug\dctx{\tyub})}{\substvars(\ty')}.
    By inversion of the assumption \vldinenvdflt{\substvars}, we know
    \subty{\AEnv'}{\substvars(\vany)}{\substvars(\tyub)}.
    Since $\substvars(\plug\dctx{\tyub}) = 
    \plug{\substvars(\dctx)}{\substvars(\tyub)}$ and
    \subty{\AEnv'}{\substvars(\dctx)}{\substvars(\dctx)} by reflexivity,
    we have \subty{\AEnv'}{\plug{\substvars(\dctx)}{\substvars(\vany)}}
    {\plug{\substvars(\dctx)}{\substvars(\tyub)}}.
    The case concludes by \thmref{thm:sub-ty-trans} (transitivity) with the
    middle type $\substvars(\plug\dctx{\tyub})$:
    \[\subty{\AEnv'}{\substvars(\plug\dctx{\vany})}{\substvars(\ty')}.\]
\end{proof}


\subsection{Properties of Join and Meet}%
\label{subsec:props-join-meet-proof}
%% ======================================================================

\begin{theorem}{Soundness of join.}%
\label{thm:join-sound}
    Join produces a valid upper bound, i.e., 
    $\forall \AEnv, \ty, \ty'$ s.t. \tyvlddflt{\ty, \ty'}.
    \[
        \tyvlddflt{\tyjoindflt{\ty}{\ty'}}
        \quad\text{and}\quad
        \subtydflt{\ty}{(\tyjoindflt{\ty}{\ty'})}
        \quad\text{and}\quad
        \subtydflt{\ty'}{(\tyjoindflt{\ty}{\ty'})}.
    \]
    Furthermore, \tyjoindflt{\ty}{\ty'} is the least upper bound
    of \ty and $\ty'$, i.e.,
    $\forall \tyub$ s.t. \tyvlddflt{\tyub},
    \subtydflt{\ty}{\tyub} and \subtydflt{\ty'}{\tyub}.
    \[
        \subtydflt{(\tyjoindflt{\ty}{\ty'})}{\tyub}.
    \]
\end{theorem}
\begin{proof}
    Straightforward by the definition of $\sqcup_{\AEnv}$,
    using \thmref{thm:sub-ty-refl} (reflexivity), 
    \RST{UnionLeft}, and \RST{UnionRight}.
\end{proof}

\begin{lemma}{Symmetry of join.}
    $\forall \AEnv, \ty, \ty'$ s.t. \tyvlddflt{\ty, \ty'}.
    \[
        \subtydflt{(\tyjoindflt{\ty}{\ty'})}{(\tyjoindflt{\ty'}{\ty})}
        \quad\text{and}\quad
        \subtydflt{(\tyjoindflt{\ty'}{\ty})}{(\tyjoindflt{\ty}{\ty'})}.
    \]
\end{lemma}
\begin{proof}
    Straightforward by the definition of $\sqcup_{\AEnv}$,
    using \thmref{thm:sub-ty-refl} (reflexivity), 
    \RST{UnionLeft}, and \RST{UnionRight}.
\end{proof}

\begin{theorem}{Soundness of meet.}%
\label{thm:meet-sound}
    Meet produces a valid lower bound, i.e., 
    $\forall \AEnv, \ty, \ty'$ s.t. \tyvlddflt{\ty, \ty'}.
    \[
        \tyvlddflt{\tymeetdflt{\ty}{\ty'}}
        \quad\text{and}\quad
        \subtydflt{(\tymeetdflt{\ty}{\ty'})}{\ty}
        \quad\text{and}\quad
        \subtydflt{(\tymeetdflt{\ty}{\ty'})}{\ty'}.
    \]
\end{theorem}
\begin{proof}
    Straightforward by strong induction on
    $\tymsrdflt{\ty} + \tymsrdflt{\ty'}$.

    For example, the case where \subtydflt{\ty}{\ty'} follows from
    the assumptions and reflexivity \subtydflt{\ty}{\ty},
    and the catch-all case $\tymeetdflt{\ty}{\ty'} = \tybot$
    follows from \RST{Bot}.

    In the case \tymeetdflt{(\tyunion{\ty_1}{\ty_2})}{\ty'},
    we know by IH that \tymeetdflt{\ty_1}{\ty'} and \tymeetdflt{\ty_2}{\ty'}
    are valid lower bounds of $\ty_1,\ty'$ and $\ty_2,\ty'$, respectively.
    Thus, the case concludes by \RST{UnionLeft}.

    In the case \tymeetdflt{\vany}{\ty'}, we have \subtydflt{\tylb}{\vany}
    by \thmref{thm:sub-ty-refl} (reflexivity) and \RST{VarRight}.
    Then, by IH, \subtydflt{\tymeetdflt{\tylb}{\ty'}}{\tylb}.
    The case concludes by \thmref{thm:sub-ty-trans} (transitivity):
    \subtydflt{\tymeetdflt{\tylb}{\ty'}}{\vany}.
    
    The remaining cases for tuples and invariant constructors
    use the induction hypotheses and constructors \RST{Tuple} and \RST{Inv},
    respectively, as well as \thmref{thm:join-sound} (soundness of join)
    in the invariant case.
\end{proof}

\begin{lemma}{Symmetry of meet.}
    $\forall \AEnv, \ty, \ty'$ s.t. \tyvlddflt{\ty, \ty'}.
    \[
        \subtydflt{(\tymeetdflt{\ty}{\ty'})}{(\tymeetdflt{\ty'}{\ty})}
        \quad\text{and}\quad
        \subtydflt{(\tymeetdflt{\ty'}{\ty})}{(\tymeetdflt{\ty}{\ty'})}.
    \]
\end{lemma}
\begin{proof}
    Straightforward by strong induction on
    $\tymsrdflt{\ty} + \tymsrdflt{\ty'}$,
    using the definition of $\sqcap_{\AEnv}$,
    \thmref{thm:sub-ty-refl} (reflexivity),
    and \RST{} constructors.
\end{proof}



\subsection{Properties of Constrained Subtyping of Types}%
\label{subsec:props-subtyctr-proof}
%% ======================================================================

\begin{lemma}{Constrained subtyping coincides with subtyping on
    unification-free types.}%
\label{lem:subtyctr-subty}
    $\forall \AEnv, \ty, \ty' \text{ s.t. } 
    \tyvld{}{\AEnv} \ \land\ \tyvlddflt{\ty} \ \land\ \tyvlddflt{\ty'}$ 
    the following holds:
    \begin{enumerate}
        \item $\forall \UEnvD.\ \subtyctrdflt{\ty}{\ty'} \quad\implies\quad
            \CSet = \EmptyCSet\ \land\ \subtydflt{\ty}{\ty'};$
        \item $\subtydflt{\ty}{\ty'} \quad\implies\quad 
            \forall \UEnvD.\ \subtyctrdfltenv{\ty}{\ty'}{\EmptyCSet}.$
    \end{enumerate}
\end{lemma}
\begin{proof}
    Straightforward by induction on the derivation of:
    \begin{enumerate}
        \item \subtyctrdflt{\ty}{\ty'} (more precisely, by mutual induction
            on \subtyctrLdflt{\ty}{\ty'} and \subtyctrRdflt{\ty}{\ty'});
        \item \subtydflt{\ty}{\ty'}.
    \end{enumerate}

    In the first case, the rules \RSC{UBot}, \RSC{UVarLeft}, \RSC{UVarRight},
    and \RSC{UVar-UnionRight} could not have been used to build the derivation
    because they refer a unification variable \va from \UEnvD,
    and both \ty and $\ty'$ are valid in \AEnv alone.
    All other rules of constrained subtyping have matching subtyping rules.

    In the second case, all subtyping rules have matching rules of constrained
    subtyping.
    The assumption \tyvld{}{\AEnv} and \lemref{lem:subty-weakening} (context
    weakening) allow for concluding 
    \tyvlddflt{\plug\dctx\tyub} and \tyvlddflt{\tylb} 
    to apply the induction hypothesis in the rules
    \RSC{VarLeft}/\RST{VarLeft} and \RSC{VarRight}/\RST{VarRight}.
\end{proof}

\begin{theorem}{Soundness of constrained subtyping.}%
\label{thm:subtyctr-sound}
    $\forall \AEnv, \UEnvD, \ty, \ty' \text{ s.t. } \tyvld{}{\AEnv},$ if
    \[\subtyctrRdflt{\ty}{\ty'} \ \land\ 
        \tyvlddflt{\ty} \ \land\  \tyunfvlddflt{\ty'}\] 
    or
    \[\subtyctrLdflt{\ty}{\ty'} \ \land\ 
        \tyunfvlddflt{\ty} \ \land\  \tyvlddflt{\ty'},\]
    then $\forall \substvars \text{ s.t. } 
        \dom{\substvars} \supseteq \UEnvD \ \land\ 
        \dom{\substvars} \cap \dom{\AEnv} = \varnothing \ \land\ 
        \vldinenv{\AEnv}{\CSet}{\substvars}.$
     \[ 
        \subtydflt{\substvars(\ty)}{\substvars(\ty')}.
     \]
\end{theorem}
\begin{proof}
    By strong induction on $\tymsrdflt{\ty} + \tymsrdflt{\ty'}.$
    \begin{itemize}
        \item Case \RSC{Top} \subtyctrdfltenv{\ty}{\tyany}{\EmptyCSet}.
            By definition, $\substvars(\tyany) = \tyany$.
            Thus, the case concludes by \RST{Top}:
            \subtydflt{\substvars(\ty)}{\tyany}.
        \item Case \RSC{Bot} by \RST{Bot}, similarly to \RSC{Top}.
        \item Case \RSC{UBot}
            \subtyctrLdfltenv{\plug\dctx\va}{\ty'}{\ctrsngl\va\tybot}.
            Since $\substvars(\plug\dctx\va) = 
                \plug{\substvars(\dctx)}{\substvars(\va)} = 
                \plug{\dctx'}{\substvars(\va)}$ for some $\dctx'$
            and by assumption, \subtydflt{\substvars(\va)}{\tybot},
            we have \subtydflt{\plug{\dctx'}{\substvars(\va)}}{\plug{\dctx'}\tybot}.
            By \RST{Bot}, \subtydflt{\plug{\dctx'}\tybot}{\substvars(\ty')}.
            Therefore, the case concludes by transitivity:
            \subtydflt{\plug{\dctx'}{\substvars(\va)}}{\substvars(\ty')}.
        \item Case \RSC{VarRefl} by \RST{VarRefl}, for $\substvars(\vx) = \vx.$
        \item Case \RSC{UVarLeft} 
            \subtyctrLdfltenv{\va}{\ty'}{\ctrsngl{\va}{\ty'}} by assumption
            \subtydflt{\substvars(\va)}{\ty'}, since $\substvars(\ty') = \ty'$
            due to \tyvlddflt{\ty'}.
        \item Case \RSC{UVarRight} similarly to \RSC{UVarLeft}.
        \item Case \RSC{VarLeft} (\RSC{VarRight}) by inversion, IH, and
            \RST{VarLeft} (\RST{VarRight}), for $\substvars(\vx) = \vx$
            and $\substvars(\tyub) = \tyub$ ($\substvars(\tylb) = \tylb$).
        \item Cases \RSC{Tuple}, \RSC{Inv}, \RSC{UnionLeft}, and \RSC{UnionRight}
            are all similar: by inversion, IH, and the corresponding subtyping
            constructor.
        \item \textbf{Case \RSC{UVar-UnionRight}} 
            \subtyctrLdfltenv{\plug\dctx{\va}}{\tyunion{\ty'_1}{\ty'_2}}
            {\CSet'_1 \cup \CSet'_2 \cup \CSet'}.
            Since \tyvlddflt{\tyunion{\ty'_1}{\ty'_2}}, we have
            $\substvars(\tyunion{\ty'_1}{\ty'_2}) = \tyunion{\ty'_1}{\ty'_2}.$
            By assumption,
            \vldinenv{\AEnv}{\CSet'_1 \cup \CSet'_2 \cup \CSet'}{\substvars}.
            By inversion, 
            \subtyctrL{\AEnv}{\UEnvD,\va_1}{\plug\dctx{\va_1}}{\ty'_1}{\CSet_1} 
            and
            \subtyctrL{\AEnv}{\UEnvD,\va_2}{\plug\dctx{\va_2}}{\ty'_2}{\CSet_2},
            where $\CSet_1 = \CSet'_1 \mcup_{i=1}^n \ctrset{\ctrsub{\va_1}{\tyub_1^i}}$
            and $\CSet_2 = \CSet'_2 \mcup_{j=1}^m \ctrset{\ctrsub{\va_2}{\tyub_2^j}}$.
            
            When $n=0$ or $m=0$, we can take $\substvars_1 = 
                \subst{\substvars}{\substel{\va_1}{\substvars(\va)}}$
            and $\substvars_2 = 
                \subst{\substvars}{\substel{\va_2}{\substvars(\va)}}$.
            In both subcases ($n=0$ and $m=0$), we have 
            \vldinenv{\AEnv}{\CSet_1}{\substvars_1} and
            \vldinenv{\AEnv}{\CSet_2}{\substvars_2}, because $\substvars(\va)$
            satisfies the constraints on $\va_1$ and $\va_2$.
            Therefore, by the induction hypotheses,
            \subtydflt{\substvars_1(\plug\dctx{\va_1})}{\tyunion{\ty'_1}{\ty'_2}}
            and
            \subtydflt{\substvars_2(\plug\dctx{\va_2})}{\tyunion{\ty'_1}{\ty'_2}}.
            Clearly, $\substvars(\plug\dctx{\va}) = 
                \substvars_1(\plug\dctx{\va_1}) = 
                \substvars_2(\plug\dctx{\va_2})$,
            and thus, the subcases conclude by \RST{UnionRight}.

            The most interesting case is when $n \geq 1, m \geq 1$.
            Let $\msqcap_{i=1}^n \tyub_1^i$ be denoted with $\tyub_1$
            and $\msqcap_{j=1}^m \tyub_2^j$ with $\tyub_2$.
            By~\thmref{thm:meet-sound} (soundness of meet), we have
            \subtydflt{\tyub_1}{\tyub_1^i} and \subtydflt{\tyub_2}{\tyub_2^j}
            for all $i, j$. Therefore, we can take 
            $\substvars_1 = \subst{\substvars}{\substel{\va_1}{\tyub_1}}$ and
            $\substvars_2 = \subst{\substvars}{\substel{\va_2}{\tyub_2}}$,
            and it holds that \vldinenv{\AEnv}{\CSet_1}{\substvars_1} and
            \vldinenv{\AEnv}{\CSet_2}{\substvars_2}.
            Therefore, the induction hypotheses are applicable, and we get
            \subtydflt{\plug{\substvars_1(\dctx)}{\tyub_1}}{\ty'_1} (H$_1$) and
            \subtydflt{\plug{\substvars_2(\dctx)}{\tyub_2}}{\ty'_2} (H$_2$).
            By assumption on \substvars, we have
            \subtydflt{\substvars(\va)}{\tyunion{\tyub_1}{\tyub_2}}.
            Therefore, \subtydflt{\plug{\substvars(\dctx)}{\substvars(\va)}}
                {\plug{\substvars(\dctx)}{\tyunion{\tyub_1}{\tyub_2}}}.
            Since $\va_1, \va_2$ were fresh, we know
            $\substvars(\dctx) = \substvars_1(\dctx) = \substvars_2(\dctx)$.
            Therefore, by \RST{UnionLeft} applied to H$_1$, H$_2$,
            \subtydflt{\plug{\substvars(\dctx)}{\tyunion{\tyub_1}{\tyub_2}}}
                {\tyunion{\ty'_1}{\ty'_2}}.
            Finally, the subcase $n \geq 1, m \geq 1$ concludes by transitivity:
            \subtydflt{\plug{\substvars(\dctx)}{\substvars(\va)}}{\tyunion{\ty'_1}{\ty'_2}}.
    \end{itemize}
\end{proof}

\begin{theorem}{Soundness of constrains resolution.}%
\label{thm:solvectr-sound}
    $\forall \AEnv, \UEnv$ s.t. $\tyvld{}{\AEnv, \UEnv}$.
    $\forall \CSet$ s.t. $\forall \ctrsub{\tylb}{\va} \in \CSet.
        \ \tyvlddflt{\tylb}\ \land\ \tyvld{\UEnv}{\va}$ and 
        $\forall \ctrsub{\va}{\tyub} \in \CSet.
        \ \tyvld{\UEnv}{\va} \ \land\ \tyvlddflt{\tyub}.$
    \[
        \solvectrdflt = \substvars
        \quad\implies\quad
        \vldinenv{\AEnv}{\UEnv}{\substvars}\ \ \land\ \ 
        \vldinenv{\AEnv}{\CSet}{\substvars}.
    \]
\end{theorem}
\begin{proof}
    By induction on \UEnv. The base case is trivial ($\CSet = \EmptyCSet$
    because $\UEnv = \EmptyEnv$).

    In the inductive step $\UEnv,\varbound{\va}{\tylb}{\tyub}$,
    the induction hypothesis applies to the call \solvectr{\AEnv}{\UEnv}{
        \CSet' \mcup_i \CSet_{\tylb_i} \mcup_j \CSet_{\tyub_j}
    }, which returns \substvars.
    Therefore, we know that \vldinenv{\AEnv}{\UEnv}{\substvars} and
    \vldinenv{\AEnv}{\CSet' \mcup_i \CSet_{\tylb_i} \mcup_j \CSet_{\tyub_j}}{\substvars}.

    Let $\substvars(\tylb) \mcup_i \tylb_i$ be denoted with $\ty_{\va}$
    and \subst\substvars{\substel{\va}{\ty_{\va}}} with $\substvars_{\va}.$
    %be denoted with $\substvars_{\va}.$
    By~\thmref{thm:subtyctr-sound} (soundness of constrained subtyping)
    applied to \subtyctrRdfltenv{\tylb_i}{\tyub}{\CSet_{\tylb_i}} with
    \vldinenv{\AEnv}{\CSet_{\tylb_i}}{\substvars} and
    \subtyctrLdfltenv{\tylb}{\tyub_j}{\CSet_{\tyub_j}} with
    \vldinenv{\AEnv}{\CSet_{\tyub_j}}{\substvars},
    we know 
    \[
        \subtydflt{\tylb_i}{\substvars(\tyub)}\ (\textrm{H}_{l_i})\ 
        \ \text{ and }\ 
        \subtydflt{\substvars(\tylb)}{\tyub_j}\ (\textrm{H}_{u_j}).
    \]
    % Since $\lnot\occ{\va}{\tylb}, \lnot\occ{\va}{\tyub}$, we have
    % $\substvars(\tylb) = \substvars_{\va}(\tylb)$ and
    % $\substvars(\tyub) = \substvars_{\va}(\tyub)$.
    By \RST{UnionRight} and reflexivity, we know 
    \subtydflt{\substvars(\tylb)}{\ty_{\va}}.
    By \thmref{thm:subty-sound-subst} (soundness of subtyping with respect to
    substitution), \subtydflt{\substvars(\tylb)}{\substvars(\tyub)} (H),
    for \subty{\UEnv}{\tylb}{\tyub} by the \UEnv validity assumption. 
    Thus, we have %\subtydflt{\substvars(\tylb)}{\ty_{\va}} and
    \subtydflt{\ty_{\va}}{\substvars(\tyub)}
    by \RST{UnionLeft}, H, and H$_{l_i}$.
    Since \vldinenv{\AEnv}{\UEnv}{\substvars} and \va does not occur in \UEnv,
    it holds that \vldinenv{\AEnv}{\UEnv}{\substvars_{\va}}, and we get 
    \[ \vldinenv{\AEnv}{\UEnv, \varbound{\va}{\tylb}{\tyub}}{\substvars_{\va}}. \]

    Finally, we know that $\forall i,j,$ \subtydflt{\tylb_i}{\ty_{\va}} holds by
    \RST{UnionRight} and reflexivity, and \subtydflt{\ty_{\va}}{\tyub_j} holds
    by \RST{UnionLeft}, H$_{u_j}$, and \subtydflt{\tylb_i}{\tyub_j},
    which means \vldinenv{\AEnv}{\CSet_{\va}}{\substvars_{\va}}.
    Since \va does not occur in $\CSet'$, we also know 
    \vldinenv{\AEnv}{\CSet'}{\substvars_{\va}}.
    Thus, \vldinenv{\AEnv}{\CSet}{\substvars_{\va}}, which concludes the proof.
\end{proof}


