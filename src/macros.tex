%% General
%% ======================================================================

\newcommand{\TODO}[1]{\textcolor{red}{\textbf{TODO:} #1}}

\newcommand{\figref}[1]{Fig.~\ref{#1}\xspace}
\newcommand{\lemref}[1]{Lem.~\ref{#1}\xspace}
\newcommand{\thmref}[1]{Thm.~\ref{#1}\xspace}
\newcommand{\ruleref}[1]{Rule~{\small #1}\xspace}
\newcommand{\defref}[1]{Def.~\ref{#1}\xspace}
\newcommand{\secref}[1]{Section~\ref{#1}\xspace}
\newcommand{\chapref}[1]{Chapter~\ref{#1}\xspace}
\newcommand{\appref}[1]{Appendix~\ref{#1}\xspace}

\newcommand{\tdef}[1]{\textbf{#1}}

\newcommand{\CSharp}{C\texttt{\small\#}\xspace}
\newcommand{\FSub}{$F^N_{\leq}$\xspace}

%% for wide figures:
%% http://16marzo.blogspot.com/2009/09/figure-e-margini-in-classicthesis.html
\newlength\largefigure
\setlength{\largefigure}{\columnwidth+\marginparsep+\marginparwidth}

\definecolor{light-gray}{gray}{0.85}

\newcommand{\cfbox}[2]{%
    \colorlet{currentcolor}{.}%
    {\color{#1}%
    \fbox{\color{currentcolor}#2}}%
}

%% Code
%% *********************************************************

\newcommand{\code}[1]{\texttt{\small#1}\xspace}
\newcommand{\cjl}[1]{\lstinline[language=Julia]!#1!\xspace}

\lstnewenvironment{julia}
    {\lstset{language=Julia,numbers=left,}}
    {}
\newenvironment{codeenvd}
    {
    \begin{center}
    \begin{minipage}{8cm}
    }
    {
    \end{minipage}
    \end{center}
    }

%% Math
%% *********************************************************

%% make sure we are in math mode
\newcommand{\EM}[1]{\ensuremath{#1}\xspace}

\newtheorem{theorem}{Theorem}
\newtheorem{lemma}{Lemma}

\newcommand{\interp}[1]{\EM{\llbracket #1 \rrbracket}}

%% Formalization
%% ======================================================================

%% Metavariables
%% *********************************************************

\newcommand{\ty}{\EM{\tau}}                 %% type annotation τ
\newcommand{\gty}{\EM{\sigma}}              %% type tag σ

\newcommand{\nsty}{\EM{\phi}}               %% normalized simple type φ
%% normalized existential type
\newcommand{\nety}{\EM{\xi}}
%\newcommand{\nety}{\EM{\prescript{\exists}{}{\phi}}}        %% ∃φ
%% normalized union-existential type
\newcommand{\nuty}{\EM{\omega}}
%\newcommand{\nuty}{\EM{\prescript{}{\cup\exists}{\phi}}}    %% ∪∃φ       


\newcommand{\cname}{\EM{\mathit{cname}}}
\newcommand{\aname}{\EM{\mathit{aname}}}
\newcommand{\iname}{\EM{\mathit{name}}}

\newcommand{\VEnv}{\EM{\mathrm{V}}}     %% simple variable environment V
\newcommand{\AEnv}{\EM{\Gamma}}         %% forall-variable environment Γ
\newcommand{\EEnv}{\EM{\mathbf\Delta}}         %% exist-variable environment Δ
                                        %% (unification variables)
\newcommand{\EmptyEnv}{\EM{\cdot}}

\newcommand{\CSet}{\EM{\mathcal{C}}}    %% set of subtype constraints
\newcommand{\EmptySet}{\EM{\varnothing}}    %% empty set of constraints

%% Style
%% *********************************************************

\newcommand{\tyname}[1]{\EM{\mathsf{#1}}}       %% type name (like Int)
\newcommand{\var}[1]{\EM{\mathrm{#1}}}          %% type variable
\newcommand{\unvar}[1]{\EM{\mathbf{#1}}}        %% unification variable

%% overline for list
\newcommand{\ol}[1]{\EM{\overline{#1}}}

%% set
\newcommand{\cset}[1]{\EM{\{#1\}}}

%% side condition
\newcommand{\sidecond}[1]{\EM{\scriptstyle #1}}

%% inference rule names
\newcommand{\SR}[1]{\textsc{S-#1}}
\newcommand{\ER}[1]{\textsc{E-#1}}

%% Symbols
%% *********************************************************

\newcommand{\Alt}{~\vert~}                      %% |

\newcommand{\symsub}{\EM{<:}}                   %% <:
\newcommand{\symeq}{\EM{\approx}}               %% ==
%\newcommand{\vdashfr}{\EM{\vdash\!\!{\circ}}}   %% |-o
\newcommand{\vdashfr}{\EM{\Vdash}}   %% ||-

%% Type Constructors
%% *********************************************************

\newcommand{\tyany}{\tyname{Any}}               %% Any
\newcommand{\tybot}{\tyname{Bot}}               %% Bot

\newcommand{\typair}[2]{\EM{#1 \times #2}}      %% t1 × t2
\newcommand{\tyunion}[2]{\EM{#1 \cup #2}}       %% t1 ∪ t2

\newcommand{\tyinv}[2]{\EM{#1\{#2\}}}           %% t1{...}

%\newcommand{\tybound}[3]{\EM{\tyvar{#1}\!\in\!({#2},{#3})}} %% X \in (tl,tu)
\newcommand{\varbound}[3]{\EM{#2\!\symsub\!#1\!\symsub\!#3}} %% tl<:X<:tu
\newcommand{\tybound}[3]{\varbound{\tyvar{#1}}{#2}{#3}} %% tl<:X<:tu
\newcommand{\tyboundu}[2]{\EM{\tyvar{#1}\!\symsub\!#2}}     %% X<:tu
\newcommand{\gtybound}[2]{\EM{\gtyvar{#1}\!\symsub\!#2}}    %% X<:tu
%\newcommand{\untybound}[3]{\EM{\tyvar{\unvar{#1}}\!\in\!({#2},{#3})}} %% Q \in
%(tl,tu)
\newcommand{\untybound}[3]{\EM{#2\!\symsub\!\tyvar{\unvar{#1}}\!\symsub\!#3}} %% tl<:Q<:tu
\newcommand{\ungtybound}[2]{\EM{\gtyvar{\unvar{#1}}\!\symsub\!#2}}    %% Q<:tu

%% regular existential type with all bounds ∃tX_lb^ub.t
\newcommand{\tyexist}[4]{\EM{\exists\tybound{#1}{#2}{#3}.#4}}
%% concrete existential type withh all bounds ∃gX^ub.t
\newcommand{\gtyexist}[3]{\EM{\exists\gtybound{#1}{#2}.#3}}
%% regular existential type with only upper bound ∃tX^ub.t
\newcommand{\tyexistu}[3]{\EM{\exists\tyboundu{#1}{#2}.#3}}
%% regular existential type with no bounds ∃tX.t
\newcommand{\tyexistnb}[2]{\EM{\exists\tyvar{#1}.#2}}
%% concrete existential type with no bounds ∃gX.t
\newcommand{\gtyexistnb}[2]{\EM{\exists\gtyvar{#1}.#2}}
%% simple existential type with no bounds ∃X.t (variable kind is not specified)
\newcommand{\styexist}[2]{\EM{\exists\var{#1}.#2}}

%% lists of types
\newcommand{\tys}{\EM{\ol{\ty}}}
\newcommand{\gtys}{\EM{\ol{\gty}}}
\newcommand{\nutys}{\EM{\ol{\nuty}}}

%% closed types
\newcommand{\closed}[1]{\EM{\dot{#1}}}
\newcommand{\cty}{\closed{\ty}}
\newcommand{\cgty}{\closed{\gty}}
\newcommand{\cnuty}{\closed{\nuty}}

\newcommand{\tyvar}[1]{\var{\prescript{\circ}{}{#1}}}     %% type variable
\newcommand{\gtyvar}[1]{\var{\prescript{\bullet}{}{#1}}}   %% concrete type variable
\newcommand{\avar}[1]{\var{\prescript{?}{}{#1}}}        %% any type variable

%% constraints
\newcommand{\subctr}[2]{\EM{#1 \leq #2}}
\newcommand{\eqctr}[2]{\EM{#1 = #2}}

%% %%%%%%%%%%%%%%%%%%%%%%%%%%%%%%%%%% Type Examples

\newcommand{\tyint}{\tyname{Int}}
\newcommand{\tyflt}{\tyname{Flt}}
\newcommand{\tystr}{\tyname{Str}}

\newcommand{\nref}{\tyname{Ref}}
\newcommand{\ninvpair}{\tyname{Pair}}

\newcommand{\vx}{\var{X}}
\newcommand{\vy}{\var{Y}}

\newcommand{\tvx}{\tyvar{X}}
\newcommand{\tvy}{\tyvar{Y}}

\newcommand{\gvx}{\gtyvar{X}}
\newcommand{\gvy}{\gtyvar{Y}}

\newcommand{\avx}{\avar{X}}
\newcommand{\avy}{\avar{Y}}

\newcommand{\uavq}{\avar{\unvar{Q}}}
\newcommand{\utvq}{\tyvar{\unvar{Q}}}
\newcommand{\ugvq}{\gtyvar{\unvar{Q}}}

%% Meta functions
%% *********************************************************

\DeclareMathOperator{\fvop}{\mathit{FV}}
\DeclareMathOperator{\domop}{\mathit{dom}}

\DeclareMathOperator{\ubop}{\mathit{ub}}    %% upper bound
\DeclareMathOperator{\lbop}{\mathit{lb}}    %% lower bound

\DeclareMathOperator{\normop}{\mathsf{Norm}}
\DeclareMathOperator{\solvecop}{\mathbf{Solve}}

\newcommand{\fv}[1]{\EM{\fvop(#1)}}
\newcommand{\dom}[1]{\EM{\domop(#1)}}

\newcommand{\ub}[2]{\EM{\ubop(#1, #2)}}
\newcommand{\ubd}[1]{\ub{\AEnv}{#1}}
\newcommand{\lb}[2]{\EM{\lbop(#1, #2)}}
\newcommand{\lbd}[1]{\lb{\AEnv}{#1}}

\newcommand{\norm}[1]{\EM{\normop(#1)}}

\newcommand{\solvec}[4]{\EM{\solvecop(#1,#2,#3,#4)}}
\newcommand{\solvecd}[1]{\solvec{\AEnv}{\EEnv}{\CSet}{#1}}

%% Relations
%% *********************************************************

%% %%%%%%%%%%%%%%%%%%%%%%%%%%%%%%%%%% Well scopedness

\newcommand{\wlscp}[2]{\EM{#1 \vdash #2}}                %% well scoped
\newcommand{\wlscpd}[1]{\wlscp{\VEnv}{#1}}
\newcommand{\wlfrscp}[2]{\EM{#1~\vdashfr~#2}}   %% well scoped with fresh scope
\newcommand{\wlfrscpd}[1]{\wlfrscp{\VEnv}{#1}}


%% %%%%%%%%%%%%%%%%%%%%%%%%%%%%%%%%%% Subtyping

%% Constrained subtyping Γ;Δ |- t1 <: t2 -| C
\newcommand{\subtyc}[5]{\EM{#1 ; #2 \vdash\ #3\ \symsub\ #4\ \dashv #5}}
\newcommand{\subtycd}[3]{\subtyc{\AEnv}{\EEnv}{#1}{#2}{#3}}
%% Subtyping Γ |- t1 <: t2
\newcommand{\subty}[3]{\EM{#1 \vdash #2 \symsub #3}}
\newcommand{\subtyd}[2]{\subty{\AEnv}{#1}{#2}}

%% %%%%%%%%%%%%%%%%%%%%%%%%%%%%%%%%%% Equality

\newcommand{\equalc}[5]{\EM{#1 ; #2 \,\vdashfr\ #3\ \symeq\ #4\ \dashv #5}}
\newcommand{\equalcd}[3]{\equalc{\AEnv}{\EEnv}{#1}{#2}{#3}}

%% Lambda-Julia
%% *********************************************************

\newcommand{\ljsub}[4]{\EM{#1 \vdash #2 <: #3 \vdash #4}}
