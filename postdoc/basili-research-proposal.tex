\documentclass[twocolumn]{article} 

\usepackage{inconsolata}
\usepackage[scaled=0.95]{PTSerif}

\usepackage[margin=0.75in]{geometry}
\usepackage[usenames,dvipsnames]{xcolor}
\usepackage[colorlinks = true,
            urlcolor  = NavyBlue,
            citecolor = teal]{hyperref}
\usepackage{xspace}
\usepackage{listings}

\setlength{\columnsep}{0.25in}

\pagenumbering{gobble}

\lstdefinelanguage{Julia}%
  {morekeywords=[2]{Number,Vector,Int64,Float64},%
}[keywords]

\lstset{%
    language         = Julia,
    basicstyle       = \ttfamily,
    keywordstyle     = \bf,
    keywordstyle     = {[2]\color{Orchid}},
}

\newcommand{\jlc}[1]{\lstinline[language=Julia]|#1|\xspace}


\renewcommand\refname{}


\title{A Cure for Type-Unstable Code}\label{a-cure-for-type-unstable-code}
\author{\href{https://julbinb.github.io/}{Julia Belyakova}}
\date{}%{\href{https://julbinb.github.io/}{julbinb.github.io}}

\begin{document}

\maketitle

% \subsection{Background}\label{background}

% Julia (ref) is a dynamic, high-level, yet high-performance programming
% language, originally designed for scientific computing. Julia aims to
% solve the two-language problem, combining the convenience of
% productivity languages such as Python and R, and performance of
% languages such as Fortran and C (ref).

% Although Julia is dynamically typed and does not require any type
% annotations, types have a profound role in the language semantics and
% performance. Thus, Julia is designed around symmetric multiple dynamic
% dispatch. Multiple dispatch allows a function to have multiple
% implementations, called methods, tailored to different argument types;
% at run time, a function call is dispatched to the most specific method
% for the given arguments. To deliver performance, Julia relies on a
% type-specializing just-in-time compiler: with few exceptions (ref),
% every time a method is called with a new set of argument types, it is
% specialized for those types.

The key to effective compilation of \href{https://julialang.org/}{Julia}
programs, where every function call is resolved by multiple dynamic dispatch,
%with non-trivial subtyping~\cite{subtyping},
is dispatch elimination.
Dispatched calls prevent further optimizations and are especially expensive
for functions like multiplication that have hundreds of methods
and often appear in hot loops.
To deliver performance comparable to that of C, Julia relies on
a type-specializing just-in-time compiler; a function like
% \begin{verbatim}
% mul42(x) = x * 42
% \end{verbatim}
\texttt{mul42(x)\ =\ x\ *\ 42},
when called with an integer, is compiled to the following
LLVM code without any dynamic dispatch:
\begin{lstlisting}[language=LLVM]
  i64 @julia_mul42_503(i64 signext %0) 
  { %1 = mul i64 %0, 42
    ret i64 %1 }
\end{lstlisting}

% \subsection{Type stability}\label{type-stability}

However, to be compiled effectively,
Julia programs need to be written in a particular style
that is conducive to optimizations. Thus, %for example, 
Julia programmers are
\href{https://docs.julialang.org/en/v1/manual/performance-tips/#Write-%22type-stable%22-functions}{encouraged}
to write so-called \textbf{type-stable} code. Informally, a
function is type stable if the type of its output can be predicted from
the input types. For instance, consider
\begin{lstlisting}[language=Julia]
  pos(x::Number) = x < 0 ? 0 : x
\end{lstlisting}
%\texttt{pos(x::Number)\ =\ x\ \textless{}\ 0\ ?\ 0\ :\ x}
which can be called with any subtype~\cite{subtyping} of \jlc{Number}.
This function is not type stable: when called with a float,
it returns either a float or an
integer depending on the \emph{value} of the argument \texttt{x}.

Type-unstable functions impede optimizations of their callers: in
particular, instability prevents dispatch elimination. For example, a
\jlc{Float64}-specialized version of \texttt{h(x)\ =\ f(pos(x))} can
optimize only the call to \texttt{pos} but not to \texttt{f}: because of
\texttt{pos}'s instability, the run-time type of \texttt{pos(x)} cannot
be predicted, and thus, dispatch cannot be resolved for \texttt{f} ahead
of time.

Notably, type stability is a compiler-dependent property~\cite{stability}:
%formally,
a~function is type stable for the given input types
if for these types, the compiler is able to infer a \emph{concrete}
return type, such as \jlc{Int64} or \jlc{Vector\{Float64\}}.
Furthermore, type stability is not a property of the source language but
rather of an intermediate representation targeted by the type inference
algorithm. To debug type stability, Julia programmers need to inspect
type-inferred IR produced by the compiler. For this, the compiler
provides easy access to compiled code: for instance,
\jlc{code\_warntype(pos,\ (Float64,))} would show the result of type
inference for a call to \texttt{pos} with a float. Nevertheless,
type-stable development can be a tedious process that requires
understanding the IR and manually querying the compiler.

\subsubsection*{Proposed work}\label{proposed-work}

To make the development of type-stable and efficient Julia code more
accessible, we propose to devise a source code analysis, as well as
program transformations for type-unstable code.
To reason about transformations, our model of JIT-compilation~\cite{stability}
needs to be extended to account for new function definitions~\cite{worldage}.
The analysis and
transformations will both be incorporated into an interactive IDE tool.
To guide the design of the tool, we will interview Julia programmers
with different levels of expertise in Julia and familiarity with
compilers.

To give an example of a transformation, consider the unstable
\texttt{pos} function discussed above. There, type instability can be
fixed by replacing the integer literal \texttt{0} with \texttt{zero(x)},
which returns a zero value of the same type as \texttt{x}.

The impact of instability can also be mitigated by introducing so-called
\href{https://docs.julialang.org/en/v1/manual/performance-tips/#kernel-functions}{function barriers}.
A function barrier factors out the code
depending on a type-unstable call into a separate function. Thus, in the
following example,
\begin{verbatim}
  x = pos(...)     =>    h(x) = g1(x) + g2(x)
  g1(x) + g2(x)          x = pos(...)
                         h(x)
\end{verbatim}
dispatch cannot be eliminated for \texttt{g1(x)\ +\ g2(x)} in the
original program because of the call to unstable \texttt{pos}. After a
function-barrier transformation that introduces a new function
\texttt{h}, the expression \texttt{g1(x)\ +\ g2(x)} can be optimized in
a version of \texttt{h} specialized for the run-time type of \texttt{x}.
Thus, the new program will have one dynamically dispatched call to
\texttt{h} instead of the three dynamic calls to \texttt{g1},
\texttt{g2}, and \texttt{+}.

% \subsection{References}\label{references}

\vspace*{-2.7em}
\bibliographystyle{plainurl}
\bibliography{refs}

\end{document}
