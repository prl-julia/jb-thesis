\title{Decidable Subtyping\\for the Julia Language}
\subtitle{Thesis Proposal}

\author{Julia Belyakova}

\date{\normalsize%
Northeastern University\\
Khoury College of Computer Sciences\\
Boston, Massachusetts, USA\\
April 2023
}

\maketitle


\begin{abstract}

Julia is a dynamic, high-level, yet high-performance programming language
for scientific computing.
To encourage a high level of code reuse and extensibility, Julia is
designed around symmetric multiple dynamic dispatch, which allows functions
to have multiple implementations tailored to different argument types.
To resolve multiple dispatch, Julia relies on a subtype relation over a complex
language of run-time types and type annotations, which include parametric types,
untagged unions that distribute over covariant tuples, and impredicative
existential types.
Notably, Julia's subtyping is undecidable.
Unlike in statically typed languages
with undecidable subtyping, e.g. Scala, where the undecidability manifests at
compile time, in Julia, it leads to a run-time stack overflow error. 
This can happen
at almost any point during the program execution, as subtyping is used to
resolve every function call.
In this thesis, I propose to develop a decidable subtype relation
for the Julia language,
such that migrating existing code would require minimal effort.
In particular, to provide for decidability, I will restrict the language of type
annotations. To evaluate the migration effort, I will check whether
type annotations used in registered Julia packages conform to the proposed
restriction.

\end{abstract}


\tableofcontents
