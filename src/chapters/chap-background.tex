\chapter{Background} %: Types and Multiple Dispatch in Julia}%
\label{chap:background}

%\section{Overview of the Julia Language}
%% ======================================================================

Julia is a high-level, dynamic programming language, which was originally
designed for scientific computing~\cite{bib:bezanson:julia-fresh:2017}
and released in~2012.
It aims to solve the so-called ``two-language problem''
by providing both good performance and productivity
features~\cite{bib:bezanson:julia-dyn-perf:oopsla:2018}.
For performance, Julia relies on an optimizing, type-specializing JIT compiler.
For productivity, the language provides garbage collection, dynamic typing, and
multiple dynamic dispatch.

\begin{figure}[t]
%-(x::Date, y::Week) = Date(UTD(value(x) - 7*value(y)))    
\begin{julia}
-(x::BigInt) = MPZ.neg(x, y)
-(x::BigInt, y::BigInt) = MPZ.sub(x, y)
-(x::T, y::T) where T<:Union{Int16, Int32, ..., UInt128} = sub_int(x, y)
-(m::Missing, n::Number) = missing
-(A::AbstractArray{T,N}) where {T,N} = broadcast_preserving_zero_d(-, A)
\end{julia}
\caption{Several method definitions of generic function~\cjl{(-)} in~Julia
}\label{fig:code:subtraction}
\begin{tablenotes}[para]
\small
Methods in lines 1--2 call library functions.
Method in line 3 calls a built-in function.
Method in line 4 returns a singleton value \cjl{missing} of the \cjl{Missing} type.
Method in line 5 calls a higher-order function that applies \cjl{(-)} to the
elements of the array argument \cjl{A}.
\end{tablenotes}
\end{figure}

\tdef{Multiple dynamic dispatch}~\cite{bib:bobrow:common-loops:1986,%
bib:chambers:multi-cecil:1992} is at the core of the Julia language.
The dispatch mechanism allows a function, called \emph{generic
function}, to have multiple implementations, called \emph{methods}, that are
tailored to different argument types. For example, \figref{fig:code:subtraction}
shows several method definitions of function \cjl{(-)},
which is used as both unary negation (lines~1 and 5)
and binary subtraction (lines 2--4).

Unlike static\footnote{Statically typed languages often support
\emph{method overloading}. The difference between method overloading and method
dispatch is that overloading is resolved at compile time, using static types of
the arguments, whereas dispatch is resolved at run time, using the precise,
run-time types of the arguments.} method overloading,
multiple dynamic dispatch is resolved at \emph{run time}:
every function call in the program is dispatched to the
\emph{most specific applicable} method based on the run-time
types of the arguments. If such a single best method does not exist,
an error is raised.
\secref{sec:background:dispatch} describes this process in more detail, but for now,
it suffices to know that dispatch resolution relies on \tdef{subtyping}.
Notably, Julia does not allow the user to call a method of a generic function
directly: the only way to reach a method is via the dispatch mechanism.
Thus, whenever a method is called, its arguments are guaranteed to be
subtypes of the declared method signature types.

\section{Types}\label{sec:background:types-overview}
%% ======================================================================

The expressive power of Julia's multiple dispatch stems from 
its unique \tdef{language of type annotations}, as
demonstrated in~\figref{fig:code:subtraction}.
Briefly, the type language supports:
\begin{itemize}
  \item the top type \cjl{Any}, which is a supertype of all types;
  \item nominal non-parametric types, e.g. \cjl{BigInt} and \cjl{Number};
  \item nominal parametric types, which are invariant in the type parameters, 
        e.g. \cjl{AbstractArray\{T,N\}}\footnote{Julia allows unbounded type
        variables such as the dimensionality parameter \cjl{N} 
        of \cjl{AbstractArray\{T,N\}}
        to be instantiated with ``plain bits'' values such as numbers or
        booleans, for which
        \href{https://docs.julialang.org/en/v1/base/base/\#Base.isbits}{\cjl{isbits}}
        function returns \cjl{true}.
        Thus, \cjl{AbstractArray\{Int64,1\}} is a valid nominal type.
        For value arguments, the invariance check succeeds if they
        are bitwise equal.}
        in line~5;
  \item set-theoretic union types, e.g. 
    \cjl{Union\{Int16, ..., UInt128\}} in line~3;
    the empty \cjl{Union\{\}} is the bottom type, i.e. a subtype
    of all types;
  \item covariant tuple types, e.g. \cjl{Tuple\{String, Number\}};
  \item so-called union-all types, written with \cjl{where} (lines~3 and 5):
    intuitively, a union-all type \cjl{t where T} represents a union of types
    \cjl{t[t'/T]} for all valid instantiations \cjl{t'} of the type
    variable~\cjl{T}.
\end{itemize}
% A more detailed account of Julia types and subtyping is given
% in \chapref{sec:julia-sub:overview}, with \chapref{sec:julia-sub:lambda-julia} 
% providing a specification of subtyping.

% One of the key properties related to Julia types is the distinction between
% \emph{concrete} and \emph{abstract} types. A concrete type is a type that
% describes the representation of a value; for any value, its concrete type can be
% obtained with the \cjl{typeof} operator. Concrete types include primitive types,
% such as \cjl{Int128}, and composite \cjl{struct} types (either mutable or
% immutable). Any concrete type definition in Julia is final: a concrete type does
% not have non-empty subtypes other than itself.
% On the other hand, every concrete type has a single
% declared supertype, which can only be an abstract nominal type.
% Abstract nominal types can be used
% to create a nominal subtype hierarchy, but they cannot be instantiated with
% values. Other categories of abstract types include union types

Nominal types are induced by user-defined datatype declarations and constitute
a single-parent inheritance hierarchy.
Unlike mainstream object-oriented languages such as \CSharp and Java,
Julia does not support data inheritance: only \emph{abstract} types,
which cannot be instantiated with values, can be inherited,
i.e. used as declared supertypes of other nominal types.
Datatypes that can be instantiated with values are referred to as
\emph{concrete}. Concrete nominal types include primitive types
such as \cjl{Int128}, and composite \cjl{struct} types (either mutable or
immutable).
\figref{fig:code:user-def-types} provides several examples of user-defined
concrete (on the left) and abstract (on the right) types.
Parametric types (both abstract and concrete) 
can declare non-recursive lower and upper bounds on type variables;
for example, type \cjl{Rational\{T\}} requires \cjl{T} to be a subtype of
abstract \cjl{Integer}.
Supertypes in type declarations are declared to the right of~\cjl{<:}
and cannot refer to the type being declared.
For example, \cjl{Int128} is a subtype of \cjl{Signed}, and \cjl{Signed} is a
subtype of \cjl{Integer}. If the supertype declaration is omitted, like in
\cjl{AbstractSet\{T\}}, the supertype defaults to \cjl{Any}.
%which is a supertype of all types.

\begin{figure}[t] 
\begin{minipage}{5.5cm}
\begin{julia}
primitive type Int128 <: Signed 128
end

struct Rational{T<:Integer} <: Real
  num::T
  den::T
end

mutable struct
  BitSet <: AbstractSet{Int}
  ...
end
\end{julia}
% struct Missing end
% bits::Vector{UInt64}
% offset::Int
\end{minipage}
\hspace{1.2cm}
\begin{minipage}{4.8cm}
\begin{julia}
abstract type Signed <: Integer
end

abstract type AbstractSet{T}
end
\end{julia}
% RefValue{T}() where {T} = new()
% RefValue{T}(x) where {T} = new(x)    
\end{minipage}
\caption{Examples of type declarations: concrete (left) and abstract (right)
}\label{fig:code:user-def-types}
\begin{tablenotes}[para]
\small
  \cjl{Int128} is a primitive, 128-bit type.
  \cjl{Rational} is an immutable parametric composite type.
  \cjl{BitSet} is a mutable composite type.
  \cjl{Signed} is an abstract type.
  \cjl{AbstractSet} is an abstract parametric type.
\end{tablenotes}
\end{figure}

The distinction between \emph{concrete} and \emph{abstract} types
is important to Julia's runtime.
\emph{Concrete} types are run-time types of values, which can be
obtained with the \cjl{typeof} operator. Because concrete types are final, 
i.e. cannot be inherited, the compiler knows their size and memory layout
and can thus optimize code efficiently.
In Julia, concrete types include concrete nominal types and their tuples.
\emph{Abstract} types, on the other hand,
are widely used as type annotations in method definitions.
They also account for heterogeneous data when used as field type annotations
or type arguments of parametric constructors, e.g. in \cjl{Vector\{Any\}}.
Values of abstract types are stored as references and are not conducive
to optimizations.
Besides abstract nominal types, abstract types include union types
such as \cjl{Union\{Int16,}$\ldots$\cjl{, UInt128\}}, and union-all types
such as \cjl{Vector\{T\} where T} (a shorthand for this type 
is simply \cjl{Vector}). Note, however, that every particular instantiation of
\cjl{Vector}, e.g. \cjl{Vector\{Number\}}, is a concrete type
regardless of the concreteness of the type argument.
The bottom type represented with the empty union, \cjl{Union\{\}}, is neither
concrete nor abstract: it has no values and is a subtype of all types.

It is worth noting that the Julia language does not have a structural function type,
which would typically be written as \cjl{T} $\rightarrow$ \cjl{S}. As mentioned above,
all function calls are dynamically dispatched to a method of that generic
function, and there is no way to directly call a particular method or
pass it as an argument.
However, generic functions are first-class values and can be used as
method arguments: for example, \cjl{(-)} is passed to a function call in line~5
of~\figref{fig:code:subtraction}.
Every generic function~\cjl{f} has the concrete type \cjl{typeof(f)},
which is a subtype of the type of all functions \cjl{Function}.

In Julia, types are also values: they can be stored in variables, 
passed as function arguments, and introspected.
Depending on the kind of a type, the \cjl{typeof} operator returns
one of four values:
\cjl{DataType} for fully instantiated nominal types and tuples,
\cjl{Union} for finite union types,
\cjl{UnionAll} for union-all (\cjl{where}) types,
and \cjl{Core.TypeofBottom} for the bottom type \cjl{Union\{\}}.
There is also a special abstract parametric type \cjl{Type\{T\}}
such that type \cjl{t} is considered an instance\footnote{\cjl{v isa t}
checks if value \cjl{v} is an instance of type \cjl{t}} of \cjl{Type\{t\}}:
as exemplified by \figref{fig:code:dispatch-on-type},
\cjl{Type\{T\}} facilitates multiple dispatch on type values.

\begin{figure}[t]
\centering
\begin{minipage}{8cm}
\begin{julia}
myzero(::Type{Int})               = 0
myzero(::Type{Float64})           = 0.0
myzero(::Type{Vector{T}}) where T = Vector{T}()

myzero(Int)           # returns 0
myzero(Vector{Bool})  # returns Bool[]
\end{julia}
\end{minipage}
\caption{Function~\cjl{myzero}, which dispatches on type values
using \cjl{Type\{T\}}}\label{fig:code:dispatch-on-type}
\end{figure}

\section{Multiple Dispatch}\label{sec:background:dispatch}
%% ======================================================================

Argument type annotations are the key to defining multiple distinct
methods, but the user can choose to omit some or all annotations in a method.
For example, in the first \cjl{sub5} definition below, the omitted type
of~\cjl{x} defaults to \cjl{Any}, a supertype of all types:
%\begin{wrapfigure}{R}{4cm}
%\vspace{-6mm}
\begin{codeenvd}
\begin{julia}
sub5(x) = x - 5             # same as sub5(x::Any) = x - 5
sub5(x::String) = x * "-5"
\end{julia}
\end{codeenvd}
%\vspace{-6mm}
%\end{wrapfigure}
If \cjl{sub5} is called with a string, the call will dispatch to the method 
definition in line 2 and always succeed. 
If \cjl{sub5} is called with anything else,
the call will dispatch to the method definition in line 1. 
In the latter case, the execution will succeed
as long as there is an implementation of \cjl{(-)} for the type of \cjl{x}
and \cjl{Int64}, which is the type of \cjl{5}.
For example, \cjl{sub5(10.0)} evaluates to \cjl{5.0},
whereas \cjl{sub5([])} fails:
%\begin{wrapfigure}{R}{4cm}
%\vspace{-6mm}
\begin{codeenvd}
\begin{julia}
julia> sub5([])
ERROR: MethodError: no method matching -(::Vector{Any}, ::Int64)
\end{julia}
\end{codeenvd}
%\vspace{-6mm}
%\end{wrapfigure}
To achieve good performance, Julia generates method instances
specialized to concrete argument types the function was called
with~\cite{bib:pelenitsyn:type-stability:oopsla:2021}.
Thus, next time a function is called with the same argument types,
method dispatch will have more definitions to choose from.
\figref{fig:code:specializations} shows an example REPL session for \cjl{sub5}.

\begin{figure}[t]
\centering
\begin{minipage}{10cm}
\begin{julia}
julia> sub5(x) = x - 5
julia> sub5(x::String) = x * "-5"

julia> methods(sub5)
# 2 methods for generic function "sub5":
[1] sub5(x::String) in Main at REPL[2]:1
[2] sub5(x) in Main at REPL[1]:1

julia> methods(sub5)[2].specializations
svec()

julia> sub5(10.0)
5.0

julia> methods(sub5)[2].specializations
svec(MethodInstance for sub5(::Float64), nothing, ...)
\end{julia}
\end{minipage}
\caption{A REPL session demonstrating method specialization
}\label{fig:code:specializations}
\begin{tablenotes}[para]
\small
Function \cjl{methods} lists all methods of a generic function.
Field \cjl{specializations} stores compiled method specializations.
\end{tablenotes}
\end{figure}

Method resolution for a dispatched function call \cjl{f(a1, a2, ...)}
relies on tuple subtyping~\cite{bib:leavens:tuple-dispatch:1998}
between run-time argument types and method signatures.
Namely, argument types are combined into a tuple of concrete types
\begin{center}
  $\gty =$ \cjl{Tuple\{typeof(a1), typeof(a2), ...\}},
\end{center}
and every method signature is either a tuple of declared argument types
if the method definition is not parametric, or a union-all of that tuple
if the definition is parametric.
For example, consider \figref{fig:code:subtraction}: method in line~1
is represented by the signature \cjl{Tuple\{BigInt\}}, line~4 by
\cjl{Tuple\{Missing,Number\}}, and line~3 by
\cjl{Tuple\{T, T\} where T <: Union\{...\}}.
%Thus, for the purpose of dispatch resolution, arguments of a function call
%are represented with a single concrete type \gty,
%and every method signature is represented as a type \ty.
Because tuple subtyping gives all elements equal consideration,
multiple dispatch in Julia is \emph{symmetric}.

Dispatch resolution, if successful, returns the \emph{most specific applicable
method} for the given arguments.
Namely, for a call to a generic function~\cjl{f} with a concrete argument
type~\gty, the process can be described as follows:
\begin{enumerate}
  \item Find all applicable methods, i.e., all method signatures \ty of the
    function~\cjl{f} that are supertypes of the argument type $\gty <: \ty$.
    If no methods are applicable, a method-not-found error is raised.
  \item\label{item:dispatch:specific} Among 
    the applicable methods $\{\ty_1, \ldots, \ty_n\}$,
    pick the method with the most specific type signature,
    i.e., $\ty_k$ such that $\forall i\in1..n, \ty_k <: \ty_i$.
    %(method signature is a subtype of all other applicable method
    %signatures).
    If there is no such single best method, a method-is-ambiguous
    error is raised.\footnote{Requested by language users over the years,
    Julia has additional specificity rules to resolve ambiguities in some cases,
    but those are not relevant to this work.}
\end{enumerate}
Thus, subtyping is an integral part of Julia's \emph{dynamic semantics},
and due to step \ref{item:dispatch:specific}, it is used with
arbitrarily complex types.

%While there are ways to optimize dispatch resolution to limit the number of
%subtyping checks, 

%Furthermore, the following section will show that because of invariant
%constructors and union-all types, even subtyping with a concrete type on the left
%(i.e., $\gty <: \ty$) can require an arbitrary subtyping check $\ty_1 <: \ty_2$.

