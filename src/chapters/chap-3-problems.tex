\chapter{Thesis Question}\label{chap:3}

The undecidability of subtyping in Julia
can manifest itself at almost any point during the program execution,
as subtyping over a complex language of run-time types and type annotations
is an integral part of Julia's dynamic semantics.
Namely, subtyping is used to resolve function calls,
process new method definitions,
manipulate data (e.g. when adding an element to a container),
as well as during the JIT compilation.
In practice, the undecidability %of subtyping
leads to 
a run-time crash with a \cjl{StackOverflowError}.
Such an error can be particularly hard to debug,
because neither the problematic subtyping query nor its origins are available
to the user.

A number of issues related to subtyping have been reported
on the Julia bug tracker. For example,
\href{https://github.com/JuliaLang/julia/issues/41948}{\code{\#41948}}\footnote{
    \url{https://github.com/JuliaLang/julia/issues/41948}
} reports a \cjl{StackOverflowError} caused by a function definition,
which is likely linked to the undecidability;
\href{https://github.com/JuliaLang/julia/issues/33137}{\code{\#33137}}\footnote{
    \url{https://github.com/JuliaLang/julia/issues/33137}
} points out an inconsistency in subtyping; % related to the diagonal rule.
\href{https://github.com/JuliaLang/julia/issues/24166}{\code{\#24166}}\footnote{
    \url{https://github.com/JuliaLang/julia/issues/24166} 
} (now fixed) reports a problem with reflexivity and transitivity.
Overall, there are 105 open/704 closed issues labeled with ``types and
dispatch'' as of March 2023,
with 13 open/138 closed being also labeled with ``bug''
(not every issue is properly labeled as a bug,
e.g. the aforementioned
\href{https://github.com/JuliaLang/julia/issues/24166}{\code{\#24166}}).
\tabref{tab:julia-issues-stats} provides a few more data points for comparison:
for example, there are 8 open/86 closed ``codegen'' bugs
and 1 open/15 closed ``GC'' bugs.
% For context, ``bug'' and ``codegen'' are assigned to 8 open/86 closed issues
% out of 71 open/654 closed ``codegen'' issues.
% In total, ``bug'' is assigned to 226 open/2644 closed issues.
Thus, type-related concerns, including the undecidability of subtyping,
are not purely theoretical and
constitute a non-negligible portion of problems in the Julia implementation.
%Overall, there are 22 open and 114 closed issues labeled with both ``bug''
%and ``types and dispatch'' (as of December~2021). For context, ``bug'' and
%``codegen'' are assigned to 6 open and 76 closed issues, and overall, ``bug''
%is assigned to 213 open and 2477 closed issues.
%This demonstrates that type-related concerns, including undecidability of
%subtyping, are not purely theoretical: they manifest in the user code and
%constitute a non-negligible portion of problems in the Julia implementation.

\begin{table}[t]
\caption{Statistics of issues on the Julia bug tracker: open/closed (March 2023)}\label{tab:julia-issues-stats}
\vspace*{0.25em}
\centering\footnotesize
\begin{tabular}{c|ccccc}
 & types and dispatch & codegen & GC & macros & <any label> \\
\midrule
<any label> &
  92/414 & 56/246 & 24/51 & 27/36 & 3551/19238 \\
bug &
  13/138 & 8/86 & 1/15 & 5/11 & 226/2664
\end{tabular}
\end{table}

The main question addressed by my thesis research is:
\begin{quote}
\emph{Can Julia's type language be altered in a practical way
to allow for a decidable subtype relation?}
\end{quote}
Here, \emph{practical} means that most of the existing type annotations
in Julia programs would conform to the proposed type language.
The resulting relation is intended to be reflexive and transitive, as is
typical for a subtype relation. 

Furthermore, I suggest that Julia types are given a set-theoretic interpretation,
with the subtype relation matching set inclusion on the interpretations.
Although Julia was inspired by semantic subtyping, the existing subtype relation
is not consistent with the semantic approach: for example, the type 
\cjl{Tuple\{Int, Union\{\}\}} (a covariant tuple of an integer and the bottom type)
is not considered a subtype of the bottom type despite the fact that there are
no values of type \cjl{Tuple\{Int, Union\{\}\}}.



% The main research problem I am going to focus on is
% \tdef{decidable subtyping for the Julia language}.
% The goal is to find a decidable subtyping specification for a type language
% that is close enough to the one currently used in Julia
% and is not unreasonably restrictive.
% The latter means that the majority of types in the existing Julia packages
% are supported by the proposed specification.
% Furthermore, the type language should be suitable for the use in the rest of
% the Julia compiler.
% To tackle the problem, I will answer the following questions:
% \begin{enumerate}
%     \item \emph{How are types used in Julia and what operations on types
%       need to be supported?}
%       Clearly, types are used as type annotations in method definitions, with
%       subtyping being part of the dispatch resolution.
%       But beyond that, Julia is known to rely on type inference for optimizations
%       during JIT compilation. According to \citet{TODO}, which describes the
%       original Julia design, the type inference algorithm needs subtyping
%       but also meet, join, and widening operators.
%       As \cite{TODO} is generally outdated, the current state of Julia needs
%       to be reviewed to identify what operations on types the compiler relies on.
%     \item \emph{How complicated are types used in practice?}
%       To make subtyping decidable, I will likely need to restrict the type
%       language in some way. At the same time, the restriction should not be
%       prohibitively strong for the existing code base. Thus, an analysis of types
%       currently used in practice can provide guidance for possible restrictions.
% \end{enumerate}
