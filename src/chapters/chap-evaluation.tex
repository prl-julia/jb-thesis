\chapter{Evaluation of Migration Effort}\label{chap:eval}

To estimate the effort required for migrating existing Julia code
to a version of Julia with the restriction on existential types,
I perform a corpus analysis of all 9K packages
from the official Julia package registry.
In particular, the analysis extracts type annotations
from the source code of the packages
and checks how many of those satisfy the restriction
described in \secref{chap:dec-sub}.

Out of more than 2M successfully extracted, \emph{statically} identifiable
type annotations, only two do not have an equivalent type signature
that is expressible under the restriction;
one more annotation has an expressible semantic\footnote{
  The equivalence is not currently derivable in Julia, but nonetheless,
  the type can be replaced without affecting the semantics of
  the analyzed program.} equivalent.
These results are encouraging and suggest that %in practice,
%the proposed restriction does not lose much expressiveness in exchange
%for decidability.
%the cost of expressiveness lost in exchange for decidability is very low.
%allows for a desirable trade-off between expressiveness and decidability.
with the restriction, decidability of subtyping comes at a minimal practical
cost to expressiveness.

\paragraph{Corpus.}
The chosen corpus of packages is the entire official \emph{General
registry}\footnote{\url{https://github.com/JuliaRegistries/General}}:
this registry is the default source of packages used by Julia programs
and contains the majority of all public Julia packages.
The list of packages is obtained using
the JuliaHub service\footnote{
  JuliaHub (\url{https://juliahub.com/ui/Packages}) is listed on
  \url{https://julialang.org/packages/}
  as one of the services for browsing Julia packages.
}:
JuliaHub conveniently provides a JSON file\footnote{
  \url{https://juliahub.com/app/packages/info}
}, which I process with \code{JuliaPkgsList.jl}\footnote{
  \url{https://github.com/julbinb/JuliaPkgsList.jl}
} to extract information about packages and latest registered versions.
There were \numprint{9355} entries listed on JuliaHub as of 2023-05-20.
Out of those, using \code{JuliaPkgDownloader.jl}\footnote{
  \url{https://github.com/julbinb/JuliaPkgDownloader.jl/}
}, \numprint{9315} packages were downloaded successfully.
Some JuliaHub entries were not valid registered packages,
e.g. the entry for Julia itself, some did not contain a version,
some were no longer publicly available or could not be processed
for other reasons.

Thus, the resulting corpus consists of 172K files with 19.5M lines of code
across 9K packages (\tabref{tab:corpus}).

\begin{table}[h]
\centering
\begin{tabular}{c|c|c}
 packages & files & lines of code \\ 
 \midrule
 \numprint{9315} & \numprint{172024} & \numprint{19476938}
\end{tabular}
\caption{Corpus of Julia packages from General registry (May 2023)
}\label{tab:corpus}
\begin{tablenotes}[para]
\small
Lines of code exclude comments.
\end{tablenotes}
\end{table}
%as determined by \code{cloc}.

%There were 9355 entries listed as of 2022-05-28.
%The tool extracts package name and latest version.
%Some entries in the JSON file were not proper registered packages,
%e.g. it contained the Julia repo itself, which is not a registered package.
%Some repositories are no longer open and cannot be accessed.
%Some repos did not have a version (31).

\section{Methodology} 

The analysis\footnote{
  \url{https://github.com/prl-julia/julia-sub
  }
}, written in Julia and executed with Julia 1.8.5, 
extracts type annotations from all \code{.jl}
files in the corpus and reports annotations that cannot be trivially rewritten 
into equivalent ones admitted by the restricted type language.
To extract type annotations, the analysis relies on the Julia parser and 
the MacroTools.jl\footnote{
  \url{https://github.com/FluxML/MacroTools.jl}
} package, which allows for convenient pattern matching over Julia's abstract
syntax tree. 
The analysis is static and as such, does not process dynamically generated code.

I identify and extract the following kinds of type annotations:
\begin{itemize}
    \item method signatures: for example, 
      \cjl{Tuple\{T, Vector\{T\}\} where T} is extracted from method
      \cjl{f(x::T, v::Vector\{T\}) where T = ...};
    \item return type annotations: for example, \cjl{Bool} from 
      \cjl{f(x) :: Bool = ...};
    \item all uses of explicit type assertions \cjl{:: t} outside method
      signatures: these include field
      type annotations, run-time type assertions, and local variable
      type annotations.
\end{itemize}

Type annotations are analyzed as follows:
\begin{itemize}
  \item For each bound type variable, the analysis records information about the
    position of the \cjl{where}-binding in a type, as well as positions of all
    occurrences of the variable relative to the binding.
    Position is defined by a list of constructors.
    For example, in the type \cjl{Tuple\{Ref\{T\}, T, S where S\} where T},
    variable \cjl{T} is bound in \cjl{[]} and occurs in 
    \cjl{\{[Tuple, Ref], [Tuple]\}};
    \cjl{S} is bound in \cjl{[Tuple]} and occurs in \cjl{\{[]\}}.
  \item All variable occurrences are checked against the restriction
    using the binding position to distinguish between top-level 
    and inner-to-invariant-constructor variables, which need to satisfy
    different requirements.
\end{itemize}
%The shorthand notation is unrolled ahead of time
\figref{fig:tests:ta-analysis} provides a list of examples from 
the test suit of the analysis implementation.

Some types not syntactically representable by the restriced grammar of \ty
(\figref{fig:ty-grammar})
have an \emph{equivalent} representable counterpart (according
to Julia subtyping).
Such unrepresentable types are not flagged by the analysis, and they
fall into one of the following two categories:
\begin{itemize}
  \item An existential variable occurs once covariantly, for example,
    \cjl{Tuple\{T\} where T<:u}: this type is equivalent to \cjl{Tuple\{u\}},
    which is a valid \ty.
  \item An existential variable occurs only once in an invariant position, 
    but the occurrence
    is separated from the binding by covariant constructors, for example,
    \cjl{Tuple\{Vector\{T\}\} where T} or
    \cjl{Union\{Vector\{T\}, Missing\} where T}: these types are equivalent
    to \cjl{Tuple\{Vector\{T\} where T\}} and
    \cjl{Union\{Vector\{T\} where T, Missing\}}, respectively,
    which are valid restricted types \ty.
\end{itemize}
In Julia, the first kind of benign unrepresentable types is already 
automatically rewritten into the equivalent existential-free form: for example,
\cjl{Vector\{Tuple\{T\} where T<:Number\}} is evaluated
to \cjl{Vector\{Tuple\{Number\}\}} in the REPL.
% \begin{center}
% \begin{minipage}{8cm}
% \begin{lstlisting}
% julia> Vector{Tuple{T} where T<:Number}
% Vector{Tuple{Number}}
% \end{lstlisting}
% \end{minipage}
% \end{center}

\section{Results}\label{sec:eval:results}

The majority of the source code was processed without failures,
with 206 packages having at least one file that could not be parsed by Julia (an 
example of a parsing error is given in \figref{fig:evaluation-parse-errors}).
In successfully parsed files, the analysis identified a total of
\numprint{2136609} type annotations.
Out of these annotations:
\begin{itemize}
  \item \numprint{1573} were not be processed at all, usually because 
    a type variable binding contained a macro or quoted expression;
    several examples of that are given
    in \figref{fig:evaluation-process-errors};
  \item \numprint{22396} were partially processed: this happens when all 
    type variable bindings are processed successfully, but a part of the type
    cannot not be processed due to a macro or quoted expression; in this case, 
    the analysis can miss some occurrences of type variables.
  %For example, partial processing occurs when an unquote is used to
  %splice an AST into a generated type, in which
  %case we do not attempt binding analysis into that unquoted expression.
\end{itemize}
In total, 1629 packages had at least one %macro-related type 
parsing or processing error.

Of the \numprint{2135036} statically identifiable, 
fully-or-partially analyzable type annotations,
\numprint{2135030} (i.e. \textbf{99.999\%}) were identified as satisfying
the proposed restriction,
and 6 annotations were flagged as potentially being impacted---they are listed
in \figref{fig:evaluation-unsup-ty-anns}.

\begin{figure}
\begin{minipage}{6cm}
\begin{lstlisting}
# false positive, package ForneyLab.jl
# src/algorithms/
#   posterior_factorization.jl
Tuple{
  Vararg{
    Union{
      T, Set{T}, Vector{T}
    } where T <: Variable
  }, Any}
# equivalent to:
Tuple{
  Vararg{
    Union{
      Variable,
      Set{<:Variable},
      Vector{<:Variable}
    }
  }, Any}


# false positive, package Tries.jl
# src/Tries.jl
Tuple{
  Vararg{
    Pair{NTuple{N, K}, T} where N
  }
} where T where K
# equivalent to:
Tuple{
  Vararg{
    Pair{NTuple{N, K}, T} 
  } where N
} where T where K


# false positive, package Tries.jl
# src/Tries.jl
Tuple{
  Vararg{
    Pair{NTuple{N, K}, <:Any} where N
  }
} where K
# equivalent to:
Tuple{
  Vararg{
    Pair{NTuple{N, K}, <:Any} 
  } where N
} where K
\end{lstlisting}
\end{minipage}
\hspace{0.5cm}
\begin{minipage}{6.4cm}
\begin{lstlisting}
# false positive (semantically)
# package Muon.jl
# src/alignedmapping.jl
Dict{K,
  Union{
    AbstractArray{<:Number}, 
    AbstractArray{
      Union{Missing, T}
    } where T <: Number, 
    DataFrame
}}
# semantically equivalent to,
# and admitted by the code:
Dict{K,
  Union{
    AbstractArray{<:Number}, 
    AbstractArray{T} where 
      Missing<:T<:Union{Missing, Number}, 
    DataFrame
}}

# true positive, package Alicorn.jl
# test/Utils/UtilsTests.jl 
Array{
  Tuple{T, 
    Array{T, N} where N, 
    Bool
  } where T
}
# admitted by the code:
Array{
  Tuple{Any, 
    Array{<:Any, <:Any},
    Bool
}}

# true positive
# package UnitfulEquivalences.jl
# src/UnitfulEquivalences.jl
Tuple{Type{
  Union{
    Quantity{T, D, U}, 
    Level{L, S, 
      Quantity{T, D, U}
    } where {L, S}
  } where {T, U}
}} where D
# cannot be easily rewritten
\end{lstlisting}
\end{minipage}
\caption{Type annotations flagged by the analysis:\\ 
false positives (left) and true positive (right)
}\label{fig:evaluation-unsup-ty-anns}
\end{figure}

Three of these six annotations %identified as contravening our restriction were
were false positives related to \cjl{Vararg}. 
In Julia, variadic arguments are represented as
\cjl{Vararg}-in-\cjl{Tuple} types: for example, \cjl{Tuple\{Vararg\{Int\}\}}
stands for a tuple of arbitrarily many integers. 
According to Julia subtyping, \cjl{Vararg} is covariant in its type argument,
whereas the analysis reported it as if it were invariant.

Furthermore, one of the remaining three type annotations can be rewritten
into a semantically equivalent
type that does satisfy our restriction. The flagged type in 
question---the \cjl{Dict} type at the top right of
\figref{fig:evaluation-unsup-ty-anns}---contains an unrepresentable type:
\begin{codeenvd}
\begin{lstlisting}
      AbstractArray{Union{Missing, T}} where T<:Number
\end{lstlisting}
\end{codeenvd}
The above type is semantically equivalent to:
\begin{codeenvd}
\begin{lstlisting}
    AbstractArray{T} where Missing<:T<:Union{Missing, Number}
\end{lstlisting}
\end{codeenvd}
However, In Julia 1.8.5, the former type is a subtype of the latter
but not vice versa.

Thus, only two remaining types truly cannot be expressed under
the proposed restriction.
%
In the first case, 
\begin{codeenvd}
\begin{lstlisting}
    Array{Tuple{Any, Array{T, N} where N, Bool} where T}
\end{lstlisting}
\end{codeenvd}
the problem is that variable \cjl{T} occurs twice.
%
In the second case, which seemingly has the same problem as above,
\begin{codeenvd}
\begin{lstlisting}
    Tuple{Type{Union{
        Quantity{T, D, U}, 
        Level{L, S, Quantity{T, D, U}} where {L, S}
      } where {T, U}
    }} where D
\end{lstlisting}
\end{codeenvd}
the type is equivalent to
\begin{codeenvd}
\begin{lstlisting}
    Tuple{Type{Union{
        Quantity{T, D, U} where {T, U}, 
        Level{L, S, Quantity{T, D, U}} where {L, S, T, U}
      }
    }} where D
\end{lstlisting}
\end{codeenvd}
where all variables occur exactly once.
However, in the second component of the union, variables \cjl{T, U}
are not bound immediately outside the invariant \cjl{Quantity}.

Additionally, I extract and analyze type declarations.
There were a total of \numprint{141232} declarations,
239 of which could not be processed and \numprint{3804} were partially
processed (similarly to the case of type annotations). 
Only one type declaration did not satisfy the restriction in its supertype
declaration, which was in the same package and with the same type as the
false positive \cjl{Dict\{...Union\{T,Missing\}...\}}.

%% BARE DATA
%% ======================================================================

% \paragraph{Packages.}
% Used \url{https://github.com/julbinb/JuliaPkgsList.jl} to get a list of
% Julia packages.
% The info is taken from a JSON file at
% \url{https://juliahub.com/app/packages/info}.
% (JuliaHub is listed as a service for searching all registered packages
% here: \url{https://julialang.org/packages/}).

% There were 9355 entries listed as of 2022-05-28.
% The tool extracts package name and latest version.
% Some entries in the JSON file were not proper registered packages,
% e.g. it contained the Julia repo itself, which is not a registered package.
% Some repositories are no longer open and cannot be accessed.
% Some repos did not have a version (31).

% 9315 packages were successfully downloaded using
% \url{https://github.com/julbinb/JuliaPkgDownloader.jl/}.

% \paragraph{Analysis.}
% Static analysis of type annotations in the downloaded packages.
% Here is cloc info:

% \begin{verbatim}
% -------------------------------------------------------------------
% Language         files          blank        comment           code
% -------------------------------------------------------------------
% Julia            172024        3830817        1937753       19476938
% \end{verbatim}

% All Julia files (i.e. files with .jl extension) were parsed using
% the Julia parser.

% 206 packages had files that could not be parsed.

% After parsing, type annotations were extracted using MacroTools.jl:
% \begin{itemize}
%     \item method signatures, e.g. \cjl{f(x::T, v::Vector\{T\}) where T}
%       is transformed into \cjl{Tuple\{T, Vector\{T\}\} where T}
%     \item if there is a return type anotation, it is counted as a separate type
%     \item with the exception of method signatures, all \cjl{t} from
%         \cjl{::t}, which includes field type annotations, run-time type assertions,
%         and local variable type annotations
% \end{itemize}

% Some type annotations could not be properly processed, e.g. because of macros
% \begin{verbatim}
% Error: Couldn't process type annotation
%   tastr = "(Tuple{A} where Base.IteratorSize(A)::Base.SizeUnknown) where A"
%   err = AssertionError: Unsupported lb-var-ub format

% Error: Couldn't process type annotation
%   tastr = "(((Tuple{(\$T_nameparam){\$N, \$M, \$FT}} where \$FT) where \$M) where \$N) where \$(T_params...)"
%   err = AssertionError: Unsupported lb-var-ub format
% \end{verbatim}
% Total of 1629 packages had at least one error like that.

% Total 2136609 type annotations were found. Out of those:
% \begin{itemize}
%     \item 1573 couldn't be processed at all (e.g. as described above): usually
%         because a type variable binding couldn't be parsed
%     \item 22396 were partially processed, e.g. if one element of a tuple was
%         a \$-thingy, it was ignored when type variable occurrence was analyzed
% \end{itemize}

% 2135030 satisfy the restriction, 6 were reported as not satisfying.

% 2135030 + 6 + 1573 (errors) = 2136609, checks out.

% Types that syntactically do not fit the restriction but are trivially
% equivalent (involve tuples) were considered good by the analysis.

\section{Migration Strategies}

In general, for cases where an equivalent representable type does not exist,
I suggest two migration strategies:
\begin{enumerate}
  \item Use more permissive types. For example, replace unrepresentable
    \cjl{Ref\{Dict\{T, T\} where T\}} with \cjl{Ref\{Dict\{<:Any,<:Any\}\}}.
    This approach may require changing type annotations in method signatures as
    well as code where type-annotated values are created:
    \begin{codeenvd}
    \begin{lstlisting}
f(r :: Ref{Dict{T,T} where T}) = ...
r = Ref{Dict{T,T} where T}(...)
f(r)

# transforms to

f(r :: Ref{Dict{<:Any,<:Any}}) = ...
r = Ref{Dict{<:Any,<:Any}}(...)
f(r)
    \end{lstlisting}
    \end{codeenvd}
    If it is necessary that the type arguments of the dictionary are the same,
    a dynamic check can be added. For instance, if \cjl{d} is a dictionary
    value, the following code ensures that the type parameters are
    equivalent:
    \begin{codeenvd}
    \begin{lstlisting}
typeof(d).parameters[1] == typeof(d).parameters[2]
    \end{lstlisting}
    \end{codeenvd}
    If there is already a method definition for the more permissive type,
    it needs to be merged with the no-longer-supported definition:
    to ``dispatch'' to the right code, a dynamic check akin to the above
    can be used.
  \item Define a wrapper type and use it instead of the unrepresentable one.
    For example:
    \begin{codeenvd}
    \begin{lstlisting}
struct EqDict{T}
  d :: Dict{T,T}
end
    \end{lstlisting}
    \end{codeenvd}
    Then, instead of \cjl{Ref\{Dict\{T, T\} where T\}}, the representable type
    \cjl{Ref\{EqDict\{<:Any\}\}} can be used, but the original data needs
    to be wrapped.
    %All sites 
    %This approach may require changing sites where data flows into the method,
    %since the data needs to be wrapped.
\end{enumerate}

\paragraph{Migrating the three examples.}

For the three examples identified by the analysis (listed on the right of
\figref{fig:evaluation-unsup-ty-anns} in \chapref{sec:eval:results}),
I performed a manual inspection and attempted migrating the code,
which included the following steps:
\begin{itemize}
  \item run tests;
  \item insert \cjl{@info "inside"} near the impacted type annotation and run
    tests again to ensure that the affected part of the program is exercised by
    the tests (macro \cjl{@info} simply prints its argument);
  \item modify the code and run tests again, compare the result.
\end{itemize}

Two out of the three examples were successfully migrated with a single
change---in the problematic type annotation.
In both cases, the problematic type annotations were
field or type variable annotations, and one of those cases also included
the above mentioned problematic supertype declaration.

The first successful migration was done for \code{Muon.jl} Julia package:
\figref{fig:evaluation-migrate-muon} depicts relevant code
and shows two necessary changes in comments.
There, one change is in a supertype declaration, and another
is in a field type annotation.

The second successful migration was done for \code{Alicorn.jl}:
\figref{fig:evaluation-migrate-alicorn} depicts relevant code
and shows one necessary change in a comment.
There, the change is in the type annotation of local variable \cjl{examples}.

\begin{figure}
\begin{minipage}{12cm}
\begin{lstlisting}
struct AlignedMapping{T <: Tuple, K, R} <: 
AbstractAlignedMapping{
  T,
  K,
  Union{
    AbstractArray{<:Number},
    # MIGRATION: replace with
    # AbstractArray{T} where Missing <: T <: Union{Number,T}
    AbstractArray{Union{Missing, T}} where T <: Number,
    AbstractDataFrame,
  },
}
  ref::R # any type as long as it supports size()
  d::Dict{K,
    Union{
        AbstractArray{<:Number},
        # MIGRATION: replace with
        # AbstractArray{T} where Missing <: T <: Union{Number,T}
        AbstractArray{Union{Missing, T}} where T <: Number,
        DataFrame,
    },
  }
  
  function AlignedMapping{T, K}(r, d::AbstractDict{K}) 
      where {T <: Tuple, K}
    @info "inside AlignedMapping"
    for (k, v) in d
        checkdim(T, v, r, k)
    end
    return new{T, K, typeof(r)}(r, d)
  end
end
\end{lstlisting}
\end{minipage}
\caption{\code{Muon.jl} code fragment with unrepresentable types
}\label{fig:evaluation-migrate-muon}
\end{figure}

\begin{figure}
\begin{minipage}{12cm}
\begin{lstlisting}
function _getExamplesFor_isElementOf()
  # MIGRATION: replace with
  # examples::Array{ Tuple{Any, Array, Bool} } = [
  examples::Array{ Tuple{T, Array{T,N} where N, Bool} where T } = [
    ("a", ["a" "b" "c" "d"], true),
    (1, [2 3; 4 5], false),
  ]
  @info "inside _getExamplesFor_isElementOf"
  return examples
end
\end{lstlisting}
\end{minipage}
\caption{\code{Alicorn.jl} code fragment with unrepresentable type
}\label{fig:evaluation-migrate-alicorn}
\end{figure}

The final example, depicted in \figref{fig:evaluation-migrate-unitful}, 
involves \cjl{Type} and dispatch on type values (discussed in the first
paragraph on page \pageref{fig:code:dispatch-on-type}).
Because the problematic type annotation appears in a method that is used
to macro-generate other methods, I could not easily
track all the uses of the method and migrate it.
The example also suggests that the case of \cjl{Type\{T\}} might need
special treatment to allow \cjl{T} to be instantiated with type signatures
\tysig rather than \ty. To ensure decidability, an incomplete equality
relation (discussed on page \pageref{item:use-equality})
could likely be used in place of the left-to-right and right-to-left
subtyping checks.

\begin{figure}
\begin{minipage}{12cm}
\begin{lstlisting}
# COULD NOT MIGRATE
dimtype(::Type{
  Union{
    Quantity{T,D,U}, 
    Level{L,S,Quantity{T,D,U}} where {L,S}
  } where {T,U}
}) where D = begin
	@info "inside dimtype"
	typeof(D)
end

macro eqrelation(name, relation)
  ...
  quote
    UnitfulEquivalences.edconvert(
      ::dimtype($(esc(a))), x::$(esc(b)), ::$(esc(name))
    ) = x * $(esc(rhs))
    UnitfulEquivalences.edconvert(
      ::dimtype($(esc(b))), x::$(esc(a)), ::$(esc(name))
    ) = x / $(esc(rhs))
    nothing
  end
  ...
end

\end{lstlisting}
\end{minipage}
\caption{\code{UnitfulEquivalences.jl} code fragment with unrepresentable type
(could not be migrated easily)
}\label{fig:evaluation-migrate-unitful}
\end{figure}

Overall, the results of the static analysis of type annotations
are encouraging: in practice, the impact of the restriction appears to be
limited.
