\chapter{Thesis Statement}\label{chap:3}

% The main question addressed by my thesis research is:
% \begin{quote}
% \emph{Can Julia's type language be altered in a practical way
% to allow for a decidable subtype relation?}
% \end{quote}
% Here, \emph{practical} means that most of the existing type annotations
% in Julia programs would conform to the proposed type language.
% The resulting relation is intended to be reflexive and transitive, as is
% typical for a subtype relation. 

\begin{quotation}\emph{
  The Julia language can be evolved to provide for decidable subtyping while
  requiring minimal effort to migrate existing code.
}\end{quotation}

In the thesis, I will show that
the decidability of subtyping can be achieved
for a type language reminiscent of Julia's type annotations language,
with a restriction on existential types used to instantiate parametric types.
The resulting subtype relation is intended to be reflexive and transitive,
as is typical for a subtype relation. 
Furthermore, I will define a set-theoretic interpretation of types
in terms of run-time type tags and explore the relationship between
the set-theoretic model
and the decidable, syntactic subtype relation.
%with the subtype relation matching set inclusion on the interpretations.
Although Julia was inspired by semantic subtyping, the existing subtype relation
is not consistent with the semantic approach: for example, the type 
\cjl{Tuple\{Int, Union\{\}\}} (a covariant tuple of an integer and the bottom type)
is not considered a subtype of the bottom type despite the fact that there are
no values of type \cjl{Tuple\{Int, Union\{\}\}}.
This inconsistency will be addressed in the subtype relation proposed
in the thesis.

To evaluate the effort required to migrate existing code to a language
with the new subtype relation, I will conduct a static analysis of type
annotations in the official registry of Julia packages
(about 9000 packages as of March 2023).
In addition, I will manually inspect a random sample of packages containing
type annotations that do not conform to the new type language
and propose possible migration strategies.

I intend to submit a POPL 2024 paper on the formalization and evaluation of
the decidable subtype relation and defend my thesis in August 2023.

An implementation of the subtype relation is beyond the scope of this thesis and
is left for future work.

% Furthermore, I suggest that Julia types are given a set-theoretic interpretation,
% with the subtype relation matching set inclusion on the interpretations.
% Although Julia was inspired by semantic subtyping, the existing subtype relation
% is not consistent with the semantic approach: for example, the type 
% \cjl{Tuple\{Int, Union\{\}\}} (a covariant tuple of an integer and the bottom type)
% is not considered a subtype of the bottom type despite the fact that there are
% no values of type \cjl{Tuple\{Int, Union\{\}\}}.

\paragraph*{Preliminary work.}
In my thesis research so far,
I have collaborated on reconstructing a specification of Julia subtyping
(OOPSLA 2018~\cite{TODO}),
defined and mechanized a set-theoretic model of a subset of Julia types
(FTfJP 2019~\cite{TODO}),
and collaborated on modeling Julia's type-specializing JIT compiler
(OOPSLA 2021~\cite{TODO}).

\paragraph*{Plan of work.}
I am currently working on proving the decidability, reflexivity,
and transitivity of the new subtype relation,
and intend to complete proofs in May 2023.
After that, I will perform the evaluation
and study an extension of the set-theoretic interpretation
for types with type variables.
I will concurrently work on writing the thesis and the POPL 2024 paper
throughout May and June, and will finish the thesis in July.

\begin{table}[h]
  \caption{Schedule}
  \vspace*{0.25em}
  \centering\footnotesize
  \begin{tabular}{c|ccccc}
  \toprule
  & May & June & July & August \\
  \midrule
  proofs \& evaluation & X & X & & \\
  paper & X & X & & \\
  thesis & X & X & X & \\
  defense & & & & X \\
\end{tabular}
\end{table}

% The main research problem I am going to focus on is
% \tdef{decidable subtyping for the Julia language}.
% The goal is to find a decidable subtyping specification for a type language
% that is close enough to the one currently used in Julia
% and is not unreasonably restrictive.
% The latter means that the majority of types in the existing Julia packages
% are supported by the proposed specification.
% Furthermore, the type language should be suitable for the use in the rest of
% the Julia compiler.
% To tackle the problem, I will answer the following questions:
% \begin{enumerate}
%     \item \emph{How are types used in Julia and what operations on types
%       need to be supported?}
%       Clearly, types are used as type annotations in method definitions, with
%       subtyping being part of the dispatch resolution.
%       But beyond that, Julia is known to rely on type inference for optimizations
%       during JIT compilation. According to \citet{TODO}, which describes the
%       original Julia design, the type inference algorithm needs subtyping
%       but also meet, join, and widening operators.
%       As \cite{TODO} is generally outdated, the current state of Julia needs
%       to be reviewed to identify what operations on types the compiler relies on.
%     \item \emph{How complicated are types used in practice?}
%       To make subtyping decidable, I will likely need to restrict the type
%       language in some way. At the same time, the restriction should not be
%       prohibitively strong for the existing code base. Thus, an analysis of types
%       currently used in practice can provide guidance for possible restrictions.
% \end{enumerate}
