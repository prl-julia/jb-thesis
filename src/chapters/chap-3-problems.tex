\chapter{Thesis Statement}\label{chap:3}

% The main question addressed by my thesis research is:
% \begin{quote}
% \emph{Can Julia's type language be altered in a practical way
% to allow for a decidable subtype relation?}
% \end{quote}
% Here, \emph{practical} means that most of the existing type annotations
% in Julia programs would conform to the proposed type language.
% The resulting relation is intended to be reflexive and transitive, as is
% typical for a subtype relation. 

Because of the undecidability of subtyping, which is an integral part of Julia's
dynamic semantics, a Julia program can
unexpectedly crash during the execution.
One way to address this problem would be to use a decidable subtype relation
in place of the existing, undecidable one.
However, as subtyping impacts the set of valid programs and their semantics,
this can have profound effects on user experience with the language.
%the language expressiveness, usability, and performance.
%Thus, not any decidable subtype relation would be worthwhile a major change in
%the language.

My thesis is:
\begin{quotation}\emph{
  The Julia language can be evolved to provide for decidable subtyping while
  requiring minimal effort for migrating existing code.
}\end{quotation}

To validate the thesis, I will:
\begin{enumerate}
  \item design a new subtype relation based on the one currently used by Julia
    and prove it decidable;
  \item estimate the migration effort by conducting a static analysis of
    registered Julia packages and suggesting code migration strategies.
\end{enumerate}

% I conjecture that the migration effort can be used to assess the impact of the
% new subtype relation on expressiveness and usability:
% if most of the existing code remains valid in the new language, 
% then the impact on the expressiveness and usability is limited.

An implementation of the subtype relation and understanding its impact on
Julia's performance are beyond the scope of this thesis and
remain future work.

% Furthermore, I suggest that Julia types are given a set-theoretic interpretation,
% with the subtype relation matching set inclusion on the interpretations.
% Although Julia was inspired by semantic subtyping, the existing subtype relation
% is not consistent with the semantic approach: for example, the type 
% \cjl{Tuple\{Int, Union\{\}\}} (a covariant tuple of an integer and the bottom type)
% is not considered a subtype of the bottom type despite the fact that there are
% no values of type \cjl{Tuple\{Int, Union\{\}\}}.

\paragraph*{Preliminary work.}
In my thesis research so far,
I have collaborated on reconstructing a specification of Julia subtyping
(OOPSLA 2018~\cite{oopsla18b}),
and defined and mechanized a set-theoretic model of a subset of Julia types
(FTfJP 2019~\cite{Belyakova:2019:minijl-sub});
%and collaborated on modeling Julia's type-specializing JIT compiler
%(OOPSLA 2021~\cite{TODO}).
more details about these efforts can be found in \chapref{chap:4}.
I have also collaborated on modeling Julia's dynamic semantics, including
\cjl{eval}~\cite{oopsla20a}
and the JIT compiler~\cite{oopsla21jules},
which illuminated the role of types and subtyping
in the language.

% The main research problem I am going to focus on is
% \tdef{decidable subtyping for the Julia language}.
% The goal is to find a decidable subtyping specification for a type language
% that is close enough to the one currently used in Julia
% and is not unreasonably restrictive.
% The latter means that the majority of types in the existing Julia packages
% are supported by the proposed specification.
% Furthermore, the type language should be suitable for the use in the rest of
% the Julia compiler.
% To tackle the problem, I will answer the following questions:
% \begin{enumerate}
%     \item \emph{How are types used in Julia and what operations on types
%       need to be supported?}
%       Clearly, types are used as type annotations in method definitions, with
%       subtyping being part of the dispatch resolution.
%       But beyond that, Julia is known to rely on type inference for optimizations
%       during JIT compilation. According to \citet{TODO}, which describes the
%       original Julia design, the type inference algorithm needs subtyping
%       but also meet, join, and widening operators.
%       As \cite{TODO} is generally outdated, the current state of Julia needs
%       to be reviewed to identify what operations on types the compiler relies on.
%     \item \emph{How complicated are types used in practice?}
%       To make subtyping decidable, I will likely need to restrict the type
%       language in some way. At the same time, the restriction should not be
%       prohibitively strong for the existing code base. Thus, an analysis of types
%       currently used in practice can provide guidance for possible restrictions.
% \end{enumerate}
