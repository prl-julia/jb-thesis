\chapter{Background: Dispatch and Subtyping in Julia}\label{chap:2}

\section{Overview of the Julia Language}
%% ======================================================================

Julia is a high-level, dynamic programming language, originally designed 
for scientific computing~\cite{bezanson:julia-fresh:2017}.
It aims to solve a so-called ``two-language problem''
by providing both good performance and productivity features~\cite{bezanson:julia-dyn-perf:oopsla:2018}.
For performance, Julia relies on an optimizing, type-specializing JIT compiler.
For productivity, the language provides garbage collection, dynamic typing, and
multiple dynamic dispatch.

\begin{figure}[t]
%-(x::Date, y::Week) = Date(UTD(value(x) - 7*value(y)))    
\begin{julia}
-(x::BigInt) = MPZ.neg(x, y)
-(x::BigInt, y::BigInt) = MPZ.sub(x, y)
-(x::T, y::T) where T<:Union{Int16, Int32, ..., UInt128} = sub_int(x, y)
-(m::Missing, n::Number) = missing
-(A::AbstractArray{T,N} where {T,N}) = broadcast_preserving_zero_d(-, A)
\end{julia}
\caption{Several method definitions of generic function~\cjl{(-)}
in~Julia}\label{fig:code:subtraction}
\end{figure}

\tdef{Multiple dynamic dispatch}~\cite{bobrow:common-loops:1986,%
chambers:multi-cecil:1992} is at the core of the Julia language.
The dispatch mechanism allows a function, called \emph{generic
function}, to have multiple implementations, called \emph{methods}, that are
tailored to different argument types. For example, \figref{fig:code:subtraction}
shows several method definitions of function \cjl{(-)},
which is used as both unary negation (lines~1 and 5)
and binary subtraction (lines 2--4).
At \emph{run time}\footnote{Statically typed languages often support
\emph{method overloading}. The difference between method overloading and method
dispatch is that overloading is resolved at compile time, using static types of
the arguments, whereas dispatch is resolved at run time, using the precise,
run-time types of the arguments.},
every function call in the program is dispatched to the
\emph{most specific applicable} method, which is determined based on the
types of the argument values. If such a single best method does not exist,
an error is raised.
\secref{sec:2:dispatch} describes this process in more detail, but for now,
it suffices to know that dispatch resolution relies on \tdef{subtyping}.
Notably, Julia does not allow the user to call a method of a generic function
directly: the only way to reach a method is via the dispatch mechanism.
Thus, whenever a method is called, its arguments are guaranteed to be
subtypes of the declared method signature types.

Julia has a non-trivial \tdef{language of types}, as
demonstrated in~\figref{fig:code:subtraction}. In addition to nominal types such
as \cjl{BigInt} and \cjl{Number}, the language also supports set-theoretic union
types (e.g. \cjl{Union\{Int16, ..., UInt128\}} in line~3),
covariant tuple types (e.g. \cjl{Tuple\{String, Number\}}),
invariant parametric types (e.g. \cjl{Vector\{T\}}),
and so-called union-all types, written with \cjl{where} (lines~3 and 5).
Intuitively, a union-all type \cjl{t where T} represents a union of types
\cjl{t[t'/T]} for all valid instantiations \cjl{t'} of the type variable \cjl{T}.
\citet{zappa-nardelli:julia-sub:oopsla:2018} provide a detailed
account of Julia types and their subtype relation, and we give a brief
overview of subtyping in \secref{sec:2:subtyping}.

One of the key properties related to Julia types is the distinction between
\emph{concrete} and \emph{abstract} types. A concrete type is a type that
describes the representation of a value; for any value, its concrete type can be
obtained with the \cjl{typeof} operator. Concrete types include primitive types,
such as \cjl{Int128}, and composite \cjl{struct} types (either mutable or
immutable). Any concrete type definition in Julia is final: a concrete type does
not have non-empty subtypes other than itself.
On the other hand, every concrete type has a single
declared supertype, which can only be an abstract nominal type.
Abstract nominal types can be used
to create a nominal subtype hierarchy, but they cannot be instantiated with
values. Other categories of abstract types include union types
such as \cjl{Union\{Int16,}$\ldots$\cjl{, UInt128\}}, and union-all types
such as \cjl{Vector\{T\} where T} (a shorthand for this type 
is simply \cjl{Vector}). Note, however, that every particular instantiation of
\cjl{Vector}, e.g. \cjl{Vector\{Number\}}, is a concrete type.
The bottom type represented with the empty union, \cjl{Union\{\}}, is neither
concrete nor abstract: it has no values and is a subtype of all types.

\figref{fig:code:user-def-types} provides several examples of user-defined
concrete (on the left) and abstract (on the right) types. Parametric types (both
abstract and concrete) can declare (non-recursive) lower and upper bounds on type variables;
for example, type \cjl{Rational\{T\}} requires \cjl{T} to be a subtype of
abstract \cjl{Integer}.
Supertypes in type definitions are declared to the right of~\cjl{<:}.
For example, \cjl{Int128} is a subtype of \cjl{Signed}, and \cjl{Signed} is a
subtype of \cjl{Integer}. If the supertype declaration is omitted, like in
\cjl{AbstractSet\{T\}}, the supertype defaults to \cjl{Any},
which is a supertype of all types.

It is worth noting that the Julia language does not have a structural function type,
which would typically be written as \cjl{T} $\rightarrow$ \cjl{S}. As mentioned above,
all function calls are dynamically dispatched to a method of the generic
function, and there is no way to directly call or
pass a particular method as an argument.
However, generic functions are first-class values and can be used as
method arguments: for example, \cjl{(-)} is passed to a function call in line~5
of~\figref{fig:code:subtraction}.
Every generic function~\cjl{f} has the concrete type \cjl{typeof(f)},
which is a subtype of the type of all functions \cjl{Function}.

\begin{figure}[t] 
\begin{minipage}{6cm}
\begin{julia}
primitive type Int128 <: Signed 128
end

struct Rational{T<:Integer} <: Real
  num::T
  den::T
end

mutable struct
  BitSet <: AbstractSet{Int}
  ...
end
\end{julia}
% struct Missing end
% bits::Vector{UInt64}
% offset::Int
\end{minipage}
\hspace{1.2cm}
\begin{minipage}{5.5cm}
\begin{julia}
abstract type Signed <: Integer
end

abstract type AbstractSet{T}
end
\end{julia}
% RefValue{T}() where {T} = new()
% RefValue{T}(x) where {T} = new(x)    
\end{minipage}
\caption{Examples of type definitions:
  concrete (left) and abstract (right)}\label{fig:code:user-def-types}
\end{figure}


\section{Multiple dispatch resolution}\label{sec:2:dispatch}
%% ======================================================================

Method resolution for a dispatched function call \cjl{f(a1, a2, ...)}
relies on tuple subtyping~\cite{leavens:tuple-dispatch:1998}
between run-time argument types and method type signatures.
Namely, argument types are combined into a tuple of concrete types
\begin{center}
  $\gty =$ \cjl{Tuple\{typeof(a1), typeof(a2), ...\}},
\end{center}
whereas every method type signature is either a tuple of declared argument types
if the method definition is not parametric, or a union-all of that tuple
if the definition is parametric.
For example, consider \figref{fig:code:subtraction}: method in line~1
is represented by the signature \cjl{Tuple\{BigInt\}}, line~4 by
\cjl{Tuple\{Missing,Number\}}, and line~3 by
\cjl{Tuple\{T, T\} where T <: Union\{...\}}.
%Thus, for the purpose of dispatch resolution, arguments of a function call
%are represented with a single concrete type \gty,
%and every method type signature is represented as a type \ty.

Dispatch resolution, if successful, returns the \emph{most specific applicable
method} for the given arguments.
Namely, for a call to a generic function~\cjl{f} with a concrete argument
type~\gty, the process can be described as follows:
\begin{enumerate}
  \item Find all applicable methods, i.e., all method signatures \ty of the
    function~\cjl{f} that are supertypes of the argument type $\gty <: \ty$.
    If no methods are applicable, a method-not-found error is raised.
  \item Among the applicable methods $\{\ty_1, \ldots, \ty_n\}$,
    pick the method with the most specific type signature,
    i.e., $\ty_k$ such that $\forall i\in1..n, \ty_k <: \ty_i$.
    %(method type signature is a subtype of all other applicable method
    %signatures).
    If there is no such single best method, a method-is-ambiguous
    error is raised.\footnote{Requested by language users over the years,
    Julia has additional specificity rules to resolve ambiguities in some cases,
    but those are not relevant to this work.}
\end{enumerate}
Notably, Julia's dispatch mechanism is \emph{symmetric}:
during dispatch resolution, all arguments are given
equal consideration as a part of subtyping on tuples.

While there are ways to optimize dispatch resolution to limit the number of
subtyping checks, the key takeaway is that %from this section 
in Julia, \emph{subtyping} is checked \emph{at run time} as a part of
resolving multiple dynamic dispatch.
%Furthermore, the following section will show that because of invariant
%constructors and union-all types, even subtyping with a concrete type on the left
%(i.e., $\gty <: \ty$) can require an arbitrary subtyping check $\ty_1 <: \ty_2$.

\section{Julia subtyping}\label{sec:2:subtyping}
%% ======================================================================

Subtyping in Julia largely follows the combination of
\tdef{nominal subtyping} for user-defined nominal types and
\tdef{semantic subtyping} for covariant tuple and union types.
For example, using types from \figref{fig:code:user-def-types}, we note that
\cjl{Int128} is a subtype of \cjl{Signed}, and, transitively, of \cjl{Integer};
\cjl{BitSet} is a subtype of \cjl{AbstractSet\{Int\}}.
%but not \cjl{AbstractSet\{String\}}.
A~union type \cjl{Union\{t1, t2, ...\}} describes a set-theoretic union of
types \cjl{t1}, \cjl{t2}, \cjl{...}; for example, \cjl{Int} is a subtype of
\cjl{Union\{Signed, String\}}, and \cjl{Union\{t1, t2, ...\} <: t} if all
components \cjl{t1 <: t, t2 <: t, ...}.
Tuples in Julia are immutable, and tuple types are covariant:
\cjl{Tuple\{t1, t2, ...\}} is a subtype of \cjl{Tuple\{t1', t2'...\}} if
their corresponding components are subtypes, i.e., \cjl{t1 <: t1', t2 <: t2', ...}.
Following semantic subtyping, tuple types distribute over unions,
so types \cjl{Tuple\{Union\{Int,String\}\}} and
\cjl{Union\{Tuple\{Int\},Tuple\{String\}\}} are equivalent.

User-defined parametric structs are invariant in the type parameter
regardless of whether the struct is
mutable or immutable, meaning that \cjl{Name\{t11, t12, ...\}} is a subtype of 
\cjl{Name\{t21, t22, ...\}} only if the corresponding type arguments are equivalent,
i.e., \cjl{t11 <:> t21, t21 <:> t22, ...}.
Thus, the immutable invariant struct \cjl{Rational\{Int\}} is \emph{not}
a subtype of \cjl{Rational\{Signed\}},
while the covariant tuple \cjl{Tuple\{Int\} <: Tuple\{Signed\}}.

Abstract union-all types \cjl{t where tl<:T<:tu} are better known
in literature as \tdef{bounded existential types}, which also model
Java wildcards~\cite{torgersen:wildcards:2004}\footnote{In Julia syntax, 
a Java wildcard type
\cjl{Foo<?>} can be written as \cjl{Foo\{T\} where T}.}.
In what follows, we will call \cjl{t where tl<:T<:tu} an existential type;
if lower (upper) bound on the type variable is omitted, it defaults to the
bottom type \cjl{Union\{\}} (top type \cjl{Any}).
Intuitively, an existential type denotes a union of \cjl{t[t'/T]} for all
instantiations of the type variable \cjl{T} such that \cjl{tl <: t' <: tu}.
Similarly to subtyping of finite union types, the intent is that:
\begin{itemize}
  \item \cjl{(t where tl<:T<:tu) <: t2} if \emph{for all} valid instantiations
    \cjl{t'}, it holds that\\ \cjl{t[t'/T] <: t2}, and
  \item \cjl{t1 <: (t where tl<:T<:tu)} if \emph{there exists} at least one 
    \cjl{t'} such that \cjl{t1 <: t[t'/T]}.
\end{itemize}
For example, \cjl{Vector\{Int\}} is a subtype of \cjl{Vector\{T\} where
T<:Integer} because \cjl{T} can be instantiated with \cjl{Int},
and \cjl{Vector\{T\} where T<:Integer} is a subtype of
\cjl{Vector\{S\} where S} because for all valid instantiations \cjl{t'} of~\cjl{T},
type variable \cjl{S} can be instantiated with the same type \cjl{t'}.
Just like unions, existential types distribute over tuples:
for example, types \cjl{Tuple\{Vector\{T\} where T\}}
and \cjl{Tuple\{Vector\{T\}\} where T} are equivalent.

Existential types in Julia are \emph{impredicative}:
existential quantifiers can appear anywhere in a type,
and type variables can be instantiated with arbitrary existential types.
For example, type \cjl{Vector\{Matrix\{T\} where T\}} denotes a vector
of matrices with arbitrary element types.
In contrast, \cjl{Vector\{Matrix\{S\}\} where S} denotes a set of vectors where
elements are matrices with the same element type.
Thus, a vector of integer matrices \cjl{Vector\{Matrix\{Int\}\}} is a subtype
of the latter---existential---type, because \cjl{S} can be instantiated with
\cjl{Int}. But it is not a subtype of the former---invariant parametric---type
\cjl{Vector\{Matrix\{T\} where T\}}, because type arguments \cjl{Matrix\{Int\}}
and \cjl{Matrix\{T\} where T} are not equivalent.
Note that because of the impredicativity, type arguments of invariant type
constructors such as \cjl{Vector} are arbitrarily complex. Thus, even though any
particular \cjl{Vector\{t\}} is a concrete type, checking subtyping for this
type requires a subtyping check for an arbitrary \cjl{t}.

Existential types serve at least two distinct purposes in the Julia language.
First, parametric types with existential parameters,
such as \cjl{Vector\{Matrix\{T\} where T\}},
are useful for representing heterogeneous data.
Second, top-level existential types, such as
\cjl{Tuple\{T, Vector\{T\}\} where T}, represent
type signatures of parametric method definitions.
It may be surprising that Julia uses existential rather than universal types,
but recall that the primary purpose of types is to serve multiple dispatch.
In Julia, it is impossible to directly invoke a parametric method definition
and provide it with a type argument. Instead, the method is being dispatched to if
subtyping for the corresponding existential type succeeds. Then, in the body of
the method, the existential type is implicitly unpacked, with the witness type
being some valid instantiation induced by subtyping.
Consider the following code snippet as an example:
\begin{codeenvd}
\begin{julia}
f(v :: Vector{T}) where T = Set{T}(v)

f([5, 7, 5]) # Set{Int} with 2 elements: 5, 7
\end{julia}
\end{codeenvd}
Because \cjl{[5, 7, 5]} is a \cjl{Vector\{Int\}} and
\cjl{Tuple\{Vector\{Int\}\}} is a subtype of the existential
\cjl{Tuple\{Vector\{T\}\} where T},
as witnessed by the instantiation of \cjl{T} with \cjl{Int},
the call \cjl{f([5, 7, 5])} dispatches to the method in line~1,
and \cjl{T} in the body of the method becomes \cjl{Int}.
However, when multiple instantiations of the variable are possible,
Julia sometimes gives up on assigning the witness type.
In the example below, subtyping succeeds for the call \cjl{g(5.0)},
because \cjl{Tuple\{Bool\} <: Tuple\{T\} where T>:Int}, as witnessed
by multiple instantiations of \cjl{T}, e.g. \cjl{Any} or
\cjl{Union\{Int, Bool\}}.
But rather than pick one instantiation, Julia throws an error.
\begin{codeenvd}
\begin{julia}
> g(x::T) where T>:Int = begin
    println(x)
    println(T)
  end

> g(true)
true
ERROR: UndefVarError: T not defined
\end{julia}
\end{codeenvd}
On the other hand, for the call \cjl{g(5)}, \cjl{T} is assigned the smallest
possible type, \cjl{Int}:
\begin{codeenvd}
\begin{julia}
> g(5)
5
Int
\end{julia}
\end{codeenvd}

Subtyping of existential types includes a special case, called 
the \tdef{diagonal rule}, which provides the support for
a generic programming pattern where method arguments are expected
to be of the same concrete type.
Consider the following method definition, which defines equality
\cjl{(==)} in terms of the built-in equality of bit representations \cjl{(===)}:
\begin{codeenvd}
\begin{julia}
==(x::T, y::T) where T<:Number = x === y
\end{julia}
\end{codeenvd}
If it were possible to instantiate \cjl{T} with an abstract type such as
\cjl{Integer}, the method could be called with a pair of a signed and unsigned
integer. This would be an incorrect implementation of equality, for the same bit
representation corresponds to different numbers when interpreted with and
without the sign.
To prevent such behavior, the diagonal rule states: if a type variable
appears in the type (1) only covariantly and (2) more than once,
it can be instantiated
only with concrete types.\footnote{A similar rule applies to static resolution
of method overloading in \CSharp. An example can be found on this page:
\href{https://fzn.fr/projects/lambdajulia/diagonalcsharp.pdf}{https://fzn.fr/projects/lambdajulia/diagonalcsharp.pdf}}
Thus, the type signature of \cjl{(==)} above,
\cjl{Tuple\{T, T\} where T<:Number}, represents a restricted existential type:
it is a union of tuples \cjl{Tuple\{t, t\}} where \cjl{t} is a concrete subtype
of \cjl{Number}. The same rule applies in line~3
of~\figref{fig:code:subtraction}: built-in integer subtraction \cjl{sub_int} is
guaranteed to be called only with primitive integers of the same concrete type.

\section{Undecidability of Julia subtyping}\label{sec:2:undecidable}
%% ======================================================================

\begin{figure}
\[
\begin{array}{rcl}
  \interp{\mathit{Top}} &=& \text{\cjl{Union\{\}}}\\
  \interp{\alpha} &=& \alpha\\
  \interp{\forall\alpha\leq\ty.\ty'} &=&
    \text{\cjl{Tuple\{Ref\{}}\alpha\text{\cjl{\}, }}
    \interp{\ty'}\text{\cjl{\} where\ }}\alpha\text{\cjl{\ >:\ }}\interp{\ty}\\
\end{array}
\]
\caption{Encoding of \FSubN types in Julia}\label{fig:FSub-encoding}
\end{figure}

It is not unusual for a statically typed programming language 
to have undecidable subtyping,
as witnessed by Java and Scala~\cite{grigore:java-undec:2017,hu:dot-undec:2020}.
In practice, the undecidability means that the
compiler might not terminate on some programs. Although undesirable,
such property can be acceptable if it manifests rarely and
allows for an expressive type system.

In a dynamically typed Julia, however, subtyping is used at run time---for
dispatch resolution. Even on rare occasions, consequences of
undecidability at run time are of greater concern than at compile time.
That is why the decidability of subtyping was one of the explicit goals
of the original design of the Julia language~\cite{bezanson:julia:2015}.
To this end, Julia disallowed several features that were known to cause
undecidability, such as recursive constraints on type variables and
circularities in the inheritance hierarchy~\cite{tate:taming-wildcards:2011}.

Despite the intentional simplifications in the type language,
Julia subtyping is in fact \emph{undecidable}.
As discovered by Ben Chung and Ross Tate, Julia can encode system
\FSubN, which is known to be undecidable~\cite{pierce:bound-sub-undec:1992}.
\figref{fig:FSub-encoding} shows the encoding\footnote{Arrow %of \FSubN types
type is dropped as irrelevant to the undecidabiltiy result.},
with $\ty_1 \leq \ty_2$ in \FSubN defined as
$\interp{\ty_2} <: \interp{\ty_1}$ in Julia.

In practice, the undecidability of Julia subtyping manifests itself with
a \cjl{StackOverflowError}.
The reason is that internally, Julia relies on a dedicated stack
to resolve subtyping
and terminates the program when the stack reaches a certain limit.
%Thus, in practice, subtyping checks that showcase undecidability, such as
%the Ghelli example, often manifest with a \cjl{StackOverflowError}
%rather than non-termination.
