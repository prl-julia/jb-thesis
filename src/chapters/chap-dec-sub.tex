\chapter{Decidable Subtyping of Existential Types}%
\label{chap:dec-sub}

This chapter presents a decidable subtype relation for the core language
of Julia types that includes covariant tuples, invariant type constructors,
unions, existential types, and distributivity.
As discussed in \secref{sec:julia-sub:undec}, it is the interaction of
impredicative existential types, invariant constructors, and unions that makes
Julia subtyping so challenging to decide, which is why I focus
on this sublanguage first.

The decidability of subtyping for the core language is achieved by
restricting existential types allowed to appear inside invariant 
constructors. In particular, the restriction limits such inner existentials 
to the ones expressible with Java wildcards.
This reduces the space of types representing heterogeneous data,
but top-level existential types, which are used to represent parametric method
definitions, remain intact.

\TODO{Evaluation and extensions for diagonals and nominals later}

% In this chapter, I present a decidable subtype relation for the following 
% sublanguage of Julia types:
% \begin{itemize}
%     \item covariant tuples;
%     \item invariant type constructors 
%         (without nominal subtyping hierarchy);
%     \item unions that distribute over tuples;
%     \item top-level bounded existential types that distribute over tuples;
%     \item restricted bounded existential types for invariant constructors.
% \end{itemize}

As outlined in~\secref{sec:julia-sub:undec}
(strategy \ref{item:remove-uni-vars} on page~\pageref{item:remove-uni-vars}),
the key to obtaining a decidable subtype relation for Julia is to
\emph{eliminate variable-introducing existential types inside invariant
constructors}. Notably, such existentials do not need to be eliminated
entirely. Instead, they can be \textbf{restricted} to the type grammar
of \emph{Java wildcards}~\cite{bib:torgersen:wildcards:2004}.

In Java, the wildcards mechanism provides \emph{use-site
variance}~\cite{bib:thorup:unif-genericity:1999}, a restricted form
of bounded existential types~\cite{bib:igarashi:variance:2002}.
For example, the wildcard type 
\[ \code{List<?\ \textbf{extends} Number>} \]
represents an existential type 
\[ \code{$\exists$X<:Number.List<X>}. \]
Thus, the type \code{List<List<?>{>}} represents a heterogeneous
list of lists where inner lists may have different element types.
With existential types, the above type can be written as
\[ \code{List<$\exists$X.List<X>{>}}, \]
which corresponds to the following Julia type:
\[ \text{\cjl{Vector\{Vector\{T\} where T\}}}. \]

Note that with wildcards, every occurrence of \code{``?''} introduces a 
fresh existential variable that \emph{cannot be referred to by name},
and as such, has the following properties:
\begin{enumerate}
    \item occurs in the type exactly once---as an argument of a type
        constructor;
    \item is bound immediately outside the containing type constructor;
    \item cannot have recursive constraints.
\end{enumerate}
For example, the type \code{List<Pair<?, ?>{>}} represents a heterogeneous
list of pairs \code{List<$\exists$X.$\exists$Y.Pair<X,Y>{>}},
but there is no way of restricting both elements of the pair to the same type
like with full-fledged existential types: 
\code{List<$\exists$X.Pair<X,X>{>}}.

Due to this ``namelessness'' property,
subtyping of wildcards-induced existential types can be checked without 
explicitly introducing existential variables.
For example, the following subtyping
\[
    \code{List<?\ \textbf{super}\ l\ \textbf{extends}\ u>}
    \ \ <:\ \ 
    \code{List<?\ \textbf{super}\ l'\ \textbf{extends}\ u'>}
\]
holds as long as the bounds of the implicit variable on the right-hand side
contain the bounds of the left-hand side one, that is,
\[
    \code{l'} <: \code{l} \quad\text{and}\quad \code{u} <: \code{u'}.
\]
Thus, in the case of Julia subtyping,
\begin{quotation}\emph{
    wildcards-induced existential types can be allowed inside
    invariant constructors without jeopardizing the decidability of subtyping,
}\end{quotation}
as they will not introduce new unification variables.

Conveniently, the Julia language already supports a shorthand notation
corresponding to the proposed restriction\footnote{As of June 2023,
it is impossible to specify both lower and upper bounds with this notation.}.
For instance, types \cjl{Vector}, \cjl{Vector\{<:Number\}}, and \cjl{Vector\{>:Int\}}
represent \cjl{Vector\{T\} where T}, \cjl{Vector\{T\} where T<:Number},
and \cjl{Vector\{T\} where T>:Int}, respectively.

The next section provides a complete definition of decidable subtyping
for the core language of Julia types, with the above described restriction
on existential types inside invariant constructors.
In the remaining sections of this chapter, I prove several properties
of subtyping, in particular,
decidability of subtyping (\thmref{thm:subtysig-terminates})
and
soundness of constraints resolution (\thmref{thm:solvectr-sound}).

\section{Definition of Subtyping}%
\label{sec:dec-sub:defs}
%% ======================================================================

%% Grammar
%% *********************************************************

\begin{figure}[t]
\footnotesize
\makebox[\textwidth][c]{
\begin{tabular}{l@{\hspace{4mm}}l}
    $\begin{array}{rcll}
        \tysig
        &::=& & \textit{Type signature} \\
        &\Alt& \tyany & \text{top} \\
        &\Alt& \tybot & \text{bottom} \\
        &\Alt& \vany
            & \text{type variable} \\
        &\Alt& \typair{\tysig_1}{\tysig_2}
            & \text{covariant tuple} \\        
        &\Alt& \tyinv\iname\rexvars
            & \text{invariant constr.} \\
        &\Alt& \tyunion{\tysig_1}{\tysig_2}
            & \text{union} \\
        &\Alt& \tyexist{\vany}{\tylb}{\tyub}{\tysig}
            & \text{existential} \\
        \\
        \vany
        &::=& & \textit{Type variable} \\
        &\Alt& \vx, \vy, \ldots & \text{universal var.} \\
        &\Alt& \va, \vb, \ldots & \text{unification var.} \\
        \\
        \rexvar
        &::=& \rexvarbound{\tylb}{\tyub} & \textit{Restricted existential var.} \\
    \end{array}$ 
    &
    $\begin{array}{rcll}
        \ty, l, u
        &::=& & \textit{Type} \\
        &\Alt& \tyany & \text{top} \\
        &\Alt& \tybot & \text{bottom} \\
        &\Alt& \vany
            & \text{type variable} \\
        &\Alt& \typair{\ty_1}{\ty_2}
            & \text{covariant tuple} \\        
        &\Alt& \tyinv\iname\rexvars
            & \text{invariant constr.} \\
        &\Alt& \tyunion{\ty_1}{\ty_2}
            & \text{union} \\
        \\ \\ \\ \\ \\ \\ \\ \\
    \end{array}$
\end{tabular}
}
\caption{Grammar of type signatures and types}\label{fig:ty-grammar}
\end{figure}

The restricted type language is given in \figref{fig:ty-grammar}.
For brevity, I switch to a shorter notation, with \tyany, \tybot, 
$\times$, $\cup$, and $\exists$ used instead of
\cjl{Any}, \cjl{Union\{\}}, \cjl{Tuple}, \cjl{Union}, and \cjl{where},
respectively. \iname (a shorthand for \tyinv\iname{}) and 
\tyinv\iname{\ldots} represent
non-parametric and invariant parametric datatypes, where datatype
declarations are implicit and do not impose restrictions on type parameters.

The type language distinguishes between more expressive
\textbf{type signatures} \tysig
and less expressive \textbf{types} \ty:
\begin{itemize}
    \item type signatures \tysig correspond to method type signatures
      and allow for explicit existential types bound outside
      invariant constructors; variable bounds cannot be recursive but are
      allowed to refer to other variables in scope;
    \item types \ty describe data; they are similar to type signatures
      but support only a restricted form of existential types
      \tyinv\iname\rexvars.
\end{itemize}
Semantically, an existential $\tyexist{\vany}{\tylb}{\tyub}{\tysig}$ 
represents a union of $\subst{\tysig}{\substel{\vany}{\ty}}$ for all valid
\emph{type} instantiations $\tylb <: \ty <: \tyub$ of the type variable \vany.
In the spirit of Java wildcards, a \textbf{restricted existential type} 
\tyinv\iname{\rexvarbound{\tylb_1}{\tyub_1},\ldots} %\footnote{%
%in \figref{fig:ty-grammar}, it is written
%as \tyinv\iname\rexvars, with \rexvar being \rexvarbound\tylb\tyub.}
represents
\tyexist{\vx}{\tylb_1}{\tyub_1}{\exists\ldots\tyinv\iname{\vx_1,\ldots}}.
\tyinv\iname{\ty_1,\ldots} is a shorthand for the tightly-bound existential
\tyinv\iname{\rexvarbound{\ty_1}{\ty_1},\ldots}.
Note that the same semantic type can have multiple syntactic representations.
For example, the following pairs of types are equivalent:
\[
\begin{array}{ccc}
    \tyint &\approx& 
        \tyunion{\tyint}{\tyint} \\
    \typair{(\tyunion{\tyint}{\tyflt})}{\tystr} &\approx&
        \tyunion{(\typair{\tyint}{\tystr})}{(\typair{\tyflt}{\tystr})} \\
    \tyexist{\vany}{\tybot}{\tystr}{\typair{\vany}{\tyint}} &\approx& 
        \typair{\tystr}{\tyint} \\
    \tyexist{\vany}{\tybot}{\tyany}{\tyinv\nref\vany} &\approx&
        \tyinv\nref{\rexvarbound\tybot\tyany} \\
\end{array}
\]

% The key to decidable subtyping is to restrict existential types within
% invariant constructors to use-site variance of the form
% \tyinv\iname{\rexvarbound{\tylb}{\tyub},\ldots},
% similar to Java wildcards.
% Unrestricted existential types are supported only at the top level,
% in type signatures \tysig.

% Types and type signatures are defined in \figref{fig:ty-grammar}.

To visually aid the perception of inference rules that define subtyping,
I use two styles of type variables: $\vx, \vy, \ldots$ (referred to as
universal variables) and $\va, \vb, \ldots$ (referred to as unification
variables), with \vany used when the distinction is irrelevant.
\textbf{Unification} variables, discussed extensively
in \secref{sec:julia-sub:undec}, are variables introduced by existential
types on the right-hand side of a subtyping judgment.
Universal variables are variables introduced on the left.
In \secref{sec:julia-sub:lambda-julia}, universal and unification variables
were called left/forall and right/exist, respectively.

In the next section, I give an overview of the subtyping
algorithm for the type language in~\figref{fig:ty-grammar},
with the exact definition and decidability discussed separately.
\textbf{The subtyping algorithm} is given by subtyping rules in
Figures~\ref{fig:subtyping-base}, \ref{fig:subtyping-constrained}, and
\ref{fig:subtyping-tysigs},
along with the constraints resolution algorithm in \figref{fig:ctr-solve}.
The rules are not syntax-directed, i.e., there may be multiple rules
applicable to a pair of types or type signatures; however, the rules
are \emph{analytic}~\cite{bib:martin-lof:analytic-synthetic:1994}:
there is a finite number of applicable rules, and the premises of each rule
are comprised of the subcomponents of its conclusion.
%Detailed proofs are provided in the following sections.
%Although the overview points to figures with exact definitions,
I suggest that the reader skips figures with subtyping rules
when reading the overview for the first time.

% we use \vx, \vy, etc. to denote universal variables
% and \va, \vb, etc. to denote unification variables.
% Unification variables are constrained variables.

\subsection{Overview}
%% *********************************************************

Subtyping starts with subtyping of type signatures
(\figref{fig:subtyping-tysigs}):
\[ \subtysigdflt\tysig{\tysig'}. \]

Here, all explicit existential variables from \tysig (universal variables)
are introduced into environment \AEnv (defined in \figref{fig:subty-aux}), 
and variables from $\tysig'$ (unification variables) are introduced into \UEnv.
To reach all existential types, it may be necessary to apply distributivity
and go through union types. The following derivation provides an example:
\makebox[\textwidth][c]{
\begin{minipage}{\ruleswidth}
\begin{mathpar}
\small
\inferrule*[right=\footnotesize\RSS{UnionRight}]
{ 
    \inferrule*[right=\footnotesize\RSS{ExistLeft}]
    { 
        \inferrule*[right=\footnotesize\RSS{ExistRight}]
        { 
            \inferrule[\footnotesize\RSS{Types}]
            { \text{<discussed below>} }
            { \subtysig{\vx{\symsub}\tyint}{\va}  
                {\typair{\tystr}{\tyinv\nref\vx}}
                {\typair{\tyany}{\tyinv\nref\va}}
            }   
        }
        { \subtysig{\vx{\symsub}\tyint}{\EmptyEnv}  
            {\typair{\tystr}{\tyinv\nref\vx}}
            {\tyexistnob{\va}{(\typair{\tyany}{\tyinv\nref\va})}}
        }
    }
    { \subtysig{\EmptyEnv}{\EmptyEnv} 
        {\typair{\tystr}{(\tyexistnob{\vx{\symsub}\tyint}{\tyinv\nref\vx})}}
        {\tyexistnob{\va}{(\typair{\tyany}{\tyinv\nref\va})}}
    }
}
{ \subtysig{\EmptyEnv}{\EmptyEnv}
    {\typair{\tystr}{(\tyexistnob{\vx{\symsub}\tyint}{\tyinv\nref\vx})}}
    {\tyunion{\tyint}{\tyexistnob{\va}{(\typair{\tyany}{\tyinv\nref\va})}}} 
}
\\
\end{mathpar}
\end{minipage}}
First, \RSS{UnionRight} picks the second type on the right.
Second, because of the distributivity, the existential binding on the left
can be pulled through the tuple by \RSS{ExistLeft}.
Finally, \RSS{ExistRight} opens the existential on the right.

Once the algorithm reaches types on both sides,
i.e. \[ \subtysigdflt\ty{\ty'}, \]
subtyping should succeed if there is 
a valid substitution \substvars of unification variables from \UEnv
such that $\substvars(\ty) \symsub \substvars(\ty')$.
This is done in two steps (\RSS{Types}):
\begin{enumerate}
    \item constrained subtyping \subtyctrR{\AEnv}{\dom{\UEnv}}{\ty}{\ty'}{\CSet}
      generates constraints \CSet;
    \item \solvectrdflt resolves the constraints.
\end{enumerate}

When constrained subtyping (\figref{fig:subtyping-constrained}) takes over,
\[ \subtyctrRdflt{\ty}{\ty'}\ \text{ where }\ \UEnvD = \dom{\UEnv}, \]
the algorithm checks subtyping, possibly generating constraints 
\ctrsub{\tylb}{\va} and \ctrsub{\va}{\tyub} on
unification variables \va from \UEnvD, which may appear in $\ty'$.
This step ignores the declared unification variable bounds,
as emphasized by the environment \UEnvD as opposed to \UEnv.
Initially, all unification variables are located on the right,
which is indicated with $\bullet$ on the right of~$\symsubctrR$.
In the case of invariant constructors, the right-to-left subtyping check
is performed by \subtyctrLdflt{\ty'}{\ty}, with all unification variables
being located on the left.
Thus, constrained subtyping maintains the invariant that unification
variables always appear on at most one side of the subtyping judgment.
Because of this, generated constraints \tylb, \tyub are guaranteed to be 
\textbf{free from unification variables}.
Thus, the running example above continues as follows,
generating the constraint set {\ctrset{\ctrsub{\vx}{\va}, \ctrsub{\va}{\vx}}}.
\makebox[\textwidth][c]{
\begin{minipage}{\ruleswidth}
\begin{mathpar}
\small
\inferrule[]
{ 
    \inferrule[\footnotesize\RSC{Top}]
    { }
    { \subtyctrR{\vx{\symsub}\tyint}{\va}{\tyint}{\tyany}{\EmptyCSet} }
    \and
    \inferrule[]
    { 
        \inferrule[\footnotesize\RSC{UVarRight}]
        { }
        { \subtyctrR{\vx{\symsub}\tyint}{\va}
            {\vx}{\va}{\ctrset{\ctrsub{\vx}{\va}}}               
        }
        \and
        \inferrule[\footnotesize\RSC{UVarLeft}]
        { }
        { \subtyctrL{\vx{\symsub}\tyint}{\va}
            {\va}{\vx}{\ctrset{\ctrsub{\va}{\vx}}}
        }
    }
    { \subtyctrR{\vx{\symsub}\tyint}{\va}
        {\tyinv\nref\vx}
        {\tyinv\nref\va}
        {\ctrset{\ctrsub{\vx}{\va}, \ctrsub{\va}{\vx}}} }
}
{ \subtyctrR{\vx{\symsub}\tyint}{\va}
    {\typair{\tystr}{\tyinv\nref\vx}}
    {\typair{\tyany}{\tyinv\nref\va}}
    {\ctrset{\ctrsub{\vx}{\va}, \ctrsub{\va}{\vx}}}
}
\\
\end{mathpar}
\end{minipage}}

If constrained subtyping succeeds and generates a constraint set
\CSet, the constraints are resolved %with respect to \AEnv and \UEnv
by \solvectrdflt (\figref{fig:ctr-solve}).
Namely, all constraints on the same unification variable are checked for
consistency with each other and with the declared variable bounds from
\UEnv. If all the constraints are consistent, the variable is instantiated
with the smallest type---a union of all (declared and generated) lower bounds.
The consistency of constraints is checked with:
\begin{itemize}
    \item constrained subtyping \subtyctrLdflt{\tylb}{\ty} or
        \subtyctrRdflt{\ty}{\tyub}, to check that generated constraints \ty
        are consistent with declared bounds \tylb and \tyub;
        constrained subtyping is necessary because unification variable
        bounds may refer to other unification variables;
    \item unification-free subtyping of types \subtydflt{\tylb}{\tyub}
        (\figref{fig:subtyping-base}),
        to check that generated constraints are consistent with each other.
\end{itemize}
In the running example, the generated bound \vx of the unification variable \va
is trivially consistent with the declared bounds \tybot and \tyany;
furthermore, \subty{\vx{\symsub}\tyint}{\vx}{\vx} by reflexivity,
so the unification variable \va is instantiated with \vx.

In unification-free subtyping \subtydflt{\ty}{\ty'}, 
which is used for consistency checks,
variables are treated as universal. For example, with the exception of
the reflexive case, a variable \vany is a subtype of \ty only if the upper bound
\tyub of \vany is a subtype of \ty. This is similar to subtyping of
left/forall variables in \figref{fig:jlsubex}, \secref{sec:julia-sub:lambda-julia}.

% Once all existentials are processed, we run constraints-generating
% subtyping $\subtyctrdflt\ty{\ty'}$. It stands for either:
% \begin{itemize}
%     \item $\subtyctrRdflt\ty{\ty'}$ when unification variables are on the right, or
%     \item $\subtyctrLdflt\ty{\ty'}$ when unification variables are on the left.
% \end{itemize}
% We use \UEnvD rather than \UEnv to emphasize that declared bounds of unification
% variables are irrelevant for the judgment.

% Once all constraints are generated, they are resolved by \solvectrdflt
% primarily with $\subtydflt\ty{\ty'}$,
% subtyping of types that are free from unification vars.
% There is a caveat that declared bounds of unification vars need to be consistent
% with collected constraints, which is why constrained subtyping is used
% in constraints resolution.

\begin{figure}
\footnotesize
\[
\begin{array}{lccccccl}
    \dctxtysig &::=& \square &
        \Alt& \typair{\dctxtysig}{\tysig} &
        \Alt& \typair{\tysig}{\dctxtysig} &
        \textit{Distributivity context for type signatures}
    \\
    \dctxty &::=& \square &
        \Alt& \typair{\dctxty}{\ty} &
        \Alt& \typair{\ty}{\dctxty} &
        \textit{Distributivity context for types}
    \\
    \\
    \AEnv, \UEnv &::=& \EmptyEnv &
        \Alt& \AEnv, \varbound{\vany}{\tylb}{\tyub} &
        & & 
        \textit{Type variable environment}
    \\
    \UEnvD &::=& \EmptyEnv &
        \Alt& \UEnvD, \vany &
        & & 
        \textit{Type variable list}
    \\
    \\
    \CSet &::=& \multicolumn{3}{c}{
            \ctrset{\ctrsub{\tylb}{\va}, \ctrsub{\va}{\tyub}, \ldots}
        } & & &
        \textit{Constraint set}
\end{array}
\]
\caption{Auxiliary definitions}\label{fig:subty-aux}
\end{figure}
% \\
% \\
% \sty &::=& \tyany \Alt \tybot \Alt \vany &
%     \Alt& \typair{\sty_1}{\sty_2} &
%     \Alt& \tyinv\iname\rexvars &
%     \textit{Type without top-level unions}


Next, I present the exact definitions of:
\begin{itemize}
    \item unification-free subtyping of types \subtydflt{\ty}{\ty'};
    \item constrained subtyping of types \subtyctrdflt{\ty}{\ty'};
    \item constraints resolution \solvectrdflt;
    \item signature subtyping \subtysigdflt{\tysig}{\tysig'}.
\end{itemize}

For simplicity of the presentation, I assume the following variable names 
convention based on Barendregt's convention:

\begin{definition}[Variable names convention]\label{def:var-names}
    Everywhere in definitions and proofs, the following conditions hold:
    \begin{itemize}
        \item all bound variables in \ty, \tysig are different from each other 
            and from free variables;
        \item all variables in \AEnv, \UEnv, \UEnvD are different 
            from each other;
        \item whenever both \AEnv and \UEnv or \AEnv and \UEnvD
            appear in the same judgment, 
            $\dom{\AEnv} \cap \dom{\UEnv} = \varnothing$ and
                $\dom{\AEnv} \cap \dom{\UEnvD} = \varnothing$;
        % \item whenever both \AEnv and \UEnv or \AEnv and \UEnvD
        %     appear in the same judgment:
        %     \begin{itemize}
        %         \item $\dom{\AEnv} \cap \dom{\UEnv} = \varnothing$ and
        %             $\dom{\AEnv} \cap \dom{\UEnvD} = \varnothing$;
        %         \item \AEnv contains only universal variables \vx, \vy, etc.,
        %             with their bounds also containing only universal variables;
        %         \item \UEnv and \UEnvD contain only unification variables
        %             \va, \vb, etc., with their bounds (for \UEnv) also
        %             containing only unification variables.
        %     \end{itemize}
        \item whenever a variable environment and a type/type signature
            appear in the same judgment, bound variables of the type/type
            signature are different from variables in the environment;
        \item whenever multiple types/type signatures appear in the same
            judgment, their bound variables are different.
    \end{itemize}
    These conditions can be maintained by alpha-renaming.
\end{definition}

% The convention about using different sets of variables for \AEnv and \UEnv/\UEnvD
% is not necessary, but it visually aids the perception of inference rules:
% whenever \va or \vb is used in a judgment,
% it is known to belong to \UEnv or \UEnvD.
% Without this convention, we would need to always check which environment
% the variable belongs to in order to discern its meaning.

%% Subtyping
%% *********************************************************

\subsection{Unification-Free Subtyping of Types}
%% *********************************************************

\begin{figure}
\footnotesize
\begin{mathpar}
    \fbox{\subtydflt{\ty}{\ty}}
    \\

    \inferrule[\RST{Top}]
    { }
    { \subtydflt{\ty}{\tyany} }

    \inferrule[\RST{Bot}]
    { }
    { \subtydflt{\plug\dctx\tybot}{\ty'} }
    \\

    \inferrule[\RST{VarRefl}]
    { \varbound{\vany}{\tylb}{\tyub} \in \AEnv }
    { \subtydflt{\vany}{\vany} }

    \inferrule[\RST{VarLeft}]
    { \varbound{\vany}{\tylb}{\tyub} \in \AEnv \and
        \subtydflt{\plug\dctx{\tyub}}{\ty'} }
    { \subtydflt{\plug{\dctx}\vany}{\ty'} }

    \inferrule[\RST{VarRight}]
    { \varbound{\vany}{\tylb}{\tyub} \in \AEnv \and
        \subtydflt\ty{\tylb} }
    { \subtydflt{\ty}{\vany} }
    \\

    \inferrule[\RST{Tuple}]
    { \subtydflt{\ty_1}{\ty'_1} \and \subtydflt{\ty_2}{\ty'_2} }
    { \subtydflt{\typair{\ty_1}{\ty_2}}{\typair{\ty'_1}{\ty'_2}} }

    % \inferrule*[]
    % { \inherits
    %     {\tyinv{\iname}{\vx_1,\ldots,\vx_n}}
    %     {\tyinv{\iname'}{\ty_1,\ldots,\ty_m}} \and
    %   \forall j \in 1..m.\ 
    %   \subtydflt
    %     {\subst{\ty_j}
    %         {\substel{\vx_1}{\rexvar_1},\ldots,\substel{\vx_n}{\rexvar_n}}}
    %     {\rexvar'_j} }
    % { \subtydflt
    %     {\tyinv\iname{\rexvar_1,\ldots,\rexvar_n}}
    %     {\tyinv{\iname'}{\rexvar'_1,\ldots,\rexvar'_m}} }

    \inferrule[\RST{Inv}]
    { \forall i \in 1..n. \and \subtydflt{\rexvar_i}{\rexvar'_i} }
    { \subtydflt
        {\tyinv\iname{\rexvar_1,\ldots,\rexvar_n}}
        {\tyinv\iname{\rexvar'_1,\ldots,\rexvar'_n}} }

    \inferrule[\RST{UnionLeft}]
    { \subtydflt{\plug\dctx{\ty_1}}{\ty'} \and 
        \subtydflt{\plug\dctx{\ty_2}}{\ty'} }
    { \subtydflt{\plug\dctx{\tyunion{\ty_1}{\ty_2}}}{\ty'} }

    \inferrule[\RST{UnionRight}]
    { \exists i.\ \ \subtydflt\ty{\ty'_i} }
    { \subtydflt{\ty}{\tyunion{\ty'_1}{\ty'_2}} }

    \\
    \fbox{\subtydflt{\rexvar}{\rexvar}}
    \\
    \inferrule*[]
    { \subtydflt{\tylb'}{\tylb} \and \subtydflt{\tyub}{\tyub'} }
    { \subtydflt
        {\rexvarbound{\tylb}{\tyub}}
        {\rexvarbound{\tylb'}{\tyub'}} }
    %
    \\
    \fbox{\subtydflt{\dctx}{\dctx}}
    \\

    \inferrule*[]
    { }
    { \subtydflt\square\square }

    \inferrule*[]
    { \subtydflt{\dctx_1}{\dctx'_1} \and \subtydflt{\ty_2}{\ty'_2} }
    { \subtydflt{\typair{\dctx_1}{\ty_2}}{\typair{\dctx'_1}{\ty'_2}} }

    \inferrule*[]
    { \subtydflt{\ty_1}{\ty'_1} \and \subtydflt{\dctx_2}{\dctx'_2} }
    { \subtydflt{\typair{\ty_1}{\dctx_2}}{\typair{\ty'_1}{\dctx'_2}} }
\end{mathpar}
\caption{Subtyping of types (free from unification variables)
    %Note: \dctx stands for \dctxty.
}\label{fig:subtyping-base}
\end{figure}

Unification-free subtyping of types is given in \figref{fig:subtyping-base}.

Following the semantic subtyping approach,
union types distribute over covariant tuples (\RST{UnionLeft}), 
and tuples containing the bottom type are subtypes of all types (\RST{Bot}).
Both of these rules are expressed with the distributivity context \dctx
defined in \figref{fig:subty-aux}, which allows unions and \tybot to be
``pulled through'' covariant tuples.
Although in Julia, only the proper bottom type \tybot is a subtype of all types
(but not \plug\dctx\tybot), I consider the more general \RST{Bot}
in accordance with semantic subtyping. To obtain Julia's semantics,
\RST{Bot} should be written as \subtydflt{\tybot}{\ty'}.

Subtyping of variables corresponds to the intuition that subtyping should
hold for all their valid instantiations. Thus, when a variable appears
covariantly on the left, it is replaced with the upper bound (\RST{VarLeft}).
When a variable appears on the right, it is replaced with the lower bound
(\RST{VarRight}).
The \RST{VarLeft} rule has to use the distributivity context, 
for the upper bound can be a union or bottom type. Without \dctx, the following
judgment would not be derivable:
\[
    \subty{\varbound{\vany}{\tybot}{(\tyunion{\tyint}{\tyflt})}}
        {\typair{\vany}{\tyany}}
        {\tyunion{\typair{\tyint}{\tyany}}{\typair{\tyflt}{\tyany}}}.
\]

Finally, subtyping of invariant constructors/restricted existential types
ensures that for each type parameter, the bounds on the left are contained
within the bounds on the right. This corresponds to the intuition that
for all valid instantiations of a variable on the left, there should be
a valid instantiation of a variable on the right.
For example, subtyping
\[
    \subty{}{\tyinv\nref{\rexvarbound{\tybot}{\tyint}}}
        {\tyinv\nref{\rexvarbound{\tybot}{\tyany}}}
\]
holds, whereas
\[
    \vdash \tyinv\nref{\rexvarbound{\tybot}{\tyint}} \nless:
    \tyinv\nref{\rexvarbound{\tyint}{\tyany}}
\]
does not.

Subtyping of distributivity contexts \subtydflt{\dctx}{\dctx'} is 
useful for proofs but is not used in the definition of \subtydflt{\ty}{\ty'}.

\subsection{Constrained Subtyping of Types}
%% *********************************************************

\begin{figure}
\footnotesize
\makebox[\textwidth][c]{
\begin{minipage}{\ruleswidth}  
\begin{mathpar}
    \fbox{\subtyctrdflt{\ty}{\ty}} 
    \\

    \inferrule[\RSC{Top}]
    { }
    { \subtyctrdfltenv{\ty}{\tyany}{\EmptyCSet} }

    \inferrule[\RSC{Bot}]
    { }
    { \subtyctrdfltenv{\plug\dctx\tybot}{\ty'}{\EmptyCSet} }

    \colorbox{light-gray}{$
    \inferrule[\RSC{UBot}]
    { \va \in \UEnvD }
    { \subtyctrLdfltenv{\plug\dctx\va}{\ty'}{\ctrsngl\va\tybot} }
    $}
    \\

    \inferrule[\RSC{VarRefl}]
    { \varbound{\vx}{\tylb}{\tyub} \in \AEnv }
    { \subtyctrdfltenv{\vx}{\vx}{\EmptyCSet} }

    \colorbox{light-gray}{$
    \inferrule[\RSC{UVarLeft}]
    { \va \in \UEnvD }
    { \subtyctrLdfltenv{\va}{\ty'}{\ctrsngl{\va}{\ty'}} }
    $}

    \colorbox{light-gray}{$
    \inferrule[\RSC{UVarRight}]
    { \va \in \UEnvD }
    { \subtyctrRdfltenv{\ty}{\va}{\ctrset{\ctrsub{\ty}{\va}}} }
    $}

    \inferrule[\RSC{VarLeft}]
    { \varbound{\vx}{\tylb}{\tyub} \in \AEnv \and
        \subtyctrdflt{\plug\dctx{\tyub}}{\ty'} }
    { \subtyctrdflt{\plug\dctx\vx}{\ty'} }

    \inferrule[\RSC{VarRight}]
    { \varbound{\vx}{\tylb}{\tyub} \in \AEnv \and
        \subtyctrdflt\ty{\tylb} }
    { \subtyctrdflt{\ty}{\vx} }
    \\

    \inferrule[\RSC{Tuple}]
    { \subtyctrdfltenv{\ty_1}{\ty'_1}{\CSet_1} \and 
        \subtyctrdfltenv{\ty_2}{\ty'_2}{\CSet_2} }
    { \subtyctrdfltenv{\typair{\ty_1}{\ty_2}}{\typair{\ty'_1}{\ty'_2}}
        {\CSet_1 \cup \CSet_2} }

    \inferrule[\RSC{Inv}]
    { \forall i \in 1..n. \and 
        \subtyctrdfltenv{\rexvar_i}{\rexvar'_i}{\CSet_i} }
    { \subtyctrdfltenv
        {\tyinv\iname{\rexvar_1,\ldots,\rexvar_n}}
        {\tyinv\iname{\rexvar'_1,\ldots,\rexvar'_n}}
        {\mcup_{i=1}^n \CSet_i} }
    \\

    \inferrule[\RSC{UnionLeft}]
    { \subtyctrdfltenv{\plug\dctx{\ty_1}}{\ty'}{\CSet_1} \and 
        \subtyctrdfltenv{\plug\dctx{\ty_2}}{\ty'}{\CSet_2} }
    { \subtyctrdfltenv{\plug\dctx{\tyunion{\ty_1}{\ty_2}}}{\ty'}
        {\CSet_1 \cup \CSet_2} }

    \inferrule[\RSC{UnionRight}]
    { \exists i.\ \ \subtyctrdflt\ty{\ty'_i} }
    { \subtyctrdflt{\ty}{\tyunion{\ty'_1}{\ty'_2}} }

    \colorbox{light-gray}{$
    \inferrule[\RSC{UVar-UnionRight}]
    {   \va \in \UEnvD \and \va_1, \va_2 \text{ fresh} \and
        \subtyctrL{\AEnv}{\UEnvD,\va_1}{\plug\dctx{\va_1}}{\ty'_1}{\CSet_1} \and
        \subtyctrL{\AEnv}{\UEnvD,\va_2}{\plug\dctx{\va_2}}{\ty'_2}{\CSet_2} \\ 
        \CSet_1 = \CSet'_1 \mcup_{i=1}^n \ctrset{\ctrsub{\va_1}{\tyub_1^i}} 
            \and \va_1 \notin \CSet'_1 \and
        \CSet_2 = \CSet'_2 \mcup_{j=1}^m \ctrset{\ctrsub{\va_2}{\tyub_2^j}} 
            \and \va_2 \notin \CSet'_2  \\
        \CSet'\ \ =\ \ 
            \ctrsngl{\va}{\tyunion{\msqcap_{i=1}^n \tyub_1^i}
                {\msqcap_{j=1}^m \tyub_2^j}} 
            \text{ if } n\geq1,m\geq1
        \text{\ or\ }
        \mcup_{i=1}^n \ctrset{\ctrsub{\va}{u_1^i}}\text{ if } m=0
        \text{\ or\ }
        \mcup_{j=1}^m \ctrset{\ctrsub{\va}{u_2^j}}\text{ if } n=0
    }
    { \subtyctrLdfltenv{\plug\dctx{\va}}{\tyunion{\ty'_1}{\ty'_2}}
        {\CSet'_1 \cup \CSet'_2 \cup \CSet'} }
    $}
%
    \\
    \fbox{\subtyctrdflt{\rexvar}{\rexvar}}
    \\

    \inferrule*[]
    { \subtyctrRdfltenv{\tylb'}{\tylb}{\CSet_l} \and 
        \subtyctrLdfltenv{\tyub}{\tyub'}{\CSet_u} }
    { \subtyctrLdfltenv
        {\rexvarbound{\tylb}{\tyub}}
        {\rexvarbound{\tylb'}{\tyub'}}
        {\CSet_l \cup \CSet_u} }

    \inferrule*[]
    { \subtyctrLdfltenv{\tylb'}{\tylb}{\CSet_l} \and 
        \subtyctrRdfltenv{\tyub}{\tyub'}{\CSet_u} }
    { \subtyctrRdfltenv
        {\rexvarbound{\tylb}{\tyub}}
        {\rexvarbound{\tylb'}{\tyub'}}
        {\CSet_l \cup \CSet_u} }
\end{mathpar}
\end{minipage}}
\caption{Constrained subtyping of types
    % Note: %\dctx stands for \dctxty;
    % every rule with the symbol $\symsubctr$ is a shorthand for two rules, 
    % one where all occurrences of $\symsubctr$ are replaced with 
    % $\symsubctrL$, and another with $\symsubctrR$.
    % Thus, the figure defines two mutually recursive relations,
    % \subtyctrLdflt{\ty}{\ty'} and \subtyctrRdflt{\ty}{\ty'}.
    %$\symsubctr$ stands for either $\symsubctrL$ or $\symsubctrR$,
    %and within one rule, all occurrences of $\symsubctr$ denote the same relation
}\label{fig:subtyping-constrained}
\begin{tablenotes}[para]
\small
    Every rule with the symbol $\symsubctr$ is a shorthand for two rules, 
    one where all occurrences of $\symsubctr$ are replaced with 
    $\symsubctrL$, and another where all occurrences of $\symsubctr$ are
    replaced with $\symsubctrR$.
    Thus, the figure defines two mutually recursive relations,
    \subtyctrLdflt{\ty}{\ty'} and \subtyctrRdflt{\ty}{\ty'}.
\end{tablenotes}
\end{figure}
%% old definition of UVar-UnionRight: unclear how to prove its soundness
% \colorbox{light-gray}{$
% \inferrule[\RSC{UVar-UnionRight}]
% {   \va \in \UEnvD \and \va_1, \va_2 \text{ fresh} \and
%     \subtyctrL{\AEnv}{\UEnvD,\va_1}{\plug\dctx{\va_1}}{\ty'_1}{\CSet_1} \and
%     \subtyctrL{\AEnv}{\UEnvD,\va_2}{\plug\dctx{\va_2}}{\ty'_2}{\CSet_2} \\ 
%     \CSet_1 = \CSet'_1 \mcup_{i=1}^n \ctrset{\ctrsub{\va_1}{\tyub_1^i}} 
%         \and \va_1 \notin \CSet'_1 \and
%     \CSet_2 = \CSet'_2 \mcup_{j=1}^m \ctrset{\ctrsub{\va_2}{\tyub_2^j}} 
%         \and \va_2 \notin \CSet'_2  \\
%     \CSet'\ \ =\ \ \mcup_{i=1,j=1}^{i=n,j=m}
%         \ctrset{\ctrsub{\va}{\tyunion{\tyub_1^i}{\tyub_2^j}}}
%         \text{ if } n\geq1,m\geq1
%     \text{\ or\ }
%     \mcup_{i=1}^n \ctrset{\ctrsub{\va_1}{u_1^i}}\text{ if } m=0
%     \text{\ or\ }
%     \mcup_{j=1}^m \ctrset{\ctrsub{\va_2}{u_2^j}}\text{ if } n=0
% }
% { \subtyctrLdfltenv{\plug\dctx{\va}}{\tyunion{\ty'_1}{\ty'_2}}
%     {\CSet'_1 \cup \CSet'_2 \cup \CSet'} }
% $}

Constrained subtyping of types is given in \figref{fig:subtyping-constrained}.
The figure defines two mutually recursive relations,
\[
    \subtyctrLdflt{\ty}{\ty'} \quad\text{and}\quad \subtyctrRdflt{\ty}{\ty'},
\]
where the former is used when unification variables appear on the left,
and the latter is used when unification variables appear on the right.
For brevity, similar rules of the two relations are abbreviated with
a single rule
\[
    \subtyctrdflt{\ty}{\ty'}.
\]

Most of the rules of constrained subtyping match the rules of unification-free
subtyping, with the only difference being the propagation of constraints
from recursive calls. For instance, compare subtyping of tuples:
\begin{mathpar}
\small
\inferrule[]
    { \subtydflt{\ty_1}{\ty'_1} \and \subtydflt{\ty_2}{\ty'_2} }
    { \subtydflt{\typair{\ty_1}{\ty_2}}{\typair{\ty'_1}{\ty'_2}} }

\inferrule[]
    { \subtyctrdfltenv{\ty_1}{\ty'_1}{\CSet_1} \and 
        \subtyctrdfltenv{\ty_2}{\ty'_2}{\CSet_2} }
    { \subtyctrdfltenv{\typair{\ty_1}{\ty_2}}{\typair{\ty'_1}{\ty'_2}}
        {\CSet_1 \cup \CSet_2} }
\end{mathpar}
In fact, when types do not contain unification variables,
constrained and unification-free subtyping coincide
(\lemref{lem:subtyctr-subty}).

The only case where \subtyctrLdflt{\ty}{\ty'} needs to call
\subtyctrRdflt{\ty}{\ty'} and vice versa is subtyping of invariant
constructors/restricted existential types (\RSC{Inv}):
here, the right-to-left check \subtyctrdflt{\tylb'}{\tylb} moves unification
variables to the opposite side.

All the rules unique to constrained subtyping are highlighted in gray:
\RSC{UBot}, \RSC{UVarLeft}, \RSC{UVarRight}, \RSC{UVar-UnionRight}.
These are the rules that \emph{generate} new constraints on unification
variables rather than simply propagate them.
The most interesting rule is \RSC{UVar-UnionRight}: it addresses the case
where the unification variable could be instantiated with a union.
Consider the following example:\\
\makebox[\textwidth][c]{
\begin{minipage}{\ruleswidth}
\begin{mathpar}
\small
    \inferrule[]
    {
        \inferrule[]
        { }
        { \subtyctrL{\EmptyEnv}{\va}
            {\typair{\va}{\tyany}}
            {\tyunion{\typair{\tyint}{\tyany}}{\typair{\tyflt}{\tyany}}}
            {?} }
        \and
        \inferrule[]
        { 
            \inferrule[]
            { \ldots }
            { \subtyctrR{\EmptyEnv}{\va}
                {\typair{\tyint}{\tyany}}
                {\typair{\va}{\tyany}}
                { \ctrsngl{\tyint}{\va} } } 
            \and
            \ldots
        }
        { \subtyctrR{\EmptyEnv}{\va}
            {\tyunion{\typair{\tyint}{\tyany}}{\typair{\tyflt}{\tyany}}}
            {\typair{\va}{\tyany}}
            { \ctrset{\ctrsub{\tyint}{\va}, \ctrsub{\tyflt}{\va}} } } 
    }
    { \subtyctrR{\EmptyEnv}{\va}
        {\tyinv\nref{\tyunion{\typair{\tyint}{\tyany}}
            {\typair{\tyflt}{\tyany}}}}
        {\tyinv\nref{\typair{\va}{\tyany}}}
        {?} }
\\
\end{mathpar}
\end{minipage}}
Without \RSC{UVar-UnionRight}, the best way to proceed with
\[
    \subtyctrL{\EmptyEnv}{\va}
            {\typair{\va}{\tyany}}
            {\tyunion{\typair{\tyint}{\tyany}}{\typair{\tyflt}{\tyany}}}
            {?}
\]
(besides \RSC{UBot}, which requires \ctrsngl{\va}{\tybot})
would be to use \RSC{UnionRight} and pick either \typair{\tyint}{\tyany}
or \typair{\tyflt}{\tyany} on the right. This would result in constraining \va 
with either \ctrsngl{\va}{\tyint} or \ctrsngl{\va}{\tyflt}.
Taken together with \ctrset{\ctrsub{\tyint}{\va}, \ctrsub{\tyflt}{\va}},
either of these constraints would render the resulting set of constraints
unsatisfiable.
However, subtyping
\[
    \subtyctrR{\EmptyEnv}{\va}
        {\tyinv\nref{\tyunion{\typair{\tyint}{\tyany}}
            {\typair{\tyflt}{\tyany}}}}
        {\tyinv\nref{\typair{\va}{\tyany}}}
        {?}
\]
clearly holds if \va is instantiated with
\tyunion{\tyint}{\tyflt}.
Thus, when \va occurs on the left covariantly, in a position where 
unification-free subtyping could appeal to \RST{UnionLeft},
the rule \RSC{UVar-UnionRight} allows for a more permissive, union upper bound.
Consider the application of \RSC{UVar-UnionRight} to the right-to-left check
in the example:\\
\makebox[\textwidth][c]{
\begin{minipage}{\ruleswidth}
\begin{mathpar}
\small
    \inferrule[]
    {
        \inferrule[]
        { \ldots }
        { \subtyctrL{\EmptyEnv}{\va, \va_1}
            {\typair{\va_1}{\tyany}}
            {\typair{\tyint}{\tyany}}
            {\ctrsngl{\va_1}{\tyint}} }
        \and
        \inferrule[]
        { \ldots }
        { \subtyctrL{\EmptyEnv}{\va, \va_2}
            {\typair{\va_2}{\tyany}}
            {\typair{\tyflt}{\tyany}}
            {\ctrsngl{\va_2}{\tyflt}} }
    }
    { \subtyctrL{\EmptyEnv}{\va}
        {\typair{\va}{\tyany}}
        {\tyunion{\typair{\tyint}{\tyany}}{\typair{\tyflt}{\tyany}}}
        {\ctrsngl{\va}{\tyunion\tyint\tyflt} } }
\\
\end{mathpar}
\end{minipage}}
Now, the resulting constraint set
\ctrset{\ctrsub{\va}{\tyunion\tyint\tyflt}, 
\ctrsub{\tyint}{\va}, \ctrsub{\tyflt}{\va}}
is satisfiable.
Note that the rule \RSC{UVar-UnionRight} needs to be careful when 
the constraint set $\CSet_i$ produced by the premise
\[\subtyctrL{\AEnv}{\UEnvD,\va_i}{\plug\dctx{\va_i}}{\ty'_i}{\CSet_i}\]
contains multiple upper-bound constraints on $\va_i$, because all those
constraints need to hold at the same time.
For instance, consider the following example,
where $\ty_a \symsub \ty'_a$ and $\ty_a$ is incomparable with $\ty_b$:
\[
    \subtyctrL{\EmptyEnv}{\va}
        {\typair{\va}{\va}}
        {\tyunion{(\tyunion{\typair{\ty_a}{\ty_a}}{\typair{\ty'_a}{\ty_b}})}
            {\typair{\ty_b}{\tyany}}}
        { ? }
\]
There are multiple ways to instantiate \va, e.g. with \tybot,
but the most permissive constraint set can be obtained by first
applying \RSC{UVar-UnionRight} to the left occurrence of \va:\\
\makebox[\textwidth][c]{
\begin{minipage}{\ruleswidth}
\begin{mathpar}
\small
    \inferrule[]
    {
        \inferrule[]
        { \ldots }
        { \subtyctrL{\EmptyEnv}{\va, \va_1}
            {\typair{\va_1}{\va}}
            {\tyunion{\typair{\ty_a}{\ty_a}}{\typair{\ty'_a}{\ty_b}}}
            { ? } }
        \and
        \inferrule[]
        { \ldots }
        { \subtyctrL{\EmptyEnv}{\va, \va_2}
            {\typair{\va_2}{\va}}
            {\typair{\ty_b}{\tyany}}
            {\ctrsngl{\va_2}{\ty_b}} }
    }
    { \subtyctrL{\EmptyEnv}{\va}
        {\typair{\va}{\va}}
        {\tyunion{(\tyunion{\typair{\ty_a}{\ty_a}}{\typair{\ty'_a}{\ty_b}})}
            {\typair{\ty_b}{\tyany}}}
        { ? } }
\\
\end{mathpar}
\end{minipage}}
Next, let us focus on the left premise and apply the same rule to \va:\\
\makebox[\textwidth][c]{
\begin{minipage}{\ruleswidth}
\begin{mathpar}
\small
\inferrule[]
{ 
    \inferrule[]
    { \ldots }
    { \subtyctrL{\EmptyEnv}{\va, \va_1, \va_3}
        {\typair{\va_1}{\va_3}}
        {\typair{\ty_a}{\ty_a}}
        {\ctrset{\ctrsub{\va_1}{\ty_a}, \ctrsub{\va_3}{\ty_a}}} }
    \and
    \inferrule[]
    { \ldots }
    { \subtyctrL{\EmptyEnv}{\va, \va_1, \va_4}
        {\typair{\va_1}{\va_4}}
        {\typair{\ty'_a}{\ty_b}}
        {\ctrset{\ctrsub{\va_1}{\ty'_a}, \ctrsub{\va_4}{\ty_b}}} }
}
{ \subtyctrL{\EmptyEnv}{\va, \va_1}
    {\typair{\va_1}{\va}}
    {\tyunion{\typair{\ty_a}{\ty_a}}{\typair{\ty'_a}{\ty_b}}}
    {\ctrset{\ctrsub{\va_1}{\ty_a}, \ctrsub{\va_1}{\ty'_a}, \ctrsub{\va}{\tyunion{\ty_a}{\ty_b}}}} }
\\
\end{mathpar}
\end{minipage}}
There are two constraints on $\va_1$, both of which have to be satisfied.
Thus, \RSC{UVar-UnionRight} merges them into one type using the intersection
function \tymeet{\EmptyEnv}{\ty_a}{\ty'_a}, defined in \figref{fig:ty-join-meet}.
The intersection returns a type that is a subtype of both $\ty_a$ and $\ty'_a$:
in this case, the smaller type $\ty_a$.
Thus, the original subtyping judgment succeeds:
\[
    \subtyctrL{\EmptyEnv}{\va}
        {\typair{\va}{\va}}
        {\tyunion{(\tyunion{\typair{\ty_a}{\ty_a}}{\typair{\ty'_a}{\ty_b}})}
            {\typair{\ty_b}{\tyany}}}
        { \ctrset{\ctrsub{\va}{\tyunion{\ty_a}{\ty_b}}, \ctrsub{\va}{\tyunion{\ty_a}{\ty_b}}} }
\]
The two occurrences of the same constraint are due to the two separate
applications of \RSC{UVar-UnionRight}, $\va_1/\va_2$ and $\va_3/\va_4$.

% \hline
% \multicolumn{6}{c}{\tyjoindflt{\ty}{\ty'}} \\ 
% \hline 
% \ty &\sqcup_{\AEnv}& \ty' &=& \ty' & \text{if } \subtydflt{\ty}{\ty'} \\
% \ty &\sqcup_{\AEnv}& \ty' &=& \ty  & \text{if } \subtydflt{\ty'}{\ty} \\
% \ty &\sqcup_{\AEnv}& \ty' &=& \tyunion{\ty}{\ty'} & \text{otherwise} \\
% \\
\begin{figure}
\footnotesize
\makebox[\textwidth][c]{
\begin{minipage}{\ruleswidth}  
\[
\begin{array}{ccccll}
    \hline
    \multicolumn{6}{c}{\tymeetdflt{\ty}{\ty'}} \\ 
    \hline 
    \ty &\sqcap_{\AEnv}& \ty' &=& \ty  & \text{ if } \subtydflt{\ty}{\ty'} \\
    \ty &\sqcap_{\AEnv}& \ty' &=& \ty' & \text{ if } \subtydflt{\ty'}{\ty} \\
    \tyunion{\ty_1}{\ty_2} &\sqcap_{\AEnv}& \ty' &=& 
        \tyunion{(\tymeetdflt{\ty_1}{\ty'})}{(\tymeetdflt{\ty_2}{\ty'})} &  \\
    \ty &\sqcap_{\AEnv}& \tyunion{\ty'_1}{\ty'_2} &=& 
        \tyunion{(\tymeetdflt{\ty}{\ty'_1})}{(\tymeetdflt{\ty}{\ty'_2})} &  \\
    \vany &\sqcap_{\AEnv}& \ty' &=& 
        \tymeetdflt{\tylb}{\ty'} & 
        \text{ where } \varbound{\vany}{\tylb}{\tyub} \in \AEnv \\
    \ty &\sqcap_{\AEnv}& \vany &=& 
        \tymeetdflt{\ty}{\tylb} & 
        \text{ where } \varbound{\vany}{\tylb}{\tyub} \in \AEnv \\
    \typair{\ty_1}{\ty_2} &\sqcap_{\AEnv}& \typair{\ty'_1}{\ty'_2} &=& 
        \typair{(\tymeetdflt{\ty_1}{\ty'_1})}{(\tymeetdflt{\ty_2}{\ty'_2})} &  \\
    \tyinv\iname{\ldots,\rexvarbound{\tylb_i}{\tyub_i},\ldots} 
        &\sqcap_{\AEnv}& \tyinv\iname{\ldots,\rexvarbound{\tylb'_i}{\tyub'_i},\ldots} &=& 
        \tyinv\iname{\ldots,\rexvarbound{(\tyunion{\tylb_i}{\tylb'_i})}
            {(\tymeetdflt{\tyub_i}{\tyub'_i})},\ldots} &  \\
    & & & & \multicolumn{2}{l}{\text{where } \forall i.\ 
        \subtydflt{(\tyjoindflt{\tylb_i}{\tylb'_i})}
        {(\tymeetdflt{\tyub_i}{\tyub'_i})} } \\
    \ty &\sqcap_{\AEnv}& \ty' &=& \tybot & \text{otherwise} \\
\end{array}
\]
\end{minipage}}
% \caption{Join ($\sqcup_{\AEnv}$) and meet ($\sqcap_{\AEnv}$) of types.
% Both definitions should be read top to bottom, i.e. earlier cases have
% precedence over later cases.
% Some cases (e.g. where \ty or $\ty'$ are \tybot or \tyany, 
% and where both are the same type variable)
% are omitted because they are covered by \subtydflt{\ty}{\ty'} and 
% \subtydflt{\ty'}{\ty}.}\label{fig:ty-join-meet}
\caption{Intersection ($\sqcap_{\AEnv}$) of types}\label{fig:ty-join-meet}
\begin{tablenotes}[para]
\small
Some cases, such as \tybot, \tyany, and the same type variable,
are absent because they are covered by the cases \subtydflt{\ty}{\ty'} and 
\subtydflt{\ty'}{\ty}.
\end{tablenotes}
\end{figure}

\subsection{Signature Subtyping}
%% *********************************************************

\begin{figure}
\footnotesize
\makebox[\textwidth][c]{
\begin{minipage}{\ruleswidth}
\begin{mathpar}
    \fbox{\subtysigdflt{\tysig}{\tysig}}
    \\

    \inferrule[\RSS{Top}]
    { }
    { \subtysigdflt{\tysig}{\tyany} }

    \inferrule[\RSS{Bot}]
    { }
    { \subtysigdflt{\plug\dctxsig\tybot}{\tysig'} }
    \\

    \inferrule[\RSS{VarLeft}]
    { \varbound{\vx}{\tylb}{\tyub} \in \AEnv \and
        \subtysigdflt{\plug\dctxsig{\tyub}}{\tysig'} }
    { \subtysigdflt{\plug\dctxsig\vx}{\tysig'} }

    \inferrule[\RSS{UnionLeft}]
    { \subtysigdflt{\plug\dctxsig{\tysig_1}}{\tysig'} \and 
        \subtysigdflt{\plug\dctxsig{\tysig_2}}{\tysig'}}
    { \subtysigdflt{\plug\dctxsig{\tyunion{\tysig_1}{\tysig_2}}}{\tysig'} }
    \\

    \inferrule[\RSS{InvLeft}]
    { \vx \text{ fresh} \and 
        \subtysig{\AEnv, \varbound{\vx}{\tylb}{\tyub}}{\UEnv}
        {\plug\dctxsig{\tyinv\iname{\ldots,\vx,\ldots}}}{\tysig'} }
    { \subtysigdflt{\plug\dctxsig{
        \tyinv\iname{\ldots,\rexvarbound{\tylb}{\tyub},\ldots}}}{\tysig'} }

    \inferrule[\RSS{ExistLeft}]
    { \subtysig{\AEnv, \varbound{\vx}{\tylb}{\tyub}}{\UEnv}{\plug\dctxsig\tysig}{\tysig'} }
    { \subtysigdflt{\plug\dctxsig{\tyexist{\vx}{\tylb}{\tyub}{\tysig}}}{\tysig'} }


    \inferrule[\RSS{UnionRight}]
    { \exists i.\ \subtysigdflt{\tysig}{\plug\dctxsig{\tysig'_i}} }
    { \subtysigdflt{\tysig}{\plug\dctxsig{\tyunion{\tysig'_1}{\tysig'_2}}} }

    \inferrule[\RSS{ExistRight}]
    { \subtysig{\AEnv}{\UEnv, \varbound{\va}{\tylb}{\tyub}}{\tysig}{\plug\dctxsig{\tysig'}} }
    { \subtysigdflt{\tysig}{\plug\dctxsig{\tyexist{\va}{\tylb}{\tyub}{\tysig'}}} }
    \\

    \inferrule[\RSS{Types}]
    { \subtyctrR{\AEnv}{\dom\UEnv}{\ty}{\ty'}{\CSet} \and
        \solvectrdflt = \substvars }
    { \subtysigdflt{\ty}{\ty'} }
\end{mathpar}
\end{minipage}}
\caption{Subtyping of type signatures
    %Note: \dctx stands for \dctxtysig.
}\label{fig:subtyping-tysigs}
\end{figure}

The final piece of the subtyping algorithm is subtyping of top-level signatures,
which is given in \figref{fig:subtyping-tysigs}:
\[
    \subtysigdflt{\tysig}{\tysig'}.
\]
This step introduces all explicitly bound existential variables to
environments \AEnv and \UEnv.
Similarly to subtyping of types, the rules \RSS{Bot}, \RSS{VarLeft},
\RSS{UnionLeft}, and \RSS{InvLeft} use the distributivity context---in
the case of type signatures, \dctxsig.
The rule \RSS{InvLeft} is needed to account for the distributivity of restricted
existential types, in the same way as explicit existentials distribute
over tuples. For example:\\
\makebox[\textwidth][c]{
\begin{minipage}{\ruleswidth}
\begin{mathpar}
\small
\inferrule*[right=\footnotesize\RSS{ExistRight}]
{ 
    \inferrule*[right=\footnotesize\RSS{InvLeft}]
    { 
        \inferrule[\footnotesize\RSS{Types}]
        { \subtyctrR{\varbound{\va}{\tybot}{\tyany}}{\va}
            {\typair{\tyinv\nref\vx}{\tyint}}
            {\typair{\tyinv\nref\va}{\tyany}}
            { \ctrset{\ctrsub{\vx}{\va}, \ctrsub{\va}{\vx}} }
          \and \ldots }
        { \subtysig{\varbound{\vx}{\tybot}{\tyany}}
            {\varbound{\va}{\tybot}{\tyany}}
            {\typair{\tyinv\nref\vx}{\tyint}}
            {\typair{\tyinv\nref\va}{\tyany}} }
    }
    { \subtysig{\EmptyEnv}{\varbound{\va}{\tybot}{\tyany}}
        {\typair{\tyinv\nref{\rexvarbound{\tybot}{\tyany}}}{\tyint}}
        {\typair{\tyinv\nref\va}{\tyany}} }
}
{ \subtysig{\EmptyEnv}{\EmptyEnv}
    {\typair{\tyinv\nref{\rexvarbound{\tybot}{\tyany}}}{\tyint}}
    { \tyexistnob{\va}{\typair{\tyinv\nref\va}{\tyany}} } }
\\
\end{mathpar}
\end{minipage}}
Without \RSS{InvLeft}, constrained subtyping
\[
    \subtyctrR{\EmptyEnv}{\va}
    {\typair{\tyinv\nref{\rexvarbound{\tybot}{\tyany}}}{\tyint}}
    {\typair{\tyinv\nref\va}{\tyany}}
    { \ctrset{\ctrsub{\tyany}{\va}, \ctrsub{\va}{\tybot}} }
\]
would generate unsatisfiable constraints.

Notice the absence of rules for subtyping tuples and invariant constructors:
once types are reached on both sides, the rule \RSS{Types} delegates further
checks to constrained subtyping \subtyctrRdflt{\ty}{\ty'},
followed by constraints resolution \solvectrdflt.
The algorithm $\solvectrop$, defined in \figref{fig:ctr-solve}, 
checks for consistency of all the constraints on each
unification variable, starting with the last introduced one in \UEnv.
Because variable bounds can refer to earlier-introduced variables,
generated constraints are checked for consistency with declared constraints
using constrained subtyping. The resulting constraints are then checked
recursively for the smaller environment \UEnv.
% Substitution is defined in the usual manner.

The rule \RSS{UnionLeft} may seem surprising because the two premises
\subtysigdflt{\plug\dctxsig{\tysig_1}}{\tysig'} and
\subtysigdflt{\plug\dctxsig{\tysig_2}}{\tysig'} are not required
to have consistent instantiations of \UEnv.
For example, in the following derivation,\\
\makebox[\textwidth][c]{
\begin{minipage}{\ruleswidth}
\begin{mathpar}
\small
\inferrule[]
{ 
    \inferrule[]
    { \subtysig{\EmptyEnv}{\va}{\tyinv\nref\tyint}{\tyinv\nref\va} }
    { \subtysig{\EmptyEnv}{\va}
        {\tyinv\nref\tyint}
        {\tyunion{\tyinv\nref\va}{\tyinv\nvec\va}} }
    \and
    \inferrule[]
    { \subtysig{\EmptyEnv}{\va}{\tyinv\nvec\tystr}{\tyinv\nvec\va} }
    { \subtysig{\EmptyEnv}{\va}
        {\tyinv\nvec\tystr}
        {\tyunion{\tyinv\nref\va}{\tyinv\nvec\va}} }
}
{ \subtysig{\EmptyEnv}{\va}
    {\tyunion{\tyinv\nref\tyint}{\tyinv\nvec\tystr}}
    {\tyunion{\tyinv\nref\va}{\tyinv\nvec\va}} }  
\\  
\end{mathpar}
\end{minipage}}
\va gets instantiated with \tyint in the first premise and
\tystr in the second.
This is correct because the types
\tyexistnob{\va}{(\tyunion{\tysig_1}{\tysig_2})} and
\tyunion{(\tyexistnob{\va}{\tysig_1})}{(\tyexistnob{\va}{\tysig_2})}
are semantically equivalent, which is also the case in Julia:
\begin{center}
\begin{minipage}{8cm}
\begin{lstlisting}
julia> Union{Ref{Int}, Vector{String}} <: 
    Union{Ref{T}, Vector{T}} where T
true

julia> Union{Ref{T} where T, Vector{T} where T} == 
    Union{Ref{T}, Vector{T}} where T
true
\end{lstlisting}
\end{minipage}
\end{center}

\begin{figure}
\footnotesize
\centering
\begin{minipage}{.7\linewidth}
\begin{algorithm}[H]
    \SetKwProg{SolveCtrs}{Solve}{}{}

    \SolveCtrs{$(\AEnv;\,\EmptyEnv;\,\CSet)$}{
        \KwRet{$\emptysubst$}
    }
    \BlankLine
    \SolveCtrs{$(\AEnv;\,\UEnv,\varbound{\va}{\tylb}{\tyub};\,\CSet)$}{
        $\CSet_{\va} \gets 
            \ctrset{\ctrsub{\tylb'}{\va} \,|\, \ctrsub{\tylb'}{\va} \in \CSet} 
            \cup 
            \ctrset{\ctrsub{\va}{\tyub'} \,|\, \ctrsub{\va}{\tyub'} \in \CSet} $ \;
        $\CSet' \gets \CSet \setminus \CSet_{\va}$ \;
        $\UEnvD \gets \dom{\UEnv}$\;
        \lForEach{$\ctrsub{\tylb_i}{\va}, \ctrsub{\va}{\tyub_j} \in \CSet_{\va}$}{%
            \subtydflt{\tylb_i}{\tyub_j}
        }
        \lForEach{$\ctrsub{\tylb_i}{\va} \in \CSet_{\va}$}{%
            \subtyctrRdfltenv{\tylb_i}{\tyub}{\CSet_{\tylb_i}}
        }
        \lForEach{$\ctrsub{\va}{\tyub_j} \in \CSet_{\va}$}{%
            \subtyctrLdfltenv{\tylb}{\tyub_j}{\CSet_{\tyub_j}}
        }
        $\substvars \gets \solvectr{\AEnv}{\UEnv}{
            \CSet' \mcup_i \CSet_{\tylb_i} \mcup_j \CSet_{\tyub_j}
        }$\;
        \KwRet{$\subst\substvars{\substel{\va}{
            \substvars(\tylb) \mcup_i \tylb_i
        }}$}
    }
    %$\substvars \gets \emptysubst$
\end{algorithm}  
\end{minipage}
\caption{Constraints resolution algorithm \solvectrdflt}\label{fig:ctr-solve}      
\end{figure}

\subsection{Validity of Types and Type Signatures}
%% *********************************************************

\begin{figure}
\footnotesize
\begin{mathpar}
    \fbox{\tyvlddflt{\ty}}
    \\

    \inferrule*[]
    { }
    { \tyvlddflt{\tyany} }

    \inferrule*[]
    { }
    { \tyvlddflt{\tybot} }

    \inferrule*[]
    { \varbound{\vany}{\tylb}{\tyub} \in \AEnv }
    { \tyvlddflt{\vany} }
    \\

    \inferrule*[]
    { \tyvlddflt{\ty_1} \and \tyvlddflt{\ty_2} }
    { \tyvlddflt{\typair{\ty_1}{\ty_2}} }

    \inferrule*[]
    { \iname \text{ has arity } n \and
      \forall i \in 1..n.\ \tyvlddflt{\rexvar_i} }
    { \tyvlddflt{\tyinv\iname{\rexvar_1,\ldots,\rexvar_n}} }

    \inferrule*[]
    { \tyvlddflt{\ty_1} \and \tyvlddflt{\ty_2} }
    { \tyvlddflt{\tyunion{\ty_1}{\ty_2}} }
    %
    \\
    \fbox{\tyvlddflt{\rexvar}}
    \\

    \inferrule*[]
    { \tyvlddflt{\tylb} \and \tyvlddflt{\tyub} \and 
        \subtydflt{\tylb}{\tyub} }
    { \tyvlddflt{\rexvarbound{\tylb}{\tyub}} }
    %
    \\
    \fbox{\tyvlddflt{\dctx}}
    \\

    \inferrule*[]
    { }
    { \tyvlddflt\square }

    \inferrule*[]
    { \tyvlddflt{\dctx} \and \tyvlddflt{\ty} }
    { \tyvlddflt{\typair{\dctx}{\ty}} }

    \inferrule*[]
    { \tyvlddflt{\ty} \and \tyvlddflt{\dctx} }
    { \tyvlddflt{\typair{\ty}{\dctx}} }
    %
    \\
    \fbox{\tyvlddflt{\tysig}}
    \\

    \inferrule*[]
    { }
    { \tyvlddflt{\tyany} }

    \ldots

    \inferrule*[]
    { \tyvlddflt{\tysig_1} \and \tyvlddflt{\tysig_2} }
    { \tyvlddflt{\tyunion{\tysig_1}{\tysig_2}} }

    \inferrule*[]
    { \tyvlddflt{\tylb} \and \tyvlddflt{\tyub} \and
        \subtydflt{\tylb}{\tyub} \and
        \tyvld{\AEnv, \varbound{\vany}{\tylb}{\tyub}}{\tysig} }
    { \tyvlddflt{\tyexist{\vany}{\tylb}{\tyub}{\tysig}} }
    %
    \\
    \fbox{\tyunfvlddflt{\ty}}
    \\

    \inferrule*[]
    { \varbound{\vx}{\tylb}{\tyub} \in \AEnv }
    { \tyunfvlddflt{\vx} }

    \inferrule*[]
    { \va \in \UEnvD }
    { \tyunfvlddflt{\va} }

    \inferrule*[]
    { }
    { \tyunfvlddflt{\tyany} }

    \ldots

    \inferrule*[]
    { \tyunfvlddflt{\ty_1} \and \tyunfvlddflt{\ty_2} }
    { \tyunfvlddflt{\tyunion{\ty_1}{\ty_2}} }
    %
    \\
    \fbox{\tyvld{}{\AEnv}}
    \\

    \inferrule*[]
    { }
    { \tyvld{}{\EmptyEnv} }

    \inferrule*[]
    { \tyvld{}{\AEnv} \and 
        \tyvlddflt{\tylb} \and \tyvlddflt{\tyub} \and
        \subtydflt{\tylb}{\tyub} }
    { \tyvld{}{\AEnv, \varbound{\vany}{\tylb}{\tyub}} }
\end{mathpar}
\caption{Validity of types and type signatures}\label{fig:ty-tysig-validity}
\end{figure}
% \inferrule*[]
% { }
% { \tyvlddflt{\tybot} }

% \inferrule*[]
% { \varbound{\vany}{\tylb}{\tyub} \in \AEnv }
% { \tyvlddflt{\vany} }

% \inferrule*[]
% { \iname \text{ has arity } n \and
%   \forall i \in 1..n.\ \tyvlddflt{\rexvar_i} }
% { \tyvlddflt{\tyinv\iname{\rexvar_1,\ldots,\rexvar_n}} }

% \inferrule*[]
% { \tyvlddflt{\tysig_1} \and \tyvlddflt{\tysig_2} }
% { \tyvlddflt{\typair{\tysig_1}{\tysig_2}} }

The subtyping algorithm should be called with valid types and type signatures,
as defined in \figref{fig:ty-tysig-validity}.
In particular:
\begin{itemize}
    \item \subtydflt{\ty}{\ty'} requires that \tyvlddflt{\ty, \ty'};
    \item \subtyctrRdflt{\ty}{\ty'} requires that \tyvlddflt{\ty}
        and \tyunfvlddflt{\ty'};
    \item \subtyctrLdflt{\ty}{\ty'} requires that \tyunfvlddflt{\ty}
        and \tyvlddflt{\ty'};
    \item \subtysigdflt{\tysig}{\tysig'} requires that \tyvlddflt{\tysig}
        and \tyvld{\concat{\AEnv}{\UEnv}}{\tysig'}.
\end{itemize}
The validity check ensures that free variables are bound in 
corresponding environments,
and that variable bounds are non-recursive and consistent.
%i.e., the lower bound is a subtype of the upper bound.

The consistency of declared variable bounds is necessary for transitivity.
This requirement is a departure from Julia, where types like
\[ \text{\cjl{Ref\{T\} where Any<:T<:Int}} \]
are considered valid.
Although semantically, such types denote empty sets and cannot be instantiated,
Julia does not consider them to be subtypes of \cjl{Union\{\}}:
\begin{center}
\begin{minipage}{10cm}
\begin{lstlisting}
julia> (Ref{T} where Any<:T<:Int){Int}
ERROR: TypeError: in Ref, in T, expected Any<:T<:Int

julia> (Ref{T} where Any<:T<:Int) <: Union{}
false
\end{lstlisting}
\end{minipage}
\end{center}

% t where Ref{Int}<:S<:Ref{T} where T isn't valid.
% In a purely semantic approach, the piece of the infinite union with T->Int
% would be non-empty, whereas all other instantiations of T would result in 
% the empty sets.

\section{Proofs}

\subsection{Decidability of Subtyping}\label{subsec:dec-proof}
%% ======================================================================

\begin{figure}
\footnotesize
\[
\begin{array}{lcl}
    \hline
    \multicolumn{3}{c}{\tymsrdflt{\ty}} \\ 
    \hline 
    \tymsrdflt{\tyany} &::=& 1\\
    \tymsrdflt{\tybot} &::=& 1\\
    \tymsr{\,\EmptyEnv\,}{\vany} &::=& 1\\
    \tymsr{\AEnv, \varbound{\vany}{\tylb}{\tyub}}{\vany} &::=& 
        1 + \tymsrdflt{\tylb} + \tymsrdflt{\tyub}\\
    \tymsr{\AEnv, \varbound{\vany'}{\tylb}{\tyub}}{\vany} &::=& 
        \tymsrdflt{\vany} \\
    \tymsrdflt{\typair{\ty_1}{\ty_2}} &::=& 
        1 + \tymsrdflt{\ty_1} + \tymsrdflt{\ty_2}\\
    \tymsrdflt{\tyinv\iname{\rexvar_1,\ldots,\rexvar_n}} &::=&
        1 + \tymsrdflt{\rexvar_1} + \ldots + \tymsrdflt{\rexvar_n}\\
    \tymsrdflt{\tyunion{\ty_1}{\ty_2}} &::=& 
        1 + \tymsrdflt{\ty_1} + \tymsrdflt{\ty_2}\\
    \\
    \hline
    \multicolumn{3}{c}{\tymsrdflt{\rexvar}} \\ 
    \hline 
    \tymsrdflt{\rexvarbound{\ty}{\ty}} &::=& \tymsrdflt{\ty}\\
    \tymsrdflt{\rexvarbound{\tylb}{\tyub}} &::=& 
        2\times(1 + \tymsrdflt{\tylb} + \tymsrdflt{\tyub})\\
    \\
    \hline
    \multicolumn{3}{c}{\tymsrdflt{\dctx}} \\ 
    \hline 
    \tymsrdflt{\square} &::=& 0\\
    \tymsrdflt{\typair{\dctx}{\ty}} &::=& 
        1 + \tymsrdflt{\dctx} + \tymsrdflt{\ty}\\
    \tymsrdflt{\typair{\ty}{\dctx}} &::=& 
        1 + \tymsrdflt{\ty} + \tymsrdflt{\dctx}\\
    \\
    \hline
    \multicolumn{3}{c}{\tymsrdflt{\tysig}} \\ 
    \hline 
    \tymsrdflt{\tyany} &::=& 1\\
    \multicolumn{3}{l}{\ldots} \\
    \tymsrdflt{\tyexist{\vany}{\tylb}{\tyub}{\tysig}} &::=& 
        1 + \tymsrdflt{\tylb} + \tymsrdflt{\tyub} + 
        \tymsr{\AEnv,\varbound{\vany}{\tylb}{\tyub}}{\tysig}\\
    \\
    \hline
    \multicolumn{3}{c}{\tymsrdflt{\dctxsig}} \\ 
    \hline 
    \tymsrdflt{\square} &::=& 0\\
    \tymsrdflt{\typair{\dctxsig}{\tysig}} &::=& 
        1 + \tymsrdflt{\dctxsig} + \tymsrdflt{\tysig}\\
    \tymsrdflt{\typair{\tysig}{\dctxsig}} &::=& 
        1 + \tymsrdflt{\tysig} + \tymsrdflt{\dctxsig}\\
\end{array}
\]
\caption{Measure of types and type signatures}\label{fig:ty-measure}
\end{figure}

To show the decidability of the subtyping algorithm,
we will use the measure $\msrop$ of types and type signatures,
as defined in \figref{fig:ty-measure}.
The measure function is defined recursively and is
similar to the syntactic size,
except for the treatment of type variables: for variables from~\AEnv,
the measure of a variable includes the measures of its bounds.
The definition of the measure for restricted existential variables \rexvar
reflects the fact that \tyinv\iname{\rexvar_1,\ldots,\rexvar_n} represents both
invariant constructors and restricted existential types.
When $\rexvar_i$ represents a single type $\ty_i$,
i.e. $\rexvar_i = \rexvarbound{\ty_i}{\ty_i}$,
its measure is simply the measure of the type $\ty_i$ in
\tyinv\iname{\ldots,\ty_i,\ldots}.
Otherwise, $\rexvar_i = \rexvarbound{\tylb_i}{\tyub_i}$ represents
an existential type with a single occurrence of the bound variable,
\tyexist{\vany_i}{\tylb_i}{\tyub_i}{\tyinv\iname{\ldots,\vany_i,\ldots}},
and measures accordingly, comprising both the binding and its single occurrence.

Note that the measure function itself always terminates and evaluates to a
positive integer. This is the case because for every recursive call
\tymsr{\AEnv'}{\ty'} of \tymsrdflt{\ty}, 
the combined syntactic size of the arguments $\size{\ty'} + \size{\AEnv'}$
is strictly smaller than $\size{\ty} + \size{\AEnv}$.
The same is true for recursive calls
\tymsr{\AEnv'}{\tysig'} of \tymsrdflt{\tysig}.
The syntactic size is defined in~\figref{fig:ty-size}.

\begin{figure}
\footnotesize
\[
\begin{array}{lcl}
    \hline
    \multicolumn{3}{c}{\size{\ty}} \\ 
    \hline 
    \size{\tyany} &::=& 1\\
    \size{\tybot} &::=& 1\\
    \size{\vany}  &::=& 1\\
    \size{\typair{\ty_1}{\ty_2}}  &::=& 1 + \size{\ty_1} + \size{\ty_2}\\
    \size{\tyinv\iname{\rexvar_1,\ldots,\rexvar_n}} &::=&
        1 + \size{\rexvar_1} + \ldots + \size{\rexvar_n}\\
    \size{\tyunion{\ty_1}{\ty_2}} &::=& 1 + \size{\ty_1} + \size{\ty_2}\\
    \\
    \hline
    \multicolumn{3}{c}{\size{\rexvar}} \\ 
    \hline 
    \size{\rexvarbound{\tylb}{\tyub}} &::=& \size{\tylb} + \size{\tyub}\\
    \\
    \hline
    \multicolumn{3}{c}{\size{\dctx}} \\ 
    \hline 
    \size{\square} &::=& 0\\
    \size{\typair{\dctx}{\ty}} &::=& 
        1 + \size{\dctx} + \size{\ty}\\
    \size{\typair{\ty}{\dctx}} &::=& 
        1 + \size{\ty} + \size{\dctx}\\
    \\
    \hline
    \multicolumn{3}{c}{\size{\tysig}} \\ 
    \hline 
    \size{\tyany} &::=& 1\\
    \multicolumn{3}{l}{\ldots} \\
    \size{\tyexist{\vany}{\tylb}{\tyub}{\tysig}} &::=& 
        1 + \size{\tylb} + \size{\tyub} + \size{\tysig}\\
    \\
    \hline
    \multicolumn{3}{c}{\size{\AEnv}} \\ 
    \hline 
    \size{\EmptyEnv} &::=& 0 \\
    \size{\AEnv, \varbound{\vany}{\tylb}{\tyub}} &::=& 
        \size{\AEnv} + \size{\tylb} + \size{\tyub}\\
    \\
    \hline
    \multicolumn{3}{c}{\size{\dctxsig}} \\ 
    \hline 
    \size{\square} &::=& 0\\
    \size{\typair{\dctxsig}{\tysig}} &::=& 
        1 + \size{\dctxsig} + \size{\tysig}\\
    \size{\typair{\tysig}{\dctxsig}} &::=& 
        1 + \size{\tysig} + \size{\dctxsig}\\
\end{array}
\]
\caption{Syntactic size}\label{fig:ty-size}
\end{figure}

In what follows, we will implicitly use the following facts about
distributivity contexts (proofs are straightforward by induction):
\begin{itemize}
    \item $\plug{\dctx}{\ty} = \ty';$
    \item $\plug{\dctx}{\dctx'} = \dctx'';$
    \item $\plug{\dctx}{\plug{\dctx'}{\ty}} = 
        \plug{(\plug{\dctx}{\dctx'})}{\ty};$
    \item $\subtydflt{\dctx}{\dctx'} \land \subtydflt{\ty}{\ty'}
        \implies \subtydflt{\plug{\dctx}{\ty}}{\plug{\dctx'}{\ty'}};$
    \item $\size{\plug{\dctx}{\ty}} = \size{\dctx} + \size{\ty};$
    \item $\tymsrdflt{\plug{\dctx}{\ty}} = \tymsrdflt{\dctx} + \tymsrdflt{\ty};$
    \item $\tymsrdflt{\plug{\dctxsig}{\tysig}} = 
        \tymsrdflt{\dctxsig} + \tymsrdflt{\tysig};$
    \item $\substvars(\plug\dctx\ty) = 
        \plug{\substvars(\dctx)}{\substvars(\ty)}; $    
    \item \TODO{more as needed}
\end{itemize}


\begin{theorem}{Termination of\ \ \subtydflt{\ty}{\ty'}.}%
\label{thm:subty-terminates}
    The subtyping algorithm built from the rules of subtyping of types
    \subtydflt{\ty}{\ty'} terminates.
\end{theorem}
\begin{proof}
    It follows from the fact that for each subtyping rule, 
    the measure of each premise \subtydflt{\ty_p}{\ty'_p}
    is strictly smaller than the measure 
    of the conclusion \subtydflt{\ty}{\ty'}, i.e.
    \[\tymsrdflt{\ty_p} + \tymsrdflt{\ty'_p} \quad<\quad 
    \tymsrdflt{\ty} + \tymsrdflt{\ty'}.\]

    For example, in the case \RST{VarLeft},
    \[\tymsrdflt{\plug\dctx\tyub} + \tymsrdflt{\ty'} < 
    \tymsrdflt{\plug\dctx\vany} + \tymsrdflt{\ty'}\]
    because \[\tymsrdflt{\tyub} < \tymsrdflt{\vany} = 
        1 + \tymsrdflt{\tylb} + \tymsrdflt{\tyub}.\]
\end{proof}


\begin{lemma}{Termination of join ($\sqcup_{\AEnv}$) and meet ($\sqcap_{\AEnv}$).}%
\label{lem:meet-terminates}
    $\forall \AEnv, \ty, \ty'$ both
    \tyjoindflt{\ty}{\ty'} and \tymeetdflt{\ty}{\ty'} terminate.
\end{lemma}
\begin{proof}
    \tyjoindflt{\ty}{\ty'} terminates because subtyping of types terminates
    by~\thmref{thm:subty-terminates}.

    \tymeetdflt{\ty}{\ty'} terminates because subtyping of types terminates
    and for each recursive call \tymeetdflt{\ty_r}{\ty'_r} of 
    \tymeetdflt{\ty}{\ty'}, the measure $\tymsrdflt{\ty_r} + \tymsrdflt{\ty'_r}$ 
    is strictly smaller than $\tymsrdflt{\ty} + \tymsrdflt{\ty'}$.
\end{proof}

\begin{theorem}{Termination of\ \ \subtyctrdflt{\ty}{\ty'}.}%
\label{thm:subtyctr-terminates}
    The subtyping algorithm built from the rules of
    constrained subtyping of types
    \subtyctrdflt{\ty}{\ty'} terminates.
\end{theorem}
\begin{proof}
    Similarly to the previous theorem, the measure decreases, i.e.
    \[\tymsrdflt{\ty_p} + \tymsrdflt{\ty'_p} \quad<\quad 
    \tymsrdflt{\ty} + \tymsrdflt{\ty'}\]
    for each premise \subtyctrdflt{\ty_p}{\ty'_p}
    of the conclusion \subtyctrdflt{\ty}{\ty'}.
    
    The only interesting case is \RSC{UVar-UnionRight}.
    By the variable names convention~\ref{def:var-names}, $\va \notin \AEnv$.
    Therefore, $\tymsrdflt{\va} = \tymsrdflt{\va_1} = \tymsrdflt{\va_2}$
    and the measure of the left-hand side type is the same in both premises
    and the conclusion,
    while the measure of the right-hand side type in both premises 
    is strictly smaller than in conclusion.
    Furthermore, $\msqcap_{i=1}^n \tyub_1^i$ and $\msqcap_{j=1}^m \tyub_2^j$
    terminate by~\lemref{lem:meet-terminates}.
\end{proof}

\begin{theorem}{Termination of\ \solvectrdflt.}%
\label{thm:solvectr-terminates}
    The constraints resolution algorithm \solvectrdflt terminates.
\end{theorem}
\begin{proof}
    It follows from the fact that:
    \begin{enumerate}
        \item subtyping algorithms \subtydflt{\ty}{\ty'} and 
            \subtyctrdflt{\ty}{\ty'} used to check the consistency of 
            constraints terminate;
        \item the argument \UEnv of the only recursive call to $\solvectrop$
            is strictly smaller than that of the original call.
    \end{enumerate} 
\end{proof}


\begin{figure}
\footnotesize
\[
\begin{array}{lcl}
    \hline
    \multicolumn{3}{c}{\occdflt{\ty}} \\ 
    \hline 
    \occdflt{\tyany} &::=& \false\\
    \occdflt{\tybot} &::=& \false\\
    \occdflt{\vany}  &::=& \true\\
    \occdflt{\vany'}  &::=& \false\\
    \occdflt{\typair{\ty_1}{\ty_2}}  &::=& \occdflt{\ty_1} \lor \occdflt{\ty_2}\\
    \occdflt{\tyinv\iname{\rexvar_1,\ldots,\rexvar_n}} &::=&
        \occdflt{\rexvar_1} \lor \ldots \lor \occdflt{\rexvar_n}\\
    \occdflt{\tyunion{\ty_1}{\ty_2}} &::=& \occdflt{\ty_1} \lor \occdflt{\ty_2}\\
    \\
    \hline
    \multicolumn{3}{c}{\occdflt{\rexvar}} \\ 
    \hline 
    \occdflt{\rexvarbound{\tylb}{\tyub}} &::=& \occdflt{\tylb} \lor \occdflt{\tyub}\\
    \\
    \hline
    \multicolumn{3}{c}{\occdflt{\tysig}} \\ 
    \hline 
    \occdflt{\tyany} &::=& \false\\
    \multicolumn{3}{l}{\ldots} \\
    \occdflt{\tyexist{\vany}{\tylb}{\tyub}{\tysig}} &::=& \true\\
    \occdflt{\tyexist{\vany'}{\tylb}{\tyub}{\tysig}} &::=& 
        \occdflt{\tylb} \lor \occdflt{\tyub} \lor \occdflt{\tysig}\\
    \\
    \hline
    \multicolumn{3}{c}{\occdflt{\AEnv}} \\ 
    \hline 
    \occdflt{\EmptyEnv} &::=& \false \\
    \occdflt{\AEnv, \varbound{\vany}{\tylb}{\tyub}} &::=& \true\\
    \occdflt{\AEnv, \varbound{\vany'}{\tylb}{\tyub}} &::=& 
        \occdflt{\AEnv} \lor \occdflt{\tylb} \lor \occdflt{\tyub}\\
\end{array}
\]
\caption{Occurrence of a variable}\label{fig:var-occ}
\end{figure}

\begin{lemma}{Context weakening in $\msrop$.}%
\label{lem:msr-weakening}
    The measure of a type signature does not change if the environment
    is extended (in any position) with a variable that occurs neither
    in the signature nor in the environment, i.e.,
    $\forall \tysig, \AEnv, \AEnv'. 
    \forall \varbound{\vany}{\tylb}{\tyub} \text{ s.t. } 
    \lnot \occdflt{\tysig} \land 
    \lnot \occdflt{\AEnv} \land \lnot \occdflt{\AEnv'}.$
    \[\tymsr{\concat{\AEnv}{\AEnv'}}{\tysig} = 
        \tymsr{\concat{\AEnv,\varbound{\vany}{\tylb}{\tyub}}{\AEnv'}}{\tysig},\]
    where \concat{\AEnv}{\AEnv''} denotes the concatenation of lists,
    and $\occop$ is defined in~\figref{fig:var-occ}.
\end{lemma}
\begin{proof}
    By strong induction on $n = \size{\AEnv} + \size{\AEnv'} + \size{\tysig}$.

    Case $n = 0$ is not possible, as the minimal size of a type signature is 1.
    
    In the inductive step for $n$, the induction hypothesis (IH) states that
    $\forall n'<n. \forall \tysig', \AEnv'', \AEnv'''  \text{ s.t. }
    n' = \size{\AEnv''} + \size{\AEnv'''} + \size{\tysig'}.
    \forall \varbound{\vany}{\tylb}{\tyub} \text{ s.t. } 
    \lnot \occdflt{\tysig'} \land 
    \lnot \occdflt{\AEnv''} \land \lnot \occdflt{\AEnv'''}.$
    \[\tymsr{\concat{\AEnv''}{\AEnv'''}}{\tysig'} = 
    \tymsr{\concat{\AEnv'',\varbound{\vany}{\tylb}{\tyub}}{\AEnv'''}}{\tysig'}.\]
    
    Case analysis on \tysig. Base cases \tyany and \tybot are straightforward.
    Cases $\times$, \tyinv\iname{\ldots}, and $\cup$ are also straightforward
    using the induction hypothesis for components of \tysig.
    The remaining cases are:
    \begin{itemize}
        \item Case $\vany'$. Case analysis on $\AEnv'$.
            \begin{itemize}
                \item Case \EmptyEnv. Because $\lnot \occdflt{\vany'},$ we know
                    $\vany \neq \vany'$. Thus,
                    $\tymsr{\AEnv, \varbound{\vany}{\tylb'}{\tyub'}}{\vany'} =
                    \tymsr{\AEnv}{\vany'}$ by definition of $\msrop$.
                \item Case $\AEnv', \varbound{\vany'}{\tylb'}{\tyub'}$.
                    By definition,
                    \[\tymsr{\concat{\AEnv}{\AEnv', \varbound{\vany'}{\tylb'}{\tyub'}}}{\vany'} =
                    1 + \tymsr{\concat{\AEnv}{\AEnv'}}{\tylb'} + 
                    \tymsr{\concat{\AEnv}{\AEnv'}}{\tyub'}.\]
                    Since $\size{\AEnv} + \size{\AEnv'} + \size{\tylb'}\ <\ 
                    \size{\AEnv} + \size{\AEnv', \varbound{\vany'}{\tylb'}{\tyub'}} + \size{\vany'} =
                    \size{\AEnv} + \size{\AEnv'} + \size{\tylb'} + \size{\tyub'} + 1$,
                    the IH applies with $\AEnv'' = \AEnv, \AEnv''' = \AEnv', 
                    \tysig' = \tylb'$, which gives 
                    $\tymsr{\concat{\AEnv}{\AEnv'}}{\tylb'} = 
                    \tymsr{\concat{\AEnv, \varbound{\vany}{\tylb}{\tyub}}{\AEnv'}}{\tylb'}$,
                    and similarly for $\tyub'$. Thus,
                    \[ \tymsr{\concat{\AEnv}{\AEnv', \varbound{\vany'}{\tylb'}{\tyub'}}}{\vany'} =
                    \tymsr{\concat{\AEnv, \varbound{\vany}{\tylb}{\tyub}}{\AEnv', \varbound{\vany'}{\tylb'}{\tyub'}}}{\vany'}. \]
            \end{itemize}
        \item Case \tyexist{\vany'}{\tylb'}{\tyub'}{\tysig}.
            By definition,
            \[\tymsr{\concat{\AEnv}{\AEnv'}}{\tyexist{\vany'}{\tylb'}{\tyub'}{\tysig}} =
            1 + \tymsr{\concat{\AEnv}{\AEnv'}}{\tylb'} + \msrop(\ldots\tyub') +
            \tymsr{\concat{\AEnv}{\AEnv', \varbound{\vany'}{\tylb'}{\tyub'}}}{\tysig}.\]
            Similarly to the last subcase of the $\vany'$ case, the IH applies
            to $\tylb'$ and $\tyub'$.
            Furthermore, since $\size{\AEnv} + \size{\AEnv'} + 
            \size{\tylb'} + \size{\tyub'} + \size{\tysig} <
            \size{\AEnv} + \size{\AEnv'} + 1 + \size{\tylb'} + \size{\tyub'}
            + \size{\tysig},$
            the IH applies to \tysig with $\AEnv'' = \AEnv, 
            \AEnv''' = (\AEnv', \varbound{\vany'}{\tylb'}{\tyub'}), 
            \tysig' = \tysig$.
            All pieces combined, 
            \[\tymsr{\concat{\AEnv}{\AEnv'}}{\tyexist{\vany'}{\tylb'}{\tyub'}{\tysig}} =
            \tymsr{\concat{\AEnv, \varbound{\vany}{\tylb}{\tyub}}{\AEnv'}}{\tyexist{\vany'}{\tylb'}{\tyub'}{\tysig}}.\]
            % and
            % \[\tymsr{\concat{\AEnv, \varbound{\vany}{\tylb}{\tyub}}{\AEnv'}}{\tyexist{\vany'}{\tylb'}{\tyub'}{\tysig}} =
            % 1 + \tymsr{\concat{\AEnv, \varbound{\vany}{\tylb}{\tyub}}{\AEnv'}}{\tylb'} + 
            % \tymsr{\concat{\AEnv, \varbound{\vany}{\tylb}{\tyub}}{\AEnv'}}{\tyub'} +
            % \tymsr{\concat{\AEnv, \varbound{\vany}{\tylb}{\tyub}}{\AEnv', \varbound{\vany'}{\tylb'}{\tyub'}}}{\tysig}.\]
    \end{itemize}
\end{proof}

\begin{theorem}{Termination of\ \ \subtysigdflt{\tysig}{\tysig'}.}%
\label{thm:subtysig-terminates}
    The subtyping algorithm built from the rules of
    subtyping of type signatures
    \subtysigdflt{\tysig}{\tysig'} terminates.
\end{theorem}
\begin{proof}
    It follows from the fact that for each subtyping rule, 
    the measure of each premise \subtysigdflt{\tysig_p}{\tysig'_p}
    is strictly smaller than the measure 
    of the conclusion \subtysigdflt{\tysig}{\tysig'}, i.e.
    \[\tymsr{\AEnv'}{\tysig_p} + \tymsr{\concat{\AEnv'}{\UEnv'}}{\tysig'_p} \quad<\quad 
    \tymsr{\AEnv}{\tysig} + \tymsr{\concat{\AEnv}{\UEnv}}{\tysig'}.\]

    Most of the cases are similar to the cases of~\thmref{thm:subty-terminates}
    on the termination of \subtydflt{\ty}{\ty'}.
    The remaining cases are:
    \begin{itemize}
        \item \RSS{InvLeft}. Since \vx is a fresh variable,
            by~\lemref{lem:msr-weakening} (weakening),
            $\tymsr{\AEnv, \varbound{\vx}{\tylb}{\tyub}}
                {\plug\dctxsig{\tyinv\iname{\ldots}}} = 
            \tymsrdflt{\plug\dctxsig{\tyinv\iname{\ldots}}},$ and also
            $\tymsr{\concat{\AEnv, \varbound{\vx}{\tylb}{\tyub}}{\UEnv}}{\tysig'}
            = \tymsr{\concat{\AEnv}{\UEnv}}{\tysig'}.$

            By the definition of $\msrop$,
            \[\tymsr{\AEnv, \varbound{\vx}{\tylb}{\tyub}}{\vx} =
            1 + \tymsrdflt{\tylb} + \tymsrdflt{\tyub} <
            2\times(1 + \tymsrdflt{\tylb} + \tymsrdflt{\tyub}) =
            \tymsrdflt{\rexvarbound{\tylb}{\tyub}},\]
            which concludes the case.
        \item \RSS{ExistLeft}. By the variables names
            convention~\defref{def:var-names}, \vx is different from variables
            in \AEnv, \UEnv, as well as bound variables of \dctxsig, \tysig,
            and $\tysig'$. Therefore, by~\lemref{lem:msr-weakening} (weakening),
            $\tymsr{\AEnv, \varbound{\vx}{\tylb}{\tyub}}{\dctxsig} = 
            \tymsrdflt{\dctxsig}$, and similarly for $\tysig'$.
            By the definition of $\msrop$,
            \[\tymsrdflt{\tyexist{\vx}{\tylb}{\tyub}{\tysig}} = 1 +
                \tymsrdflt{\tylb} + \tymsrdflt{\tyub} + 
                \tymsr{\AEnv, \varbound{\vx}{\tylb}{\tyub}}{\tysig},\]
            which is strictly larger than
            \tymsr{\AEnv, \varbound{\vx}{\tylb}{\tyub}}{\tysig} in the premise,
            which concludes the case.
        \item \RSS{ExistRight}. Similarly to \RSS{ExistLeft}.
        \item \RSS{Types}. The first premise,
            \subtyctrR{\AEnv}{\dom\UEnv}{\ty}{\ty'}{\CSet},
            terminates by \thmref{thm:subtyctr-terminates}.
            Since the constraints resolution \solvectrdflt terminates
            by \thmref{thm:solvectr-terminates}, the entire step also terminates.
    \end{itemize}
\end{proof}


\subsection{Properties of Subtyping of Types}%
\label{subsec:props-subty-proof}
%% ======================================================================

%We will use an implicit assumption that all types are valid
%in the given context.

\begin{theorem}{Reflexivity of subtyping of types.}\label{thm:sub-ty-refl}
    $
        \forall \ty, \AEnv, \text{ s.t. } \tyvlddflt{\ty}.\quad
        \subtydflt{\ty}{\ty}.
    $
\end{theorem}
\begin{proof}
    By induction on the structure of \ty.
    \begin{itemize}
        \item Case \tyany by \RST{Top}.
        \item Case \tybot by \RST{Bot}.
        \item Case \vany by \RST{VarRefl}.
        \item Case \typair{\ty_1}{\ty_2} by IH and \RST{Tuple}.
        \item Case \tyinv\iname{\rexvar_1,\ldots,\rexvar_n} by IH on
            $\tylb_i, \tyub_i$, and \RST{Inv}.
        \item Case \tyunion{\ty_1}{\ty_2} by IH, \RST{UnionRight},
            and \RST{UnionLeft}.
    \end{itemize}
\end{proof}

\begin{lemma}{Subtyping of \tybot implies arbitrary subtyping.}\label{lem:sub-of-bot}
    \[
    \forall \ty, \dctx_{\tybot}, \AEnv.\quad 
    \subtydflt{\ty}{\plug{\dctx_{\tybot}}\tybot}
    \quad\implies\quad
    (\forall \ty', \dctx'.\quad \subtydflt{\plug{\dctx'}{\ty}}{\ty'}).
    \]
\end{lemma}
\begin{proof}
    By induction on the derivation of 
    \subtydflt{\ty}{\plug{\dctx_{\tybot}}\tybot}.
    \begin{itemize}
        \item Case \RST{Bot}
            \subtydflt{\plug\dctx\tybot}{\plug{\dctx_{\tybot}}{\tybot}}
            where $\ty = \plug\dctx\tybot$.

            The case concludes by \RST{Bot}:
            \subtydflt{\plug{\dctx'}{\plug\dctx\tybot}}{\ty'}. 
        \item Case \RST{VarLeft}
            \subtydflt{\plug\dctx\vany}{\plug{\dctx_{\tybot}}{\tybot}}.

            By inversion, \subtydflt{\plug\dctx\tyub}{\plug{\dctx_{\tybot}}{\tybot}}.
            By IH, \subtydflt{\plug{\dctx'}{\plug\dctx\tyub}}{\ty'}.
            Thus, the case concludes by \RST{VarLeft}: 
            \subtydflt{\plug{\dctx'}{\plug\dctx\vany}}{\ty'}.
        \item Case \RST{Tuple}, subcase where
            $\dctx_{\tybot} = \typair{\dctx'_{\tybot}}{\ty'_2}$
            ($\dctx_{\tybot} = \square$ is not possible, and
            $\dctx_{\tybot} = \typair{\ty_1}{\dctx'_{\tybot}}$
            is proved analogously),
            $\ty = \typair{\ty_1}{\ty_2}$:
            \subtydflt{\typair{\ty_1}{\ty_2}}
            {\typair{\plug{\dctx'_{\tybot}}{\tybot}}{\ty'_2}}.

            By inversion, \subtydflt{\ty_1}{\plug{\dctx'_{\tybot}}{\tybot}}.
            By IH, \subtydflt{\plug{\dctx'^h}{\ty_1}}{\ty'} for all $\dctx'^h$,
            so we can take it to be \plug{\dctx'}{\typair{\square}{\ty_2}}.
            Thus, the case concludes by IH: 
            \subtydflt{\plug{\dctx'}{\typair{\ty_1}{\ty_2}}}{\ty'}.
        \item Case \RST{UnionLeft}
            \subtydflt{\plug\dctx{\tyunion{\ty_1}{\ty_2}}}{\plug{\dctx_{\tybot}}{\tybot}}
            where $\ty = \tyunion{\ty_1}{\ty_2}$.
            By inversion, 
            \subtydflt{\plug\dctx{\ty_1}}{\plug{\dctx_{\tybot}}{\tybot}} and
            \subtydflt{\plug\dctx{\ty_2}}{\plug{\dctx_{\tybot}}{\tybot}}.
            By IH, \subtydflt{\plug{\dctx'}{\plug\dctx{\ty_1}}}{\ty'} and
            \subtydflt{\plug{\dctx'}{\plug\dctx{\ty_2}}}{\ty'}.
            Thus, the case concludes by \RST{UnionLeft}: 
            \subtydflt{\plug{\dctx'}{\plug\dctx{\tyunion{\ty_1}{\ty_2}}}}{\ty'}.
    \end{itemize}
    The remaining cases 
    (\RST{Top}, \RST{VarRefl}, \RST{VarRight}, \RST{Inv}, \RST{UnionRight}) 
    are not possible.
\end{proof}

\begin{lemma}{Subtyping of inner union on the right.}%
\label{lem:sub-inner-union-right}
    $\forall \ty, \dctx', \ty'_1, \ty'_2, \AEnv, \text{ s.t. }
    \tyvlddflt{\ty, \dctx', \ty'_1, \ty'_2}.$
    \[
        \begin{array}{ccc}
        \subtydflt{\ty}{\plug{\dctx'}{\tyunion{\ty'_1}{\ty'_2}}}\\
        \quad\implies\quad\\
        (\forall \dctx_1, \dctx_2, \text{ s.t. }
        \tyvlddflt{\dctx_1, \dctx_2} \land
        \subtydflt{\dctx_1}{\dctx_2}.\quad
        \subtydflt
            {\plug{\dctx_1}{\ty}}
            {\tyunion
                {\plug{\dctx_2}{\plug{\dctx'}{\ty'_1}}}
                {\plug{\dctx_2}{\plug{\dctx'}{\ty'_2}}}
            }).
        \end{array}
    \]
\end{lemma}
\begin{proof}
    By induction on the derivation of
    \subtydflt{\ty}{\plug{\dctx'}{\tyunion{\ty'_1}{\ty'_2}}}.
    \begin{itemize}
        \item Case \RST{Bot} by \RST{Bot}.
        \item Case \RST{VarLeft} by inversion, IH, and \RST{VarLeft}.
        \item Case \RST{Tuple}, subcase where
            $\dctx' = \typair{\dctx''}{\ty'}$:
            \subtydflt{\typair{\ty_1}{\ty_2}}
                {\typair{\plug{\dctx''}{\tyunion{\ty'_1}{\ty'_2}}}{\ty'}}.
            By inversion,
            \subtydflt{\ty_1}{\plug{\dctx''}{\tyunion{\ty'_1}{\ty'_2}}} and
            \subtydflt{\ty_2}{\ty'}. 
            
            By IH applied to 
            \subtydflt{\ty_1}{\plug{\dctx''}{\tyunion{\ty'_1}{\ty'_2}}},
            \subtydflt{\plug{\dctx^h_1}{\ty_1}}
                {\tyunion
                    {\plug{\dctx^h_2}{\plug{\dctx''}{\ty'_1}}}
                    {\plug{\dctx^h_2}{\plug{\dctx''}{\ty'_2}}}
                }
            for all $\dctx^h_1, \dctx^h_2$ s.t. $\subtydflt{\dctx^h_1}{\dctx^h_2}$.
            Thus, we can take them to be \plug{\dctx_2}{\typair{\square}{\ty_2}} 
            and \plug{\dctx_2}{\typair{\square}{\ty'}}, 
            respectively, which concludes the case with
            \subtydflt{\plug{\dctx_1}{\typair{\ty_1}{\ty_2}}}
                {\tyunion
                    {\plug{\dctx_2}{\typair{\plug{\dctx''}{\ty'_1}}{\ty_2}}}
                    {\plug{\dctx_2}{\typair{\plug{\dctx''}{\ty'_2}}{\ty'}}}
                }
        \item Case \RST{UnionLeft} by inversion, IH, and \RST{UnionLeft}.
        \item Case \RST{UnionRight}, subcase $i = 1$ where $\dctx' = \square$:
            \subtydflt{\ty}{\tyunion{\ty'_1}{\ty'_2}}.
            By inversion, \subtydflt\ty{\ty'_1}.
            By assumption, \subtydflt{\dctx_1}{\dctx_2}, and thus,
            \subtydflt{\plug{\dctx_1}{\ty}}{\plug{\dctx_2}{\ty'_1}}.
            The case concludes by \RST{UnionRight} with $i=1$:
            \subtydflt{\plug{\dctx_1}{\ty}}
                {\tyunion{\plug{\dctx_2}{\ty'_1}}{\plug{\dctx_2}{\ty'_2}}}.
    \end{itemize}
    The remaining cases 
    (\RST{Top}, \RST{VarRefl}, \RST{VarRight}, \RST{Inv}) 
    are not possible.
\end{proof}

\begin{lemma}{Adding inner union on the right.}%
\label{lem:add-inner-union-right}
    $\forall \ty, \dctx', \ty', \AEnv, \text{ s.t. }
    \tyvlddflt{\ty, \dctx', \ty'}.$
    \[
        \subtydflt{\ty}{\plug{\dctx'}{\ty'}}
        \quad\implies\quad
        (\forall \ty''.\ \subtydflt{\ty}{\plug{\dctx'}{\tyunion{\ty'}{\ty''}}}).
    \]
\end{lemma}
\begin{proof}
    By induction on the derivation of
    \subtydflt{\ty}{\plug{\dctx'}{\ty'}}.
    \begin{itemize}
        \item Case \RST{Top} where $\dctx'=\square$. By assumption
            \subtydflt{\ty}{\tyany} and \RST{UnionRight} with $i=1$,
            \subtydflt{\ty}{\tyunion{\tyany}{\ty''}}.
        \item Case \RST{Bot} by \RST{Bot}.
        \item Case \RST{VarRefl} where $\dctx'=\square$
            by assumption and \RST{UnionRight} with $i=1$.
        \item Case \RST{VarLeft} by inversion, IH, and \RST{VarLeft}.
        \item Case \RST{VarRight} where $\dctx'=\square$
            by assumption and \RST{UnionRight} with $i=1$.
        \item Case \RST{Tuple}. 
            Subcase $\dctx'$ by assumption and \RST{UnionRight} with $i=1$.
            The other two subcases by inversion, IH, and \RST{Tuple}.
        \item Case \RST{Inv} where $\dctx'=\square$
            by assumption and \RST{UnionRight} with $i=1$.
        \item Case \RST{UnionLeft} by inversion, IH, and \RST{UnionLeft}.
        \item Case \RST{UnionRight} where $\dctx'=\square$
            by assumption and \RST{UnionRight} with $i=1$.
    \end{itemize}
\end{proof}

\begin{lemma}{Subtyping of union on the right.}%
\label{lem:sub-union-right}
    $\forall \ty, \dctx', \ty'_1, \ty'_2, \AEnv, \text{ s.t. }
    \tyvlddflt{\ty, \dctx', \ty'_1, \ty'_2}.$
    \[
        \subtydflt{\ty}{\tyunion{\plug{\dctx'}{\ty'_1}}{\plug{\dctx'}{\ty'_2}}}
        \quad\implies\quad
        \subtydflt{\ty}{\plug{\dctx'}{\tyunion{\ty'_1}{\ty'_2}}}.
    \]
\end{lemma}
\begin{proof}
    By induction on the derivation of
    \subtydflt{\ty}{\tyunion{\plug{\dctx'}{\ty'_1}}{\plug{\dctx'}{\ty'_2}}}.
    Four cases are possible: \RST{Bot}, \RST{VarLeft}, \RST{UnionLeft}, and
    \RST{UnionRight}. The first three are analogous to the cases of
    \lemref{lem:sub-inner-union-right} (subtyping inner union on the right): 
    inversion, IH, constructor.
    The remaining case is \RST{UnionRight}, subcase $i=1$.
    By inversion, \subtydflt{\ty}{\plug{\dctx'}{\ty'_1}}.
    Thus, the case concludes by \lemref{lem:add-inner-union-right} (adding inner
    union on the right):
    \subtydflt{\ty}{\plug{\dctx'}{\tyunion{\ty'_1}{\ty'_2}}}.
    % bot by bot
    % VarLeft by inversion, IH, VarLeft
    % UnionLeft by inversion, IH, UnionLeft
\end{proof}

\begin{lemma}{Context weakening in subtyping of types.}%
\label{lem:subty-weakening}
    $\forall \AEnv, \ty, \ty'.\ \forall \AEnv' \text{ s.t. }$\\
    $\dom{\AEnv'} \cap \dom{\AEnv} = \varnothing \land 
    (\forall \vany \in \dom{\AEnv'}. 
        \lnot \occ{\vany}{\ty} \land \lnot \occ{\vany}{\ty}),$
    \[ \subtydflt{\ty}{\ty'} \quad\implies\quad 
    \subty{\concat{\AEnv}{\AEnv'}}{\ty}{\ty'}. \]
\end{lemma}
\begin{proof}
    Straightforward induction on the derivation of \subtydflt{\ty}{\ty'}.
\end{proof}


\begin{theorem}{Transitivity of subtyping of types.}\label{thm:sub-ty-trans}
    $\forall \ty_1, \ty_2, \ty_3, \AEnv, \text{ s.t. }
    \tyvlddflt{\ty_1, \ty_2, \ty_3} \land \tyvld{\,}{\AEnv}.$
    \[
        \subtydflt{\ty_1}{\ty_2} \land \subtydflt{\ty_2}{\ty_3}
        \quad\implies\quad
        \subtydflt{\ty_1}{\ty_3}.
    \]
\end{theorem}
\begin{proof}
    By strong induction on
    $n = \tymsrdflt{\ty_1} + 2\times\tymsrdflt{\ty_2} + \tymsrdflt{\ty_3}$.
    Cases $n = 1..3$ are not possible as the minimal measure of a type is 1.
    In the inductive step for $n$, the induction hypothesis (IH) states that
    $\forall n'<n. \forall \ty'_1, \ty'_2, \ty'_3 \text{ s.t. }
    n' = \tymsrdflt{\ty'_1} + 2\times\tymsrdflt{\ty'_2} + \tymsrdflt{\ty'_3},$
    it holds that
    \[
        \subtydflt{\ty'_1}{\ty'_2} \land \subtydflt{\ty'_2}{\ty'_3}
        \quad\implies\quad
        \subtydflt{\ty'_1}{\ty'_3}.
    \]
    Case analysis on \subtydflt{\ty_2}{\ty_3} (the most interesting cases are
    highlighted in bold).
    \begin{itemize}
        \item Case \RST{Top} \subtydflt{\ty_2}{\tyany} where $\ty_3 = \tyany$
            concludes by \RST{Top}: \subtydflt{\ty_1}{\tyany}.
        \item Case \RSS{Bot} \subtydflt{\plug\dctx\tybot}{\ty_3}
            where $\ty_2=\plug\dctx\tybot$.
            The case concludes by \lemref{lem:sub-of-bot}
            (subtyping of \tybot) applied
            to the assumption \subtydflt{\ty_1}{\plug\dctx\tybot}
            with $\dctx' = \square, \ty' = \ty_3$:
            \subtydflt{\ty_1}{\ty_3}.
        \item Case \RST{VarRefl} \subtydflt{\vany}{\vany}.
            Since $\ty_2=\ty_3=\vany$, the case concludes by the assumption
            \subtydflt{\ty_1}{\vany}.
        \item \textbf{Case \RST{VarLeft}} \subtydflt{\plug\dctx\vany}{\ty_3} 
            where $\ty_2 = \plug\dctx\vany$.
            By inversion, $\varbound{\vany}{\tylb}{\tyub} \in \AEnv$ and
            \subtydflt{\plug\dctx{\tyub}}{\ty_3}.
            By assumption, \subtydflt{\ty_1}{\plug\dctx\vany}.
            We will need the following auxiliary fact.

            \begin{lemma}\label{lem:sub-var-right-sub-ub}
                \subtydflt{\ty_1}{\plug\dctx\tyub}.
            \end{lemma}
            \begin{proof}
                By induction on \subtydflt{\ty_1}{\plug\dctx\vany}.
                \begin{itemize}
                    \item Case \RST{Bot} by \RST{Bot}.
                    \item Case \RST{VarRefl} \subtydflt{\vany}{\vany}.
                        By \thmref{thm:sub-ty-refl} (reflexivity),
                        \subtydflt{\tyub}{\tyub}.
                        Thus, by \RST{VarLeft}, \subtydflt{\vany}{\tyub}.
                    \item Case \RST{VarLeft} 
                        \subtydflt{\plug{\dctx'}{\vany'}}{\plug\dctx\vany}.
                        By inversion, $\varbound{\vany'}{\tylb'}{\tyub'} \in \AEnv$
                        and \subtydflt{\plug{\dctx'}{\tyub'}}{\plug\dctx\vany}.
                        By IH,
                        \subtydflt{\plug{\dctx'}{\tyub'}}{\plug\dctx\tyub}.
                        Thus, by \RST{VarLeft},
                        \subtydflt{\plug{\dctx'}{\vany'}}{\plug\dctx\tyub}.
                    \item \textbf{Case \RST{VarRight}} \subtydflt{\ty_1}{\vany}.
                        By inversion, $\varbound{\vany}{\tylb}{\tyub} \in \AEnv$
                        and \subtydflt{\ty_1}{\tylb}.
                        By inversion of the assumptions \tyvlddflt{\vany} and
                        \tyvld{\,}{\AEnv} and by~\lemref{lem:subty-weakening}
                        (context weakening), we have \subtydflt{\tylb}{\tyub}.

                        Since $\ty_2 = \vany$ and $\tymsrdflt{\vany} = 
                        1 + \tymsrdflt{\tylb} + \tymsrdflt{\tyub}$, we have
                        $\tymsrdflt{\ty_1} + 2\times\tymsrdflt{\tylb} + 
                        \tymsrdflt{\tyub} < \tymsrdflt{\ty_1} + 
                        2\times\tymsrdflt{\vany} + \tymsrdflt{\ty_3}$. Thus,
                        IH for \emph{transitivity} is applicable to 
                        \subtydflt{\ty_1}{\tylb} and \subtydflt{\tylb}{\tyub},
                        which concludes the case with \subtydflt{\ty_1}{\tyub}.
                    \item Case \RST{Tuple}, subcase
                        $\dctx = \typair{\dctx'}{\ty_{22}}$
                        ($\dctx = \square$ is not possible, and
                        $\dctx = \typair{\ty_{21}}{\dctx'}$
                        is proved analogously), i.e.
                        \subtydflt
                            {\typair{\ty_{11}}{\ty_{12}}}
                            {\typair{\plug{\dctx'}\vany}{\ty_{22}}}.
                        By inversion, \subtydflt{\ty_{11}}{\plug{\dctx'}\vany}
                        and \subtydflt{\ty_{12}}{\ty_{22}}.
                        By IH, \subtydflt{\ty_{11}}{\plug{\dctx'}\tyub}.
                        Thus, by \RST{Tuple},
                        \subtydflt
                            {\typair{\ty_{11}}{\ty_{12}}}
                            {\typair{\plug{\dctx'}\tyub}{\ty_{22}}}.
                    \item Case \RST{UnionLeft} is proved analogously 
                        to \RST{Tuple}: by inversion, IH, and \RST{UnionLeft}.
                \end{itemize}
                The remaining cases
                (\RST{Top}, \RST{Tuple}, \RST{Inv}, \RST{UnionRight}) 
                are not possible.
            \end{proof}

            Be \lemref{lem:sub-var-right-sub-ub} above,
            \subtydflt{\ty_1}{\plug\dctx\tyub}.
            Since $\ty_2 = \plug\dctx\vany$ and 
            $\tymsrdflt{\tyub} < \tymsrdflt{\vany}$,
            IH is applicable to \subtydflt{\ty_1}{\plug\dctx\tyub} and 
            \subtydflt{\plug\dctx\tyub}{\ty_3}, which concludes the case with
            \subtydflt{\ty_1}{\ty_3}.
        \item Case \RST{VarRight} \subtydflt{\ty_2}{\vany}.
            By inversion, $\varbound{\vany}{\tylb}{\tyub} \in \AEnv$ and
            \subtydflt{\ty_2}{\tylb}.
            Since $\tymsrdflt{\tylb} < \tymsrdflt{\vany}$, by IH,
            \subtydflt{\ty_1}{\tylb}.
            Thus, by \RST{VarRight}, \subtydflt{\ty_1}{\vany}.
        \item Case \RST{Tuple} 
            \subtydflt{\typair{\ty_{21}}{\ty_{22}}}{\typair{\ty_{31}}{\ty_{32}}}
            where $\ty_i = \typair{\ty_{i1}}{\ty_{i2}}$.
            Case analysis on \subtydflt{\ty_1}{\typair{\ty_{21}}{\ty_{22}}}.
            \begin{itemize}
                \item Case \RST{Bot} by \RST{Bot}.
                \item Case \RST{VarLeft} by inversion, IH on 
                    \subtydflt{\plug\dctx\tyub}{\ty_2} and
                    \subtydflt{\ty_2}{\ty_3}, and \RST{VarLeft} on
                    \subtydflt{\plug\dctx\tyub}{\ty_3}.
                \item Case \RST{Tuple} by inversion, IH on
                    \subtydflt{\ty_{1j}}{\ty_{2j}} and 
                    \subtydflt{\ty_{2j}}{\ty_{3j}}, and \RST{Tuple}.
                \item Case \RST{UnionLeft} 
                    \subtydflt{\plug\dctx{\tyunion{\ty_{11}}{\ty_{12}}}}{\ty_2}
                    by inversion, IH on
                    \subtydflt{\plug\dctx{\ty_{1j}}}{\ty_2} and
                    \subtydflt{\ty_2}{\ty_3}, and \RST{UnionLeft}.
            \end{itemize}
            The remaining cases
            (\RST{Top}, \RST{VarRefl}, \RST{VarRight}, \RST{Inv}, \RST{UnionRight}) 
            are not possible.
        \item Case \RST{Inv} is proved similarly to \RST{Tuple}, with
            possible cases of \subtydflt{\ty_1}{\ty_2} being
            \RST{Bot}, \RST{VarLeft}, \RST{Inv}, and \RST{UnionLeft}.
        \item \textbf{Case \RST{UnionLeft}} 
            \subtydflt{\plug\dctx{\tyunion{\ty_{21}}{\ty_{22}}}}{\ty_3}.
            By \lemref{lem:sub-inner-union-right} applied to
            \subtydflt{\ty_1}{\plug\dctx{\tyunion{\ty_{21}}{\ty_{22}}}}
            with $\dctx_1=\square,\dctx_2=\square$,
            \subtydflt{\ty_1}{\tyunion{\plug\dctx{\ty_{21}}}{\plug\dctx{\ty_{22}}}}.
            Case analysis on the latter.
            \begin{itemize}
                \item Case \RST{Bot} by \RST{Bot}.
                \item Case \RST{VarLeft} where $\ty_1 = \plug{\dctx'}\vany$.
                    By inversion, $\varbound{\vany}{\tylb}{\tyub} \in \AEnv$ and
                    \subtydflt{\plug{\dctx'}\tyub}
                    {\tyunion{\plug\dctx{\ty_{21}}}{\plug\dctx{\ty_{22}}}}.
                    By \lemref{lem:sub-union-right} applied to the latter,
                    \subtydflt{\plug{\dctx'}\tyub}
                    {\plug\dctx{\tyunion{\ty_{21}}{\ty_{22}}}}.

                    Since $\tymsrdflt{\tyub} < \tymsrdflt{\vany}$, by IH,
                    \subtydflt{\plug{\dctx'}\tyub}{\ty_3}.
                    Thus, the case concludes by \RST{VarLeft}:
                    \subtydflt{\plug{\dctx'}\vany}{\ty_3}.
                \item Case \RST{UnionLeft} similarly to \RST{VarLeft}.
                \item Case \RST{UnionRight}, subcase $i=1$.
                    By inversion, \subtydflt{\ty_1}{\plug\dctx{\ty_{21}}}.
                    By inversion of the outer case assumption
                    \subtydflt{\plug\dctx{\tyunion{\ty_{21}}{\ty_{22}}}}{\ty_3},
                    \subtydflt{\plug\dctx{\ty_{21}}}{\ty_3}.
                    Since $\tymsrdflt{\plug\dctx{\ty_{21}}} < 
                    \tymsrdflt{\plug\dctx{\tyunion{\ty_{21}}{\ty_{22}}}}$,
                    by IH, \subtydflt{\ty_1}{\ty_3}.
            \end{itemize}
            The remaining cases
            (\RST{Top}, \RST{VarRefl}, \RST{VarRight}, \RST{Tuple}, \RST{Inv}) 
            are not possible.
        \item Case \RST{UnionRight} 
            \subtydflt{\ty_2}{\tyunion{\ty_{31}}{\ty_{32}}} where 
            $\ty_3 = \tyunion{\ty_{31}}{\ty_{32}}$, subcase $i=1$.
            By inversion, \subtydflt{\ty_2}{\ty_{31}}. Since
            $\tymsrdflt{\ty_{31}} < \tymsrdflt{\tyunion{\ty_{31}}{\ty_{32}}}$,
            by IH, \subtydflt{\ty_1}{\ty_{31}}.
            Thus, the case concludes by \RST{UnionRight}.
    \end{itemize}
\end{proof}

%% This fact we use implicitly
% \begin{lemma}{Preservation of subtyping by distributivity context.}%
%     \[
%         \forall \AEnv, \dctx, \ty, \ty'.\quad
%         \subtydflt{\ty}{\ty'} \quad\implies\quad
%         \subtydflt{\plug\dctx\ty}{\plug\dctx{\ty'}}.
%      \]
% \end{lemma}
% \begin{proof}
%     Straightforward by induction on the structure of \dctx,
%     using the assumption \subtydflt{\ty}{\ty'} in the base case
%     and \thmref{thm:sub-ty-refl} (reflexivity) in the inductive cases.
% \end{proof}

\begin{theorem}{Soundness of subtyping of types with respect to subsitution.}%
\label{thm:subty-sound-subst}
    $\forall \AEnv, \ty, \ty' \text{ s.t. } \tyvlddflt{\ty, \ty'}.\ 
     \forall \AEnv', \substvars \text{ s.t. } \vldinenvdflt{\substvars}.$
     \[ 
        \subtydflt{\ty}{\ty'} \quad\implies\quad
        \subty{\AEnv'}{\substvars(\ty)}{\substvars(\ty')}.
     \]
\end{theorem}
\begin{proof}
    By induction on the derivation of \subtydflt{\ty}{\ty'}.

    Cases \RST{Top} and \RST{Bot} are straightforward by \RST{Top} and
    \RST{Bot}, respectively. 

    Case \RST{VarRefl} by \thmref{thm:sub-ty-refl} (reflexivity):
    \subty{\AEnv'}{\substvars(\vany)}{\substvars(\vany)}.

    Cases \RST{Tuple}, \RST{Inv}, \RST{UnionLeft}, and \RST{UnionRight} are
    straightforward using inversion, the induction hypothesis, and constructor.
    For example, consider the case \RST{Tuple}
    \subtydflt{\typair{\ty_1}{\ty_2}}{\typair{\ty'_1}{\ty'_2}}.
    By inversion, \subtydflt{\ty_i}{\ty'_i}.
    By IH, \subty{\AEnv'}{\substvars(\ty_i)}{\substvars(\ty'_i)}.
    The case concludes by \RST{Tuple} with 
    \[\subty{\AEnv'}{\typair{\substvars(\ty_1)}{\substvars(\ty_2)}}
    {\typair{\substvars(\ty'_1)}{\substvars(\ty'_2)}}\]
    and the fact that $\substvars(\typair{\ty_1}{\ty_2}) = 
    \typair{\substvars(\ty_1)}{\substvars(\ty_2)}.$

    The remaining cases \RST{VarLeft} and \RST{VarRight} are similar.
    For example, consider \RST{VarLeft} \subtydflt{\plug{\dctx}\vany}{\ty'}.
    By inversion, $\varbound{\vany}{\tylb}{\tyub} \in \AEnv$ and
    \subtydflt{\plug\dctx{\tyub}}{\ty'}.
    By IH, \subty{\AEnv'}{\substvars(\plug\dctx{\tyub})}{\substvars(\ty')}.
    By inversion of the assumption \vldinenvdflt{\substvars}, we know
    \subty{\AEnv'}{\substvars(\vany)}{\substvars(\tyub)}.
    Since $\substvars(\plug\dctx{\tyub}) = 
    \plug{\substvars(\dctx)}{\substvars(\tyub)}$ and
    \subty{\AEnv'}{\substvars(\dctx)}{\substvars(\dctx)} by reflexivity,
    we have \subty{\AEnv'}{\plug{\substvars(\dctx)}{\substvars(\vany)}}
    {\plug{\substvars(\dctx)}{\substvars(\tyub)}}.
    The case concludes by \thmref{thm:sub-ty-trans} (transitivity) with the
    middle type $\substvars(\plug\dctx{\tyub})$:
    \[\subty{\AEnv'}{\substvars(\plug\dctx{\vany})}{\substvars(\ty')}.\]
\end{proof}


\subsection{Properties of Join and Meet}%
\label{subsec:props-join-meet-proof}
%% ======================================================================

\begin{theorem}{Soundness of join.}%
\label{thm:join-sound}
    Join produces a valid upper bound, i.e., 
    $\forall \AEnv, \ty, \ty'$ s.t. \tyvlddflt{\ty, \ty'}.
    \[
        \tyvlddflt{\tyjoindflt{\ty}{\ty'}}
        \quad\text{and}\quad
        \subtydflt{\ty}{(\tyjoindflt{\ty}{\ty'})}
        \quad\text{and}\quad
        \subtydflt{\ty'}{(\tyjoindflt{\ty}{\ty'})}.
    \]
    Furthermore, \tyjoindflt{\ty}{\ty'} is the least upper bound
    of \ty and $\ty'$, i.e.,
    $\forall \tyub$ s.t. \tyvlddflt{\tyub},
    \subtydflt{\ty}{\tyub} and \subtydflt{\ty'}{\tyub}.
    \[
        \subtydflt{(\tyjoindflt{\ty}{\ty'})}{\tyub}.
    \]
\end{theorem}
\begin{proof}
    Straightforward by the definition of $\sqcup_{\AEnv}$,
    using \thmref{thm:sub-ty-refl} (reflexivity), 
    \RST{UnionLeft}, and \RST{UnionRight}.
\end{proof}

\begin{lemma}{Symmetry of join.}
    $\forall \AEnv, \ty, \ty'$ s.t. \tyvlddflt{\ty, \ty'}.
    \[
        \subtydflt{(\tyjoindflt{\ty}{\ty'})}{(\tyjoindflt{\ty'}{\ty})}
        \quad\text{and}\quad
        \subtydflt{(\tyjoindflt{\ty'}{\ty})}{(\tyjoindflt{\ty}{\ty'})}.
    \]
\end{lemma}
\begin{proof}
    Straightforward by the definition of $\sqcup_{\AEnv}$,
    using \thmref{thm:sub-ty-refl} (reflexivity), 
    \RST{UnionLeft}, and \RST{UnionRight}.
\end{proof}

\begin{theorem}{Soundness of meet.}%
\label{thm:meet-sound}
    Meet produces a valid lower bound, i.e., 
    $\forall \AEnv, \ty, \ty'$ s.t. \tyvlddflt{\ty, \ty'}.
    \[
        \tyvlddflt{\tymeetdflt{\ty}{\ty'}}
        \quad\text{and}\quad
        \subtydflt{(\tymeetdflt{\ty}{\ty'})}{\ty}
        \quad\text{and}\quad
        \subtydflt{(\tymeetdflt{\ty}{\ty'})}{\ty'}.
    \]
\end{theorem}
\begin{proof}
    Straightforward by strong induction on
    $\tymsrdflt{\ty} + \tymsrdflt{\ty'}$.

    For example, the case where \subtydflt{\ty}{\ty'} follows from
    the assumptions and reflexivity \subtydflt{\ty}{\ty},
    and the catch-all case $\tymeetdflt{\ty}{\ty'} = \tybot$
    follows from \RST{Bot}.

    In the case \tymeetdflt{(\tyunion{\ty_1}{\ty_2})}{\ty'},
    we know by IH that \tymeetdflt{\ty_1}{\ty'} and \tymeetdflt{\ty_2}{\ty'}
    are valid lower bounds of $\ty_1,\ty'$ and $\ty_2,\ty'$, respectively.
    Thus, the case concludes by \RST{UnionLeft}.

    In the case \tymeetdflt{\vany}{\ty'}, we have \subtydflt{\tylb}{\vany}
    by \thmref{thm:sub-ty-refl} (reflexivity) and \RST{VarRight}.
    Then, by IH, \subtydflt{\tymeetdflt{\tylb}{\ty'}}{\tylb}.
    The case concludes by \thmref{thm:sub-ty-trans} (transitivity):
    \subtydflt{\tymeetdflt{\tylb}{\ty'}}{\vany}.
    
    The remaining cases for tuples and invariant constructors
    use the induction hypotheses and constructors \RST{Tuple} and \RST{Inv},
    respectively, as well as \thmref{thm:join-sound} (soundness of join)
    in the invariant case.
\end{proof}

\begin{lemma}{Symmetry of meet.}
    $\forall \AEnv, \ty, \ty'$ s.t. \tyvlddflt{\ty, \ty'}.
    \[
        \subtydflt{(\tymeetdflt{\ty}{\ty'})}{(\tymeetdflt{\ty'}{\ty})}
        \quad\text{and}\quad
        \subtydflt{(\tymeetdflt{\ty'}{\ty})}{(\tymeetdflt{\ty}{\ty'})}.
    \]
\end{lemma}
\begin{proof}
    Straightforward by strong induction on
    $\tymsrdflt{\ty} + \tymsrdflt{\ty'}$,
    using the definition of $\sqcap_{\AEnv}$,
    \thmref{thm:sub-ty-refl} (reflexivity),
    and \RST{} constructors.
\end{proof}



\subsection{Properties of Constrained Subtyping of Types}%
\label{subsec:props-subtyctr-proof}
%% ======================================================================

\begin{lemma}{Constrained subtyping coincides with subtyping on
    unification-free types.}%
\label{lem:subtyctr-subty}
    $\forall \AEnv, \ty, \ty' \text{ s.t. } 
    \tyvld{}{\AEnv} \ \land\ \tyvlddflt{\ty} \ \land\ \tyvlddflt{\ty'}$ 
    the following holds:
    \begin{enumerate}
        \item $\forall \UEnvD.\ \subtyctrdflt{\ty}{\ty'} \quad\implies\quad
            \CSet = \EmptyCSet\ \land\ \subtydflt{\ty}{\ty'};$
        \item $\subtydflt{\ty}{\ty'} \quad\implies\quad 
            \forall \UEnvD.\ \subtyctrdfltenv{\ty}{\ty'}{\EmptyCSet}.$
    \end{enumerate}
\end{lemma}
\begin{proof}
    Straightforward by induction on the derivation of:
    \begin{enumerate}
        \item \subtyctrdflt{\ty}{\ty'} (more precisely, by mutual induction
            on \subtyctrLdflt{\ty}{\ty'} and \subtyctrRdflt{\ty}{\ty'});
        \item \subtydflt{\ty}{\ty'}.
    \end{enumerate}

    In the first case, the rules \RSC{UBot}, \RSC{UVarLeft}, \RSC{UVarRight},
    and \RSC{UVar-UnionRight} could not have been used to build the derivation
    because they refer a unification variable \va from \UEnvD,
    and both \ty and $\ty'$ are valid in \AEnv alone.
    All other rules of constrained subtyping have matching subtyping rules.

    In the second case, all subtyping rules have matching rules of constrained
    subtyping.
    The assumption \tyvld{}{\AEnv} and \lemref{lem:subty-weakening} (context
    weakening) allow for concluding 
    \tyvlddflt{\plug\dctx\tyub} and \tyvlddflt{\tylb} 
    to apply the induction hypothesis in the rules
    \RSC{VarLeft}/\RST{VarLeft} and \RSC{VarRight}/\RST{VarRight}.
\end{proof}

\begin{theorem}{Soundness of constrained subtyping.}%
\label{thm:subtyctr-sound}
    $\forall \AEnv, \UEnvD, \ty, \ty' \text{ s.t. } \tyvld{}{\AEnv},$ if
    \[\subtyctrRdflt{\ty}{\ty'} \ \land\ 
        \tyvlddflt{\ty} \ \land\  \tyunfvlddflt{\ty'}\] 
    or
    \[\subtyctrLdflt{\ty}{\ty'} \ \land\ 
        \tyunfvlddflt{\ty} \ \land\  \tyvlddflt{\ty'},\]
    then $\forall \substvars \text{ s.t. } 
        \dom{\substvars} \supseteq \UEnvD \ \land\ 
        \dom{\substvars} \cap \dom{\AEnv} = \varnothing \ \land\ 
        \vldinenv{\AEnv}{\CSet}{\substvars}.$
     \[ 
        \subtydflt{\substvars(\ty)}{\substvars(\ty')}.
     \]
\end{theorem}
\begin{proof}
    By strong induction on $\tymsrdflt{\ty} + \tymsrdflt{\ty'}.$
    \begin{itemize}
        \item Case \RSC{Top} \subtyctrdfltenv{\ty}{\tyany}{\EmptyCSet}.
            By definition, $\substvars(\tyany) = \tyany$.
            Thus, the case concludes by \RST{Top}:
            \subtydflt{\substvars(\ty)}{\tyany}.
        \item Case \RSC{Bot} by \RST{Bot}, similarly to \RSC{Top}.
        \item Case \RSC{UBot}
            \subtyctrLdfltenv{\plug\dctx\va}{\ty'}{\ctrsngl\va\tybot}.
            Since $\substvars(\plug\dctx\va) = 
                \plug{\substvars(\dctx)}{\substvars(\va)} = 
                \plug{\dctx'}{\substvars(\va)}$ for some $\dctx'$
            and by assumption, \subtydflt{\substvars(\va)}{\tybot},
            we have \subtydflt{\plug{\dctx'}{\substvars(\va)}}{\plug{\dctx'}\tybot}.
            By \RST{Bot}, \subtydflt{\plug{\dctx'}\tybot}{\substvars(\ty')}.
            Therefore, the case concludes by transitivity:
            \subtydflt{\plug{\dctx'}{\substvars(\va)}}{\substvars(\ty')}.
        \item Case \RSC{VarRefl} by \RST{VarRefl}, for $\substvars(\vx) = \vx.$
        \item Case \RSC{UVarLeft} 
            \subtyctrLdfltenv{\va}{\ty'}{\ctrsngl{\va}{\ty'}} by assumption
            \subtydflt{\substvars(\va)}{\ty'}, since $\substvars(\ty') = \ty'$
            due to \tyvlddflt{\ty'}.
        \item Case \RSC{UVarRight} similarly to \RSC{UVarLeft}.
        \item Case \RSC{VarLeft} (\RSC{VarRight}) by inversion, IH, and
            \RST{VarLeft} (\RST{VarRight}), for $\substvars(\vx) = \vx$
            and $\substvars(\tyub) = \tyub$ ($\substvars(\tylb) = \tylb$).
        \item Cases \RSC{Tuple}, \RSC{Inv}, \RSC{UnionLeft}, and \RSC{UnionRight}
            are all similar: by inversion, IH, and the corresponding subtyping
            constructor.
        \item \textbf{Case \RSC{UVar-UnionRight}} 
            \subtyctrLdfltenv{\plug\dctx{\va}}{\tyunion{\ty'_1}{\ty'_2}}
            {\CSet'_1 \cup \CSet'_2 \cup \CSet'}.
            Since \tyvlddflt{\tyunion{\ty'_1}{\ty'_2}}, we have
            $\substvars(\tyunion{\ty'_1}{\ty'_2}) = \tyunion{\ty'_1}{\ty'_2}.$
            By assumption,
            \vldinenv{\AEnv}{\CSet'_1 \cup \CSet'_2 \cup \CSet'}{\substvars}.
            By inversion, 
            \subtyctrL{\AEnv}{\UEnvD,\va_1}{\plug\dctx{\va_1}}{\ty'_1}{\CSet_1} 
            and
            \subtyctrL{\AEnv}{\UEnvD,\va_2}{\plug\dctx{\va_2}}{\ty'_2}{\CSet_2},
            where $\CSet_1 = \CSet'_1 \mcup_{i=1}^n \ctrset{\ctrsub{\va_1}{\tyub_1^i}}$
            and $\CSet_2 = \CSet'_2 \mcup_{j=1}^m \ctrset{\ctrsub{\va_2}{\tyub_2^j}}$.
            
            When $n=0$ or $m=0$, we can take $\substvars_1 = 
                \subst{\substvars}{\substel{\va_1}{\substvars(\va)}}$
            and $\substvars_2 = 
                \subst{\substvars}{\substel{\va_2}{\substvars(\va)}}$.
            In both subcases ($n=0$ and $m=0$), we have 
            \vldinenv{\AEnv}{\CSet_1}{\substvars_1} and
            \vldinenv{\AEnv}{\CSet_2}{\substvars_2}, because $\substvars(\va)$
            satisfies the constraints on $\va_1$ and $\va_2$.
            Therefore, by the induction hypotheses,
            \subtydflt{\substvars_1(\plug\dctx{\va_1})}{\tyunion{\ty'_1}{\ty'_2}}
            and
            \subtydflt{\substvars_2(\plug\dctx{\va_2})}{\tyunion{\ty'_1}{\ty'_2}}.
            Clearly, $\substvars(\plug\dctx{\va}) = 
                \substvars_1(\plug\dctx{\va_1}) = 
                \substvars_2(\plug\dctx{\va_2})$,
            and thus, the subcases conclude by \RST{UnionRight}.

            The most interesting case is when $n \geq 1, m \geq 1$.
            Let $\msqcap_{i=1}^n \tyub_1^i$ be denoted with $\tyub_1$
            and $\msqcap_{j=1}^m \tyub_2^j$ with $\tyub_2$.
            By~\thmref{thm:meet-sound} (soundness of meet), we have
            \subtydflt{\tyub_1}{\tyub_1^i} and \subtydflt{\tyub_2}{\tyub_2^j}
            for all $i, j$. Therefore, we can take 
            $\substvars_1 = \subst{\substvars}{\substel{\va_1}{\tyub_1}}$ and
            $\substvars_2 = \subst{\substvars}{\substel{\va_2}{\tyub_2}}$,
            and it holds that \vldinenv{\AEnv}{\CSet_1}{\substvars_1} and
            \vldinenv{\AEnv}{\CSet_2}{\substvars_2}.
            Therefore, the induction hypotheses are applicable, and we get
            \subtydflt{\plug{\substvars_1(\dctx)}{\tyub_1}}{\ty'_1} (H$_1$) and
            \subtydflt{\plug{\substvars_2(\dctx)}{\tyub_2}}{\ty'_2} (H$_2$).
            By assumption on \substvars, we have
            \subtydflt{\substvars(\va)}{\tyunion{\tyub_1}{\tyub_2}}.
            Therefore, \subtydflt{\plug{\substvars(\dctx)}{\substvars(\va)}}
                {\plug{\substvars(\dctx)}{\tyunion{\tyub_1}{\tyub_2}}}.
            Since $\va_1, \va_2$ were fresh, we know
            $\substvars(\dctx) = \substvars_1(\dctx) = \substvars_2(\dctx)$.
            Therefore, by \RST{UnionLeft} applied to H$_1$, H$_2$,
            \subtydflt{\plug{\substvars(\dctx)}{\tyunion{\tyub_1}{\tyub_2}}}
                {\tyunion{\ty'_1}{\ty'_2}}.
            Finally, the subcase $n \geq 1, m \geq 1$ concludes by transitivity:
            \subtydflt{\plug{\substvars(\dctx)}{\substvars(\va)}}{\tyunion{\ty'_1}{\ty'_2}}.
    \end{itemize}
\end{proof}

\begin{theorem}{Soundness of constrains resolution.}%
\label{thm:solvectr-sound}
    $\forall \AEnv, \UEnv$ s.t. $\tyvld{}{\AEnv, \UEnv}$.
    $\forall \CSet$ s.t. $\forall \ctrsub{\tylb}{\va} \in \CSet.
        \ \tyvlddflt{\tylb}\ \land\ \tyvld{\UEnv}{\va}$ and 
        $\forall \ctrsub{\va}{\tyub} \in \CSet.
        \ \tyvld{\UEnv}{\va} \ \land\ \tyvlddflt{\tyub}.$
    \[
        \solvectrdflt = \substvars
        \quad\implies\quad
        \vldinenv{\AEnv}{\UEnv}{\substvars}\ \ \land\ \ 
        \vldinenv{\AEnv}{\CSet}{\substvars}.
    \]
\end{theorem}
\begin{proof}
    By induction on \UEnv. The base case is trivial ($\CSet = \EmptyCSet$
    because $\UEnv = \EmptyEnv$).

    In the inductive step $\UEnv,\varbound{\va}{\tylb}{\tyub}$,
    the induction hypothesis applies to the call \solvectr{\AEnv}{\UEnv}{
        \CSet' \mcup_i \CSet_{\tylb_i} \mcup_j \CSet_{\tyub_j}
    }, which returns \substvars.
    Therefore, we know that \vldinenv{\AEnv}{\UEnv}{\substvars} and
    \vldinenv{\AEnv}{\CSet' \mcup_i \CSet_{\tylb_i} \mcup_j \CSet_{\tyub_j}}{\substvars}.

    Let $\substvars(\tylb) \mcup_i \tylb_i$ be denoted with $\ty_{\va}$
    and \subst\substvars{\substel{\va}{\ty_{\va}}} with $\substvars_{\va}.$
    %be denoted with $\substvars_{\va}.$
    By~\thmref{thm:subtyctr-sound} (soundness of constrained subtyping)
    applied to \subtyctrRdfltenv{\tylb_i}{\tyub}{\CSet_{\tylb_i}} with
    \vldinenv{\AEnv}{\CSet_{\tylb_i}}{\substvars} and
    \subtyctrLdfltenv{\tylb}{\tyub_j}{\CSet_{\tyub_j}} with
    \vldinenv{\AEnv}{\CSet_{\tyub_j}}{\substvars},
    we know 
    \[
        \subtydflt{\tylb_i}{\substvars(\tyub)}\ (\textrm{H}_{l_i})\ 
        \ \text{ and }\ 
        \subtydflt{\substvars(\tylb)}{\tyub_j}\ (\textrm{H}_{u_j}).
    \]
    % Since $\lnot\occ{\va}{\tylb}, \lnot\occ{\va}{\tyub}$, we have
    % $\substvars(\tylb) = \substvars_{\va}(\tylb)$ and
    % $\substvars(\tyub) = \substvars_{\va}(\tyub)$.
    By \RST{UnionRight} and reflexivity, we know 
    \subtydflt{\substvars(\tylb)}{\ty_{\va}}.
    By \thmref{thm:subty-sound-subst} (soundness of subtyping with respect to
    substitution), \subtydflt{\substvars(\tylb)}{\substvars(\tyub)} (H),
    for \subty{\UEnv}{\tylb}{\tyub} by the \UEnv validity assumption. 
    Thus, we have %\subtydflt{\substvars(\tylb)}{\ty_{\va}} and
    \subtydflt{\ty_{\va}}{\substvars(\tyub)}
    by \RST{UnionLeft}, H, and H$_{l_i}$.
    Since \vldinenv{\AEnv}{\UEnv}{\substvars} and \va does not occur in \UEnv,
    it holds that \vldinenv{\AEnv}{\UEnv}{\substvars_{\va}}, and we get 
    \[ \vldinenv{\AEnv}{\UEnv, \varbound{\va}{\tylb}{\tyub}}{\substvars_{\va}}. \]

    Finally, we know that $\forall i,j,$ \subtydflt{\tylb_i}{\ty_{\va}} holds by
    \RST{UnionRight} and reflexivity, and \subtydflt{\ty_{\va}}{\tyub_j} holds
    by \RST{UnionLeft}, H$_{u_j}$, and \subtydflt{\tylb_i}{\tyub_j},
    which means \vldinenv{\AEnv}{\CSet_{\va}}{\substvars_{\va}}.
    Since \va does not occur in $\CSet'$, we also know 
    \vldinenv{\AEnv}{\CSet'}{\substvars_{\va}}.
    Thus, \vldinenv{\AEnv}{\CSet}{\substvars_{\va}}, which concludes the proof.
\end{proof}



\section{Extended Subtyping}\label{sec:dec-sub:ext}
%% ======================================================================

In this section, I discuss two Julia features that were
omitted from the core language in~\figref{fig:ty-grammar}:
nominal subtyping (\secref{subsec:dec-sub:inheritance}) and
the diagonal rule (\secref{subsec:dec-sub:diagonal}).

\subsection{Nominal subtyping}\label{subsec:dec-sub:inheritance}
%% *********************************************************

\begin{figure}[t] 
\begin{minipage}{5cm}
\begin{lstlisting}
abstract type Real <: Number
end

struct Rational{
    T<:Integer
} <: Real
    num::T
    den::T
end
\end{lstlisting}
\end{minipage}
\hspace{1cm}
\begin{minipage}{5cm}
\begin{lstlisting}
abstract type Ref{T} <: Any
end

struct RefArray{
    T, A<:AbstractArray{T}, R
} <: Ref{T}
    x::A
    ...
end
\end{lstlisting}  
\end{minipage}
\caption{Examples of inheritance in type declarations
}\label{fig:code:inheritance}
\end{figure}

The Julia language supports a limited form of single-parent inheritance:
abstract nominal types can be inherited by both abstract and concrete
types, but concrete types are final and cannot be inherited.
Type parameters of parametric nominal types are invariant;
they can be constrained by non-recursive lower and upper bounds,
and can be referenced from the supertype declaration.
For instance, \cjl{Point\{X<:Real\}} can be declared a subtype of
\cjl{AbstractPoint\{X\}}.
\figref{fig:code:inheritance} provides a few more examples.
A type that is being inherited needs to be defined before the inheriting type,
and mututally recursive type declarations are not supported.

Because of its fairly restricted nature, inheritance in Julia does not
interfere with the decidability of subtyping, unlike, for example,
in Java~\cite{bib:tate:taming-wildcards:2011}.
In particular, the lack of variance annotations, recursive inheritance,
and recursive constraints prevents properties such as expansive
inheritance~\cite{bib:kennedy:nom-sub-var-dec:2007}, which are known to cause
undecidability.
Thus, in Julia, subtyping of nominal types is straightforward: the subtyping
algorithm simply walks the finite inheritance hierarchy, substituting
type arguments of parametric types.
In the specification of Julia
subtyping~\cite{bib:zappa-nardelli:julia-sub:oopsla:2018},
this step is encoded with one extra rule \RLJ{App\_Super},
which is given in~\figref{fig:jlsubex-inheritance}.
In the context of the decidability proof from \secref{subsec:dec-proof},
the measure of nominal types needs to be modified to include the 
measure of the declared supertype, akin to
how~\citet{bib:greenman:f-bound-material-shape:2014}
proved decidability of subtyping for Java with Material-Shape Separation.

Without changes, the framework of restricted existential types presented
in \secref{sec:dec-sub:defs} is suitable for \emph{partial} handling of
inheritance in Julia. For example, consider the following derivation,
where \tyinv\nvec\vx is a declared subtype of \tyinv\nabsvec\vx:\\
\makebox[\textwidth][c]{
\begin{minipage}{\ruleswidth}
\begin{mathpar}
\small
\inferrule[]
{ 
    \inherits{\tyinv\nvec\vx}{\tyinv\nabsvec\vx} \and
    \inferrule[]
    { 
        \subtydflt{\tybot}{\tybot} \and
        \subtydflt{\tyint}{\tyany} \and
    }
    { \subtydflt
        {\tyinv\nabsvec{\rexvarbound{\tybot}{\tyint}}}
        {\tyinv\nabsvec{\rexvarbound{\tybot}{\tyany}}} 
    }
}
{ \subtydflt
    {\tyinv\nvec{\rexvarbound{\tybot}{\tyint}}}
    {\tyinv\nabsvec{\rexvarbound{\tybot}{\tyany}}} 
}
\\
\end{mathpar}
\end{minipage}}
Similarly to \RLJ{App\_Inv}, subtyping is checked by substituting
\vx with \rexvarbound{\tybot}{\tyint} in the supertype declaration
\tyinv\nabsvec\vx.
However, type parameters do not have to be immediate arguments of the 
supertype declaration. For example, the following type declaration is
allowed:
\begin{center}
\begin{minipage}{7cm}
\begin{lstlisting}
struct ZooVec{X} <: AbstractVector{Zoo{X}}
...
end
\end{lstlisting}  
\end{minipage}
\end{center}
In this case, simply substituting \vx with a restricted existential variable
would be \textbf{incorrect}:\\
\makebox[\textwidth][c]{
\begin{minipage}{\ruleswidth}
\begin{mathpar}
\small
\inferrule[]
{ 
    \inherits{\tyinv{\tyname{ZooVec}}\vx}
        {\tyinv\nabsvec{\tyinv{\tyname{Zoo}}{\vx}}} 
    \and
    \AEnv \vdash 
       \tyinv\nabsvec{\tyinv{\tyname{Zoo}}{\rexvarbound{\tybot}{\tyint}}}
       \ ???
}
{ \AEnv \vdash 
    \tyinv{\tyname{ZooVec}}{\rexvarbound{\tybot}{\tyint}}
    \nless:
    \tyinv\nabsvec{\tyinv{\tyname{Zoo}}{\rexvarbound{\tybot}{\tyany}}} 
}
\\
\end{mathpar}
\end{minipage}}
Recall that \tyinv{\tyname{ZooVec}}{\rexvarbound{\tybot}{\tyint}}
denotes an existential type, namely:
\[
    \tyexist{\vy}{\tybot}{\tyint}{\tyinv{\tyname{ZooVec}}\vy}.
\]
Therefore, the proper supertype of this type is
\[
    \tyexist{\vy}{\tybot}{\tyint}
        {\tyinv\nabsvec{\tyinv{\tyname{Zoo}}{\vy}}},
\]
which is different from 
\[
    \tyinv\nabsvec{\tyexist{\vy}{\tybot}{\tyint}{\tyinv{\tyname{Zoo}}\vy}}
\]
denoted by \tyinv\nabsvec{\tyinv{\tyname{Zoo}}{\rexvarbound{\tybot}{\tyint}}}.

\begin{figure}[t!]
\footnotesize
\makebox[\textwidth][c]{
\begin{minipage}{\ruleswidth}
\begin{mathpar}
\\
\fbox{\ljsub{E}{t}{t}{E}}
\\
\inferrule[App\_Inv]
{ E_0 = \mathit{add}(E, \text{Barrier}) \\ 
    \forall\, 0 < i \leq n. \  
    E_{i-1} \vdash \t_i <: \t'_i \vdash E'_i \ \wedge\  
    E'_i \vdash \t'_i <: \t_i \vdash E_i \\
}
{ E \vdash \cstrt{name}{\t_1, \ldots, \t_n} <: 
    \cstrt{name}{\t'_1, \ldots, \t'_n} \vdash \mathit{del}(\text{Barrier}, E_n)}

\inferrule[App\_Super]
{ 
    \cstrt{name}{\T_1, \ldots, \T_n} <: \t'' \ \in\ tds \\
    E \vdash \t''[\t_1/\T_1, \ldots, \t_n/\T_n] <: \t' \vdash E'
}
{ E \vdash \cstrt{name}{\t_1, \ldots, \t_n} <: 
    \t' \vdash E'}
\end{mathpar}
\end{minipage}}
\caption{Julia subtyping: nominal types}\label{fig:jlsubex-inheritance}
\begin{tablenotes}[para]
\small
\centering
An excerpt from \cite{bib:zappa-nardelli:julia-sub:oopsla:2018}, extends
\figref{fig:jlsubex}.
\end{tablenotes}
\end{figure}

The simplest solution to this problem would be to disallow instantiating
type variables such as \vx in \tyname{ZooVec} (that is, variables that
occur in the supertype but not as immediate arguments)
with restricted existentials.
Thus, \tyinv{\tyname{ZooVec}}\tyint and
\tyexist{\vx}{\tybot}{\tyint}{\tyinv{\tyname{ZooVec}}\vx} would be allowed,
whereas \tyinv{\tyname{ZooVec}}{\rexvarbound{\tybot}{\tyint}} would not.

\begin{figure}
\footnotesize
\makebox[\textwidth][c]{
\begin{minipage}{\ruleswidth}
\begin{mathpar}
    \fbox{\subtydflt{\ty}{\ty}}
    \\

    \inferrule[\RST{InvLeft}]
    { \vx \text{ fresh} \\
      \subty{\AEnv, \varbound{\vx}{\tylb}{\tyub}}
        {\tyinv\iname{\ldots,\vx,\ldots}}
        {\tyinv{\iname'}{\rexvar'_1,\ldots,\rexvar'_m}} }
    { \subtydflt
        {\tyinv\iname{\ldots,\rexvarbound{\tylb}{\tyub},\ldots}}
        {\tyinv{\iname'}{\rexvar'_1,\ldots,\rexvar'_m}} }

    \inferrule[\RST{InvSuper}]
    { \inherits{\tyinv\iname{\vy_1,\ldots,\vy_n}}{\ty''}
      \\
      \subtydflt
        {\subst{\ty''}{\substel{\vy_1}{\ty_1},\ldots,\substel{\vy_1}{\ty_n}}}
        {\tyinv{\iname'}{\rexvar'_1,\ldots,\rexvar'_m}} } 
    { \subtydflt
        {\tyinv\iname{\ty_1,\ldots,\ty_n}}
        {\tyinv{\iname'}{\rexvar'_1,\ldots,\rexvar'_m}} }
    \\
    \fbox{\subtyctrdflt{\ty}{\ty}} 
    \\

    \inferrule[\RSC{InvLeft-UL}]
    { \vx \text{ fresh} \\
      \subtyctrL{\AEnv, \varbound{\vx}{\tylb}{\tyub}}{\UEnvD}
        {\tyinv\iname{\ldots,\vx,\ldots}}
        {\tyinv{\iname'}{\rexvar'_1,\ldots,\rexvar'_m}}
        {\CSet} }
    { \subtyctrLdflt
        {\tyinv\iname{\ldots,\rexvarbound{\tylb}{\tyub},\ldots}}
        {\tyinv{\iname'}{\rexvar'_1,\ldots,\rexvar'_m}} }

    \inferrule[\RSC{InvLeft-UR}]
    { \vx \text{ fresh} \\
      \subtyctrR{\AEnv, \varbound{\vx}{\tylb}{\tyub}}{\UEnvD}
        {\tyinv\iname{\ldots,\vx,\ldots}}
        {\tyinv{\iname'}{\rexvar'_1,\ldots,\rexvar'_m}}
        {\CSet}  \\
      \CSet' = \dischctrdflt }
    { \subtyctrRdfltenv
        {\tyinv\iname{\ldots,\rexvarbound{\tylb}{\tyub},\ldots}}
        {\tyinv{\iname'}{\rexvar'_1,\ldots,\rexvar'_m}}
        {\CSet'} }

    \inferrule[\RSC{InvSuper}]
    { \inherits{\tyinv\iname{\vy_1,\ldots,\vy_n}}{\ty''}
      \\
      \subtyctrdflt
        {\tyinv{\iname''}{\subst{\ty''_1}{\substel{\vy_1}{\ty_1},\ldots},\ldots}}
        {\tyinv{\iname'}{\rexvar'_1,\ldots,\rexvar'_m}} }
    { \subtyctrdflt
        {\tyinv\iname{\ty_1,\ldots,\ty_n}}
        {\tyinv{\iname'}{\rexvar'_1,\ldots,\rexvar'_m}} }
\end{mathpar}
\end{minipage}}
\caption{Additional subtyping rules to support inheritance
    %Note: \dctx stands for \dctxty.
}\label{fig:subtyping-inheritance}
\begin{tablenotes}[para]
\small
This figure extends unification-free subtyping (\figref{fig:subtyping-base})
and constrained subtyping (\figref{fig:subtyping-constrained}).
Signature subtyping (\figref{fig:subtyping-tysigs}) remains unchanged.
$\dischctrop$ is defined in \figref{fig:ctr-discharge}.
\end{tablenotes}
\end{figure}

A more general solution would be to open all restricted existential types
before walking up the inheritance hierarchy, similar
to~\cite{bib:tate:mixed-site:2013}. Thus, the example above would
proceed as follows:\\
\makebox[\textwidth][c]{
\begin{minipage}{\ruleswidth}
\begin{mathpar}
\small
\inferrule[]
{ 
    \inferrule[]
    { 
        \inherits{\tyinv{\tyname{ZooVec}}\vx}
            {\tyinv\nabsvec{\tyinv{\tyname{Zoo}}{\vx}}}
        \\
        \AEnv, \varbound{\vy}{\tybot}{\tyint} \vdash
            \tyinv\nabsvec{\tyinv{\tyname{Zoo}}{\vy}}
            \nless:
            \tyinv\nabsvec{\tyinv{\tyname{Zoo}}{\rexvarbound{\tybot}{\tyany}}} 
    }
    { \vy \text{ fresh} \and
      \AEnv, \varbound{\vy}{\tybot}{\tyint} \vdash
        \tyinv{\tyname{ZooVec}}{\vy}    
        \nless:
        \tyinv\nabsvec{\tyinv{\tyname{Zoo}}{\rexvarbound{\tybot}{\tyany}}} }
}
{ \AEnv \vdash 
    \tyinv{\tyname{ZooVec}}{\rexvarbound{\tybot}{\tyint}}
    \nless:
    \tyinv\nabsvec{\tyinv{\tyname{Zoo}}{\rexvarbound{\tybot}{\tyany}}} 
}
\\
\end{mathpar}
\end{minipage}}
Additional subtyping rules that support this approach to handling inheritance
are given in \figref{fig:subtyping-inheritance}.
The rules \RST{InvSuper} and \RSC{InvSuper} are similar to \RLJ{App\_Super}:
they walk the inheritance hierarchy, substituting type arguments;
importantly, these rules can be applied only when all the arguments of the
left-hand side invariant constructor are tight,
i.e., $\ty_i$ rather than $\rexvar_i$.
The rule \RST{InvLeft} of unification-free subtyping is similar to
\RSS{InvLeft}: it simply opens a restricted existential type variable.
The remaining rules of constrained subtyping are more interesting:
\begin{itemize}
    \item \RSC{InvLeft-UL} handles the case when unification variables are on
        the left. Although the bounds \tylb and \tyub may contain unification
        variables, the fresh variable \vx is treated as universal and added
        to \AEnv. Since \vx does not occur in the right-hand side type, 
        unification variables from its bounds cannot ``leak'' to the right.
    \item \RSC{InvLeft-UR} handles the case when unification variables are on
        the right. The fresh variable \vx can occur in generated constraints,
        so it needs to be discharged before the constraints are propagated.
        The discharge algorithm is given in \figref{fig:ctr-discharge}.
        Intuitively, since \vx is introduced after unification variables,
        constraints need to be satisfied for all possible valid instantiations
        of \vx. To reflect this, covariant occurrences of \vx in lower-bound
        (upper-bound) constraints are replaced with the upper bound
        (lower bound) of \vx. If \vx occurs invariantly, it has to be
        tightly bound (\subtydflt{\tyub}{\tylb}), or else discharge 
        and subtyping fail.
\end{itemize}

\begin{figure}
\footnotesize
\centering
\begin{minipage}{.7\linewidth}
\begin{algorithm}[H]
    \SetKwProg{DischargeCtrs}{Discharge}{}{}

    \DischargeCtrs{$(\AEnv;\varbound{\vx}{\tylb}{\tyub};\CSet)$}{
        \uIf{$\subtydflt{\tyub}{\tylb}$}{
            \KwRet{$\subst{\CSet}{\substel{\vx}{\tylb}}$} 
        }
        \Else{
            $\CSet_{\tyub} \gets 
                \ctrset{\ctrsub{\covsubst{\vx}{\tyub}{\tylb'}}{\va} \,|\, \ctrsub{\tylb'}{\va} \in \CSet} $ \;
            $\CSet_{\tylb} \gets 
                \ctrset{\ctrsub{\va}{\covsubst{\vx}{\tylb}{\tyub'}} \,|\, \ctrsub{\va}{\tyub'} \in \CSet} $ \;
            \KwRet{$\CSet_{\tyub} \cup \CSet_{\tylb}$}
        }
    }
    %$\substvars \gets \emptysubst$
\end{algorithm}  
\end{minipage}
\[
\begin{array}{lcl}
    \hline
    \multicolumn{3}{c}{\covsubstdflt{\ty}} \\ 
    \hline 
    \covsubstdflt{\tyany} &=& \tyany\\
    \covsubstdflt{\tybot} &=& \tybot\\
    \covsubstdflt{\vany} &=& \ty_{\vany}\\
    \covsubstdflt{\vany'} &=& \vany'\\
    \covsubstdflt{\typair{\ty_1}{\ty_2}} &=& 
        \typair{\covsubstdflt{\ty_1}}{\covsubstdflt{\ty_2}}\\
    \covsubstdflt{\tyinv\iname{\ldots}} &=&
        \text{if } \occdflt{\tyinv\iname{\ldots}}
        \text{ then } \failure
        \text{ else } \tyinv\iname{\ldots}\\
    \covsubstdflt{\tyunion{\ty_1}{\ty_2}} &=& 
        \tyunion{\covsubstdflt{\ty_1}}{\covsubstdflt{\ty_2}}\\
\end{array}
\]
\caption{Variable discharge algorithm \dischctrdflt}\label{fig:ctr-discharge}      
\end{figure}

The following example illustrates \RSC{InvLeft-UL}:\\
\makebox[\textwidth][c]{
\begin{minipage}{\ruleswidth}
\begin{mathpar}
\small
\inferrule[]
{ 
    \inferrule[]
    {
        \inferrule[]
        {  
            \subtyctrR{\AEnv, \varbound{\vx}{\va}{\tyint}}{\va}
                {\tyint}{\va}
                {\ctrsngl{\tyint}{\va}}  
        }
        { 
            \subtyctrR{\AEnv, \varbound{\vx}{\va}{\tyint}}{\va}
                {\tyint}{\vx}
                {\ctrsngl{\tyint}{\va}}  
        }
        \and
        \subtyctrL{\AEnv, \varbound{\vx}{\va}{\tyint}}{\va}
            {\vx}{\tyany}{\EmptyCSet} 
    }
    {
        \subtyctrL{\AEnv, \varbound{\vx}{\va}{\tyint}}{\va}
            {\tyinv\nref\vx}
            {\tyinv\nref{\rexvarbound{\tyint}{\tyany}}}
            {\ctrsngl{\tyint}{\va}}  
    }
}
{   
    \subtyctrL{\AEnv}{\va}{\tyinv\nref{\rexvarbound{\va}{\tyint}}}
        {\tyinv\nref{\rexvarbound{\tyint}{\tyany}}}
        {\ctrsngl{\tyint}{\va}}
}
\\
\end{mathpar}
\end{minipage}}

The following example illustrates \RSC{InvLeft-UR} where
the variable discharge succeeds:\\
\makebox[\textwidth][c]{
\begin{minipage}{\ruleswidth}
\begin{mathpar}
\small
\inferrule[]
{ 
    \inferrule[]
    {
        \subtyctrL{\AEnv, \varbound{\vx}{\tyint}{\tyany}}{\va}
            {\tybot}{\vx}{\EmptyCSet} 
        \and
        \subtyctrR{\AEnv, \varbound{\vx}{\tyint}{\tyany}}{\va}
            {\vx}{\va}
            {\ctrsngl{\vx}{\va}}  
    }
    {
        \subtyctrR{\AEnv, \varbound{\vx}{\tyint}{\tyany}}{\va}
            {\tyinv\nref\vx}
            {\tyinv\nref{\rexvarbound{\tybot}{\va}}}
            {\ctrsngl{\vx}{\va}}  
    }
    \\\\
    \dischctr{\AEnv}{\varbound{\vx}{\tyint}{\tyany}}{\ctrsngl{\vx}{\va}} =
        \ctrsngl{\tyany}{\va}
}
{   
    \subtyctrR{\AEnv}{\va}{\tyinv\nref{\rexvarbound{\tyint}{\tyany}}}
        {\tyinv\nref{\rexvarbound{\tybot}{\va}}}
        {\ctrsngl{\tyany}{\va}}
}
\\
\end{mathpar}
\end{minipage}}

The following example illustrates \RSC{InvLeft-UR} where
the variable discharge fails:\\
\makebox[\textwidth][c]{
\begin{minipage}{\ruleswidth}
\begin{mathpar}
\small
\inferrule[]
{ 
    \inferrule[]
    {
        \subtyctrR{\AEnv, \varbound{\vx}{\tyint}{\tyany}}{\va}
            {\tyinv\nabsvec{\tyinv{\tyname{Zoo}}{\vx}}}
            {\tyinv\nabsvec\va}
            {\ctrset{\ctrsub{\tyinv{\tyname{Zoo}}{\vx}}{\va}, 
                \ctrsub{\va}{\tyinv{\tyname{Zoo}}{\vx}}}}
    }
    {
        \subtyctrR{\AEnv, \varbound{\vx}{\tyint}{\tyany}}{\va}
            {\tyinv{\tyname{ZooVec}}{\vx}}
            {\tyinv\nabsvec\va}
            {\ctrset{\ctrsub{\tyinv{\tyname{Zoo}}{\vx}}{\va}, 
                \ctrsub{\va}{\tyinv{\tyname{Zoo}}{\vx}}}}
    }
    \\\\
    \dischctr{\AEnv}{\varbound{\vx}{\tyint}{\tyany}}
        {\ctrset{\ctrsub{\tyinv{\tyname{Zoo}}{\vx}}{\va}, 
            \ctrsub{\va}{\tyinv{\tyname{Zoo}}{\vx}}}} = \failure
}
{   
    \AEnv \Alt \va \ \vdash\ \tyinv{\tyname{ZooVec}}{\rexvarbound{\tyint}{\tyany}}
        \nlessdot\!\!\bullet \tyinv\nabsvec\va
}
\\
\end{mathpar}
\end{minipage}}

The syntax of type declarations and extended validity rules
are given in \figref{fig:validity-inheritance}.
Note that the supertype cannot be a restricted existential type:
\tyinv{\aname}{\ty_1,\ldots,\ty_m} is an invariant constructor with
tight type arguments.
The validity of types is checked with respect to an implicit datatype table.
In a type declaration, type parameters $\vx_i$ may reference previous
parameters: this is handled by checking the validity of type parameters
in \tyvld{\TyTable}{\tydecl} using \tyvld{}{\AEnv} from
\figref{fig:ty-tysig-validity}.
Similarly to declared bounds of existential variables, bounds of type parameters
need to be consistent.

\begin{figure}
\footnotesize
\[
\begin{array}{rcll}
    \iname
    &::=& & \textit{Invariant constructor name} \\
    &\Alt& \cname & \text{concrete} \\
    &\Alt& \aname & \text{abstract} \\
\end{array}
\]
\[
\begin{array}{rcll}
    \TyTable &::=& \varnothing \Alt \TyTable,\tydecl 
        & \textit{Datatype table} \\
    \tydecl 
        &::=&
        \inherits{\tyinv{\iname}{\varbound{\vx_1}{\tylb_1}{\tyub_1},
            \,\ldots,\,\varbound{\vx_n}{\tylb_n}{\tyub_n}}
        }{\ubdecl}
        & \textit{Type declaration} \\
    \ubdecl
        &::=& \tyany 
        \Alt \tyinv{\aname}{\ty_1,\ldots,\ty_m}
        & \textit{Supertype} \\ 
\end{array}
\]
\begin{mathpar}
    \fbox{\tyvld{}{\TyTable}}
    \\

    \inferrule*[]
    { }
    { \tyvld{}{\varnothing} }

    \inferrule*[]
    { \tyvld{}{\TyTable} \and \tyvld{\TyTable}{\tydecl} }
    { \tyvld{}{\TyTable,\tydecl } }
    \\

    \fbox{\tyvld{\TyTable}{\tydecl}}
    \\

    \inferrule*[]
    { \tyvld{}{\varbound{\vx_1}{\tylb_1}{\tyub_1},\,\ldots}
      \and 
      \ubdecl=\tyinv\aname{\ldots} \implies \aname \text{ is defined in } \TyTable
      \and
      \tyvld{\varbound{\vx_1}{\tylb_1}{\tyub_1},\,\ldots}{\ubdecl}   
    }
    { \tyvld{\TyTable}{\inherits{\tyinv{\iname}{
            \varbound{\vx_1}{\tylb_1}{\tyub_1},\,\ldots}
        }{\ubdecl}} }
    \\

    \fbox{\tyvlddflt{\ty}}
    \\

    \inferrule*[]
    {   
        \tyvlddflt{\tylb} \and \tyvlddflt{\tyub} \and
        \vx \text{ fresh} \and
        \tyvld{\AEnv, \varbound{\vx}{\tylb}{\tyub}}
            {\tyinv\iname{\ldots,\vx,\ldots}} 
    }
    { 
      \tyvlddflt{\tyinv\iname{\ldots,\rexvarbound{\tylb}{\tyub},\ldots}}
    }

    \inferrule*[]
    { 
        \inherits{\tyinv{\iname}{\varbound{\vx_1}{\tylb_1}{\tyub_1},\,\ldots}
        }{\ubdecl} 
        \and
        \tyvlddflt{\subst{\ubdecl}{\substel{\vx_1}{\ty_1},\,\ldots}}
    }
    { 
        \tyvlddflt{\tyinv\iname{\ty_1,\,\ldots,}} 
    }
\end{mathpar}
\caption{Validity of type declarations and types in the presence of inheritance
(extends \figref{fig:ty-grammar} and \figref{fig:ty-tysig-validity})
}\label{fig:validity-inheritance}
\end{figure}


\subsection{Diagonal Rule}\label{subsec:dec-sub:diagonal}
%% *********************************************************

As discussed in \secref{sec:julia-sub:overview},
Julia provides special support for a generic programming pattern
where method arguments are expected to be of the same concrete type.
For example, in the existential type \cjl{Tuple\{T,T\} where T},
type variable \cjl{T} can be instantiated only with concrete types.
This is called the \textbf{diagonal rule}. 
%The rule applies only to top-level existential types.
% JB: this is no longer true!

In this work, the diagonal rule is modeled explicitly, using a new kind of
type variables---\textbf{concrete variable}. 
In type signatures, existential types specify the kind of the bound variable
explicitly:
\[
    \tysig ::= \ldots \Alt
        \tyexist{\vany{:}\varkind}{\tylb}{\tyub}{\tysig}
\]
The kind annotation,
\[\varkind ::= \kindany \Alt \kindval,\]
allows for two options: 
\begin{itemize}
    \item kind \kindany denotes a regular variable that can be instantiated with any type;
    \item kind \kindval denotes a {concrete} variable that can be instantiated 
        only with a concrete type.
\end{itemize}

All earlier-presented definitions and reasoning apply
to \kindany-kinded vars without modifications.

Because concrete variables have to be instantiated with concrete types,
the only allowed declared lower bound is \tybot. To achieve tight bounds,
a concrete type can be chosen for the upper bound.
Thus, the validity is extended as follows:
\begin{mathpar}
\inferrule*[]
    { \tyvlddflt{\tyub} \and
        \tyvld{\AEnv, \varbound{\varval{\vany}}{\tybot}{\tyub}}{\tysig} }
    { \tyvlddflt{\tyexist{\varval{\vany}}{\tybot}{\tyub}{\tysig}} }
\end{mathpar}

\figref{fig:concrete} defines concreteness \tyvaldflt{\ty}:
type \ty is concrete if it is
a concrete variable,
a concrete invariant constructor with tight arguments,
a tuple of concrete types,
a union of equivalent concrete types,
or a union of a concrete and a bottom type.
The concreteness judgment is used by the subtyping algorithm.

\begin{figure}[t]
\footnotesize
\makebox[\textwidth][c]{
\begin{minipage}{\ruleswidth}
\begin{mathpar}
    \fbox{\tyvaldflt{\ty}}
    \\

    \inferrule*[]
    { \varbound{\varval{\vany}}{\tylb}{\tyub} \in \AEnv }
    { \tyvaldflt{\vany} }

    \inferrule*[]
    { }
    { \tyvaldflt{\tyinv\cname{\ty_1,\ldots,\ty_n}} }

    \inferrule*[]
    { \tyvaldflt{\ty_1} \and \tyvaldflt{\ty_2} }
    { \tyvaldflt{\typair{\ty_1}{\ty_2}} }

    \inferrule*[]
    { 
        \tyvaldflt{\ty_1} \and \tyvaldflt{\ty_2} \and 
        \subtydflt{\ty_1}{\ty_2} \and \subtydflt{\ty_2}{\ty_1}
    }
    { \tyvaldflt{\tyunion{\ty_1}{\ty_2}} }

    \inferrule*[]
    { 
        i \neq j \and \tyvaldflt{\ty_i} \and \subtydflt{\ty_j}{\tybot}
    }
    { \tyvaldflt{\tyunion{\ty_1}{\ty_2}} }
\end{mathpar}
\end{minipage}}
\caption{Type concreteness}\label{fig:concrete}
\end{figure}

The subtyping algorithm requires minimal modifications, 
as defined in \figref{fig:subty-concrete}.
All previously defined rules are applicable to both kinds of variables.
The two new rules cover the tight-bound case for a concrete variable: as the only 
valid instantiation of the variable is \tyub, it can be used
instead of \tybot to check subtyping with the concrete variable on the right.
The same reasoning explains the new cases for intersection $\sqcap_\AEnv$.
Constraints for concrete unification variables are generated in the same way
as for regular variables. The key difference is in the constraint resolution
algorithm: the smallest type satisfying the constraints, 
$\ty_{\va} = \mcup_i \tylb_i$,
needs to be concrete\footnote{
    It is possible that there are no lower-bound constraints $\tylb_i$,
    in which case instantiating \va with \tybot would technically be incorrect.
    However, the absence of generated lower bounds means that any instantiation
    within the declared bounds would satisfy the constraints, so it can 
    be picked arbitrarily; in this case,
    the only necessary condition is that
    $\lnot (\subtydflt{\substvars(\tyub)}{\tybot})$,
    as otherwise, no concrete instantiations would be valid.
} in order for constraint resolution to succeed.

\begin{figure}[t]
\footnotesize
\makebox[\textwidth][c]{
\begin{minipage}{\ruleswidth}  
\begin{mathpar}
    \fbox{\subtydflt{\ty}{\ty}}
    \qquad\qquad
    \fbox{\subtyctrdflt{\ty}{\ty}}
    \\

    \inferrule[\RST{VarRight-ConcreteTight}]
    { \varbound{\varval{\vany}}{\tybot}{\tyub} \in \AEnv \\
        \tyvaldflt{\tyub} \\\\
        \subtydflt\ty{\tyub} }
    { \subtydflt{\ty}{\vany} }

    \inferrule[\RSC{VarRight-ConcreteTight}]
    { \varbound{\varval{\vany}}{\tybot}{\tyub} \in \AEnv \\
        \tyvaldflt{\tyub} \\\\
        \subtyctrdflt\ty{\tyub} }
    { \subtyctrdflt{\ty}{\vany} }
\\
\end{mathpar}
\[
\begin{array}{ccccll}
    \hline
    \multicolumn{6}{c}{\tymeetdflt{\ty}{\ty'}} \\ 
    \hline 
    \vany &\sqcap_{\AEnv}& \ty' &=& 
        \tymeetdflt{\tyub}{\ty'} & 
        \text{ where } \varbound{\varval{\vany}}{\tybot}{\tyub} \in \AEnv 
        \text{ and } \tyvaldflt{\tyub} \\
    \ty &\sqcap_{\AEnv}& \vany &=& 
        \tymeetdflt{\ty}{\tyub} & 
        \text{ where } \varbound{\varval{\vany}}{\tybot}{\tyub} \in \AEnv 
        \text{ and } \tyvaldflt{\tyub} \\
    \\
\end{array}
\]
\end{minipage}}
\begin{minipage}{6.5cm}
\begin{algorithm}[H]
    \SetKwProg{SolveCtrs}{Solve}{}{}

    \SolveCtrs{$(\AEnv;\,\UEnv,\varbound{\va{:}\!\kindany\!}{\tylb}{\tyub};\,\CSet)$}{
        $\CSet_{\va} \gets 
            \ctrset{\ctrsub{\tylb'}{\va} \in \CSet} 
            \cup 
            \ctrset{\ctrsub{\va}{\tyub'} \in \CSet} $ \;
        $\CSet' \gets \CSet \setminus \CSet_{\va}$ \;
        $\UEnvD \gets \dom{\UEnv}$\;
        \ForEach{$\ctrsub{\tylb_i}{\va}, \ctrsub{\va}{\tyub_j} \in \CSet_{\va}$}{%
            \subtydflt{\tylb_i}{\tyub_j}
        }
        \ForEach{$\ctrsub{\tylb_i}{\va} \in \CSet_{\va}$}{%
            \subtyctrRdfltenv{\tylb_i}{\tyub}{\CSet_{\tylb_i}}
        }
        \ForEach{$\ctrsub{\va}{\tyub_j} \in \CSet_{\va}$}{%
            \subtyctrLdfltenv{\tylb}{\tyub_j}{\CSet_{\tyub_j}}
        }
        $\substvars \gets \solvectr{\AEnv}{\UEnv}{
            \CSet' \mcup_i \CSet_{\tylb_i} \mcup_j \CSet_{\tyub_j}
        }$\;
        $\ty_{\va} \gets \substvars(\tylb) \mcup_i \tylb_i$\;
        \BlankLine
        \KwRet{$\subst\substvars{\substel{\va}{\ty_{\va}}}$}
    }
    %$\substvars \gets \emptysubst$
\end{algorithm}  
\end{minipage}
\hspace{0.2cm}
\begin{minipage}{6.5cm}
\begin{algorithm}[H]
    \SetKwProg{SolveCtrs}{Solve}{}{}

    \SolveCtrs{$(\AEnv;\,\UEnv,\varbound{\va{:}\!\kindval\!}{\tybot}{\tyub};\,\CSet)$}{
        $\CSet_{\va} \gets 
            \ctrset{\ctrsub{\tylb'}{\va} \in \CSet} 
            \cup 
            \ctrset{\ctrsub{\va}{\tyub'} \in \CSet} $ \;
        $\CSet' \gets \CSet \setminus \CSet_{\va}$ \;
        $\UEnvD \gets \dom{\UEnv}$\;
        \ForEach{$\ctrsub{\tylb_i}{\va}, \ctrsub{\va}{\tyub_j} \in \CSet_{\va}$}{%
            \subtydflt{\tylb_i}{\tyub_j}
        }
        \ForEach{$\ctrsub{\tylb_i}{\va} \in \CSet_{\va}$}{%
            \subtyctrRdfltenv{\tylb_i}{\tyub}{\CSet_{\tylb_i}}
        }
        \BlankLine\BlankLine
        \BlankLine\BlankLine
        $\substvars \gets \solvectr{\AEnv}{\UEnv}{
            \CSet' \mcup_i \CSet_{\tylb_i} \mcup_j \CSet_{\tyub_j}
        }$\;
        $\ty_{\va} \gets \mcup_i \tylb_i$\;
        \BlankLine
        $\tyvaldflt{\ty_{\va}}$\;
        \BlankLine
        \KwRet{$\subst\substvars{\substel{\va}{\ty_{\va}}}$}
    }
    %$\substvars \gets \emptysubst$
\end{algorithm}  
\end{minipage}
\caption{Subtyping with concrete variables}\label{fig:subty-concrete}
\begin{tablenotes}[para]
\small
    Extends \figref{fig:subtyping-constrained} and \figref{fig:ty-join-meet},
    and replaces \figref{fig:ctr-solve}.
    Constraints resolution algorithm for regular variables is exactly the same
    as the algorithm in \figref{fig:ctr-solve}; it is provided here for
    convenience of comparison.
\end{tablenotes}
\end{figure}
