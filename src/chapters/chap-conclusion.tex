\chapter{Conclusions and Future Work}

\textbf{The specification of Julia subtyping}
(\chapref{sec:julia-sub:lambda-julia})
proved essential for revising Julia's subtyping algorithm, 
finding bugs, and detecting its undecidability.
The specification can still be used by Julia programmers to better understand
subtyping and multiple dispatch, although it no longer captures all aspects of 
subtyping due to changes between Julia versions 0.6.2 and 1.9.2.

Because the undecidability of subtyping affects program execution,
finding a decidable subtype relation without sacrificing \emph{too much}
expressive power is a desirable direction for evolving the Julia language.
The proposed \textbf{restriction} and its corresponding 
\textbf{decidable subtyping} (\chapref{chap:dec-sub})
provide a good candidate for such evolution: 
\textbf{less than 0.01\%} of statically
identifiable type annotations do not satisfy the restriction, and most of them
can be rewritten without affecting program behavior
(\chapref{chap:eval}).
Furthermore, the proposed subtype relation can serve as a specification
to evaluate the correctness of an implementation of subtyping,
should Julia maintainers choose to use this work.

The results of the static corpus analysis are promising,
but further evaluation of the restriction is needed. 
%First, since macro-generated code is prevalent in Julia's codebase,
%some of that code might be relying on the full expressive power
%of impredicative existential types.
First, some Julia programs generate code with macros and \texttt{eval},
and such code is not processed by the static analysis.
Second, Julia internally relies on a data-flow analysis
to infer types and generate optimized, type-specialized code. Therefore,
not only user-defined code can be impacted by the restriction, but multiple 
components of the compiler, too. 
Thus, there are several avenues for \textbf{future work}:
\begin{itemize}
    \item Conduct a dynamic corpus analysis to check the restriction for all
        types encountered during program execution. The analysis should be able
        to detect restriction violations in both dynamically generated code
        and the results of type inference.
    \item Examine the interaction of the restricted type language 
        %and subtype relation 
        with relevant parts of the Julia compiler.
    \item Provide a correct and efficient implementation of the subtyping
        algorithm from \chapref{chap:dec-sub}.
\end{itemize}

Decidable subtyping proposed in this work can be of interest for
languages other than Julia, as it extends use-site-variance subtyping 
with more expressive, top-level existential types.
However, the framework I presented should be applied with care:
for example, its interaction with more expressive nominal subtyping
may not be straightforward.
