\chapter{Preliminary work}\label{chap:4}

%In this chapter, I briefly describe preliminary work
%related to subtyping and the role of types in Julia. %run-time semantics of Julia.

\paragraph{A reconstruction of Julia subtyping.} 
This work is the result of a collaborative effort to provide a
readable specification of Julia subtyping.
Initially, an incomplete (and soon outdated) definition of subtyping existed only
in~\cite{bezanson:julia:2015}, with the actual implementation of subtyping ($\sim$3000 lines of
heavily optimized C~code) being the only reference point.

In~\cite{zappa-nardelli:julia-sub:oopsla:2018}, we define subtyping in the form of a judgment
\[
  \ljsub{E}{t}{t'}{E'}.
\]
Here, $E$ is a type variable environment that contains two kinds of variables,
forall (also called left) and exist (also called right).
A forall variable is added to the environment
when an existential type appears on the left of the subtyping judgment,
as in \cjl{(Vector\{X\} where X) <: t'}. Such variables never change in the
environment, which corresponds to the intuition that for a left-hand side
existential type,
subtyping should hold for all possible instantiations of the type variable.
However, when an existential type appears on the right,
as in \cjl{t <: Vector\{Q\} where Q}, the variable is
added as an exist variable. During subtype checking, such variables may accrue
subtype constraints in addition to their declared bounds; if all the constraints
are consistent, subtyping succeeds, which corresponds to the intuition that
for a right-hand side existential type, subtyping holds
if there exists a valid instantiation of the type variable. 
%Constraints on possible instantiations take the form of updated
%variable bounds, and these updates produce the environment~$E'$.
Exist-variables with updated bounds are recorded in the output environment~$E'$.
For example, if \cjl{Q} is an exist variable with bounds
\cjl{Union\{\}<:Q<:Any}, then
\[
  \ljsub{\mathtt{Q}_\text{\cjl{Union\{\}}}^\text{\cjl{Any}}}
    {\text{\cjl{Int}}}{\mathtt{Q}}
    {\mathtt{Q}_\text{\cjl{Int}}^\text{\cjl{Any}}},
\]
i.e., \cjl{Int<:Q} results in updating \cjl{Q} to \cjl{Int<:Q<:Any}.

Having reconstructed the specification of Julia subtyping,
we were able to identify its undecidability. Furthermore, this work highlighted
several other problems, e.g. the lack of transitivity,
which I plan to address in the thesis.
%For example, the diagonal rule (discussed in \secref{sec:2:subtyping})
%is applied not to an isolated type, but based on the subtyping check with
%another type. As a result, the same existential type may be constrained to
%a concrete type variable in one subtyping judgment but not the other.
%This makes it hard, if not impossible, to reason about properties of subtyping
%such as reflexivity and transitivity.

\paragraph{Tag-based semantic subtyping for non-existential types.}
Julia's subtype relation was inspired by semantic subtyping.
The description of the original Julia design~\cite{bezanson:julia:2015}
suggested an intuitive interpretation of types as sets of values,
but the interpretation is not well defined.
Furthermore, the treatment of invariant parametric types
in the interpretation does not match
the subtype relation. In particular, for concrete nominal \cjl{Name},
\[
  \interp{\text{\cjl{Name\{}}\ty\text{\cjl{\}}}} =
  \{ x\ |\ \text{\cjl{typeof}}(x) = \text{\cjl{Name\{}}\ty\text{\cjl{\}}} \}   
\]
does not account for the fact that there are multiple syntactic
representations $\ty'$ corresponding to the same interpretation as \ty.
Thus, for example, types \cjl{Vector\{Union\{Int,Any\}\}} and \cjl{Vector\{Any\}}
are not equivalent according to the interpretation,
but they are equivalent in Julia.

In \cite{belyakova:minijl-sub:ftfjp:2019}, I proposed and mechanized in Coq
a semantic interpretation of (a subset of) Julia
types \ty as sets of type tags \gty (i.e. concrete types) rather than values.
For the type language of non-parametric nominal types, tuples, and unions,
a decidable syntactic subtyping based on disjunctive normal form
coincides with the set inclusion of interpretations
$\interp{\ty} \subseteq \interp{\ty'}$.
% While the treatment of tuples and unions is completely standard, nominal types
% are not as straightforward: interpreting an abstract type as a set of
% its concrete subtypes (e.g. $\interp{\text{\cjl{Signed}}} =
% \{\text{\cjl{Int8, Int16, ..., Int128}}\}$) would lead to brittle programs.
% Namely, because subtyping is used at run time to process method definitions and 
% dynamic dispatch, defining a new concrete subtype of an abstract type would
% change the subtyping relation and
% could cause the existing code to execute differently.
%It is worth emphasizing that Julia subtyping contradicts the set-theoretic
%interpretation in its treatment of the bottom type \cjl{Union\{\}}.
%Semantically, both \cjl{Union\{\}} and \cjl{Tuple\{Union\{\}\}} represent the
%empty set and are thus equivalent. However, in Julia,
%\cjl{Tuple\{Union\{\}\}} is not a subtype of \cjl{Union\{\}}.

Following up on this work, I extended the interpretation to support
invariant type constructors such as \cjl{Vector\{...\}}.
To account for the issue of multiple syntactic representations of types,
in the new system, types are given an indexed interpretation $\interp{\cdot}_n$.
In particular, a concrete type constructor \cjl{Name\{...\}} is interpreted as
\[
\begin{array}{lcl}
  \interp{\text{\cjl{Name\{}}\ty\text{\cjl{\}}}}_{n+1} & = &
  \{ \text{\cjl{Name\{}}\ty'\text{\cjl{\}}}
  \ |\ \interp{\ty'}_n = \interp{\ty}_n \}    \\
  \interp{\text{\cjl{Name\{}}\ty\text{\cjl{\}}}}_{0} & = &
  \{ \text{\cjl{Name\{}}\ty'\text{\cjl{\}}} \},
\end{array}
\]
and subtyping is defined as
\[
  \ty <: \ty' \quad \equiv \quad
  \forall n.\ \interp{\ty}_n \subseteq \interp{\ty'}_n.
\]
Despite the more complex indexed interpretation, the subtype relation has
an equivalent, decidable syntactic definition $\ty \leq \ty'$ where
$\text{\cjl{Name\{}}\ty\text{\cjl{\}}} \leq
\text{\cjl{Name\{}}\ty'\text{\cjl{\}}}$
holds if $\ty \leq \ty'$ and $\ty' \leq \ty$.
This extension is also mechanized in Coq.

% \paragraph{A model of Julia's JIT~\cite{TODO}.}
% Using an abstract machine modeled after Julia's JIT compiler, this work formally
% defines the property of a method definition that impacts the ability of
% the compiler to optimize the code: the property is called type stability.
% Although type stability is not directly related to the topic of this thesis
% proposal, having defined a model of the JIT compiler, I found that
% (1)~the correctness of JIT compilation depends on the soundness of type
% inference, and (2)~concrete types and the ability of type inference to infer
% concrete types of expressions are conducive to optimizations.
% These observations emphasize that despite Julia being a dynamic language,
% subtyping and the type language are an integral part of the compiler
% and Julia's performance.
